\section{Multiplication}\label{sec 2.3}

\newcommand{\X}{\times}

\begin{note}
這節會直接使用已知的自然數加法還有\ order 的性質,例如我們不會再證明\ \(a + b + c = c + b + a\).

另外會定義自然數乘法。類似自然數加法是\ "iterated increment operation"(重複的增量運算),自然數乘法是\ "iterated addition"(重複的加法運算)。
\end{note}

\begin{definition}[Multiplication of natural numbers]\label{def 2.3.1} Let \(m\) be a natural number. To multiply zero to \(m\), we \emph{define} \(0 \X m := 0\). Now suppose inductively that we have defined how to multiply \(n\) to \(m\). Then we can multiply \(n\INC\) to \(m\) by defining \((n\INC) \X m\) := \((n \X m) + m\).
\end{definition}

\begin{note}
根據\ Definition \ref{def 2.3.1}
    \begin{align*}
    0 \X m & = & 0 \\
    1 \X m & = (0\INC) \X m = (0 \X m) + m = 0 + m = & m \\
    2 \X m & = (1\INC) \X m = (1 \X m) + m = 0 + m + m = & m + m
    \end{align*}
and so on;到此刻我們才正式定義了乘法跟加法的關聯,以及之後會證明的「乘對加有分配律」等等。
\end{note}

\begin{additional corollary}[on top of page 30] \label{ac 2.3.1}
The product of two natural numbers is a natural number.
\end{additional corollary}
\begin{proof}
Let \(m, n\) be natural numbers. We use induction on \(n\).

Base case: Let \(n = 0\). Then
\begin{align*}
    n \X m & = 0 \X m & \\
           & = m      & \text{by Definition \ref{def 2.3.1}}
\end{align*}
which is a natural number.

Inductive hypothesis: Suppose \(n \X m\) is a natural number. We have to prove \((n\INC) \X m\) is also a natural number. But \((n\INC) \X m = (n \X m) + m.\) by Definition \ref{def 2.3.1}, and \(n \X m\) by inductive hypothesis is a natural number, \(m\) is also a natural number, so the sum(by Additional Corollary \ref{ac 2.2.1} is a natural number. This closes the induction.
\end{proof}

\begin{note}
這邊會再多額外幾個\ Corollary,主要是為了證明\ Lemma \ref{lem 2.3.2},其證明「流程」跟加法交換律很類似。
\end{note}

\begin{additional corollary} \label{ac 2.3.2}
Let \(n\) be a natural number. Then \(n \X 0 = 0\). This can be compared to Lemma \ref{lem 2.2.2}.
\end{additional corollary}
\begin{note}
類似於\ Lemma \ref{lem 2.2.2},乘法的定義只有定義\ \(0 \X n\) 是什麼,並沒有定義\ \(n \X 0\) 是什麼,我們必須額外證明它等於\ 0。
\end{note}
\begin{proof}
We use induction on \(n\).

Base case: Let \(n = 0\). Then
\begin{align*}
    n \X 0 & = \BLUE{0} \X \GREEN{0} \\
           & = \BLUE{0} & \text{by Definition \ref{def 2.3.1}}
\end{align*}
where \(\GREEN{0}\) corresponds to \(m\) in the Definition \ref{def 2.3.1}.

Inductive hypothesis: Suppose \(n \X 0 = 0\). We have to prove \((n\INC) \X 0 = 0\). And
\begin{align*}
     & (n\INC) \X 0 \\
     & = (n \X 0) + 0 & \text{by Definition \ref{def 2.3.1}} \\
     & = 0 + 0 &  \text{by inductive hypothesis} \\
     & = 0 &  \text{by Section \ref{sec 2.2}.}
\end{align*}
This closes the induction.
\end{proof}

\begin{additional corollary} \label{ac 2.3.3}
Let \(n, m\) be natural numbers. Then \(n \X (m\INC)\) = \((n \X m) + n\). This can be compared to Lemma \ref{lem 2.2.3}.
\end{additional corollary}
\begin{note}
類似於\ Lemma \ref{lem 2.2.3},乘法的定義只有定義\ \((m\INC) \X n\) 是什麼,並沒有定義\ \(n \X (m\INC)\) 是什麼,我們必須額外證明它等於\ \((n \X m) + n\)。
\end{note}
\begin{proof}
We use induction on \(n\), fix \(m\).

Base case: Let \(n = 0\). Then
\begin{align*}
    \text{LHS} & = n \X (m\INC) \\
               & = 0 \X (m\INC) & \text{since n = 0} \\
               & = 0 & \text{by Definition \ref{def 2.3.1}}
\end{align*}
And
\begin{align*}
    \text{RHS} & = (n \X m) + n \\
               & = (0 \X m) + 0 & \text{since n = 0} \\
               & = 0 + 0 & \text{by Definition \ref{def 2.3.1}} \\
               & = 0 & \text{by Section \ref{sec 2.2}}
\end{align*}
So both LHS and RHS equal to \(0\), so they are equal to each other.

Inductive hypothesis: Suppose \(n \X (m\INC) = (n \X m) + n\). We have to prove
\[(n\INC) \X (m\INC) = ((n\INC) \X m) + (n\INC).\]
Then
\begin{align*}
    \text{LHS} & = (n\INC) \X (m\INC) \\
               & = (n \X (m\INC)) + (m\INC) & \text{by Definition \ref{def 2.3.1}} \\
               & = ((n \X m) + n) + (m\INC) & \text{by inductive hypothesis} \\
               & = (n \X m) + (n + (m\INC)) & \text{by Proposition \ref{prop 2.2.5}} \\
               & = (n \X m) + ((n\INC) + m) & \text{by Lemma \ref{lem 2.2.3} and Definition \ref{def 2.2.1}} \\
               & = (n \X m) + (m + n\INC) & \text{by Proposition \ref{prop 2.2.4}} \\
               & = ((n \X m) + m) + (n\INC) & \text{by Proposition \ref{prop 2.2.5}} \\
               & = ((n\INC) \X m) + (n\INC) & \text{by Definition \ref{def 2.3.1}} \\
               & = \text{RHS}
\end{align*}
This closes the induction.
\end{proof}.

\begin{lemma}[Multiplication is commutative]\label{lem 2.3.2}
Let \(n, m\) be natural numbers. Then \(n \X m = m \X n\).
\end{lemma}
\begin{proof}
We use induction on \(n\).

Base case: Let \(n = 0\). Then LHS \(= n \X m = 0 \X m = 0\) by Definition \ref{def 2.3.1}, RHS \(= m \X n = m \X 0 = 0\) by Additional Corollary \ref{ac 2.3.2}, so they are equal to \(0\) so equal to each other.

Inductive hypothesis: Suppose \(n \X m = m \X n\), we have to prove
\[(n\INC) \X m = m \X (n\INC).\]
And
\begin{align*}
    \text{RHS} & = m \X (n\INC) \\
               & = (m \X n) + m & \text{by Additional Corollary \ref{ac 2.3.3}} \\
               & = (n \X m) + m & \text{by inductive hypothesis} \\
               & = ((n\INC) \X m) & \text{by Definition \ref{def 2.3.1}} \\
               & = \text{LHS.}
\end{align*}
This closes the induction.
\end{proof}

\begin{note}
We will now abbreviate \(n \X m\) as \(nm\), and use the usual convention that \emph{multiplication takes precedence over addition}, thus for instance \(ab + c\) means \((a \X b)+ c\), not \(a \X (b + c)\).

We will also use the usual notational conventions of precedence for the other arithmetic operations when they are defined later, to save on using parentheses all the time.
\end{note}

\begin{lemma}[Positive natural numbers have no zero divisors]\label{lem 2.3.3}
Let \(n, m\) be natural numbers. Then \(n \X m = 0\) if and only if at least one of \(n, m\) is equal to zero. In particular, if \(n\) and \(m\) are both positive, then \(nm\) is also positive.
\end{lemma}
\begin{note}
Equivalent: \(n \X m\) is positive if and only if \(n\) and \(m\) are both positive.
\end{note}
\begin{proof}

\(\Longrightarrow\): Suppose \(n \X m\) is positive, i.e. is not zero. Suppose for the sake of contradiction that \(n\) or \(m\) is not positive, i.e. is zero.
\begin{enumerate}
    \item If \(n = 0\) then \(n \X m = 0 \X m = 0\) by Definition \ref{def 2.3.1}, which contradicts the supposition that \(n \X m\) is positive.
    \item If \(m = 0\) then
    \begin{align*}
        n \X m & = n \X 0 \\
               & = 0 \X n & \text{by Lemma \ref{lem 2.3.2}} \\
               & = 0 & \text{by Definition \ref{def 2.3.1}}
    \end{align*}
    again contradicting \(n \X m\) is positive.
\end{enumerate}
So both \(n, m\) are positive.

\(\Longleftarrow\): Suppose \(n, m\) are positive, i.e. are not zero. Then by Lemma \ref{lem 2.2.10} there exist natural numbers \(c, d\) such that \(n = c\INC\) and \(m = d\INC\). So
\begin{align*}
    n \X m & = (c\INC) \X (d\INC) \\
           & = (c \X (d\INC)) + d\INC & \text{by Definition \ref{def 2.3.1}} \\
           & = (c \X m) + m & \text{since \(m = d\INC\)} \\
\end{align*}
which is positive by Proposition \ref{prop 2.2.8} because \(m\) is positive and \(c \X m\) is a natural number.
\end{proof}

\begin{proposition}[Distributive law]\label{prop 2.3.4}
For any natural numbers \(a, b, c\), we have \(a(b + c)= ab + ac\) and \((b + c)a = ba + ca\).
\end{proposition}

\begin{proof}
Since multiplication is commutative by Lemma \ref{lem 2.3.2} we only need to show the first identity \(a(b + c)= ab + ac\). Or
\begin{align*}
    (b + c)a & = a(b + c) & \text{by Lemma \ref{lem 2.3.2}} \\
             & = ab + ac  & \text{to be proved} \\
             & = ba + ac  & \text{by Lemma \ref{lem 2.3.2}} \\
             & = ba + ca  & \text{by Lemma \ref{lem 2.3.2}}
\end{align*}

We keep \(a\) and \(b\) fixed, and use induction on \(c\).

Base case: Let \(c = 0\), i.e., a(b + 0) = ab + a0. Then
\begin{align*}
    \text{LHS} & = a(b + 0) \\
        & = ab & \text{by Lemma \ref{lem 2.2.2}}
\end{align*}
while
\begin{align*}
    \text{RHS} & = ab + a0 \\
        & = ab + 0 & \text{by Lemma \ref{ac 2.3.2}} \\
        & = ab & \text{by Lemma \ref{lem 2.2.2}}
\end{align*}
so we are done with the base case.

Inductive hypothesis: Suppose that \(a(b + c) = ab + ac\), we have to prove that
\[a(b + (c\INC)) = ab + a(c\INC)\].
And
\begin{align*}
    \text{LHS} & = a(b + (c\INC)) \\
               & = a((b + c)\INC) & \text{by Lemma \ref{lem 2.2.3}}\\
               & = a(b + c) + a & \text{by Additional Corollary \ref{ac 2.3.3}} \\
\end{align*}
while
\begin{align*}
    \text{RHS} & = ab + a(c\INC) \\
               & = ab + ac + a & \text{by Additional Corollary \ref{ac 2.3.3}} \\ 
               & = a(b + c)+ a & \text{by the induction hypothesis,} 
\end{align*}
so LHS = RHS and we can close the induction.
\end{proof}

\begin{proposition}[Multiplication is associative] \label{prop 2.3.5}
For any natural numbers \(a, b, c\), we have
\[(a \X b) \X c = a \X (b \X c)\]
\end{proposition}

\begin{proof}
We use induction on \(c\).

Base case: Let \(c = 0\), then
\begin{align*}
    \text{LHS} & = (a \X b) \X c \\
               & = (a \X b) \X 0 \\
               & = 0 & \text{by Additional Corollary \ref{ac 2.3.2}}
\end{align*}
\begin{align*}
    \text{RHS} & = a \X (b \X c) \\
               & = a \X (b \X 0) \\
               & = a \X 0 & \text{by Additional Corollary \ref{ac 2.3.2}} \\
               & = 0 & \text{by Additional Corollary \ref{ac 2.3.2}}
\end{align*}
So LHS = RHS.

Inductive hypothesis: Suppose \((a \X b) \X c = a \X (b \X c)\), we have to prove
\[(a \X b) \X (c\INC) = a \X (b \X (c\INC))\]
Then
\begin{align*}
    \text{LHS} & = (a \X b) \X (c\INC) \\
               & = ((a \X b) \X c) + (a \X b) & \text{by Additional Corollary \ref{ac 2.3.3}} \\
\end{align*}
\begin{align*}
    \text{RHS} & = a \X (b \X (c\INC)) \\
               & = a \X ((b \X c) + b) & \text{by Additional Corollary \ref{ac 2.3.3}} \\
               & = (a \X (b \X c)) + (a \X b) & \text{by Proposition \ref{prop 2.3.4}} \\
               & = ((a \X b) \X c) + (a \X b) & \text{by the inductive hypothesis}.
\end{align*}
So LHS = RHS, and we can close the induction.
\end{proof}

\begin{proposition}[Multiplication preserves order]\label{prop 2.3.6}
If \(a, b\) are natural numbers such that \(a < b\), and \(c\) is positive, then \(ac < bc\).
\end{proposition}
\begin{proof}
Since \(a < b\), we have \(b = a + d\) for some \emph{positive} \(d\). Multiplying by \(c\) and using the distributive law we obtain \(bc = ac + dc\). Since \(d\) is positive, and \(c\) is positive, \(dc\) is positive, and hence \(ac < bc\) as desired.
\end{proof}

\begin{corollary}[Cancellation Law (for multiplication)] \label{corollary 2.3.7}
Let (a, b, c) be natural numbers such that \(ac = bc\) and \(c\) is non-zero(i.e. positive). Then \(a = b\).
\end{corollary}

\begin{remark} \label{remark 2.3.8}
Just as Proposition \ref{prop 2.2.6} will allow for a “virtual subtraction” which will eventually let us define genuine subtraction, this corollary provides a “virtual division” which will be needed to define genuine division later on.
\end{remark}
\begin{note}
我們不能用任何「除法」或者是「有理數」的概念來證明自然數乘法消去法。事實上這本書就是用這個乘法消去法來定義「虛擬除法」(virtual division),是一種「形式上」(formal)的除法,然後會進一步證明這跟有理數除法等價。可參考第四章\ Definition \ref{def 4.2.1}。
\end{note}
\begin{proof}
By the trichotomy of order (Proposition \ref{prop 2.2.13}), we have three cases: \(a < b\), \(a = b\), \(a > b\). Suppose that \(a < b\), then by Proposition \ref{prop 2.3.6} we have \(ac < bc\), a contradiction. Suppose that \(a > b\), that is, \(b < a\), then by Proposition \ref{prop 2.3.6} we have \(bc < ac\), again a contradiction. Thus \(a = b\) must be true, as desired.
\end{proof}

\begin{note}
到目前為止的\ propositions 已經可以讓我們推出各種常見的\ algebra equations 了,例如\ Exercise \ref{exercise 2.3.4} (\((a + b)^2 = a^2 +2ab + b^2\) for all natural numbers \(a, b\).) 另外從現在開始也不太會再直接用整數的\ increment 了,因為太原始(primitive),例如直接寫\ \(n + 1\) 取代\ \(n\INC\)。
\end{note}