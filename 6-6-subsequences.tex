\section{Subsequences} \label{sec 6.6}

Some sequences were convergent to a single limit, while others had multiple limit points.
For instance, the sequence
\[
    1.1, 0.1, 1.01, 0.01, 1.001, 0.001, 1.0001,...
\]
has two limit points at \(0\) and \(1\) (which are incidentally also the \(\liminf\) and \(\limsup\) respectively),
but is not actually convergent (since \(0\) and \(1\) are not equal, and by (contrapositive of) \PROP{6.4.5}).
However, while this sequence is not convergent, it does appear to contain \emph{convergent components};
it seems to be \emph{a mixture of} two convergent subsequences, namely
\[
    1.1, 1.01, 1.001,...
\]
and
\[
    0.1, 0.01, 0.001,....
\]
To make this notion more precise, we need a notion of `\emph{subsequence}.

\begin{definition} [Subsequences] \label{def 6.6.1}
Let \((a_n)_{n = 0}^{\infty}\) and \((b_n)_{n = 0}^{\infty}\) be sequences of real numbers.
We say that \((b_n)_{n = 0}^{\infty}\) is a \emph{subsequence} of \((a_n)_{n = 0}^{\infty}\)
iff there exists a function \(f : \SET{N} \rightarrow \SET{N}\) which is \emph{strictly increasing}
(i.e., \(f(n +1 ) > f(n)\) for all \(n \in \SET{N}\)) such that
\begin{center}
    \(b_n = a_{f(n)}\) for all \(n \in \SET{N}\).
\end{center}

More generally, we say that \((b_n)_{n=m'}^{\infty}\) is a subsequence of \((a_n)_{n=m}^{\infty}\) if there exists a strictly increasing function \(f: \{ n \in \SET{N}: n \ge m'\} \to \{ n \in \SET{N}: n \ge m\}\) such that \(b_n = a_{f(n)}\) for all \(n \ge m'\).
\end{definition}

\begin{note}
最後一個\ general case 是從\ Errata 補的;
目的是定義從任何\ index 起始的兩個\ sequence 之間的\ subsequence 關係。
\end{note}

\begin{example} \label{example 6.6.2}
If \((a_n)_{n = 0}^{\infty}\) is a sequence, then \((a_{2n})_{n = 0}^{\infty}\) is a subsequence of \((a_n)_{n = 0}^{\infty}\),
since the function \(f : \SET{N} \rightarrow \SET{N}\) defined by \(f(n) := 2n\) is a strictly increasing function from \(\SET{N}\) to \(\SET{N}\).
Note that we do not assume \(f\) to be bijective, although it is necessarily injective.
Why? Because ``strictly increasing'' implies injective.
Recall that injectivity: \(n \ne m \implies f(n) \ne f(m)\).
So let natural numbers \(n \ne m\).
Then by trichotomy, either \(n < m\) or \(n > m\).
If \(n < m\) then since \(f\) is strictly increasing, \(f(n) < f(m)\);
If \(n > m\) then since \(f\) is strictly increasing, \(f(m) < f(n)\);
so in all cases \(f(n) \ne f(m)\), as desired.

More informally, the sequence
\[
    a_0, a_2, a_4, a_6,...
\]
is a subsequence of
\[
    a_0, a_1, a_2, a_3, a_4,....
\]
\end{example}

\begin{example} \label{example 6.6.3}
The two sequences
\[
    1.1, 1.01, 1.001,...
\]
and
\[
    0.1, 0.01, 0.001,...
\]
mentioned earlier are both subsequences of
\[
    1.1, 0.1, 1.01, 0.01, 1.001, 1.0001,...
\]
\end{example}

The property of being a subsequence is reflexive and transitive, though not symmetric:
\begin{lemma} \label{lem 6.6.4}
Let \((a_n)_{n = 0}^{\infty}\), \((b_n)_{n = 0}^{\infty}\), and \((c_n)_{n = 0}^{\infty}\) be sequences of real numbers.
Then \((a_n)_{n = 0}^{\infty}\) is a subsequence of \((a_n)_{n = 0}^{\infty}\).
Furthermore, if \((b_n)_{n = 0}^{\infty}\) is a subsequence of \((a_n)_{n = 0}^{\infty}\), and \((c_n)_{n = 0}^{\infty}\) is a subsequence of \((b_n)_{n = 0}^{\infty}\),
then \((c_n)_{n = 0}^{\infty}\) is a subsequence of \((a_n)_{n = 0}^{\infty}\).
\end{lemma}

\begin{proof}
For reflexivity, just let \(f(n) = n\).
For transitivity, suppose \((b_n)_{n = 0}^{\infty}\) is a subsequence of \((a_n)_{n = 0}^{\infty}\), and \((c_n)_{n = 0}^{\infty}\) is a subsequence of \((b_n)_{n = 0}^{\infty}\).
Then there exists
\begin{itemize}
    \item[\MAROON{(1)}] \(f : \SET{N} \rightarrow \SET{N}\) s.t. \(f\) is strictly increasing and \(b_n = a_{f(n)}\) for all \(n \in \SET{N}\).
    \item[\MAROON{(2)}] \(g : \SET{N} \rightarrow \SET{N}\) s.t. \(g\) is strictly increasing and \(c_n = b_{g(n)}\) for all \(n \in \SET{N}\).
\end{itemize}

Now we claim that the composition function \(f \circ g\) is what we need: \(f \circ g\) is strictly increasing, and \(c_n = a_{(f \circ g)(n)}\).

Now suppose natural numbers \(n > m\).
Then
\begin{align*}
             & n > m \\
    \implies & g(n) > g(m) & \text{since \(g\) is strictly increasing} \\
    \implies & f(g(n)) > f(g(m)) & \text{since \(f\) is strictly increasing} \\
    \implies & (f \circ g)(n) > (f \circ g)(m) & \text{by \DEF{3.3.10}}
\end{align*}

Also, given any \(n \in \SET{n}\),
\begin{align*}
    c_n & = b_{g(n)} & \text{by \MAROON{(2)}} \\
        & = a_{f(g(n))} & \text{by \MAROON{(1)}} \\
        & = a_{(f \circ g)(n)} & \text{by \DEF{3.3.10}}
\end{align*}
\end{proof}

We now relate the concept of subsequences to the concept of limits and limit points.
\begin{proposition} [Subsequences related to \textbf{limits}] \label{prop 6.6.5}
Let \((a_n)_{n = 0}^{\infty}\) be a sequence of real numbers, and let \(L\) be a real number.
Then the following two statements are \emph{logically equivalent} (each one implies the other):
\begin{enumerate}
    \item The sequence \((a_n)_{n = 0}^{\infty}\) \textbf{converges to} \(L\)
    \item \textbf{Every} subsequence of \((a_n)_{n = 0}^{\infty}\) converges to \(L\).
\end{enumerate}
\end{proposition}

\begin{proof}
Suppose part(a) is true.
And suppose \((b_n)_{n = 0}^{\infty}\) is an arbitrary subsequence of \((a_n)_{n = 0}^{\infty}\), where \(b_n = a_{f(n)}\) with \(f\) being a strictly increasing function from \(\SET{N}\) to \(\SET{N}\) \MAROON{(1)}.
Let arbitrary \(\varE > 0\), we have to show that there exists \(N \ge 0\) s.t. \(\abs{b_n - L} \le \varE\) for all \(n \ge N\).

Now since \((a_n)_{n = 0}^{\infty}\) converges to \(L\), we can found that \(N_a \ge 0\) s.t. \(\abs{a_n - L} \le \varE\) for all \(n \ge N_a\) \MAROON{(2)}.
Also, there must exist \(N_b \ge 0\) s.t. \(f(n) \ge N_a\) for all \(n \ge N_b\) \MAROON{(3)}, otherwise \(f(n) < N_a\) for all \(n \in \SET{N}\), which, with the similar reason in \RMK{6.6.9}, implies \(f\) is not a strictly increasing function from \(\SET{N}\) to \(\SET{N}\).
So we have found \(N_b \ge 0\) s.t. for all \(n \ge N_b\), we have
\begin{align*}
             & f(n) \ge N_a & \text{by \MAROON{(3)}} \\
    \implies & \abs{a_{f(n)} - L} \le \varE & \text{by \MAROON{(2)}} \\
    \implies & \abs{b_n - L} \le \varE & \text{by \MAROON{(1)}},
\end{align*}
as desired.

Now suppose part(b) is true.
Then in particular by reflexivity \LEM{6.6.4}, \((a_n)_{n = 0}^{\infty}\) is a subsequence of itself, so \((a_n)_{n = 0}^{\infty}\) converges to \(L\).
\end{proof}

\begin{proposition}  [Subsequences related to \textbf{limit points}] \label{prop 6.6.6}
Let \((a_n)_{n = 0}^{\infty}\) be a sequence of real numbers, and let \(L\) be a real number.
Then the following two statements are logically equivalent.
\begin{enumerate}
    \item \(L\) is a \textbf{limit point} of \((a_n)_{n = 0}^{\infty}\).
    \item \textbf{There exists} a subsequence of \((a_n)_{n = 0}^{\infty}\) which converges to \(L\).
\end{enumerate}
\end{proposition}

\begin{proof}
See \EXEC{6.6.5}.
\end{proof}

\begin{remark} \label{remark 6.6.7}
The above two propositions give a \emph{sharp contrast} between the notion of a limit, and that of a limit point.
When a sequence has a limit \(L\), then \textbf{all subsequences} also converge to \(L\).
But when a sequence has \(L\) as a limit point, then only \textbf{some subsequences} converge to \(L\).
\end{remark}

We can now prove an important \emph{theorem} in real analysis, due to Bernard Bolzano (1781 -- 1848) and Karl Weierstrass (1815 --1897):
every \emph{bounded} sequence has a convergent \emph{sub}sequence.

\begin{theorem} [Bolzano-Weierstrass theorem] \label{thm 6.6.8}
Let \((a_n)_{n = 0}^{\infty}\) be a \emph{bounded} sequence
(i.e., there exists a real number \(M > 0\) such that \(\abs{a_n} \le M\) for all \(n \in \SET{N}\)).
Then there is at least one subsequence of \((a_n)_{n = 0}^{\infty}\) which converges.
\end{theorem}

\begin{proof}
Let \(L\) be the \emph{limit superior} of the sequence \((a_n)_{n = 0}^{\infty}\).
Since we have \(-M \le a_n \le M\) for all natural numbers \(n\), it follows from the comparison principle (\LEM{6.4.13}) that \(-M \le L \le M\).
(\(-M = \limsup_{n \toINF} -M \le \limsup_{n \toINF} a_n\) and \( \limsup_{n \toINF} a_n \le \limsup_{n \toINF} M = M\).)
In particular, \(L\) is a real number (not \(+\infty\) or \(-\infty\)).
By \PROP{6.4.12}(e), \(L\) is thus a limit point of \((a_n)_{n = 0}^{\infty}\).
Thus by \PROP{6.6.6}, there exists a subsequence of \((a_n)_{n = 0}^{\infty}\) which converges (to \(L\)).
\end{proof}

\begin{remark} \label{remark 6.6.9}
The Bolzano-Weierstrass theorem says that if a sequence is \emph{bounded}, then eventually it has no choice but to ``converge'' in some places(limit points);
it has ``no room'' to spread out and stop itself from acquiring limit points.
It is not true for unbounded sequences; for instance, the sequence \(1, 2, 3,...\) has no convergent subsequences whatsoever.
Why? Suppose there is a convergent subsequence \((a_n)_{n = 1}^{\infty}\) of \(1, 2, 3,...\).
First, any subsequence of \(1, 2, 3,...\) is also strictly increasing(trivial) \MAROON{(1)}.
And since \(1, 2, 3,...\) is a sequence of positive integers, \((a_n)_{n = 1}^{\infty}\) is also a sequence of positive integers \MAROON{(2)}.
And since \((a_n)_{n = 1}^{\infty}\) is convergent, it is \(0.5\)-steady; so there exists \(N \ge 1\) s.t. \(\abs{a_i - a_j} \le 0.5\) for all \(i, j \ge N\).
But by \MAROON{(2)} \(a_i, a_j\) in particular are integers, that must implies \(a_i - a_j = 0\), that is, \(a_i = a_j\), for all \(i, j \ge 1\), otherwise their difference is greater than or equal to \(1\).
But that implies \((a_n)_{n = 1}^{\infty}\) is in fact not a subsequence because it violates \MAROON{(1)}, so we have a contradiction.
So the sequence \(1, 2, 3,...\) has no convergent subsequences.

In the language of topology, this means that the interval \(\{ x \in \SET{R} : -M \le x \le M \}\) is \emph{compact}, whereas an unbounded set such as the real line \(\SET{R}\) is not compact.
The distinction between compact sets and non-compact sets will be very important in later chapters - of similar importance to the distinction between finite sets and infinite sets.
\end{remark}

\exercisesection

\begin{exercise} \label{exercise 6.6.1}
Prove \LEM{6.6.4}.
\end{exercise}

\begin{proof}
See \LEM{6.6.4}.
\end{proof}

\begin{exercise} \label{exercise 6.6.2}
Can you find two sequences \((a_n)_{n = 0}^{\infty}\) and \((b_n)_{n = 0}^{\infty}\) which are \emph{not} the same sequence, but such that each is a subsequence of the other?
\end{exercise}

\begin{proof}
Just let \((a_n)_{n = 0}^{\infty} = 1, -1, 1, -1, 1, -1,...\) and \((b_n)_{n = 0}^{\infty} = -1, 1, -1, 1, -1, 1,...\).
Then first, \(f(n) = n + 1\) is a strictly increasing function.
And both \(b_n = a_{f(n)} = a_{n + 1}\), and \(a_n = b_{f(n)} = b_{n + 1}\), for all \(n\).
So by \DEF{6.6.1}, the two sequences are subsequences of each other.

Also refer to \href{https://math.stackexchange.com/questions/3677059}{this}
\end{proof}

\begin{exercise} \label{exercise 6.6.3}
Let \((a_n)_{n = 0}^{\infty}\) be a sequence which is \emph{not} bounded.
Show that there exists a subsequence \((b_n)_{n = 0}^{\infty}\) of \((a_n)_{n = 0}^{\infty}\) such that \(\lim_{n \toINF} 1/b_n\) exists and is equal to zero.
(Hint: for each natural number \(j\), \emph{recursively} introduce the quantity \(n_j := \min\{n \in \SET{N} : \abs{a_n} \ge j; n > n_{j - 1}\}\)
(omitting the condition \(n > n_{j - 1}\) when \(j = 0\)), 
first explaining why the set \(\{n \in \SET{N} : \abs{a_n} \ge j; n > n_{j - 1}\}\) is non-empty.
Then set \(b_j := a_{n_j}\).)
\end{exercise}

\begin{proof}
Let \(f: \SET{N} \rightarrow \SET{N}\) be a function such that \(f(j) = n_j\) where
\begin{equation} \label{eq 6.2}
    n_j = 
    \begin{cases}
        \min\{n \in \SET{N} : \abs{a_n}\ \RED{>}\ 0\} \text{, if } j = 0 \\
        \min\{n \in \SET{N} : (\abs{a_n}\ \RED{>}\ j) \land (n > n_{j - 1})\} \text{, if } j \ge 1
    \end{cases}
\end{equation}
We have to show \(f\) is well-defined, and show that the subsequence defined by \(f\) meets the requirement.

Since \((a_n)_{n = 0}^{\infty}\) is unbounded, for any \(j \in \SET{N}\) (and any integer \(n_{j - 1}\), when \(j \ge 1\)) we can find another integer \(n\) s.t. \(\abs{a_n} > j\) and \(n > n_{j - 1}\);
for if we cannot find such integer, then it implies that for every \(n > n_{j - 1}\), \(\abs{a_n} \le j\), which implies \((a_n)_{n = 0}^{\infty}\) is bounded!
So in particular, for each \(j \in \SET{N}\), the introduced set applied by \(\min\) operator in \EQUATION{6.2} is \textbf{non-empty}.
Also, since the non-empty set is \textbf{the set of natural numbers}, by well-ordering-principle,
(well, although it is introduced in \PROP{8.1.4}, everyone struggling until now can find a wiki or something similar and know the principle is equivalent to mathematical induction) the minimum element exists.
(Note that if the non-empty set is a set of real numbers, then the ``minimum'' may not exist, although by \THM{5.5.9} the supremum/infimum exist.)
So given any \(j \in \SET{N}\), the corresponding minimum of the set in \EQUATION{6.2} exists, so \(f\) is well-defined.

Now again, from the definition of \EQUATION{6.2}, we have \(j_1 > j_2 \implies n_{j_1} > n_{j_2}\), so \(f\) is strictly increasing, so we can use \(f\) to define a subsequence.
We define \((b_n)_{n = 0}^{\infty}\) where \(b_n = a_{f(n)}\).
Then by \EQUATION{6.2}, we have \(\abs{a_{f(j)}} = \abs{b_j} > j\), which implies \(0 \le 1/\abs{b_j} < 1/j\), which trivially implies \(0 \le \abs{1/b_j} < 1/j\).
Since \(\lim_{j \toINF} 0 = 0\) and \(\lim_{j \toINF} 1/j = 0\), by squeeze test, \CORO{6.4.14}, we have \(\lim_{j \toINF} \abs{1/b_j} = 0\).
And by zero test, \CORO{6.4.17}, we have \(\lim_{n \toINF} 1/b_n = 0\), as desired.
\end{proof}

\begin{exercise} \label{exercise 6.6.4}
Prove \PROP{6.6.5}.
(Note that one of the two implications has a very short proof.)
\end{exercise}

\begin{proof}
See \PROP{6.6.5}.
\end{proof}

\begin{exercise} \label{exercise 6.6.5}
Prove \PROP{6.6.6}.
(Hint: to show that (a) implies (b), define the numbers \(n_j\) for each \emph{natural} numbers \(j\) by the formula \(n_j := \min \{ n > n_{j - 1} : \abs{a_n - L} \le 1/j \} \), with the convention \(n_0 := 0\),
explaining why the set \(\{ n > n_{j - 1} : \abs{a_n - L} \le 1/j \}\) is non-empty.
Then consider the sequence \(a_{n_j}\).)
\end{exercise}

\begin{proof}
Suppose \(L\) is a limit point of \((a_n)_{n = 0}^{\infty}\).
We have to find a subsequence \((b_n)_{n = 0}^{\infty}\) of \((a_n)_{n = 0}^{\infty}\) s.t. \((b_n)_{n = 0}^{\infty}\) converges to \(L\).
Now let \(f: \SET{N} \rightarrow \SET{N}\) be a function such that \(f(j) = n_j\) where
\begin{equation} \label{eq 6.3}
    n_j = 
    \begin{cases}
        0 \text{, if } j = 0 \\
        \min\{n > n_{j - 1} \land \abs{a_n - L } \le 1/j\} \text{, if } j \ge 1
    \end{cases}
\end{equation}
First, given arbitrary \(j \in \SET{N}\) such that \(j \ge 1\), we have \(1/j > 0\).
And since \(L\) is a limit point of \((a_n)_{n = 0}^{\infty}\),
by \DEF{6.4.1}(3), given \(n_{j - 1} \in \SET{N}\), we can find an integer \(n \ge n_{j - 1} + 1\), which \(> n_{j - 1}\), s.t. \(\abs{a_n - L} \le 1/j\).
And that implies the set of the \(\min\) operator in \EQUATION{6.3} is non-empty;
And similarly with \EXEC{6.6.4}, since the non-empty set is a set of natural numbers, by well-ordering principle, the minimum of the set exists.
Hence the \EQUATION{6.3} is well-defined.
Now again, from the definition of \EQUATION{6.3}, we have \(j_1 > j_2 \implies n_{j_1} > n_{j_2}\), so \(f\) is strictly increasing, so we can use \(f\) to define a subsequence.
We define \((b_n)_{n = 0}^{\infty}\) where \(b_n = a_{f(n)}\).
Now we show that \((b_n)_{n = 0}^{\infty}\) converges to \(L\).
So let arbitrary \(\varE > 0\).
In particular by \EXEC{5.4.4}, we can find \(j \in \SET{N}\) s.t. \(1/j < \varE\).
And from \EQUATION{6.3}, we have \(\abs{b_j - L} = \abs{a_{n_j} - L} \le 1/j\).
Also for any \(j' \ge j\), from \EQUATION{6.3} we have \(\abs{b_{j'} - L} = \abs{a_{n_{j'}} - L} \le 1/{j'} \le 1/{j}\).
All in all, we have found \(j \in \SET{N}\) s.t. \(\abs{b_{j'} - L} \le 1/j < \varE\) for all \(j' \ge j\).
So \((b_n)_{n = 0}^{\infty}\) converges to \(L\).

Now suppose there exists a subsequence \((b_n)_{n = 0}^{\infty}\) of \((a_n)_{n = 0}^{\infty}\) which converges to \(L\).
We have to show that \(L\) is a limit point of \((a_n)_{n = 0}^{\infty}\).
Let \(f\) be the strictly increasing function that is used to define \((b_n)_{n = 0}^{\infty}\).
And let arbitrary real number \(\varE > 0\), arbitrary natural number \(N \ge 0\).
We have to show that there exists \(n \ge N\) s.t. \(\abs{a_n - L} \le \varE\) \MAROON{(1)}, to show \(L\) is a limit point of \((a_n)_{n = 0}^{\infty}\).
Since \((b_n)_{n = 0}^{\infty}\) converges to \(L\), we can find an integer \(n_b \ge 0\) s.t. \(\abs{b_n - L} \le \varE\) for all \(n \ge n_b\) \MAROON{(2)}.
In particular, \(\abs{a_{f(n)} - L} \le \varE\) for all \(n \ge n_b\).
But since \(f\) is strictly increasing function from \(\SET{N}\) to \(\SET{N}\), we can find \(n_a \ge n_b\) s.t. \(f(n_a) \ge N\) (otherwise we will get a contradiction that \(f\) is not strictly increasing).
So we have found \(f(n_a) \ge N\), (where \(n_a \ge n_b\)) s.t.
\begin{align*}
             & f(n_a) \ge N \land n_a \ge n_b \\
    \implies & f(n_a) \ge N \land \abs{b_{n_a} - L} \le \varE & \text{by \MAROON{(2)}} \\
    \implies & f(n_a) \ge N \land \abs{a_{f(n_a)} - L} \le \varE
\end{align*}
So \MAROON{(1)} is satisfied, as desired.
\end{proof}