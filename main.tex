\documentclass[12pt]{book}
\usepackage{xspace}
\usepackage[T1]{fontenc}
\usepackage[margin=2cm]{geometry}
\usepackage{amsthm}
\usepackage{amssymb}
\usepackage{amsmath}
\usepackage{mathtools}
\usepackage{dirtytalk}

\usepackage[dvipsnames]{xcolor}

\usepackage{parskip}

\usepackage{CJKutf8} % 讓我寫中文

\usepackage{hyperref}
\hypersetup{
    colorlinks=true,
    linkcolor=blue,
    filecolor=magenta,
    urlcolor=cyan,
}
\urlstyle{same}

\title{Analysis I}
\author{ChienChiWang}

\begin{document}
\begin{CJK*}{UTF8}{bsmi}

\maketitle

\tableofcontents{}

\theoremstyle{definition}
\newtheorem{axiom}{Axiom}[chapter]
\newtheorem{assumption}[axiom]{Assumption}
\newtheorem{additional corollary}{Additional Corollary}[section]
\newtheorem{exercise}{Exercise}[section]
\newtheorem{theorem}{Theorem}[section]
\newtheorem{corollary}[theorem]{Corollary}
\newtheorem{definition}[theorem]{Definition}
\newtheorem{example}[theorem]{Example}
\newtheorem{lemma}[theorem]{Lemma}
\newtheorem{proposition}[theorem]{Proposition}
\newtheorem{remark}[theorem]{Remark}
\newtheorem*{note}{Note}

\theoremstyle{remark}
\newtheorem*{meta-proof}{Meta-proof}

\renewcommand{\labelenumi}{\textnormal{(}\alph{enumi}\textnormal{)}}

\DeclarePairedDelimiter\abs{\lvert}{\rvert}
\DeclarePairedDelimiter\ceil{\lceil}{\rceil}
\DeclarePairedDelimiter\floor{\lfloor}{\rfloor}

\newcommand{\exercisesection}{
    \begin{center}
        --- Exercises ---
    \end{center}
}

\newcommand{\AXM}[1]{Axiom \ref{axm #1}}
\newcommand{\AC}[1]{Additional Corollary \ref{ac #1}}
\newcommand{\DEF}[1]{Definition \ref{def #1}}
\newcommand{\LEM}[1]{Lemma \ref{lem #1}}
\newcommand{\PROP}[1]{Proposition \ref{prop #1}}
\newcommand{\CORO}[1]{Corollary \ref{corollary #1}}
\newcommand{\EXAMPLE}[1]{Example \ref{example #1}}
\newcommand{\EXEC}[1]{Exercise \ref{exercise #1}}
\newcommand{\RMK}[1]{Remark \ref{remark #1}}
\newcommand{\THM}[1]{Theorem \ref{thm #1}}
\newcommand{\SEC}[1]{Section \ref{sec #1}}
\newcommand{\CH}[1]{Chapter \ref{ch #1}}

\newcommand{\INC}{\texttt{++}}
\newcommand{\BLUE}[1]{\textcolor{RoyalBlue}{#1}}
\newcommand{\GREEN}[1]{\textcolor{YellowGreen}{#1}}
\newcommand{\MAROON}[1]{\textcolor{Maroon}{#1}}
\newcommand{\RED}[1]{\textcolor{Red}{#1}}

\newcommand{\SET}[1]{\mathbb{#1}}%

\begin{note}
The various colors in proofs are just for the notational purpose.
\end{note}

\chapter{Introlduction}\label{ch 1}
\begin{note}
This chapter gives some motivation about why we need to rigorously derive various properties we have (blindly) used, and gives example of improper usage of these properties, leading to nonsense.
\end{note}
\chapter{Natural Numbers}\label{ch 2}

\section{The Peano axioms}\label{sec 2.1}

\begin{axiom}\label{axm 2.1}
\( 0 \) is a natural number.
\end{axiom}
\begin{axiom}\label{axm 2.2}
If \( n \) is a natural number, then \( n\INC \) is also a natural number
\end{axiom}

\setcounter{theorem}{2}
\begin{definition}\label{def 2.1.3}
We define \( 1 \) to be the number \( 0\INC \), \(2\) to be the number \( (0\INC)\INC \), 3 to be the number \( ((0\INC)\INC)\INC \), etc.
In other words, \( 1 := 0\INC \), \( 2 := 1\INC \), \( 3 := 2\INC \), etc.
\end{definition}

\begin{note}
In this text the author uses ``\(x := y\)'' to denote the statement that \(x\) is \emph{defined} to equal \(y\).
\end{note}

\begin{proposition}\label{prop 2.1.4}
\(3\) is a natural number.
\end{proposition}
\begin{proof}
\( 3 = 2\INC = (1\INC)\INC = ((0\INC)\INC)\INC \) by definition of \(3\), \(2\) and \(1\). And by \AXM{2.1}, \(0\) is a natural number, so by \AXM{2.2}, \(0\INC\) is a natural number, and again by \AXM{2.2}, \((0\INC)\INC\) is a natural number, and again by \AXM{2.2}, \( ((0\INC)\INC)\INC \), which is equal to \(3\), is a natural number, so \(3\) is a natural number.
\end{proof}

\begin{example}\label{example 2.1.5}
這裡就在說明一個會\ \emph{wrap around} 的\ number system (例如 32-bit 整數, 或者時鐘上的數字) 符合\ Axiom \ref{axm 2.1}\ 跟\ Axiom \ref{axm 2.2},但不會是我們想要的\ number system。
\end{example}

\begin{axiom}\label{axm 2.3}
\(0\) is not the successor of any natural number; i.e. we have \(n\INC \neq 0\) for every natural number \(n\)
\end{axiom}

\begin{proposition}\label{prop 2.1.6}
\(4\) is not equal to \(0\)
\end{proposition}
\begin{proof}
By definition, 4 = 3\INC. By Proposition \ref{prop 2.1.4} we know 3 is a natural number, and by \AXM{2.3} \(3\INC \neq 0\), so \(4 = 3\INC \neq 0\).
\end{proof}

\begin{example}\label{example 2.1.7}
這裡又舉兩個 number system,符合\ Axiom \ref{axm 2.1} - \ref{axm 2.3},但是一樣不是自然數。第一個是\ \(\{0,1,2,3,4\}\),但\ \(4\INC = 4\)。另一種情形是 \(4\INC = 1\),沒有\ wrap around 回\ 0。
\end{example}

\begin{axiom}\label{axm 2.4}
Different natural numbers must have different successors; i.e. if \(n\), \(m\) are natural numbers and \(n \neq m\), then \(n\INC \neq m\INC\). \emph{Contrapositively}, if \(n\INC = m\INC\), then \(n = m\).
\end{axiom}

\begin{note}
Axiom \ref{axm 2.4} 的\ \emph{converse}\ 也是成立的,i.e. if \(n = m\), then \(n++ = m++\)。但那是因為整數的``等號''滿足\ Axiom of substitution \ref{axm a.7.4}。但是\dots書上好像沒有定義整數的等號是什麼意思,就參考\ \href{https://www.wikiwand.com/en/Peano_axioms#/Formulation}{wiki} 吧
\end{note}

\begin{proposition}\label{prop 2.1.8}
\(6\) is not equal to \(2\).
\end{proposition}
\begin{proof}
Suppose \(6 = 2\), then \(5\INC = 6 = 2 = 1\INC\), so \(5\INC = 1\INC \), so by \AXM{2.4}, \(5 = 1\). Similarly, \(4\INC = 0\INC\), so by \AXM{2.4}, \(4 = 0\). But this contradicts Proposition \ref{prop 2.1.6}, so \(6 \neq 2\).
\end{proof}

\begin{example}\label{example 2.1.9}(Informal, 因為實際上用到了實數)
這題令\ \( \textup{N} = \{0, 0.5, 1, 1.5, 2, 2.5, ...\}\),則\ \(\textup{N}\) 滿足\ Axiom \ref{axm 2.1}-\ref{axm 2.4},但他也不是自然數。
\end{example}

\begin{axiom}\label{axm 2.5} (Principle of mathematical induction). Let \(P(n)\) be any property pertaining to a natural number \(n\). Suppose that \(P(0)\) is true, and suppose that whenever \(P(n)\) is true, \(P(n\INC)\) is also true. Then \(P(n)\) is true for every natural number \(n\).
\end{axiom}

\begin{remark}\label{rmk 2.1.10} Axiom \ref{axm 2.5} 的敘述其實用了不少尚未定義的名詞,例如何謂\ property? 其實就是數理邏輯說的\ predicate。另外\ Axiom \ref{axm 2.5} 實際上是\ ``\href{https://www.wikiwand.com/en/Axiom_schema}{Axiom schema}'',有點像一種模板,可以生出任一個\  axiom。這必須去研究邏輯學。
\end{remark}

\begin{note}
就算有了\ Axiom \ref{axm 2.1}-\ref{axm 2.5},目前還是無法證明\  \ref{example 2.1.9} 的例子不是自然數;在目前這個階段,嚴格來說\ \ref{example 2.1.9} 對我們甚至根本就是\ ill-formed。
\end{note}

\begin{proposition}\label{prop 2.1.11}
這題根本就只是數學歸納法證明時的模板而已。
\end{proposition}

\begin{note}
除了\ Proposition \ref{prop 2.1.11} 的流程,數學歸納法還有變形,如\ backward induction (看\  \EXEC{2.2.6})、strong induction (看\ \PROP{2.2.14})、還有\ transfinite induction(看\  \LEM{8.5.15});BTW 最後一個根本不是數學歸納法...
\end{note}

\begin{assumption}\label{assumption 2.6} (Informal)
There exists a number system \(\SET{N}\), whose elements we will call natural numbers, for which Axiom \ref{axm 2.1}-\ref{axm 2.5} are true.
\end{assumption}
\begin{note}
Assumption \ref{assumption 2.6} 的嚴謹定義實際上是\ Axiom \ref{axm 3.7},這需要理解集合以及函數的定義後才有意義。這個\ Assumption 的意義是保證存在一個\ ``set'',滿足 Axiom \ref{axm 2.1}-\ref{axm 2.5},也就是皮亞諾公理。
\end{note}

\begin{remark}\label{remark 2.1.12} Assumption \ref{assumption 2.6} 的\ number system \(\SET{N}\) 我們稱作\ \emph{the} natural number system; i.e. 他是唯一的。其他的符合\ Axiom \ref{axm 2.1}-\ref{axm 2.5} 的\ number system 實際上都跟\ \(\SET{N}\) \emph{同構},請看\  \EXEC{3.5.13}。這邊的同構是指兩個\ set 存在\ 1-1 correspondence,並且這個\ 1-1 correspondence \emph{preserve} \INC\ operation。
\end{remark}

\begin{remark}\label{remark 2.1.13}
這也是幹話,所有自然數都是有限的(可根據\ Axiom \ref{axm 2.5} 得到),但是自然數這個\ number system 卻是``無限''的。這應該要到第八章才能完全理解。
\end{remark}

\begin{remark}\label{remark 2.1.14}
我們目前定義自然數的方式是\ \emph{axiomatic}(用公理定出來的),而非\ \emph{constructive}(建構出來的)。要用建構的方式兜出自然數也可以,可參考看公理化集合論。但這邊的重點就是我們假設某個集合存在,且符合皮亞諾公理,以這個為前提開始推導出各種性質。我們(暫時)不去探討為何那個集合存在。這就跟線性代數有點類似,至少在線代的課程本身,探討為何會存在集合(on a field)符合那\ \(2+8\) 個\ vector space 的性質從來不是重點,而是一種前提。
\end{remark}

\begin{remark}\label{remark 2.1.15}
這邊是說用公理化的方式來定義自然數其實是很近代的事情(就皮亞諾那個時代開始)。在這之前自然數(在哲學/直覺/etc 上?)都是依附在某種既有的物件上的概念(connected to some external concept),例如某種物理量(如長度)等等。
\end{remark}

\begin{proposition} [\RED{I don't know all this bloody hell}] \label{prop 2.1.16}
 這題我完全問號,連看懂都有問題;但他的目的應該是我們可以從自然數來定義\ sequence。可是這一頁的註腳也說這會扯到\ function 的概念,實際上要去看第三章(Exercise \ref{exercise 3.5.12})。\textbf{但重點是這個\ forward reference 在證明上不會\ circular,因為 function 的定義不依賴皮亞諾公理}。另外這題蠻多人問的,總之看完第三章之後要回來搞懂。
\end{proposition}
\section{Addition}\label{sec 2.2}

\newcommand{\BLUE}[1]{\textcolor{RoyalBlue}{#1}}
\newcommand{\GREEN}[1]{\textcolor{YellowGreen}{#1}}
\newcommand{\MAROON}[1]{\textcolor{Maroon}{#1}}
\newcommand{\RED}[1]{\textcolor{Red}{#1}}

\begin{definition} [Addition of natural numbers] \label{def 2.2.1} Let \(m\) be a natural number. To add zero to \(m\), we \emph{define} \(0 + m := m\). Now suppose \emph{inductively} that we have defined how to add \(n\) to \(m\). Then we can add \(n\INC \) to \(m\) by defining \((n\INC) + m\) := \((n + m)\INC\).
\end{definition}
\begin{note}
So,
\begin{align*}
\BLUE{0 + m = m}\\
1 + m = (0\INC) + m = \BLUE{(0 + m)}\INC = \BLUE{m}\INC \implies \GREEN{1 + m = m\INC} \\
2 + m = (1\INC) + m = (\GREEN{1 + m})\INC = \GREEN{(m\INC)}\INC \implies 2 + m = (m\INC)\INC
\end{align*}
and so on.
\end{note}
\begin{note}
注意,我們有定義\ \(0 + m\) 是什麼,但我們沒有定義\ \(m + 0\) 是什麼,i.e. 需要證明他們兩個相等。參考\ Lemma \ref{lem 2.2.2}。
\end{note}
\begin{note}
這個定義因為符合數學歸納法(Axiom \ref{axm 2.5}),所以給定一個自然數\ \(m\), 對於所有自然數\ \(n\),我們都對\ \(n + m\) 做了定義
\end{note}

\begin{additional corollary}\label{ac 2.2.1} (see page. 24 below) Using Axioms 2.1, 2.2, and induction (Axiom 2.5), that the sum of two natural numbers is again a natural number 
\end{additional corollary}
\begin{proof}
Let \(m\) be a natural number.

Base case: \(0 + m = m\) by Definition \ref{def 2.2.1}. But \(m\) is a natural number, so \(0 + m\) is a natural number. 

Inductive hypothesis: suppose \(n + m\) is a natural number. Wanted: \((n\INC) + m\) is a natural number. By Definition \ref{def 2.2.1} \BLUE{\((n\INC) + m = (n + m)\INC\)}, but by inductive hypothesis, \(n + m\) is a natural number, and by Axiom \ref{axm 2.2}, \BLUE{\((n + m)\INC\)} is a natural number, so \BLUE{\((n\INC) + m\)} is a natural number. This closes the induction.
\end{proof}

\begin{note}
Definition \ref{def 2.2.1} 就可以讓我們推出所有以前就在使用的加法的規則了,例如交換律/結合律。之後都會證明。
\end{note}

\begin{lemma}\label{lem 2.2.2}
For any natural number \(n\), \(n + 0 = n\).
\end{lemma}
\begin{proof} We use induction.

The base case \(0 + \BLUE{0} = \BLUE{0}\) follows since we know(by Definition \ref{def 2.2.1}) that \(0 + \BLUE{m} = \BLUE{m}\) for every natural number \(\BLUE{m}\), and \(\BLUE{0}\) is a natural number.

Now suppose inductively that \(n + 0 = n\). We wish to show that \((n\INC)+0 = n\INC\). But by Definition \ref{def 2.2.1}, \((n\INC) + 0\) is equal to \((n + 0)\INC\), which is equal to \(n\INC\) since \(n + 0 = n\). This closes the induction.
\end{proof}

\begin{lemma}\label{lem 2.2.3} For any natural numbers \(n\) and \(m\), \(\BLUE{n + (m\INC)} = \GREEN{(n + m)\INC}\)
\end{lemma}
\begin{note}
這個\ lemma 內等號\ LHS 的\ operand 的順序又跟\ Definition \ref{def 2.2.1} 相反,是\ \(\BLUE{n + (m\INC)}\) 而不是\  \((m\INC) + n\),所以不能直接從\ Definition \ref{def 2.2.1} 得知這個\ Lemma。
\end{note}
\begin{proof}
We induct on \(n\) (keeping \(m\) fixed).

Base case: \(n = 0\). In this case we have to prove \( \BLUE{0 + (m\INC)} = (\GREEN{0 + m})\INC\). But by Definition \ref{def 2.2.1}, \(\BLUE{0 + (m\INC)} = m\INC\) and \(\GREEN{0 + m} = m\), so both sides are equal to \(m\INC\) and are thus equal to each other.

Now we assume inductively that \(n + (m\INC) = (n + m)\INC\); we now have to show that \(\BLUE{(n\INC) + (m\INC)} = (\GREEN{(n\INC) + m})\INC\). The \BLUE{left-hand side} is \((n + (m\INC))\INC\) by Definition \ref{def 2.2.1}, which is equal to \(((n + m)\INC)\INC\) by the inductive hypothesis. Similarly, we have \(\GREEN{(n\INC) + m} =(n + m)\INC\) by the Definition \ref{def 2.2.1}, and so the right-hand side is also equal to \(((n + m)\INC)\INC\). Thus both sides are equal to each other, and we have closed the induction.
\end{proof}

\begin{additional corollary}\label{ac 2.2.2}(on page 26 above)\(n\INC = n +1\)
\begin{note}
我們可從\ Definition \ref{def 2.2.1} 得知\ \(1 + n = n\INC\),但不能得知\ \(n + 1 = n\INC\)
\end{note}
\begin{proof}
    \begin{align*}
        n\INC & = n\INC + 0 & \text{by Lemma \ref{lem 2.2.2}} \\
              & = (n + 0)\INC & \text{by Definition \ref{def 2.2.1}} \\
              & = n + (0\INC) & \text{by Lemma \ref{lem 2.2.3}} \\
              & = n + 1
    \end{align*}
\end{proof}
\end{additional corollary}

\begin{proposition}[Addition is commutative]\label{prop 2.2.4} For any natural numbers \(n\) and \(m\), \(n + m = m + n\).
\end{proposition}
\begin{proof}
We shall use induction on \(n\) (keeping \(m\) fixed).

First we do the base case \(n = 0\), i.e., we show \(0 + m = m + 0\). By the Definition \ref{def 2.2.1}, \(0 + m = m\), while by Lemma \ref{lem 2.2.2}, \(m + 0 = m\). Thus the base case is done.

Now suppose inductively that \(n + m = m + n\), now we have to prove that \(\BLUE{(n\INC) + m} = \GREEN{m +(n\INC)}\) to close the induction. By the Definition \ref{def 2.2.1}, \(\BLUE{(n\INC) + m} = (n + m)\INC\). By Lemma \ref{lem 2.2.3}, \(\GREEN{m + (n\INC)} = (m + n)\INC\), but this is equal to \((n + m)\INC\) by the inductive hypothesis \(n + m = m + n\). Thus \((n\INC) + m = m +(n\INC)\) and we have closed the induction.
\end{proof}

\begin{proposition}[Addition is associative]\label{prop 2.2.5} For any natural numbers \(a\), \(b\), \(c\), we have \((a + b) + c = a + (b + c)\).
\end{proposition}
\begin{proof}
We induct on \(c\), keep a, b fixed.

Let \(a\), \(b\) be particular but arbitrarily chosen natural numbers.

Base case: for \(c = 0\), we want to prove \((a + b) + 0 = a + (b + 0))\). For the LHS, \((\BLUE{a + b}) + 0 = \BLUE{a + b}\) by Lemma \ref{lem 2.2.2}. For the RHS, \(a + \GREEN{(b + 0)} = a + \GREEN{b}\) by Lemma \ref{lem 2.2.2}. So both sides equal to \(a + b\) and thus equal to each other.

Inductive hypothesis: suppose \((a + b) + c = a + (b + c)\), we want to prove \(\BLUE{(a + b) + (c\INC)} = \GREEN{a + (b + (c\INC))}\). By Lemma \ref{lem 2.2.3}, LHS = \(\BLUE{(a + b) + (c\INC)} = \MAROON{((a + b) + c)\INC}\). By Lemma \ref{lem 2.2.3}, RHS = \(\GREEN{a + (b + (c\INC))} = a + (b + c)\INC\), which again by Lemma \ref{lem 2.2.3} = \((a + (b + c))\INC\). But by inductive hypothesis, \((a + b) + c = a + (b + c)\), so \((a + (b + c))\INC = \MAROON{((a + b) + c)\INC}\), which equals to LHS. So RHS = LHS, so we closed the induction.
\end{proof}

\begin{note}
Because of this associativity we can write sums such as \(a + b + c\) without having to worry about which order the numbers are being added together.
\end{note}

\begin{proposition}[Cancellation law]\label{prop 2.2.6} Let \(a\), \(b\), \(c\) be natural numbers such that \(a + b = a + c\). Then we have \(b = c\).
\end{proposition}
\begin{note}
我們不能用任何「減法」或者是「負數」的概念來證明自然數消去法。事實上這本書就是用這個消去法來定義「虛擬減法」(virtual subtraction),是一種「形式上」(formal)的減法,然後會進一步證明這跟整數減法等價。可參考第四章\ Definition \ref{def 4.1.1}。
\end{note}
\begin{proof}
We proof this by induction on \(a\).

Base case: let \(a = 0\), we have to prove \(0 + b = 0 + c\ \implies b = c\). But by Definition \ref{def 2.2.1}, \(0 + b = b\), and \(0 + c = c\), so \(b = 0 + b = 0 + c = c\), so \(b = c\).

Inductive hypothesis: suppose \(a + b = a + c \implies b = c\), we have to prove \(\BLUE{(a\INC) + b} = \GREEN{(a\INC) + c} \implies b = c\). By Definition \ref{def 2.2.1}, \BLUE{LHS} = \((a + b)\INC\), \GREEN{RHS} = \((a + c)\INC\), so \((a + b)\INC = (a + c)\INC\), and by Axiom (of contrapositive of) \ref{axm 2.4}, \(a + b = a + c\), which by inductive hypothesis implies \(b = c\). This closes the induction.
\end{proof}

\begin{note}
至此加法的基本規則都證明了,接下來要討論加法跟「正數」的關係。
\end{note}

\begin{definition}[Positive natural numbers] \label{def 2.2.7} A natural number \(n\) is said to be positive if and only if it is not equal to \(0\).
\end{definition}
\begin{note}
沒錯,在自然數這個系統且定義包含了\ \(0\) 的情況下,「正數」的意思就是「不是\ \(0\) 的自然數」。
\end{note}

\begin{proposition}\label{prop 2.2.8} If \(a\) is a positive natural number and \(b\) is a natural number, then \(a + b\) is positive (and hence \(b + a\) is also, by Proposition \ref{prop 2.2.4}).
\end{proposition}

\begin{proof}
We prove by induction on \(b\). Let \(a\) be a positive natural number.

Base case: let \(b = 0\). Then
\begin{align*}
a + b & = a + 0 & \\
      & = 0 + a & \text{by Proposition \ref{prop 2.2.4} (commutative law)} \\
      & = a     & \text{by Definition \ref{def 2.2.1}}
\end{align*}
which by definition is a positive natural number.

Inductive hypothesis: Suppose \(a + b\) is positive, we have to prove \(a + (b\INC)\) is positive. Then by Lemma \ref{lem 2.2.3}, \(a + (b\INC) = (a + b)\INC\). But by inductive hypothesis \(a + b\) is positive, which by Definition \ref{def 2.2.7} is not equal to \(0\), so by Axiom \ref{axm 2.3}, \((a + b)\INC\) is also not equal to \(0\), which again by Definition \ref{def 2.2.7} is positive. This closes the induction.
\end{proof}

\begin{corollary} \label{corollary 2.2.9}
If \(a\) and \(b\) are natural numbers such that \(a + b = 0\), then \(a = 0\) and \(b = 0\).
\end{corollary}

\begin{proof}
Suppose for the sake of contradiction that \(a + b = 0\) but \(a \neq 0\) or \(a \neq 0\). If \(a \neq 0\), then by Definition \ref{def 2.2.7} \(a\) is positive and then by Proposition \ref{prop 2.2.8} \(a + b\) is positive and by Definition \ref{def 2.2.7} cannot be \(0\), which contradicts \(a + b = 0\). If \(b \neq 0\), then similarly \(b\) is positive and \(b + a\) is positive, and by Proposition \ref{prop 2.2.4} \(= a + b\), so \(a + b\) is positive, again a contradiction. Thus both \(a\) and \(b\) must be \(0\).
\end{proof}

\begin{lemma}\label{lem 2.2.10}
Let \(a\) be a positive natural number. Then there \emph{exists exactly one} natural number \(b\) such that \(b\INC = a\).
\end{lemma}
\begin{note}
注意這是個\ imply 陳述,如果前提不成立(i.e. \(a\) 不是\ positive natural number)則會直接\ vacuously true!
\end{note}
\begin{proof}
We prove by induction on \(a\).

Base case: let \(a = 0\). But by Definition \ref{def 2.2.7} \(a\) is not positive. So this Lemma is vacuously true.

Inductive hypothesis: suppose \(a\) is a positive natural number and \(b\) is the unique natural number such that \(b\INC = a\). We have to prove that there exists exact one natural number \(b'\) such that \(b'\INC = a\INC\). Let \(b' = b\INC\). Then by Axiom of Substitution \ref{axm a.7.4} \(b' = a\), and \(b'\INC = a\INC\) again by substitution, so the existence part is satisfied. Now we prove the unique part. Suppose there also exists a natural number \(c\) such that \(c\INC = a\INC\). Then by Axiom \ref{axm 2.4}, \(c = a\), and since \(b' = a\), so \(c = b'\), which means \(b'\) is unique. This closed the induction.
\end{proof}

\begin{note}
現在有了加法的定義跟一些性質,以及正數的定義,我們可以來定義自然數的「order」了。
\end{note}

\begin{definition}[Ordering of the natural numbers] \label{def 2.2.11} Let \(n\) and \(m\) be natural numbers. We say that \(n\) is \emph{greater than or equal to} \(m\), and write \(n \geq m\) or \(m \leq n\), if and only if we have \(n = m + a\) for some natural number \(a\). We say that n is \emph{strictly greater than} \(m\), and write \(n > m\) or \(m < n\), if and only if \(n \geq m\) and \(n \neq m\).
\end{definition}
\begin{note}
\(n \geq m\) 跟\ \(m \leq n\) 只有符號上的差別,他們在定義上是等價的。另外定義內的\ \(a\) 只需要是\ natural number 即可,不需要是\ "positive" natural number。
\end{note}
\begin{additional corollary} \label{ac 2.2.3}
\(n\INC > n\) for any natural number \(n\),因為\ (1) \(n\INC \neq n\)(否則會違反\ Axiom \ref{axm 2.4}) (2) \(n\INC \geq n\),因為\ \(n\INC = n + 1\)(by Additional Corollary \ref{ac 2.2.2}),而\ \(1\) 就是個\ natural number。

所以,這代表沒有最大的自然數。
\end{additional corollary}

\begin{proposition} [Basic properties of order for natural numbers] \label{prop 2.2.12}
Let \(a\), \(b\), \(c\) be natural numbers. Then
    \begin{enumerate}
        \item (Order is reflexive) \(a \geq a\).
        \item (Order is transitive) If \(a \geq b\) and \(b \geq c\), then \(a \geq c\).
        \item (Order is \emph{anti}-symmetric) If \(a \geq b\) and \(b \geq a\),then \(a = b\). 
        \item (Addition preserves order) \(a \geq b\) if and only if \(a + c \geq b + c\). 
        \item \(a < b\) if and only if \(a\INC \leq b\).
        \item \(a < b\) if and only if \(b = a + d\) for some \emph{positive} number \(d\).
    \end{enumerate}
\end{proposition}
\begin{note}
若忘了\ anti-symmetric 是什麼,可以回去翻離散課本講\ relation 的部分。
\end{note}
\begin{proof}{(a)}
Let \(a\) be a natural number. Then by Definition \ref{def 2.2.11}, \(\BLUE{a + 0} \geq \GREEN{a}\) because \(0\) is a natural number. But by Lemma \ref{lem 2.2.2}, \(\BLUE{a + 0} = \BLUE{a}\), so by Axiom of Substitution (\ref{axm a.7.4}), \(\BLUE{a} \geq \GREEN{a}\).
\end{proof}

\begin{proof}{(b)}
Let \(a, b, c\) be natural numbers s.t. \(a \geq b\) and \(b \geq c\). Then by Definition \ref{def 2.2.11}, \(\exists\ \text{natural numbers}\ m, n\) such that \(a = b + m\) and \(b = c + n\). So
\begin{align*}
    a & = b + m & \\
      & = (c + n) + m & \\
      & = c + (n + m) & \text{by Proposition \ref{prop 2.2.5}}\\
\end{align*}
So \(a = c + (n + m)\) for some natural number \(n + \), so by Definition \ref{def 2.2.11}, \(a \geq c\)
\end{proof}
\begin{proof}{(c)}
Let \(a, b\) be natural numbers s.t. \(a \geq b\) and \(b \geq a\). Then by Definition \ref{def 2.2.11}, \(\exists\ \text{natural numbers}\ m, n\) such that \(a = b + m\) and \(b = a + n\). So
\begin{align*}
    a & = b + m & \\
      & = (a + n) + m & \\
      & = a + (n + m) & \text{by Proposition \ref{prop 2.2.5}}\\
\end{align*}
But by Lemma \ref{lem 2.2.2}, \(a = a + 0\), so \(a + 0 = a + (n + m)\), and by Proposition \ref{prop 2.2.6} (cancellation law), \(0 = n + m\), and by Corollary \ref{corollary 2.2.9}, \(n = 0\) and \(m = 0\). In particular, \(m = 0\), so \(a = b + m = b + 0 = b\).
\end{proof}
\begin{proof}{(d)}
Let \(a, b, c, d\) be natural numbers. Then
\begin{align*}
         & a \geq b \\
    \iff & a = b + d             & \text{by Definition \ref{def 2.2.11}} \\
    \iff & a + c = (b + d) + c   & \text{by Axiom of Substitution \ref{axm a.7.4}} \\ 
    \iff & a + c = c + (b + d)   & \text{by Proposition(commutative law) \ref{prop 2.2.8} } \\
    \iff & a + c = (c + b) + d   & \text{by Proposition(associative law) \ref{prop 2.2.5}} \\
    \iff & a + c = (b + c) + d   & \text{by Proposition \ref{prop 2.2.4}} \\
    \iff & a + c \geq b + c      & \text{by Definition \ref{def 2.2.11}}
\end{align*}
\end{proof}
\begin{proof}{(e)}
Let \(a, b\) be natural numbers.

\(\Longrightarrow \): Suppose \(a < b\). Then by Definition \ref{def 2.2.11} \(a \leq b\) \BLUE{(1)} and \(a \neq b\) \BLUE{(2)}. \BLUE{(1)} implies \(a + m = b\) \MAROON{(1)} for some natural number \(m\), and with \BLUE{(2)} by Definition \ref{def 2.2.7} implies \(m\) is positive (otherwise \(b = a + m = a + 0 = a\) by Lemma \ref{lem 2.2.2}, contradicting \(a \neq b\)). Since \(m\) is positive, by Lemma \ref{lem 2.2.10}, \(m = n\INC\) \MAROON{(2)} for some natural number \(n\). So
\begin{align*}
    b & = a + m & \text{by \MAROON{(1)}}\\
      & = a + n\INC & \text{by \MAROON{(2)}}\\
      & = (a + n)\INC & \text{by Lemma \ref{lem 2.2.3}} \\
      & = a\INC + n & \text{by Definition \ref{def 2.2.1}}
\end{align*}
which by Definition \ref{def 2.2.11} implies \(a\INC \leq b\).

\( \Longleftarrow \): Suppose \(a\INC \leq b\). Then by Definition \ref{def 2.2.11} \(a\INC + m = b\) for some natural number \(m\), And
\begin{align*}
    & \BLUE{a\INC + m} \\
    & = (a + m)\INC & by Definition \ref{def 2.2.1} \\
    & = \BLUE{a + m\INC} & by Lemma \ref{lem 2.2.3}
\end{align*}
So \(\BLUE{a\INC + m = a + m\INC} = b\). By Axiom \ref{axm 2.3}, \(m\INC \neq 0\), so by Definition \ref{def 2.2.7} is positive, so \(a \neq b\). So \(a + m\INC = b\) for some natural number \(m\INC\) and \(a \neq b\), by Definition \ref{def 2.2.11}, \(a < b\).
\end{proof}

\begin{proof}{(f)}

\( \Longrightarrow \): This was proved in the "if part" of (e); the \(m\) in there meets the condition.

\( \Longleftarrow \): This was proved in the middle of the "only if part" of (e); the positive \(m\INC\) derived in the middle can be treated as \(d\) in (f), and the derivation in (e) derives \(a < b\).
\end{proof}

\begin{proposition}[Trichotomy of order for natural numbers] \label{prop 2.2.13}
Let \(a\) and \(b\) be natural numbers. Then exactly one of the following statements is true: \(a < b\), \(a = b\), or \(a > b\).
\end{proposition}
\begin{proof}
The sketch: (1) at most one condition holds, (2) at least one condition holds; this implies exactly one condition holds.

(1) at most one:
    \begin{enumerate}
        \item If \(a < b\) then by Definition \ref{def 2.2.11} \(a \neq b\). Now if \(a > b\) then we have \(a < b \land a > b\), which by Proposition \ref{prop 2.2.12}(c) (with some trivial implications i.e. \(a < b \implies a \leq b\)) implies \(a = b\), contradicting \(a \neq b\).
        \item If \(a > b\) then by Definition \ref{def 2.2.11} \(a \neq b\). Now if \(a < b\) then we have \(a > b \land a < b\), which by Proposition \ref{prop 2.2.12}(c) (with some trivial implications i.e. \(a < b \implies a \leq b\)) implies \(a = b\), contradicting \(a \neq b\).
        \item If \(a = b\) then by Definition \ref{def 2.2.11} \(a \not < b \) and \(b \not < a\).
    \end{enumerate}

(2) at least one: we use induction on \(a\) and keep \(b\) fixed.

Base case: Let \(a = 0\). Then we have \(a = 0 \leq b\), because \(b = \GREEN{b} + 0 = \GREEN{b} + a\), so there exists a natural number \(\GREEN{b}\) s.t. \(b = \GREEN{b} + a\), which by Definition \ref{def 2.2.11} implies \(a \leq b\) (This is the first (why?)). And this implies \(a < b\) or \(a = b\).

Inductive hypothesis: Suppose \(a\) satisfies the "at least one" proposition, i.e. at least one of \(a < b\), \(a = b\), or \(a > b\) is true. We have to prove \(a\INC\) satisfies the "at least one" proposition.
    \begin{enumerate}
        \item Suppose \(a < b\), then by Proposition \ref{prop 2.2.12}(e), \(a\INC \leq b\), which means \(a = b\) or \(a < b\).
        \item Suppose \(a = b\), then \(a\INC = b\INC\) by Axiom of Substitution \ref{axm a.7.4}, and \(b\INC > b\) by Additional Corollary \ref{ac 2.2.3}, so \(a\INC > b\) (This is the third (why?))
        \item Suppose \(a > b\). By Additional Corollary \ref{ac 2.2.3}, \(a\INC > a\). So we have \(a > b\) and \(a\INC > a\), which by Proposition \ref{prop 2.2.12}(b) implies \(a\INC \geq b\). Or more precisely \(a\INC > b\) because if \(a\INC = b\) that implies \(a + 1 = b\), which implies \(a < b\), which contradicts \(a > b\). (This is the second (why?))
    \end{enumerate}
So in either case \(a\INC\) satisfies at least one of the \(a\INC < b\), \(a\INC = b\) or \(a\INC > b\).
\end{proof}

\begin{note}
上面這些\ order 的性質可以讓我們定義一個\ "stronger" version of induction. 把\ stronger 標起來只是因為它只是看起來比較強,但實際上跟數學歸納法等價。
\end{note}

\begin{proposition}[Strong principle of induction] \label{prop 2.2.14}
Let \(m_0\) be a natural number, and let \(P(m)\) be a property pertaining to an arbitrary natural number \(m\). Suppose that \emph{for each} \(m \geq m_0\), \emph{we have the following implication}: if \(P(m')\) is true for all natural numbers \(m_0 \leq m' \RED{<}\ m\), then \(P(m)\) is also true. (In particular, this means that \(P(m_0)\) is true, since in this case the hypothesis is vacuous.) Then we can conclude that \(P(m)\) is true for all natural numbers \(m \geq m_0\).
\end{proposition}
\begin{note}
要仔細看這個\ proposition 是假設有什麼,然後有什麼結論。

這個\ proposition 是說對於任一個(for all) \(\geq m_0\) 的自然數\ \(m\),都滿足一個「包含了\ \(P\) 還有\ \(m_0\) 的\ if then 陳述句」;若這為真,則會\ implies 「對於所有\ \(\geq m_0\) 的自然數\ m,\(P(m)\) 都成立」。

另外當\ \(m = m_0\) 時,那個「被包含的\ if then 陳述句」的\ hypothesis 是\ vacuously true,因為該\ hypothesis 是個\ for all statements,但是\ for all 作用的集合此時為空集合。而如果我們又證明了「那個被包含的\ if then 陳述句」是\ true,則\ (1) if then 陳述句是\ true,\ (2) if-part 也是\ true,這代表\ then part 也是\ true;現在在\ \(m = m_0\) 的情況下,then part 就是\ \(P(m_0)\) is true。而\ \(P(m_0)\) is true 其實就是傳統\ induction 的\ base case。換句話說這代表「整個\ strong induction 自己的\ if-part」內會\ implies base case,所以只要提供了\ strong induction 需要的那種\ if-part 就不需要證明\ base case is true 了。
\end{note}

\begin{remark}\label{remark 2.2.15}
In applications we usually use Proposition \ref{prop 2.2.14} with \(m_0 = 0\) or \(m_0 = 1\).
\end{remark}
\section{Multiplication}\label{sec 2.3}

\begin{note}
這節會直接使用已知的自然數加法還有\ order 的性質,例如我們不會再證明\ \(a + b + c = c + b + a\).

另外會定義自然數乘法。類似自然數加法是\ "iterated increment operation"(重複的增量運算),自然數乘法是\ "iterated addition"(重複的加法運算)。
\end{note}
\chapter{Set Theory} \label{ch3}

For now we pause to introduce the concepts and notation of set theory, as they will be used increasingly heavily in
later chapters.

While set theory is not the main focus of this text, almost every other branch of mathematics relies on set theory as part of its foundation, so it is important to get at least some grounding in set theory before doing other advanced areas of mathematics.

In this chapter we present the more elementary aspects of axiomatic set theory, leaving more advanced topics such as a discussion of infinite sets and the axiom of choice to
\CH{8}.

\begin{note}
到第八章才講那些進階的東西?\ 因為直到第八章之前的定理都不用依靠那些東西。
\end{note}

\section{Fundamentals}\label{sec 3.1}

For pedagogical reasons, we will use a \emph{somewhat overcomplete list of axioms} for set theory, in the sense that some of the axioms can be used to deduce others, but there is \emph{no real harm} in doing this.

\begin{note}
就好像你可以說正整數除了滿足皮亞諾公理也自動滿足\ \(a + b = b + a\),但後者用皮亞諾就可證明。
\end{note}

\begin{definition}[\emph{Informal}] \label{def 3.1.1}
We define a set \(A\) to be any \emph{unordered collection} of objects, e.g., \( \{3, 8, 5, 2\} \) is a set. If \(x\) is an object, we say that \(x\) \emph{is an element of} \(A\) or \(x \in A\) if \(x\) lies in the collection; otherwise we say that \(x \notin A\). For instance, \(3 \in \{1, 2, 3, 4, 5\} \) but \(7 \notin \{1, 2, 3, 4, 5\} \).
\end{definition}

\begin{note}
\DEF{3.1.1} 有很多問題沒有回答,例如「什麼樣的\ collection」 才能被稱作集合,兩個集合怎麼判斷是否相等,怎麼對集合作操作(聯集、交集等等),集合可以做什麼,以及集合的元素(element)可以做什麼。
\end{note}

\begin{axiom}[Sets are objects]\label{axm 3.1}
If \(A\) is a set, then \(A\) is \emph{also an object}. In particular, given two sets \(A\) and \(B\), it is meaningful to ask whether \(A\) is also an element of \(B\).
\end{axiom}

\begin{example}[Informal]\label{example 3.1.2}
這個例子舉例\ \( \{3, \{3, 4\}, 4\} \) 裡面有一個元素也是集合,但敘述方式不嚴謹,要去看\ \SEC{3.6}
\end{example}

\begin{remark}\label{remark 3.1.3}
這裡在探討是否需要把所有\ object 都當成\  set。在邏輯的角度,這樣推論過程比較簡單,因為需要的東西的類型就只有一種,就是\ set,但是從概念上來看,將某些\ object 視為「不是\ set」則會比較單純,比方說給定一個自然數\ \(2\),將他視為一個集合(在\ Analysis 的範疇)沒什麼進一步的應用。是否將所有\ object 都當成集合,\ more or less 是等價的,所以,we shall take an agnostic position as to whether all objects are sets or not.
\end{remark}

\begin{note}
若已知\ \(x\) 是一個\ object 且\ \(A\) 是一個\ set,則要馬\ \(x \in A\) 為真,要馬\ \(x \notin A\) 為真。而若\ \(A\) 不是\ set,則我們視\ \(x \in A\) 為\ undefined。
\end{note}

\begin{definition}[Equality of sets] \label{def 3.1.4} 
Two sets \(A\) and \(B\) are equal, \(A = B\), if and only if every element of \(A\) is an element of \(B\) and vice versa. To put it another way, \(A = B\) if and only if every element \(x\) of \(A\) belongs also to \(B\), and every element \(y\) of \(B\) belongs also to \(A\). Or equivalently,
\[
  \forall\ x : x \in A \iff x \in B
\]
\end{definition}

\begin{example}
嘴砲。
\end{example}

\begin{note}
One can easily verify that this notion of equality is reflexive, symmetric, and transitive (See \EXEC{3.1.1}).
\end{note}

\begin{additional corollary}\label{ac 3.1.1}
The definition of equality in \DEF{3.1.4} is reflexive, symmetric and transitive.
\end{additional corollary}

\begin{proof}

Reflexive: Suppose \(A\) is a set. Then given any object \(x\), if \(x \in \GREEN{A}\), then \(x \in \BLUE{A}\), and given any object \(y\), if \(y \in \BLUE{A}\), then \(y \in \GREEN{A}\). So by \DEF{3.1.4}, \(\GREEN{A} = \BLUE{A}\).

Symmetric: Suppose \(A, B\) are sets and \(A = B\), then by \DEF{3.1.4},
\[
  \forall\ x : x \in A \iff x \in B
\]
But this statement is just equivalent to
\[
  \forall\ x : x \in B \iff x \in A
\]
and this by \DEF{3.1.4} implies \(B = A\).

Transitive: Suppose \(A, B, C\) are sets and \(A = B\) and \(B = C\). Then by \DEF{3.1.4}
\[
  \forall\ x : x \in A \iff x \in B
\]
\[
  \forall\ x : x \in B \iff x \in C
\]
And this implies
\[
  \forall\ x : x \in A \iff x \in B \iff x \in C
\]
And by logic this implies
\[
  \forall\ x : x \in A \iff x \in C
\]
By \DEF{3.1.4}, \(A = C\).
\end{proof}

\begin{note}
``is an element of'' relation \(\in\) 符合\ Axiom of Substitution \AXM{a.7.4},因為
\begin{center}
    if \(x \in A\) and \(A = B\), then \(x \in B\), by \DEF{3.1.4}.
\end{center}
這也代表那些完全以\ \(\in\) 定義的新的集合操作會自動符合\ Axiom of Substitution \AXM{a.7.4}。例如這一節剩下的所有\ operations 都是用\ \(\in\) 來定義的。
\end{note}
\begin{note}
接著\ \RMK{3.1.3},我們繼續來探討什麼\ object 是\ set,什麼不是。有點類似我們定哪些東西為自然數,哪些不是(\AXM{2.1},\(0\) 是自然數,然後用 \AXM{2.2} 來擴增/建構其他的自然數)。這邊在集合論就是先假設存在一個集合,叫「空集合」,然後再定義一些在集合上的操作來建構其他的集合。
\end{note}

\begin{axiom}[Empty set] \label{axm 3.2}
There exists a set \(\emptyset\), known as \emph{the} empty set, which \emph{contains no elements}, i.e., for every object \(x\) we have \(x \notin \emptyset\).
\end{axiom}

\begin{note}
\emph{The} empty set is also denoted \(\{\}\). Note that there can only be \textbf{one} empty set.
\end{note}

\begin{additional corollary} [The empty set is unique] \label{ac 3.1.2}
If there were two sets \(\emptyset\) and \(\emptyset'\) which were both empty, then by \DEF{3.1.4} they would be equal to each other.
\end{additional corollary}
\begin{proof}
Suppose \(\emptyset'\) is also empty. Then the statement
\[
  \forall\ x : x \in \emptyset' \implies x \in \emptyset
\]
is vacuously true because by \AXM{3.2} empty set contains no elements. And again by \AXM{3.2}, the statement
\[
  \forall\ x : x \in \emptyset \implies x \in \emptyset'
\]
is also vacuous. These imply
\[
  \forall\ x : (x \in \emptyset' \implies x \in \emptyset) \land (x \in \emptyset \implies x \in \emptyset')
\]
that is,
\[
  \forall\ x : x \in \emptyset' \iff x \in \emptyset
\]
by \DEF{3.1.4}, \(\emptyset' = \emptyset\)
\end{proof}

\begin{note}
If a set is not equal to the empty set, we call it \emph{non-empty}.
\end{note}

\begin{lemma}[Single choice]\label{lem 3.1.6}
Let \(A\) be a \emph{non-empty} set. Then there exists an object \(x\) such that \(x \in A\).
\end{lemma}
\begin{proof}
Suppose for the sake of contradiction that \(A\) be a non-empty set and for all object \(x\), \(x \notin A\). And by \AXM{3.2}, \(x \notin \emptyset\). Then similarly as \AC{3.1.2}, we can derive \(A = \emptyset\), which contradicts that \(A\) is \emph{non-empty}.
\end{proof}

\begin{remark}\label{remark 3.1.7}
\LEM{3.1.6} asserts that given any non-empty set \(A\), we are allowed to \emph{``choose''} an element \(x\) of \(A\) which demonstrates this non-emptyness. Later on (in \LEM{3.5.12}) we will show that given any \textbf{finite} number of non-empty sets, say \(A_1, \dots, A_n\), it is possible to choose one element \(x_1, \dots, x_n\) from each set \(A_1, \dots, A_n\); this is known as ``finite choice''. However, in order to choose elements from an \textbf{infinite} number of sets, we need an additional axiom, the \emph{axiom of choice} (\AXM{8.1}).
\end{remark}

\begin{note}
\RMK{3.1.7} 在講\ ``finite choice'' 的部分,看起來是說要被選的集合是有限個,但是沒有規定個別集合的元素數量要有限個,目前還不確定這意味著什麼。
\end{note}

\begin{remark} \label{remark 3.1.8}
Note that the empty set is not the same thing as the natural number \(0\). One is a set; the other is a number. However, it is true that the \emph{cardinality} of the empty set is \(0\); see \SEC{3.6}.
\end{remark}

We now present further axioms to enrich the class of sets available.

\begin{axiom}[Singleton sets and pair sets]\label{axm 3.3}
If \(a\) is an object, then there exists a set \( \{a\} \) whose \emph{only} element is \(a\), i.e., for every object \(y\), we have \(y \in \{a\}\) if and only if \(y = a\); we refer to \( \{a\} \) as the \emph{singleton set} whose element is \(a\). Furthermore, if \(a\) and \(b\) are objects, then there exists a set \( \{a, b\} \) whose only elements are \(a\) and \(b\); i.e., for every object \(y\), we have \( y \in \{a, b\} \) if and only if \(y = a\) or \(y = b\); we refer to this set as the \emph{pair set} formed by \(a\) and \(b\).
\end{axiom}

\begin{note}
\href{https://www.wikiwand.com/en/Axiom_of_pairing#/Consequences}{參考}: 這個公理實際說的是,給定兩個集合(這邊暫時當作所有物件都是集合)\ \(x\) 和\ \(y\),我們可以找到一個集合\ \(A\) ,它的成員就是\ \(x\) 和\ \(y\)。
\end{note}

\begin{note}
前方高能注意: \RMK{3.1.9} 在解釋\ \AXM{3.3} 裡面的\ singleton、\ pair,還有\ \AXM{3.4} 這三者,若假設``其中一部分''是公理,則剩下的可以從那個部分直接推得,不用當作是公理,i.e. 剩下的只須為定理,不用是公理。這種\ redundant\ 在本節開頭有講過,便於推導,but does no real harm。BTW 這個\ remark 裡面有一堆\ (why?) 全部都要自己推導。
\end{note}

\begin{remark} \label{remark 3.1.9}

Just as there is only one empty set, there is only one singleton set for each object \(a\), thanks to \DEF{3.1.4} (why? \MAROON{(1)}).

Similarly, given any two objects \(a\) and \(b\), there is only one pair set formed by \(a\) and \(b\).

Also, \DEF{3.1.4} also ensures that \( \{a, b\} = \{b, a\} \) (why? \MAROON{(2)}) and \( \{a, a\} = \{a\} \) (why? \MAROON{(3)}). Thus the \textbf{singleton} set axiom is in fact redundant, being \textbf{a consequence of} the \textbf{pair} set axiom.

\emph{Conversely}, the \textbf{pair set} axiom will \textbf{follow from} the \textbf{singleton} set axiom \textbf{and} the \textbf{pairwise union axiom \AXM{3.4}} (see  \LEM{3.1.13}).

One may wonder why we don’t go further and create triplet axioms, quadruplet axioms, etc.; however there will be no need for this once we introduce the pairwise union axiom
below.
\end{remark}

\begin{proof}
\MAROON{(1)}: given any object \(a\), suppose there exist two sets \(A\) and \(A'\) which are singleton sets of \(a\). Then we have:
\begin{align*}
         & (\forall\ x : x \in A \iff x = a) \land (\forall\ x : x \in A' \iff x = a) & \text{by \AXM{3.3}} \\
    \iff & (\forall\ x : x \in A \iff x = a) \land (\forall\ x : x = a \iff x \in A') & \text{by logic} \\
    \iff & (\forall\ x : x \in A \iff x = a \iff x \in A')                            & \text{by logic} \\
    \iff & (\forall\ x : x \in A \iff x \in A')                                       & \text{by simplifying logic} \\
    \iff & A = A'                                                                     & \text{by \DEF{3.1.4}}
\end{align*}

\MAROON{(2)}: given any objects \(a, b\), then
\begin{align*}
    & (x \in \{a, b\} \iff (x = a \lor x = b)) & \text{by \AXM{3.3}} \\
    \iff & (x \in \{a, b\} \iff (x = a \lor x = b) \iff (x = b \lor x = a)) & \text{by logic} \\
    \iff & (x \in \{a, b\} \iff (x = b \lor x = a)) & \text{by simplifying logic} \\
    \iff & (x \in \{a, b\} \iff (x = b \lor x = a) \iff x \in \{b, a\}) & \text{by \AXM{3.3}} \\
    \iff & (x \in \{a, b\} \iff x \in \{b, a\}) & \text{by simplifying logic} \\
    \iff & \{a, b\} = \{b, a\} & \text{by \DEF{3.1.4}}
\end{align*}

\MAROON{(3)}: given any object \(a\), then
\begin{align*}
    & (x \in \{a, a\} \iff (x = a \lor x = a)) & \text{by \AXM{3.3}} \\
    \iff & (x \in \{a, a\} \iff (x = a)) & \text{by logic} \\
    \iff & (x \in \{a, a\} \iff (x = a) \iff x \in \{a\})& \text{by \AXM{3.3}} \\
    \iff & (x \in \{a, a\} \iff x \in \{a\})& \text{by simplifying logic} \\
    \iff & \{a, a\} = \{a\} & \text{by \DEF{3.1.4}}
\end{align*}
\end{proof}

\begin{example} \label{example 3.1.10}
Since \(\emptyset\) is a set (and hence an object), so is singleton set \(\{ \emptyset \}\), i.e., the set whose only element is \( \emptyset \), is a set (and it is not the same set as \( \emptyset \), \( \{ \emptyset \} \neq  \emptyset \) (why? See \EXEC{3.1.2}). Similarly, the singleton set \( \{ \{ \emptyset \} \} \) and the pair set \( \{ \emptyset, \{ \emptyset \} \} \) are also sets. These three sets are not equal to each other (\EXEC{3.1.2}).
\end{example}

\begin{note}
現在有的三個公理已經可以讓我們建構出一堆集合了,但是他們的元素都不超過兩個,雖然元素本身可能長得很複雜。
\end{note}

\begin{axiom} [Pairwise union] \label{axm 3.4}
Given any two sets \(A, B\), there exists a set \(A \cup B\), called the \emph{union} \(A \cup B\) of \(A\) and \(B\), whose elements consists of all the elements which belong to \(A\) or \(B\) or both. In other words, for any object \(x\),
\[
    x \in A \cup B \iff (x \in A \lor x \in B)
\].
\end{axiom}

\begin{note}
注意\ Union 的定義完全是由\ \( \in \) (還有\ or) 兜出來的,所以符合替換公理。
\end{note}

\begin{example}
很廢。\( \{1, 2\} \cup \{ 2, 3 \} = \{ 1, 2, 3 \} \)
\end{example}

\begin{remark} \label{remark 3.1.12}
If \(A, B\) are sets, \(A'\) is also a set which is equal to \(A\), then \(A \cup B\) is equal to \(A' \cup B\) (why? \MAROON{(1)} One needs to use \AXM{3.4} and \DEF{3.1.4}). Similarly if \(B'\) is a set which is equal to \(B\), then \(A \cup B\) is equal to \(A \cup B'\). Thus the operation of union \emph{obeys the axiom of substitution}, and is thus well-defined on sets.
\end{remark}

\begin{proof}
\MAROON{(1)}: Suppose \(A, B, A'\) are sets such that \(A = A'\). Then given any object \(x\),
\begin{align*}
         & x \in A \cup B \\
    \iff & x \in A \lor x \in B & \text{by \AXM{3.4}} \\
    \iff & x \in A' \lor x \in B & \text{since \(A = A'\) and \(=\) satisfies \AXM{a.7.4}} \\
    \iff & x \in A' \cup B & \text{by \AXM{3.4}} \\
\end{align*}
So \(\forall x, x \in A \cup B \iff x \in A' \cup B\), so by \DEF{3.1.4} \(A \cup B = A' \cup B\).

Now suppose \(B' = B\), Then given any object \(x\),
\begin{align*}
         & x \in A \cup B \\
    \iff & x \in A \lor x \in B & \text{by \AXM{3.4}} \\
    \iff & x \in A \lor x \in B' & \text{since \(B = B'\) and \(=\) satisfies \AXM{a.7.4}} \\
    \iff & x \in A \cup B' & \text{by \AXM{3.4}} \\
\end{align*}
So similarly \(A \cup B = A \cup B'\).
\end{proof}

\begin{lemma} \label{lem 3.1.13}
If \(a\) and \(b\) are objects, then \( \{a, b\} = \{ a \} \cup \{ b \} \). If \(A, B, C\) are sets, then the \emph{union operation is commutative} (i.e., \(A \cup B = B \cup A\)) and \emph{associative} (i.e., \((A \cup B) \cup C = A \cup (B \cup C)\)). Also, we have \(A \cup A = A \cup \emptyset = \emptyset \cup A = A\).
\end{lemma}

\begin{proof}
\( \{a, b\} = \{ a \} \cup \{ b \} \): for any object \(x\),
\begin{align*}
         & x \in \{a, b\} \\
    \iff & x = a \lor x = b & \text{by \AXM{3.3}, pair-part} \\
    \iff & x \in \{a\} \lor x \in \{b\} & \text{by \AXM{3.3}, singleton-part} \\
    \iff & x \in \{a\} \cup \{b\} & \text{by \AXM{3.4}} \\
\end{align*}
So \(\forall x, x \in \{a, b\} \iff x \in \{a\} \cup \{b\} \), so \(\{a, b\} = \{a\} \cup \{b\} \) by \DEF{3.1.4}.

\(A \cup B = B \cup A\): for any object \(x\),
\begin{align*}
         & x \in A \cup B \\
    \iff & x \in A \lor x \in B & \text{by \AXM{3.4}} \\
    \iff & x \in B \lor x \in A & \text{by logic} \\
    \iff & x \in B \cup A & \text{by \AXM{3.4}}
\end{align*}

\((A \cup B) \cup C = A \cup (B \cup C)\): for any object \(x\),
\begin{align*}
         & x \in (A \cup B) \cup C \\
         \iff & x \in (A \cup B) \lor x \in C & \text{by \AXM{3.4}} \\
         \iff & (x \in A \lor x \in B) \lor x \in C & \text{by \AXM{3.4}} \\
         \iff & x \in A \lor (x \in B \lor x \in C) & \text{by logic} \\
         \iff & x \in A \lor (x \in B \cup C) & \text{by \AXM{3.4}} \\
         \iff & x \in A \cup (x \in B \cup C) & \text{by \AXM{3.4}} \\
\end{align*}
So first statement if and only if last statement, so \((A \cup B) \cup C = A \cup (B \cup C)\).

\(A \cup A = A \cup \emptyset = \emptyset \cup A = A\): for any object \(x\),
\begin{align*}
         & x \in A \cup A \\
    \iff & x \in A \lor x \in A & \text{by logic} \\
    \iff & x \in A & \text{by simplifying logic} \\
    \iff & \MAROON{A \cup A = A} & \text{by \DEF{3.1.4}} \\
    \iff & x \in A \lor x \in \emptyset & \text{by logic, something \(\lor\) something false = something} \\
    \iff & x \in A \cup \emptyset & \text{by \AXM{3.4}} \\
    \iff & \MAROON{A \cup A = A \cup \emptyset} & \text{by \DEF{3.1.4}} \\
    \iff & \MAROON{A \cup A = \emptyset \cup A} & \text{already proved commutative law}
\end{align*}
\end{proof}

\begin{remark}
幹話。
\end{remark}

\begin{note}
We are not yet in a position to define sets consisting of \(n\) objects for any given natural numbers \(n\). 老實說我不是很理解為什麼,文中敘述是說我們還沒有定義「做\ \(n\) 次操作」是什麼意思"(require iterating the above construction “\(n\) times”, but the concept of \(n\)-fold iteration has not yet been rigorously defined). 類似任意有限個元素的情況,我們目前也無法給出有無限個元素的集合的定義。這需要其他的公理。現在我們先定義什麼是「子集合」。
\end{note}

\begin{definition}[Subsets] \label{def 3.1.15}
Let \(A, B\) be sets. We say that \(A\) is a subset of \(B\), denoted \(A \subseteq B\), if and only if every element of \(A\) is also an element of \(B\), i.e. For any object \(x\), \(x \in A \implies x \in B\). We say that \(A\) is a proper subset of \(B\), denoted \(A \subsetneq B\), if \(A \subseteq B\) and \(A \neq B\).
\end{definition}

\begin{remark} \label{remark 3.1.16}
Because these definition\textbf{s}(both \(\subseteq\) and \(\subsetneq\)) involve only the notions of (set) equality and the “is an element of” relation, both of which already obey the axiom of substitution \AXM{a.7.4}, the notion of subset also automatically obeys the axiom of substitution. Thus for instance if \(A \subseteq B\) and \(A = A'\), then \(A' \subseteq B\).
\end{remark}

\begin{example} \label{example 3.1.17}
前面幹話。Given any set \(A\), we always have \(A \subseteq A\) (why?) and \(\emptyset \subseteq A\) (why?).
\end{example}

\begin{proof}
Let \(x\) be arbitrarily chosen object. Then if \(x \in A\) then \(x \in A\) is trivially true. By \DEF{3.1.15}, \(A \subseteq A\).

Let \(x\) be arbitrarily chosen object. Then if \(x \in \emptyset\) then \(x \in A\) is vacuously true. By \DEF{3.1.15}, \(\emptyset \subseteq A\).
\end{proof}

\begin{note}
下面的\ Proposition 要參考\ \DEF{8.5.1},旨在說明\ "set-inclusion" 這個\ relation 是\ partially ordered。也就是要證明它是\ reflexive(by \EXAMPLE{3.1.17}), anti-symmetric, transitive。
\end{note}

\begin{proposition} [Sets are partially ordered by set inclusion] \label{prop 3.1.18}
Let \(A, B, C\) be sets. If \(A \subseteq B\) and \(B \subsetneq C\) then \(A \subseteq C\). If \(A \subseteq B\) and \(B \subseteq A\), then \(A = B\). Finally, if \(A \subsetneq B\) and \(B \subsetneq C\) then \(A \subsetneq C\).
\end{proposition}

\begin{proof}
Transitive: Let \(A, B, C\) be sets such that \(A \subseteq B\) \MAROON{(1)} and \(B \subseteq C\) \MAROON{(2)}. Suppose object \(x \in A\), wanted: \(x \in C\). Then by \MAROON{(1)}, \(x \in B\), which with \MAROON{(2)} implies \(x \in C\).

Anti-Symmetric: Let \(A, B\) be sets such that \(A \subseteq B \land B \subseteq A\). Wanted: \(A = B\). Then we have
\begin{align*}
     & A \subseteq B \land B \subseteq A \\
\iff & (\forall\ x : x \in A \implies x \in B) \land (\forall\ x : x \in B \implies x \in A) & \text{by \DEF{3.1.15}} \\
\iff & (\forall\ x : x \in A \iff x \in B) & \text{by trivially simplifying logic} \\
\iff & A = B. & \text{by \DEF{3.1.4}}
\end{align*}
\end{proof}

Transitive for proper-inclusion: Let \(A, B, C\) be sets such that \(A \subsetneq B\) \MAROON{(1)} and \(B \subsetneq C\) \MAROON{(2)}. Wanted: \(A \subsetneq C\), that is,
\[
    A \subseteq C \land A \neq C
\] The former is trivially true since \(A \subsetneq B\) by definition implies \(A \subseteq B\) and \(B \subsetneq C\) by definition implies \(B \subseteq C\), and by transitivity of set-inclusion \(A \subseteq B\) and \(B \subseteq C\) implies \(A \subseteq C\). For the latter, suppose for the sake of contradiction that \(A = C\). Then by \MAROON{(1)} and Axiom of Substitution \AXM{a.7.4}, we have \(C \subsetneq B\), but that implies \(C \subseteq B\) and we have known \(B \subseteq C\), together implies \(B = C\) by anti-symmetry of set-inclusion. But this contradicts \(B \subsetneq C\) because that implies \(B \neq C\).

\begin{note}
根據本書的脈絡,我們直到\ \PROP{3.1.18} 才能用\ \(A \subseteq B \land B \subseteq A\) 的方法來證明\ \(A = B\)。
\end{note}

\begin{remark} \label{remark 3.1.19}
There is a relationship between subsets and unions: see for instance \EXEC{3.1.7}. 這就頭腦體操。
\end{remark}
\section{Russel's Paradox}\label{sec 3.2}

\begin{axiom} [Universal specification] \label{axm 3.8} (\RED{Dangerous!})
Suppose for every object \(x\) we have a property \(P(x)\) pertaining to \(x\) (so that for every \(x\), \(P(x)\) is either a true statement or a false statement). Then there exists a set \( \{x : P(x) \text{\ is true} \} \) such that for every object \(y\),
\[
    y \in \{x : P(x) \text{\ is true} \} \iff P(y) \text{\ is true}.
\]
This axiom is also called \emph{axiom of comprehension}. It asserts that \emph{every property corresponds to a set}. This axiom also implies most of the axioms in the previous section (\EXEC{3.2.1}, mind=blown).
\end{axiom}

\begin{note}
這條「公理」跟\ \AXM{3.5} \AXM{3.6} 的差別就是後兩個都是做用在一個\textbf{已知是集合}的東西上的。
\end{note}

\begin{note}
接下來的東西都是在聊羅素悖論,去查各種影片可能會比較有趣。
\end{note}

\begin{note}
We shall simply postulate an axiom which ensures that absurdities such as Russell’s paradox do not occur.
\end{note}

\begin{axiom} \label{axm 3.9}
If \(A\) is a non-empty set, then \textbf{there is} at least one element \(x\) of \(A\) which is either \textit{not a set}, \textit{or} is \textit{disjoint} from \(A\).
\end{axiom}

\begin{note}
BTW, the description of \AXM{3.9} uses the term \emph{disjoint}, which depends on the definition of intersection(\DEF{3.1.23}), which depends on the definition of \AXM{3.5}, so I assume \AXM{3.5} is also asserted when \AXM{3.9} is asserted.
\end{note}

\begin{note}
這個公理可能可以參考一下其他地方(e.g. \href{https://www.wikiwand.com/en/Axiom_of_regularity}{wiki})怎麼描述的,因為這本書是假設有些\ object 不是\ set,但是\ ZFC(或者\ pure set theory?) 實際上所有東西都是\ set。不過看起來其實就是把\ ``\(x\) is either not a set'' 拔掉而已。
\end{note}

\begin{note}
One particular consequence of this axiom is that \textbf{sets are no longer allowed to contain themselves}. (\EXEC{3.2.2}) So if something contains itself, then it is not a set.
\end{note}

\begin{note}
This axiom(\AXM{3.9}) is never needed for the purposes of doing analysis.
\end{note}

\exercisesection

\begin{exercise} \label{exercise 3.2.1}
Show that the universal specification axiom, \AXM{3.8}, if assumed to be true, would imply \AXM{3.2} (empty set), \AXM{3.3} (singleton and pair), \AXM{3.4} (union), \AXM{3.5} (specification), and \AXM{3.6} (replacement). (If we assume that all natural numbers are objects, we also obtain \AXM{3.7}.) Thus, this axiom, if permitted, would simplify the foundations of set theory tremendously (and can be viewed as one basis for an intuitive model of set theory known as ``\emph{naive set theory}''). Unfortunately, as we have seen, \AXM{3.8} is ``too good to be true''!
\end{exercise}

\begin{proof}
First the exercise does not say \AXM{3.8} implies \AXM{3.1}, so it seems that \AXM{3.1} is also needed.

For \AXM{3.2}, we just let \(P(x) := \text{false}\) for any object \(x\). Then by \AXM{3.8} there exists a set \( \emptyset := \{ x : P(x) \} \). Then we must have \(\forall x : x \notin \emptyset\), otherwise if \(x \in \emptyset\) then by definition of \(\emptyset\), \(P(x)\) is true, which contradicts that \(P(x)\) is false.

For \AXM{3.3}, given particular but arbitrary objects \(a, b\), we just let \(P(x) := x = a\) and \(Q(x) := x = a \lor x b\). Then by \AXM{3.8} there exist sets  \(A := \{ x : P(x) \} \) and \(B := \{ x : Q(x) \} \), i.e. \(A := \{ x : x = a \} \) and \(B := \{ x : x = a \lor x \ b \} \). Hence \AXM{3.3} (existence of singleton and pair set) is satisfied.

For \AXM{3.4} Let \(A, B\) be particular but arbitrary sets. And Let \(P(x) := x \in A \lor x \in B\). Then by \AXM{3.8}, there exists a set \(C := \{x : P(x)\} \), that is, \(C := \{x : x \in A \lor x \in B\}\). Thus the union of \(A, B\) exists. Hence \AXM{3.4} is satisfied.

For \AXM{3.5}. Let \(A\) be a particular but arbitrary set and \(P(x)\) be a particular but arbitrary statement that is either true or false for any object \(x \in A\). Then let \(Q(x) := x \in A \land P(x) \text{ is true}\). By \AXM{3.8}, there exists a set \(B := \{ x : Q(x) \} \), that is, \(B := \{x : x \in A \land P(x) \text{ is true}\}\). Thus the specification set \(B\) of \(A\) with statement \(P\) exists. Hence \AXM{3.5} is satisfied.

For \AXM{3.6}. Let \(A\) be a particular but arbitrary set and \(P(x, y)\) satisfy the hypothesis in \AXM{3.6}. By \AXM{3.8}, there exists a set \(B := \{y : P(x, y) \text{ is true}\} \land x \in A\), that is, \(B := \{y : P(x, y) \text{ is true for some \(x \in A\)} \} \). So B is the replacement set of \(A\). Hence \AXM{3.6} is satisfied.

Finally, for \AXM{3.7}, suppose all natural numbers are objects. Then let \(P(x)\) := ``\(x\) is a natural number'' :). Then there exists a set \( \SET{N} := \{ x : P(x) \} \).

\end{proof}

\begin{note}
The proof of implication of \AXM{3.7} is not rigorous(or escape the detail of the statement \(P\), escape what need to be satisfied to be a natural number). We just need to know they are object and hence can be elements of a set.
\end{note}

\begin{exercise} \label{exercise 3.2.2}
Use the axiom of regularity, \AXM{3.9} (and the singleton set axiom, \AXM{3.3}) to show that if \(A\) is a set, then \(A \notin A\). Furthermore, show that if \(A\) and \(B\) are two sets, then either \(A \notin B\) or \(B \notin A\) (or both).
\end{exercise}

\begin{proof}
Again, we use \AXM{3.1}. So if \(A\) is a set, then \(A\) is an object, and by \AXM{3.3}, \(\{A\}\) is a singleton set.

Suppose for the sake of contradiction that there exists an object \(A\) such that \(A \in A\) \MAROON{(1)}. Then because \(A \in \{A\} \)  \MAROON{(2)}, by \MAROON{(1) (2)} we know \(\{A\} \cap A = \{A\} \neq \emptyset\), i.e. not disjoint. But since the singleton set \( \{A\} \) has only one object \(A\), it implies there is no object \(x\) of \(\{A\}\) such that \(x\) and \( \{A\}\) are disjoint, which contradicts \AXM{3.9}. So the supposition is false, so for any object \(A\), \(A \notin A\).

Also, suppose for the sake of contradiction that there exist sets \(A, B\), such that
\begin{center}
    \(A \in B\) \MAROON{(1)} and \(B \in A\) \MAROON{(2)}.
\end{center}
Now consider the set \(\{A, B\}\) (which exists by pair set of \AXM{3.3}), it has two ``elements'' \BLUE{\(A\)} and \GREEN{\(B\)}. For element \BLUE{\(A\)}, since \(B \in \BLUE{A}\) by \MAROON{(2)} and \(B \in \{A, B\}\), \(B \in \BLUE{A} \cap \{A, B\}\), so \BLUE{\(A\)} and \(\{A, B\}\) are not disjoint. For element \GREEN{\(B\)}, since \(A \in \GREEN{B}\) by \MAROON{(1)} and \(A \in \{A, B\}\), \(A \in \GREEN{B} \cap \{A, B\}\), so \GREEN{\(B\)} and \(\{A, B\}\) are not disjoint. So for each element \(x \in \{A, B\}\), \(x\) and \(\{A, B\}\) is not disjoint, which contradicts with \AXM{3.9}.
\end{proof}

\begin{exercise} \label{exercise 3.2.3}
Show (assuming the other axioms of set theory) that the universal specification axiom, \AXM{3.8}, is equivalent to an axiom postulating the existence of a ``universal set'' \(\Omega\) consisting of all objects (i.e., for all objects \(x\), we have \(x \in \Omega\)). In other words, if \AXM{3.8} is true, then a universal set exists, and conversely, if a universal set exists, then \AXM{3.8} is true. (This may explain why \AXM{3.8} is called the axiom of universal specification.) Note that if a universal set \(\Omega\) existed, then we would have \(\Omega \in \Omega\) by \AXM{3.1} (\(\Omega\) is a set by \AXM{3.8} or universal specification, therefore by \AXM{3.1} it is also an object, and any object is \(\in \Omega\), so \(\Omega \in \Omega\)), contradicting \EXEC{3.2.2}. Thus the axiom of foundation specifically rules out the axiom of universal specification.
\end{exercise}

\begin{proof}
Suppose \AXM{3.8} is true. Then let \(P(x) := true\) for any object \(x\). Then by \AXM{3.8}, there exists at set \(\Omega := \{x : P(x) \text{ is true} \}\), and since \(P(x)\) is true for any object \(x\), \(x \in \Omega\).

Suppose the universal set \(\Omega\) exists. Then given an arbitrary property \(P(x)\) satisfiying the hypothesis of \AXM{3.8}, and by \AXM{3.5} (specification), since \(\Omega\) is a set, there exists the set \(A := \{ x \in \Omega : P(x) \text{ is true} \}\). Now we have to show that for every object \(y\),
\[
    y \in \{x \in \Omega : P(x) \text{ is true} \} \iff P(y) \text{ is true \BLUE{(1)}}
\]
to show that \(y \in \{x \in \Omega : P(x) \text{ is true} \}\) is in fact \(y \in \{x : P(x) \text{ is true} \}\) whose existence is guaranteed by \AXM{3.8}.
So let \(y\) be an object.
\begin{itemize}
    \item Suppose \(y \in \{x \in \Omega : P(x) \text{ is true} \). Then trivally \(P(y)\) is true.
    \item Suppose \(P(y)\) is true \MAROON{(1)}. Then since \(y\) is an object, \(y \in \Omega\) \MAROON{(2)}. By \MAROON{(1) (2)}, \(y \in \{x \in \Omega : P(x) \text{ is true} \} \).
\end{itemize}
So \BLUE{(1)} is proved.
\end{proof}


\renewcommand\thechapter{\Alph{chapter}}
\setcounter{chapter}{0}
\chapter{Appendix: the basics of mathematical logic}\label{ch a}

\setcounter{section}{6}
\section{Equality}\label{section a.7}

\setcounter{axiom}{3}
\begin{axiom} [Substitution axiom] \label{axm a.7.4}
Given any two objects \(x\) and \(y\) of the \emph{same type}, if \(x = y\), then \(f(x)= f(y)\) for all \emph{functions or operations} \(f\).
Similarly, for any property \(P(x)\) depending on \(x\),if \(x = y\), then \(P(x)\)and \(P(y)\) are equivalent statements.
\end{axiom}

\end{CJK*}
\end{document}
