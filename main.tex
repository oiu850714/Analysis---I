\documentclass[12pt]{book}
\usepackage{xspace}
\usepackage[T1]{fontenc}
\usepackage[margin=2cm]{geometry}
\usepackage{amsthm}
\usepackage{amssymb}
\usepackage{amsmath}
\usepackage{mathtools}
\usepackage{dirtytalk}
\usepackage[normalem]{ulem}

\usepackage[dvipsnames]{xcolor}

\usepackage{parskip}

\usepackage{CJKutf8} % 讓我寫中文

\usepackage{hyperref}
\hypersetup{
    colorlinks=true,
    linkcolor=blue,
    filecolor=magenta,
    urlcolor=cyan,
}
\urlstyle{same}

\title{Analysis I}
\author{ChienChiWang}

\begin{document}
\begin{CJK*}{UTF8}{bsmi}

\maketitle

\tableofcontents{}

\theoremstyle{definition}
\newtheorem{axiom}{Axiom}[chapter]
\newtheorem{assumption}[axiom]{Assumption}
\newtheorem{additional corollary}{Additional Corollary}[section]
\newtheorem{exercise}{Exercise}[section]
\newtheorem{theorem}{Theorem}[section]
\newtheorem{corollary}[theorem]{Corollary}
\newtheorem{definition}[theorem]{Definition}
\newtheorem{example}[theorem]{Example}
\newtheorem{lemma}[theorem]{Lemma}
\newtheorem{proposition}[theorem]{Proposition}
\newtheorem{remark}[theorem]{Remark}
\newtheorem*{note}{Note}

\theoremstyle{remark}
\newtheorem*{meta-proof}{Meta-proof}

\renewcommand{\labelenumi}{\textnormal{(}\alph{enumi}\textnormal{)}}

\DeclarePairedDelimiter\abs{\lvert}{\rvert}
\DeclarePairedDelimiter\ceil{\lceil}{\rceil}
\DeclarePairedDelimiter\floor{\lfloor}{\rfloor}

\newcommand{\exercisesection}{
    \begin{center}
        --- Exercises ---
    \end{center}
}

\newcommand{\AXM}[1]{Axiom \ref{axm #1}}
\newcommand{\AC}[1]{Additional Corollary \ref{ac #1}}
\newcommand{\DEF}[1]{Definition \ref{def #1}}
\newcommand{\LEM}[1]{Lemma \ref{lem #1}}
\newcommand{\PROP}[1]{Proposition \ref{prop #1}}
\newcommand{\CORO}[1]{Corollary \ref{corollary #1}}
\newcommand{\EXAMPLE}[1]{Example \ref{example #1}}
\newcommand{\EXEC}[1]{Exercise \ref{exercise #1}}
\newcommand{\RMK}[1]{Remark \ref{remark #1}}
\newcommand{\THM}[1]{Theorem \ref{thm #1}}
\newcommand{\SEC}[1]{Section \ref{sec #1}}
\newcommand{\CH}[1]{Chapter \ref{ch #1}}
\newcommand{\EQUATION}[1]{Equation \ref{eq #1}}

\newcommand{\INC}{\texttt{++}}
\newcommand{\X}{\times}
\newcommand{\FLOOR}[1]{\lfloor{#1}\rfloor}
\newcommand{\BLUE}[1]{\textcolor{RoyalBlue}{#1}}
\newcommand{\GREEN}[1]{\textcolor{YellowGreen}{#1}}
\newcommand{\MAROON}[1]{\textcolor{Maroon}{#1}}
\newcommand{\RED}[1]{\textcolor{Red}{#1}}

\newcommand{\SET}[1]{\mathbf{#1}}%

\begin{note}
The various colors in proofs are just for the notational purpose.
\end{note}

\chapter{Introlduction}\label{ch 1}
\begin{note}
This chapter gives some motivation about why we need to rigorously derive various properties we have (blindly) used, and gives example of improper usage of these properties, leading to nonsense.
\end{note}
\chapter{Natural Numbers}\label{ch 2}

\section{The Peano axioms}\label{sec 2.1}

\begin{axiom}\label{axm 2.1}
\( 0 \) is a natural number.
\end{axiom}
\begin{axiom}\label{axm 2.2}
If \( n \) is a natural number, then \( n\INC \) is also a natural number
\end{axiom}

\setcounter{theorem}{2}
\begin{definition}\label{def 2.1.3}
We define \( 1 \) to be the number \( 0\INC \), \(2\) to be the number \( (0\INC)\INC \), 3 to be the number \( ((0\INC)\INC)\INC \), etc.
In other words, \( 1 := 0\INC \), \( 2 := 1\INC \), \( 3 := 2\INC \), etc.
\end{definition}

\begin{note}
In this text the author uses ``\(x := y\)'' to denote the statement that \(x\) is \emph{defined} to equal \(y\).
\end{note}

\begin{proposition}\label{prop 2.1.4}
\(3\) is a natural number.
\end{proposition}
\begin{proof}
\( 3 = 2\INC = (1\INC)\INC = ((0\INC)\INC)\INC \) by definition of \(3\), \(2\) and \(1\). And by \AXM{2.1}, \(0\) is a natural number, so by \AXM{2.2}, \(0\INC\) is a natural number, and again by \AXM{2.2}, \((0\INC)\INC\) is a natural number, and again by \AXM{2.2}, \( ((0\INC)\INC)\INC \), which is equal to \(3\), is a natural number, so \(3\) is a natural number.
\end{proof}

\begin{example}\label{example 2.1.5}
這裡就在說明一個會\ \emph{wrap around} 的\ number system (例如 32-bit 整數, 或者時鐘上的數字) 符合\ Axiom \ref{axm 2.1}\ 跟\ Axiom \ref{axm 2.2},但不會是我們想要的\ number system。
\end{example}

\begin{axiom}\label{axm 2.3}
\(0\) is not the successor of any natural number; i.e. we have \(n\INC \neq 0\) for every natural number \(n\)
\end{axiom}

\begin{proposition}\label{prop 2.1.6}
\(4\) is not equal to \(0\)
\end{proposition}
\begin{proof}
By definition, 4 = 3\INC. By Proposition \ref{prop 2.1.4} we know 3 is a natural number, and by \AXM{2.3} \(3\INC \neq 0\), so \(4 = 3\INC \neq 0\).
\end{proof}

\begin{example}\label{example 2.1.7}
這裡又舉兩個 number system,符合\ Axiom \ref{axm 2.1} - \ref{axm 2.3},但是一樣不是自然數。第一個是\ \(\{0,1,2,3,4\}\),但\ \(4\INC = 4\)。另一種情形是 \(4\INC = 1\),沒有\ wrap around 回\ 0。
\end{example}

\begin{axiom}\label{axm 2.4}
Different natural numbers must have different successors; i.e. if \(n\), \(m\) are natural numbers and \(n \neq m\), then \(n\INC \neq m\INC\). \emph{Contrapositively}, if \(n\INC = m\INC\), then \(n = m\).
\end{axiom}

\begin{note}
Axiom \ref{axm 2.4} 的\ \emph{converse}\ 也是成立的,i.e. if \(n = m\), then \(n++ = m++\)。但那是因為整數的``等號''滿足\ Axiom of substitution \ref{axm a.7.4}。但是\dots書上好像沒有定義整數的等號是什麼意思,就參考\ \href{https://www.wikiwand.com/en/Peano_axioms#/Formulation}{wiki} 吧
\end{note}

\begin{proposition}\label{prop 2.1.8}
\(6\) is not equal to \(2\).
\end{proposition}
\begin{proof}
Suppose \(6 = 2\), then \(5\INC = 6 = 2 = 1\INC\), so \(5\INC = 1\INC \), so by \AXM{2.4}, \(5 = 1\). Similarly, \(4\INC = 0\INC\), so by \AXM{2.4}, \(4 = 0\). But this contradicts Proposition \ref{prop 2.1.6}, so \(6 \neq 2\).
\end{proof}

\begin{example}\label{example 2.1.9}(Informal, 因為實際上用到了實數)
這題令\ \( \textup{N} = \{0, 0.5, 1, 1.5, 2, 2.5, ...\}\),則\ \(\textup{N}\) 滿足\ Axiom \ref{axm 2.1}-\ref{axm 2.4},但他也不是自然數。
\end{example}

\begin{axiom}\label{axm 2.5} (Principle of mathematical induction). Let \(P(n)\) be any property pertaining to a natural number \(n\). Suppose that \(P(0)\) is true, and suppose that whenever \(P(n)\) is true, \(P(n\INC)\) is also true. Then \(P(n)\) is true for every natural number \(n\).
\end{axiom}

\begin{remark}\label{rmk 2.1.10} Axiom \ref{axm 2.5} 的敘述其實用了不少尚未定義的名詞,例如何謂\ property? 其實就是數理邏輯說的\ predicate。另外\ Axiom \ref{axm 2.5} 實際上是\ ``\href{https://www.wikiwand.com/en/Axiom_schema}{Axiom schema}'',有點像一種模板,可以生出任一個\  axiom。這必須去研究邏輯學。
\end{remark}

\begin{note}
就算有了\ Axiom \ref{axm 2.1}-\ref{axm 2.5},目前還是無法證明\  \ref{example 2.1.9} 的例子不是自然數;在目前這個階段,嚴格來說\ \ref{example 2.1.9} 對我們甚至根本就是\ ill-formed。
\end{note}

\begin{proposition}\label{prop 2.1.11}
這題根本就只是數學歸納法證明時的模板而已。
\end{proposition}

\begin{note}
除了\ Proposition \ref{prop 2.1.11} 的流程,數學歸納法還有變形,如\ backward induction (看\  \EXEC{2.2.6})、strong induction (看\ \PROP{2.2.14})、還有\ transfinite induction(看\  \LEM{8.5.15});BTW 最後一個根本不是數學歸納法...
\end{note}

\begin{assumption}\label{assumption 2.6} (Informal)
There exists a number system \(\SET{N}\), whose elements we will call natural numbers, for which Axiom \ref{axm 2.1}-\ref{axm 2.5} are true.
\end{assumption}
\begin{note}
Assumption \ref{assumption 2.6} 的嚴謹定義實際上是\ Axiom \ref{axm 3.7},這需要理解集合以及函數的定義後才有意義。這個\ Assumption 的意義是保證存在一個\ ``set'',滿足 Axiom \ref{axm 2.1}-\ref{axm 2.5},也就是皮亞諾公理。
\end{note}

\begin{remark}\label{remark 2.1.12} Assumption \ref{assumption 2.6} 的\ number system \(\SET{N}\) 我們稱作\ \emph{the} natural number system; i.e. 他是唯一的。其他的符合\ Axiom \ref{axm 2.1}-\ref{axm 2.5} 的\ number system 實際上都跟\ \(\SET{N}\) \emph{同構},請看\  \EXEC{3.5.13}。這邊的同構是指兩個\ set 存在\ 1-1 correspondence,並且這個\ 1-1 correspondence \emph{preserve} \INC\ operation。
\end{remark}

\begin{remark}\label{remark 2.1.13}
這也是幹話,所有自然數都是有限的(可根據\ Axiom \ref{axm 2.5} 得到),但是自然數這個\ number system 卻是``無限''的。這應該要到第八章才能完全理解。
\end{remark}

\begin{remark}\label{remark 2.1.14}
我們目前定義自然數的方式是\ \emph{axiomatic}(用公理定出來的),而非\ \emph{constructive}(建構出來的)。要用建構的方式兜出自然數也可以,可參考看公理化集合論。但這邊的重點就是我們假設某個集合存在,且符合皮亞諾公理,以這個為前提開始推導出各種性質。我們(暫時)不去探討為何那個集合存在。這就跟線性代數有點類似,至少在線代的課程本身,探討為何會存在集合(on a field)符合那\ \(2+8\) 個\ vector space 的性質從來不是重點,而是一種前提。
\end{remark}

\begin{remark}\label{remark 2.1.15}
這邊是說用公理化的方式來定義自然數其實是很近代的事情(就皮亞諾那個時代開始)。在這之前自然數(在哲學/直覺/etc 上?)都是依附在某種既有的物件上的概念(connected to some external concept),例如某種物理量(如長度)等等。
\end{remark}

\begin{proposition} [\RED{I don't know all this bloody hell}] \label{prop 2.1.16}
 這題我完全問號,連看懂都有問題;但他的目的應該是我們可以從自然數來定義\ sequence。可是這一頁的註腳也說這會扯到\ function 的概念,實際上要去看第三章(Exercise \ref{exercise 3.5.12})。\textbf{但重點是這個\ forward reference 在證明上不會\ circular,因為 function 的定義不依賴皮亞諾公理}。另外這題蠻多人問的,總之看完第三章之後要回來搞懂。
\end{proposition}
\section{Addition}\label{sec 2.2}

\newcommand{\BLUE}[1]{\textcolor{RoyalBlue}{#1}}
\newcommand{\GREEN}[1]{\textcolor{YellowGreen}{#1}}
\newcommand{\MAROON}[1]{\textcolor{Maroon}{#1}}
\newcommand{\RED}[1]{\textcolor{Red}{#1}}

\begin{definition} [Addition of natural numbers] \label{def 2.2.1} Let \(m\) be a natural number. To add zero to \(m\), we \emph{define} \(0 + m := m\). Now suppose \emph{inductively} that we have defined how to add \(n\) to \(m\). Then we can add \(n\INC \) to \(m\) by defining \((n\INC) + m\) := \((n + m)\INC\).
\end{definition}
\begin{note}
So,
\begin{align*}
\BLUE{0 + m = m}\\
1 + m = (0\INC) + m = \BLUE{(0 + m)}\INC = \BLUE{m}\INC \implies \GREEN{1 + m = m\INC} \\
2 + m = (1\INC) + m = (\GREEN{1 + m})\INC = \GREEN{(m\INC)}\INC \implies 2 + m = (m\INC)\INC
\end{align*}
and so on.
\end{note}
\begin{note}
注意,我們有定義\ \(0 + m\) 是什麼,但我們沒有定義\ \(m + 0\) 是什麼,i.e. 需要證明他們兩個相等。參考\ Lemma \ref{lem 2.2.2}。
\end{note}
\begin{note}
這個定義因為符合數學歸納法(Axiom \ref{axm 2.5}),所以給定一個自然數\ \(m\), 對於所有自然數\ \(n\),我們都對\ \(n + m\) 做了定義
\end{note}

\begin{additional corollary}\label{ac 2.2.1} (see page. 24 below) Using Axioms 2.1, 2.2, and induction (Axiom 2.5), that the sum of two natural numbers is again a natural number 
\end{additional corollary}
\begin{proof}
Let \(m\) be a natural number.

Base case: \(0 + m = m\) by Definition \ref{def 2.2.1}. But \(m\) is a natural number, so \(0 + m\) is a natural number. 

Inductive hypothesis: suppose \(n + m\) is a natural number. Wanted: \((n\INC) + m\) is a natural number. By Definition \ref{def 2.2.1} \BLUE{\((n\INC) + m = (n + m)\INC\)}, but by inductive hypothesis, \(n + m\) is a natural number, and by Axiom \ref{axm 2.2}, \BLUE{\((n + m)\INC\)} is a natural number, so \BLUE{\((n\INC) + m\)} is a natural number. This closes the induction.
\end{proof}

\begin{note}
Definition \ref{def 2.2.1} 就可以讓我們推出所有以前就在使用的加法的規則了,例如交換律/結合律。之後都會證明。
\end{note}

\begin{lemma}\label{lem 2.2.2}
For any natural number \(n\), \(n + 0 = n\).
\end{lemma}
\begin{proof} We use induction.

The base case \(0 + \BLUE{0} = \BLUE{0}\) follows since we know(by Definition \ref{def 2.2.1}) that \(0 + \BLUE{m} = \BLUE{m}\) for every natural number \(\BLUE{m}\), and \(\BLUE{0}\) is a natural number.

Now suppose inductively that \(n + 0 = n\). We wish to show that \((n\INC)+0 = n\INC\). But by Definition \ref{def 2.2.1}, \((n\INC) + 0\) is equal to \((n + 0)\INC\), which is equal to \(n\INC\) since \(n + 0 = n\). This closes the induction.
\end{proof}

\begin{lemma}\label{lem 2.2.3} For any natural numbers \(n\) and \(m\), \(\BLUE{n + (m\INC)} = \GREEN{(n + m)\INC}\)
\end{lemma}
\begin{note}
這個\ lemma 內等號\ LHS 的\ operand 的順序又跟\ Definition \ref{def 2.2.1} 相反,是\ \(\BLUE{n + (m\INC)}\) 而不是\  \((m\INC) + n\),所以不能直接從\ Definition \ref{def 2.2.1} 得知這個\ Lemma。
\end{note}
\begin{proof}
We induct on \(n\) (keeping \(m\) fixed).

Base case: \(n = 0\). In this case we have to prove \( \BLUE{0 + (m\INC)} = (\GREEN{0 + m})\INC\). But by Definition \ref{def 2.2.1}, \(\BLUE{0 + (m\INC)} = m\INC\) and \(\GREEN{0 + m} = m\), so both sides are equal to \(m\INC\) and are thus equal to each other.

Now we assume inductively that \(n + (m\INC) = (n + m)\INC\); we now have to show that \(\BLUE{(n\INC) + (m\INC)} = (\GREEN{(n\INC) + m})\INC\). The \BLUE{left-hand side} is \((n + (m\INC))\INC\) by Definition \ref{def 2.2.1}, which is equal to \(((n + m)\INC)\INC\) by the inductive hypothesis. Similarly, we have \(\GREEN{(n\INC) + m} =(n + m)\INC\) by the Definition \ref{def 2.2.1}, and so the right-hand side is also equal to \(((n + m)\INC)\INC\). Thus both sides are equal to each other, and we have closed the induction.
\end{proof}

\begin{additional corollary}\label{ac 2.2.2}(on page 26 above)\(n\INC = n +1\)
\begin{note}
我們可從\ Definition \ref{def 2.2.1} 得知\ \(1 + n = n\INC\),但不能得知\ \(n + 1 = n\INC\)
\end{note}
\begin{proof}
    \begin{align*}
        n\INC & = n\INC + 0 & \text{by Lemma \ref{lem 2.2.2}} \\
              & = (n + 0)\INC & \text{by Definition \ref{def 2.2.1}} \\
              & = n + (0\INC) & \text{by Lemma \ref{lem 2.2.3}} \\
              & = n + 1
    \end{align*}
\end{proof}
\end{additional corollary}

\begin{proposition}[Addition is commutative]\label{prop 2.2.4} For any natural numbers \(n\) and \(m\), \(n + m = m + n\).
\end{proposition}
\begin{proof}
We shall use induction on \(n\) (keeping \(m\) fixed).

First we do the base case \(n = 0\), i.e., we show \(0 + m = m + 0\). By the Definition \ref{def 2.2.1}, \(0 + m = m\), while by Lemma \ref{lem 2.2.2}, \(m + 0 = m\). Thus the base case is done.

Now suppose inductively that \(n + m = m + n\), now we have to prove that \(\BLUE{(n\INC) + m} = \GREEN{m +(n\INC)}\) to close the induction. By the Definition \ref{def 2.2.1}, \(\BLUE{(n\INC) + m} = (n + m)\INC\). By Lemma \ref{lem 2.2.3}, \(\GREEN{m + (n\INC)} = (m + n)\INC\), but this is equal to \((n + m)\INC\) by the inductive hypothesis \(n + m = m + n\). Thus \((n\INC) + m = m +(n\INC)\) and we have closed the induction.
\end{proof}

\begin{proposition}[Addition is associative]\label{prop 2.2.5} For any natural numbers \(a\), \(b\), \(c\), we have \((a + b) + c = a + (b + c)\).
\end{proposition}
\begin{proof}
We induct on \(c\), keep a, b fixed.

Let \(a\), \(b\) be particular but arbitrarily chosen natural numbers.

Base case: for \(c = 0\), we want to prove \((a + b) + 0 = a + (b + 0))\). For the LHS, \((\BLUE{a + b}) + 0 = \BLUE{a + b}\) by Lemma \ref{lem 2.2.2}. For the RHS, \(a + \GREEN{(b + 0)} = a + \GREEN{b}\) by Lemma \ref{lem 2.2.2}. So both sides equal to \(a + b\) and thus equal to each other.

Inductive hypothesis: suppose \((a + b) + c = a + (b + c)\), we want to prove \(\BLUE{(a + b) + (c\INC)} = \GREEN{a + (b + (c\INC))}\). By Lemma \ref{lem 2.2.3}, LHS = \(\BLUE{(a + b) + (c\INC)} = \MAROON{((a + b) + c)\INC}\). By Lemma \ref{lem 2.2.3}, RHS = \(\GREEN{a + (b + (c\INC))} = a + (b + c)\INC\), which again by Lemma \ref{lem 2.2.3} = \((a + (b + c))\INC\). But by inductive hypothesis, \((a + b) + c = a + (b + c)\), so \((a + (b + c))\INC = \MAROON{((a + b) + c)\INC}\), which equals to LHS. So RHS = LHS, so we closed the induction.
\end{proof}

\begin{note}
Because of this associativity we can write sums such as \(a + b + c\) without having to worry about which order the numbers are being added together.
\end{note}

\begin{proposition}[Cancellation law]\label{prop 2.2.6} Let \(a\), \(b\), \(c\) be natural numbers such that \(a + b = a + c\). Then we have \(b = c\).
\end{proposition}
\begin{note}
我們不能用任何「減法」或者是「負數」的概念來證明自然數消去法。事實上這本書就是用這個消去法來定義「虛擬減法」(virtual subtraction),是一種「形式上」(formal)的減法,然後會進一步證明這跟整數減法等價。可參考第四章\ Definition \ref{def 4.1.1}。
\end{note}
\begin{proof}
We proof this by induction on \(a\).

Base case: let \(a = 0\), we have to prove \(0 + b = 0 + c\ \implies b = c\). But by Definition \ref{def 2.2.1}, \(0 + b = b\), and \(0 + c = c\), so \(b = 0 + b = 0 + c = c\), so \(b = c\).

Inductive hypothesis: suppose \(a + b = a + c \implies b = c\), we have to prove \(\BLUE{(a\INC) + b} = \GREEN{(a\INC) + c} \implies b = c\). By Definition \ref{def 2.2.1}, \BLUE{LHS} = \((a + b)\INC\), \GREEN{RHS} = \((a + c)\INC\), so \((a + b)\INC = (a + c)\INC\), and by Axiom (of contrapositive of) \ref{axm 2.4}, \(a + b = a + c\), which by inductive hypothesis implies \(b = c\). This closes the induction.
\end{proof}

\begin{note}
至此加法的基本規則都證明了,接下來要討論加法跟「正數」的關係。
\end{note}

\begin{definition}[Positive natural numbers] \label{def 2.2.7} A natural number \(n\) is said to be positive if and only if it is not equal to \(0\).
\end{definition}
\begin{note}
沒錯,在自然數這個系統且定義包含了\ \(0\) 的情況下,「正數」的意思就是「不是\ \(0\) 的自然數」。
\end{note}

\begin{proposition}\label{prop 2.2.8} If \(a\) is a positive natural number and \(b\) is a natural number, then \(a + b\) is positive (and hence \(b + a\) is also, by Proposition \ref{prop 2.2.4}).
\end{proposition}

\begin{proof}
We prove by induction on \(b\). Let \(a\) be a positive natural number.

Base case: let \(b = 0\). Then
\begin{align*}
a + b & = a + 0 & \\
      & = 0 + a & \text{by Proposition \ref{prop 2.2.4} (commutative law)} \\
      & = a     & \text{by Definition \ref{def 2.2.1}}
\end{align*}
which by definition is a positive natural number.

Inductive hypothesis: Suppose \(a + b\) is positive, we have to prove \(a + (b\INC)\) is positive. Then by Lemma \ref{lem 2.2.3}, \(a + (b\INC) = (a + b)\INC\). But by inductive hypothesis \(a + b\) is positive, which by Definition \ref{def 2.2.7} is not equal to \(0\), so by Axiom \ref{axm 2.3}, \((a + b)\INC\) is also not equal to \(0\), which again by Definition \ref{def 2.2.7} is positive. This closes the induction.
\end{proof}

\begin{corollary} \label{corollary 2.2.9}
If \(a\) and \(b\) are natural numbers such that \(a + b = 0\), then \(a = 0\) and \(b = 0\).
\end{corollary}

\begin{proof}
Suppose for the sake of contradiction that \(a + b = 0\) but \(a \neq 0\) or \(a \neq 0\). If \(a \neq 0\), then by Definition \ref{def 2.2.7} \(a\) is positive and then by Proposition \ref{prop 2.2.8} \(a + b\) is positive and by Definition \ref{def 2.2.7} cannot be \(0\), which contradicts \(a + b = 0\). If \(b \neq 0\), then similarly \(b\) is positive and \(b + a\) is positive, and by Proposition \ref{prop 2.2.4} \(= a + b\), so \(a + b\) is positive, again a contradiction. Thus both \(a\) and \(b\) must be \(0\).
\end{proof}

\begin{lemma}\label{lem 2.2.10}
Let \(a\) be a positive natural number. Then there \emph{exists exactly one} natural number \(b\) such that \(b\INC = a\).
\end{lemma}
\begin{note}
注意這是個\ imply 陳述,如果前提不成立(i.e. \(a\) 不是\ positive natural number)則會直接\ vacuously true!
\end{note}
\begin{proof}
We prove by induction on \(a\).

Base case: let \(a = 0\). But by Definition \ref{def 2.2.7} \(a\) is not positive. So this Lemma is vacuously true.

Inductive hypothesis: suppose \(a\) is a positive natural number and \(b\) is the unique natural number such that \(b\INC = a\). We have to prove that there exists exact one natural number \(b'\) such that \(b'\INC = a\INC\). Let \(b' = b\INC\). Then by Axiom of Substitution \ref{axm a.7.4} \(b' = a\), and \(b'\INC = a\INC\) again by substitution, so the existence part is satisfied. Now we prove the unique part. Suppose there also exists a natural number \(c\) such that \(c\INC = a\INC\). Then by Axiom \ref{axm 2.4}, \(c = a\), and since \(b' = a\), so \(c = b'\), which means \(b'\) is unique. This closed the induction.
\end{proof}

\begin{note}
現在有了加法的定義跟一些性質,以及正數的定義,我們可以來定義自然數的「order」了。
\end{note}

\begin{definition}[Ordering of the natural numbers] \label{def 2.2.11} Let \(n\) and \(m\) be natural numbers. We say that \(n\) is \emph{greater than or equal to} \(m\), and write \(n \geq m\) or \(m \leq n\), if and only if we have \(n = m + a\) for some natural number \(a\). We say that n is \emph{strictly greater than} \(m\), and write \(n > m\) or \(m < n\), if and only if \(n \geq m\) and \(n \neq m\).
\end{definition}
\begin{note}
\(n \geq m\) 跟\ \(m \leq n\) 只有符號上的差別,他們在定義上是等價的。另外定義內的\ \(a\) 只需要是\ natural number 即可,不需要是\ "positive" natural number。
\end{note}
\begin{additional corollary} \label{ac 2.2.3}
\(n\INC > n\) for any natural number \(n\),因為\ (1) \(n\INC \neq n\)(否則會違反\ Axiom \ref{axm 2.4}) (2) \(n\INC \geq n\),因為\ \(n\INC = n + 1\)(by Additional Corollary \ref{ac 2.2.2}),而\ \(1\) 就是個\ natural number。

所以,這代表沒有最大的自然數。
\end{additional corollary}

\begin{proposition} [Basic properties of order for natural numbers] \label{prop 2.2.12}
Let \(a\), \(b\), \(c\) be natural numbers. Then
    \begin{enumerate}
        \item (Order is reflexive) \(a \geq a\).
        \item (Order is transitive) If \(a \geq b\) and \(b \geq c\), then \(a \geq c\).
        \item (Order is \emph{anti}-symmetric) If \(a \geq b\) and \(b \geq a\),then \(a = b\). 
        \item (Addition preserves order) \(a \geq b\) if and only if \(a + c \geq b + c\). 
        \item \(a < b\) if and only if \(a\INC \leq b\).
        \item \(a < b\) if and only if \(b = a + d\) for some \emph{positive} number \(d\).
    \end{enumerate}
\end{proposition}
\begin{note}
若忘了\ anti-symmetric 是什麼,可以回去翻離散課本講\ relation 的部分。
\end{note}
\begin{proof}{(a)}
Let \(a\) be a natural number. Then by Definition \ref{def 2.2.11}, \(\BLUE{a + 0} \geq \GREEN{a}\) because \(0\) is a natural number. But by Lemma \ref{lem 2.2.2}, \(\BLUE{a + 0} = \BLUE{a}\), so by Axiom of Substitution (\ref{axm a.7.4}), \(\BLUE{a} \geq \GREEN{a}\).
\end{proof}

\begin{proof}{(b)}
Let \(a, b, c\) be natural numbers s.t. \(a \geq b\) and \(b \geq c\). Then by Definition \ref{def 2.2.11}, \(\exists\ \text{natural numbers}\ m, n\) such that \(a = b + m\) and \(b = c + n\). So
\begin{align*}
    a & = b + m & \\
      & = (c + n) + m & \\
      & = c + (n + m) & \text{by Proposition \ref{prop 2.2.5}}\\
\end{align*}
So \(a = c + (n + m)\) for some natural number \(n + \), so by Definition \ref{def 2.2.11}, \(a \geq c\)
\end{proof}
\begin{proof}{(c)}
Let \(a, b\) be natural numbers s.t. \(a \geq b\) and \(b \geq a\). Then by Definition \ref{def 2.2.11}, \(\exists\ \text{natural numbers}\ m, n\) such that \(a = b + m\) and \(b = a + n\). So
\begin{align*}
    a & = b + m & \\
      & = (a + n) + m & \\
      & = a + (n + m) & \text{by Proposition \ref{prop 2.2.5}}\\
\end{align*}
But by Lemma \ref{lem 2.2.2}, \(a = a + 0\), so \(a + 0 = a + (n + m)\), and by Proposition \ref{prop 2.2.6} (cancellation law), \(0 = n + m\), and by Corollary \ref{corollary 2.2.9}, \(n = 0\) and \(m = 0\). In particular, \(m = 0\), so \(a = b + m = b + 0 = b\).
\end{proof}
\begin{proof}{(d)}
Let \(a, b, c, d\) be natural numbers. Then
\begin{align*}
         & a \geq b \\
    \iff & a = b + d             & \text{by Definition \ref{def 2.2.11}} \\
    \iff & a + c = (b + d) + c   & \text{by Axiom of Substitution \ref{axm a.7.4}} \\ 
    \iff & a + c = c + (b + d)   & \text{by Proposition(commutative law) \ref{prop 2.2.8} } \\
    \iff & a + c = (c + b) + d   & \text{by Proposition(associative law) \ref{prop 2.2.5}} \\
    \iff & a + c = (b + c) + d   & \text{by Proposition \ref{prop 2.2.4}} \\
    \iff & a + c \geq b + c      & \text{by Definition \ref{def 2.2.11}}
\end{align*}
\end{proof}
\begin{proof}{(e)}
Let \(a, b\) be natural numbers.

\(\Longrightarrow \): Suppose \(a < b\). Then by Definition \ref{def 2.2.11} \(a \leq b\) \BLUE{(1)} and \(a \neq b\) \BLUE{(2)}. \BLUE{(1)} implies \(a + m = b\) \MAROON{(1)} for some natural number \(m\), and with \BLUE{(2)} by Definition \ref{def 2.2.7} implies \(m\) is positive (otherwise \(b = a + m = a + 0 = a\) by Lemma \ref{lem 2.2.2}, contradicting \(a \neq b\)). Since \(m\) is positive, by Lemma \ref{lem 2.2.10}, \(m = n\INC\) \MAROON{(2)} for some natural number \(n\). So
\begin{align*}
    b & = a + m & \text{by \MAROON{(1)}}\\
      & = a + n\INC & \text{by \MAROON{(2)}}\\
      & = (a + n)\INC & \text{by Lemma \ref{lem 2.2.3}} \\
      & = a\INC + n & \text{by Definition \ref{def 2.2.1}}
\end{align*}
which by Definition \ref{def 2.2.11} implies \(a\INC \leq b\).

\( \Longleftarrow \): Suppose \(a\INC \leq b\). Then by Definition \ref{def 2.2.11} \(a\INC + m = b\) for some natural number \(m\), And
\begin{align*}
    & \BLUE{a\INC + m} \\
    & = (a + m)\INC & by Definition \ref{def 2.2.1} \\
    & = \BLUE{a + m\INC} & by Lemma \ref{lem 2.2.3}
\end{align*}
So \(\BLUE{a\INC + m = a + m\INC} = b\). By Axiom \ref{axm 2.3}, \(m\INC \neq 0\), so by Definition \ref{def 2.2.7} is positive, so \(a \neq b\). So \(a + m\INC = b\) for some natural number \(m\INC\) and \(a \neq b\), by Definition \ref{def 2.2.11}, \(a < b\).
\end{proof}

\begin{proof}{(f)}

\( \Longrightarrow \): This was proved in the "if part" of (e); the \(m\) in there meets the condition.

\( \Longleftarrow \): This was proved in the middle of the "only if part" of (e); the positive \(m\INC\) derived in the middle can be treated as \(d\) in (f), and the derivation in (e) derives \(a < b\).
\end{proof}

\begin{proposition}[Trichotomy of order for natural numbers] \label{prop 2.2.13}
Let \(a\) and \(b\) be natural numbers. Then exactly one of the following statements is true: \(a < b\), \(a = b\), or \(a > b\).
\end{proposition}
\begin{proof}
The sketch: (1) at most one condition holds, (2) at least one condition holds; this implies exactly one condition holds.

(1) at most one:
    \begin{enumerate}
        \item If \(a < b\) then by Definition \ref{def 2.2.11} \(a \neq b\). Now if \(a > b\) then we have \(a < b \land a > b\), which by Proposition \ref{prop 2.2.12}(c) (with some trivial implications i.e. \(a < b \implies a \leq b\)) implies \(a = b\), contradicting \(a \neq b\).
        \item If \(a > b\) then by Definition \ref{def 2.2.11} \(a \neq b\). Now if \(a < b\) then we have \(a > b \land a < b\), which by Proposition \ref{prop 2.2.12}(c) (with some trivial implications i.e. \(a < b \implies a \leq b\)) implies \(a = b\), contradicting \(a \neq b\).
        \item If \(a = b\) then by Definition \ref{def 2.2.11} \(a \not < b \) and \(b \not < a\).
    \end{enumerate}

(2) at least one: we use induction on \(a\) and keep \(b\) fixed.

Base case: Let \(a = 0\). Then we have \(a = 0 \leq b\), because \(b = \GREEN{b} + 0 = \GREEN{b} + a\), so there exists a natural number \(\GREEN{b}\) s.t. \(b = \GREEN{b} + a\), which by Definition \ref{def 2.2.11} implies \(a \leq b\) (This is the first (why?)). And this implies \(a < b\) or \(a = b\).

Inductive hypothesis: Suppose \(a\) satisfies the "at least one" proposition, i.e. at least one of \(a < b\), \(a = b\), or \(a > b\) is true. We have to prove \(a\INC\) satisfies the "at least one" proposition.
    \begin{enumerate}
        \item Suppose \(a < b\), then by Proposition \ref{prop 2.2.12}(e), \(a\INC \leq b\), which means \(a = b\) or \(a < b\).
        \item Suppose \(a = b\), then \(a\INC = b\INC\) by Axiom of Substitution \ref{axm a.7.4}, and \(b\INC > b\) by Additional Corollary \ref{ac 2.2.3}, so \(a\INC > b\) (This is the third (why?))
        \item Suppose \(a > b\). By Additional Corollary \ref{ac 2.2.3}, \(a\INC > a\). So we have \(a > b\) and \(a\INC > a\), which by Proposition \ref{prop 2.2.12}(b) implies \(a\INC \geq b\). Or more precisely \(a\INC > b\) because if \(a\INC = b\) that implies \(a + 1 = b\), which implies \(a < b\), which contradicts \(a > b\). (This is the second (why?))
    \end{enumerate}
So in either case \(a\INC\) satisfies at least one of the \(a\INC < b\), \(a\INC = b\) or \(a\INC > b\).
\end{proof}

\begin{note}
上面這些\ order 的性質可以讓我們定義一個\ "stronger" version of induction. 把\ stronger 標起來只是因為它只是看起來比較強,但實際上跟數學歸納法等價。
\end{note}

\begin{proposition}[Strong principle of induction] \label{prop 2.2.14}
Let \(m_0\) be a natural number, and let \(P(m)\) be a property pertaining to an arbitrary natural number \(m\). Suppose that \emph{for each} \(m \geq m_0\), \emph{we have the following implication}: if \(P(m')\) is true for all natural numbers \(m_0 \leq m' \RED{<}\ m\), then \(P(m)\) is also true. (In particular, this means that \(P(m_0)\) is true, since in this case the hypothesis is vacuous.) Then we can conclude that \(P(m)\) is true for all natural numbers \(m \geq m_0\).
\end{proposition}
\begin{note}
要仔細看這個\ proposition 是假設有什麼,然後有什麼結論。

這個\ proposition 是說對於任一個(for all) \(\geq m_0\) 的自然數\ \(m\),都滿足一個「包含了\ \(P\) 還有\ \(m_0\) 的\ if then 陳述句」;若這為真,則會\ implies 「對於所有\ \(\geq m_0\) 的自然數\ m,\(P(m)\) 都成立」。

另外當\ \(m = m_0\) 時,那個「被包含的\ if then 陳述句」的\ hypothesis 是\ vacuously true,因為該\ hypothesis 是個\ for all statements,但是\ for all 作用的集合此時為空集合。而如果我們又證明了「那個被包含的\ if then 陳述句」是\ true,則\ (1) if then 陳述句是\ true,\ (2) if-part 也是\ true,這代表\ then part 也是\ true;現在在\ \(m = m_0\) 的情況下,then part 就是\ \(P(m_0)\) is true。而\ \(P(m_0)\) is true 其實就是傳統\ induction 的\ base case。換句話說這代表「整個\ strong induction 自己的\ if-part」內會\ implies base case,所以只要提供了\ strong induction 需要的那種\ if-part 就不需要證明\ base case is true 了。
\end{note}

\begin{remark}\label{remark 2.2.15}
In applications we usually use Proposition \ref{prop 2.2.14} with \(m_0 = 0\) or \(m_0 = 1\).
\end{remark}
\section{Multiplication}\label{sec 2.3}

\begin{note}
這節會直接使用已知的自然數加法還有\ order 的性質,例如我們不會再證明\ \(a + b + c = c + b + a\).

另外會定義自然數乘法。類似自然數加法是\ "iterated increment operation"(重複的增量運算),自然數乘法是\ "iterated addition"(重複的加法運算)。
\end{note}
\chapter{Set Theory} \label{ch3}

For now we pause to introduce the concepts and notation of set theory, as they will be used increasingly heavily in
later chapters.

While set theory is not the main focus of this text, almost every other branch of mathematics relies on set theory as part of its foundation, so it is important to get at least some grounding in set theory before doing other advanced areas of mathematics.

In this chapter we present the more elementary aspects of axiomatic set theory, leaving more advanced topics such as a discussion of infinite sets and the axiom of choice to
\CH{8}.

\begin{note}
到第八章才講那些進階的東西?\ 因為直到第八章之前的定理都不用依靠那些東西。
\end{note}

\section{Fundamentals}\label{sec 3.1}

For pedagogical reasons, we will use a \emph{somewhat overcomplete list of axioms} for set theory, in the sense that some of the axioms can be used to deduce others, but there is \emph{no real harm} in doing this.

\begin{note}
就好像你可以說正整數除了滿足皮亞諾公理也自動滿足\ \(a + b = b + a\),但後者用皮亞諾就可證明。
\end{note}

\begin{definition}[\emph{Informal}] \label{def 3.1.1}
We define a set \(A\) to be any \emph{unordered collection} of objects, e.g., \( \{3, 8, 5, 2\} \) is a set. If \(x\) is an object, we say that \(x\) \emph{is an element of} \(A\) or \(x \in A\) if \(x\) lies in the collection; otherwise we say that \(x \notin A\). For instance, \(3 \in \{1, 2, 3, 4, 5\} \) but \(7 \notin \{1, 2, 3, 4, 5\} \).
\end{definition}

\begin{note}
\DEF{3.1.1} 有很多問題沒有回答,例如「什麼樣的\ collection」 才能被稱作集合,兩個集合怎麼判斷是否相等,怎麼對集合作操作(聯集、交集等等),集合可以做什麼,以及集合的元素(element)可以做什麼。
\end{note}

\begin{axiom}[Sets are objects]\label{axm 3.1}
If \(A\) is a set, then \(A\) is \emph{also an object}. In particular, given two sets \(A\) and \(B\), it is meaningful to ask whether \(A\) is also an element of \(B\).
\end{axiom}

\begin{example}[Informal]\label{example 3.1.2}
這個例子舉例\ \( \{3, \{3, 4\}, 4\} \) 裡面有一個元素也是集合,但敘述方式不嚴謹,要去看\ \SEC{3.6}
\end{example}

\begin{remark}\label{remark 3.1.3}
這裡在探討是否需要把所有\ object 都當成\  set。在邏輯的角度,這樣推論過程比較簡單,因為需要的東西的類型就只有一種,就是\ set,但是從概念上來看,將某些\ object 視為「不是\ set」則會比較單純,比方說給定一個自然數\ \(2\),將他視為一個集合(在\ Analysis 的範疇)沒什麼進一步的應用。是否將所有\ object 都當成集合,\ more or less 是等價的,所以,we shall take an agnostic position as to whether all objects are sets or not.
\end{remark}

\begin{note}
若已知\ \(x\) 是一個\ object 且\ \(A\) 是一個\ set,則要馬\ \(x \in A\) 為真,要馬\ \(x \notin A\) 為真。而若\ \(A\) 不是\ set,則我們視\ \(x \in A\) 為\ undefined。
\end{note}

\begin{definition}[Equality of sets] \label{def 3.1.4} 
Two sets \(A\) and \(B\) are equal, \(A = B\), if and only if every element of \(A\) is an element of \(B\) and vice versa. To put it another way, \(A = B\) if and only if every element \(x\) of \(A\) belongs also to \(B\), and every element \(y\) of \(B\) belongs also to \(A\). Or equivalently,
\[
  \forall\ x : x \in A \iff x \in B
\]
\end{definition}

\begin{example}
嘴砲。
\end{example}

\begin{note}
One can easily verify that this notion of equality is reflexive, symmetric, and transitive (See \EXEC{3.1.1}).
\end{note}

\begin{additional corollary}\label{ac 3.1.1}
The definition of equality in \DEF{3.1.4} is reflexive, symmetric and transitive.
\end{additional corollary}

\begin{proof}

Reflexive: Suppose \(A\) is a set. Then given any object \(x\), if \(x \in \GREEN{A}\), then \(x \in \BLUE{A}\), and given any object \(y\), if \(y \in \BLUE{A}\), then \(y \in \GREEN{A}\). So by \DEF{3.1.4}, \(\GREEN{A} = \BLUE{A}\).

Symmetric: Suppose \(A, B\) are sets and \(A = B\), then by \DEF{3.1.4},
\[
  \forall\ x : x \in A \iff x \in B
\]
But this statement is just equivalent to
\[
  \forall\ x : x \in B \iff x \in A
\]
and this by \DEF{3.1.4} implies \(B = A\).

Transitive: Suppose \(A, B, C\) are sets and \(A = B\) and \(B = C\). Then by \DEF{3.1.4}
\[
  \forall\ x : x \in A \iff x \in B
\]
\[
  \forall\ x : x \in B \iff x \in C
\]
And this implies
\[
  \forall\ x : x \in A \iff x \in B \iff x \in C
\]
And by logic this implies
\[
  \forall\ x : x \in A \iff x \in C
\]
By \DEF{3.1.4}, \(A = C\).
\end{proof}

\begin{note}
``is an element of'' relation \(\in\) 符合\ Axiom of Substitution \AXM{a.7.4},因為
\begin{center}
    if \(x \in A\) and \(A = B\), then \(x \in B\), by \DEF{3.1.4}.
\end{center}
這也代表那些完全以\ \(\in\) 定義的新的集合操作會自動符合\ Axiom of Substitution \AXM{a.7.4}。例如這一節剩下的所有\ operations 都是用\ \(\in\) 來定義的。
\end{note}
\begin{note}
接著\ \RMK{3.1.3},我們繼續來探討什麼\ object 是\ set,什麼不是。有點類似我們定哪些東西為自然數,哪些不是(\AXM{2.1},\(0\) 是自然數,然後用 \AXM{2.2} 來擴增/建構其他的自然數)。這邊在集合論就是先假設存在一個集合,叫「空集合」,然後再定義一些在集合上的操作來建構其他的集合。
\end{note}

\begin{axiom}[Empty set] \label{axm 3.2}
There exists a set \(\emptyset\), known as \emph{the} empty set, which \emph{contains no elements}, i.e., for every object \(x\) we have \(x \notin \emptyset\).
\end{axiom}

\begin{note}
\emph{The} empty set is also denoted \(\{\}\). Note that there can only be \textbf{one} empty set.
\end{note}

\begin{additional corollary} [The empty set is unique] \label{ac 3.1.2}
If there were two sets \(\emptyset\) and \(\emptyset'\) which were both empty, then by \DEF{3.1.4} they would be equal to each other.
\end{additional corollary}
\begin{proof}
Suppose \(\emptyset'\) is also empty. Then the statement
\[
  \forall\ x : x \in \emptyset' \implies x \in \emptyset
\]
is vacuously true because by \AXM{3.2} empty set contains no elements. And again by \AXM{3.2}, the statement
\[
  \forall\ x : x \in \emptyset \implies x \in \emptyset'
\]
is also vacuous. These imply
\[
  \forall\ x : (x \in \emptyset' \implies x \in \emptyset) \land (x \in \emptyset \implies x \in \emptyset')
\]
that is,
\[
  \forall\ x : x \in \emptyset' \iff x \in \emptyset
\]
by \DEF{3.1.4}, \(\emptyset' = \emptyset\)
\end{proof}

\begin{note}
If a set is not equal to the empty set, we call it \emph{non-empty}.
\end{note}

\begin{lemma}[Single choice]\label{lem 3.1.6}
Let \(A\) be a \emph{non-empty} set. Then there exists an object \(x\) such that \(x \in A\).
\end{lemma}
\begin{proof}
Suppose for the sake of contradiction that \(A\) be a non-empty set and for all object \(x\), \(x \notin A\). And by \AXM{3.2}, \(x \notin \emptyset\). Then similarly as \AC{3.1.2}, we can derive \(A = \emptyset\), which contradicts that \(A\) is \emph{non-empty}.
\end{proof}

\begin{remark}\label{remark 3.1.7}
\LEM{3.1.6} asserts that given any non-empty set \(A\), we are allowed to \emph{``choose''} an element \(x\) of \(A\) which demonstrates this non-emptyness. Later on (in \LEM{3.5.12}) we will show that given any \textbf{finite} number of non-empty sets, say \(A_1, \dots, A_n\), it is possible to choose one element \(x_1, \dots, x_n\) from each set \(A_1, \dots, A_n\); this is known as ``finite choice''. However, in order to choose elements from an \textbf{infinite} number of sets, we need an additional axiom, the \emph{axiom of choice} (\AXM{8.1}).
\end{remark}

\begin{note}
\RMK{3.1.7} 在講\ ``finite choice'' 的部分,看起來是說要被選的集合是有限個,但是沒有規定個別集合的元素數量要有限個,目前還不確定這意味著什麼。
\end{note}

\begin{remark} \label{remark 3.1.8}
Note that the empty set is not the same thing as the natural number \(0\). One is a set; the other is a number. However, it is true that the \emph{cardinality} of the empty set is \(0\); see \SEC{3.6}.
\end{remark}

We now present further axioms to enrich the class of sets available.

\begin{axiom}[Singleton sets and pair sets]\label{axm 3.3}
If \(a\) is an object, then there exists a set \( \{a\} \) whose \emph{only} element is \(a\), i.e., for every object \(y\), we have \(y \in \{a\}\) if and only if \(y = a\); we refer to \( \{a\} \) as the \emph{singleton set} whose element is \(a\). Furthermore, if \(a\) and \(b\) are objects, then there exists a set \( \{a, b\} \) whose only elements are \(a\) and \(b\); i.e., for every object \(y\), we have \( y \in \{a, b\} \) if and only if \(y = a\) or \(y = b\); we refer to this set as the \emph{pair set} formed by \(a\) and \(b\).
\end{axiom}

\begin{note}
\href{https://www.wikiwand.com/en/Axiom_of_pairing#/Consequences}{參考}: 這個公理實際說的是,給定兩個集合(這邊暫時當作所有物件都是集合)\ \(x\) 和\ \(y\),我們可以找到一個集合\ \(A\) ,它的成員就是\ \(x\) 和\ \(y\)。
\end{note}

\begin{note}
前方高能注意: \RMK{3.1.9} 在解釋\ \AXM{3.3} 裡面的\ singleton、\ pair,還有\ \AXM{3.4} 這三者,若假設``其中一部分''是公理,則剩下的可以從那個部分直接推得,不用當作是公理,i.e. 剩下的只須為定理,不用是公理。這種\ redundant\ 在本節開頭有講過,便於推導,but does no real harm。BTW 這個\ remark 裡面有一堆\ (why?) 全部都要自己推導。
\end{note}

\begin{remark} \label{remark 3.1.9}

Just as there is only one empty set, there is only one singleton set for each object \(a\), thanks to \DEF{3.1.4} (why? \MAROON{(1)}).

Similarly, given any two objects \(a\) and \(b\), there is only one pair set formed by \(a\) and \(b\).

Also, \DEF{3.1.4} also ensures that \( \{a, b\} = \{b, a\} \) (why? \MAROON{(2)}) and \( \{a, a\} = \{a\} \) (why? \MAROON{(3)}). Thus the \textbf{singleton} set axiom is in fact redundant, being \textbf{a consequence of} the \textbf{pair} set axiom.

\emph{Conversely}, the \textbf{pair set} axiom will \textbf{follow from} the \textbf{singleton} set axiom \textbf{and} the \textbf{pairwise union axiom \AXM{3.4}} (see  \LEM{3.1.13}).

One may wonder why we don’t go further and create triplet axioms, quadruplet axioms, etc.; however there will be no need for this once we introduce the pairwise union axiom
below.
\end{remark}

\begin{proof}
\MAROON{(1)}: given any object \(a\), suppose there exist two sets \(A\) and \(A'\) which are singleton sets of \(a\). Then we have:
\begin{align*}
         & (\forall\ x : x \in A \iff x = a) \land (\forall\ x : x \in A' \iff x = a) & \text{by \AXM{3.3}} \\
    \iff & (\forall\ x : x \in A \iff x = a) \land (\forall\ x : x = a \iff x \in A') & \text{by logic} \\
    \iff & (\forall\ x : x \in A \iff x = a \iff x \in A')                            & \text{by logic} \\
    \iff & (\forall\ x : x \in A \iff x \in A')                                       & \text{by simplifying logic} \\
    \iff & A = A'                                                                     & \text{by \DEF{3.1.4}}
\end{align*}

\MAROON{(2)}: given any objects \(a, b\), then
\begin{align*}
    & (x \in \{a, b\} \iff (x = a \lor x = b)) & \text{by \AXM{3.3}} \\
    \iff & (x \in \{a, b\} \iff (x = a \lor x = b) \iff (x = b \lor x = a)) & \text{by logic} \\
    \iff & (x \in \{a, b\} \iff (x = b \lor x = a)) & \text{by simplifying logic} \\
    \iff & (x \in \{a, b\} \iff (x = b \lor x = a) \iff x \in \{b, a\}) & \text{by \AXM{3.3}} \\
    \iff & (x \in \{a, b\} \iff x \in \{b, a\}) & \text{by simplifying logic} \\
    \iff & \{a, b\} = \{b, a\} & \text{by \DEF{3.1.4}}
\end{align*}

\MAROON{(3)}: given any object \(a\), then
\begin{align*}
    & (x \in \{a, a\} \iff (x = a \lor x = a)) & \text{by \AXM{3.3}} \\
    \iff & (x \in \{a, a\} \iff (x = a)) & \text{by logic} \\
    \iff & (x \in \{a, a\} \iff (x = a) \iff x \in \{a\})& \text{by \AXM{3.3}} \\
    \iff & (x \in \{a, a\} \iff x \in \{a\})& \text{by simplifying logic} \\
    \iff & \{a, a\} = \{a\} & \text{by \DEF{3.1.4}}
\end{align*}
\end{proof}

\begin{example} \label{example 3.1.10}
Since \(\emptyset\) is a set (and hence an object), so is singleton set \(\{ \emptyset \}\), i.e., the set whose only element is \( \emptyset \), is a set (and it is not the same set as \( \emptyset \), \( \{ \emptyset \} \neq  \emptyset \) (why? See \EXEC{3.1.2}). Similarly, the singleton set \( \{ \{ \emptyset \} \} \) and the pair set \( \{ \emptyset, \{ \emptyset \} \} \) are also sets. These three sets are not equal to each other (\EXEC{3.1.2}).
\end{example}

\begin{note}
現在有的三個公理已經可以讓我們建構出一堆集合了,但是他們的元素都不超過兩個,雖然元素本身可能長得很複雜。
\end{note}

\begin{axiom} [Pairwise union] \label{axm 3.4}
Given any two sets \(A, B\), there exists a set \(A \cup B\), called the \emph{union} \(A \cup B\) of \(A\) and \(B\), whose elements consists of all the elements which belong to \(A\) or \(B\) or both. In other words, for any object \(x\),
\[
    x \in A \cup B \iff (x \in A \lor x \in B)
\].
\end{axiom}

\begin{note}
注意\ Union 的定義完全是由\ \( \in \) (還有\ or) 兜出來的,所以符合替換公理。
\end{note}

\begin{example}
很廢。\( \{1, 2\} \cup \{ 2, 3 \} = \{ 1, 2, 3 \} \)
\end{example}

\begin{remark} \label{remark 3.1.12}
If \(A, B\) are sets, \(A'\) is also a set which is equal to \(A\), then \(A \cup B\) is equal to \(A' \cup B\) (why? \MAROON{(1)} One needs to use \AXM{3.4} and \DEF{3.1.4}). Similarly if \(B'\) is a set which is equal to \(B\), then \(A \cup B\) is equal to \(A \cup B'\). Thus the operation of union \emph{obeys the axiom of substitution}, and is thus well-defined on sets.
\end{remark}

\begin{proof}
\MAROON{(1)}: Suppose \(A, B, A'\) are sets such that \(A = A'\). Then given any object \(x\),
\begin{align*}
         & x \in A \cup B \\
    \iff & x \in A \lor x \in B & \text{by \AXM{3.4}} \\
    \iff & x \in A' \lor x \in B & \text{since \(A = A'\) and \(=\) satisfies \AXM{a.7.4}} \\
    \iff & x \in A' \cup B & \text{by \AXM{3.4}} \\
\end{align*}
So \(\forall x, x \in A \cup B \iff x \in A' \cup B\), so by \DEF{3.1.4} \(A \cup B = A' \cup B\).

Now suppose \(B' = B\), Then given any object \(x\),
\begin{align*}
         & x \in A \cup B \\
    \iff & x \in A \lor x \in B & \text{by \AXM{3.4}} \\
    \iff & x \in A \lor x \in B' & \text{since \(B = B'\) and \(=\) satisfies \AXM{a.7.4}} \\
    \iff & x \in A \cup B' & \text{by \AXM{3.4}} \\
\end{align*}
So similarly \(A \cup B = A \cup B'\).
\end{proof}

\begin{lemma} \label{lem 3.1.13}
If \(a\) and \(b\) are objects, then \( \{a, b\} = \{ a \} \cup \{ b \} \). If \(A, B, C\) are sets, then the \emph{union operation is commutative} (i.e., \(A \cup B = B \cup A\)) and \emph{associative} (i.e., \((A \cup B) \cup C = A \cup (B \cup C)\)). Also, we have \(A \cup A = A \cup \emptyset = \emptyset \cup A = A\).
\end{lemma}

\begin{proof}
\( \{a, b\} = \{ a \} \cup \{ b \} \): for any object \(x\),
\begin{align*}
         & x \in \{a, b\} \\
    \iff & x = a \lor x = b & \text{by \AXM{3.3}, pair-part} \\
    \iff & x \in \{a\} \lor x \in \{b\} & \text{by \AXM{3.3}, singleton-part} \\
    \iff & x \in \{a\} \cup \{b\} & \text{by \AXM{3.4}} \\
\end{align*}
So \(\forall x, x \in \{a, b\} \iff x \in \{a\} \cup \{b\} \), so \(\{a, b\} = \{a\} \cup \{b\} \) by \DEF{3.1.4}.

\(A \cup B = B \cup A\): for any object \(x\),
\begin{align*}
         & x \in A \cup B \\
    \iff & x \in A \lor x \in B & \text{by \AXM{3.4}} \\
    \iff & x \in B \lor x \in A & \text{by logic} \\
    \iff & x \in B \cup A & \text{by \AXM{3.4}}
\end{align*}

\((A \cup B) \cup C = A \cup (B \cup C)\): for any object \(x\),
\begin{align*}
         & x \in (A \cup B) \cup C \\
         \iff & x \in (A \cup B) \lor x \in C & \text{by \AXM{3.4}} \\
         \iff & (x \in A \lor x \in B) \lor x \in C & \text{by \AXM{3.4}} \\
         \iff & x \in A \lor (x \in B \lor x \in C) & \text{by logic} \\
         \iff & x \in A \lor (x \in B \cup C) & \text{by \AXM{3.4}} \\
         \iff & x \in A \cup (x \in B \cup C) & \text{by \AXM{3.4}} \\
\end{align*}
So first statement if and only if last statement, so \((A \cup B) \cup C = A \cup (B \cup C)\).

\(A \cup A = A \cup \emptyset = \emptyset \cup A = A\): for any object \(x\),
\begin{align*}
         & x \in A \cup A \\
    \iff & x \in A \lor x \in A & \text{by logic} \\
    \iff & x \in A & \text{by simplifying logic} \\
    \iff & \MAROON{A \cup A = A} & \text{by \DEF{3.1.4}} \\
    \iff & x \in A \lor x \in \emptyset & \text{by logic, something \(\lor\) something false = something} \\
    \iff & x \in A \cup \emptyset & \text{by \AXM{3.4}} \\
    \iff & \MAROON{A \cup A = A \cup \emptyset} & \text{by \DEF{3.1.4}} \\
    \iff & \MAROON{A \cup A = \emptyset \cup A} & \text{already proved commutative law}
\end{align*}
\end{proof}

\begin{remark}
幹話。
\end{remark}

\begin{note}
We are not yet in a position to define sets consisting of \(n\) objects for any given natural numbers \(n\). 老實說我不是很理解為什麼,文中敘述是說我們還沒有定義「做\ \(n\) 次操作」是什麼意思"(require iterating the above construction “\(n\) times”, but the concept of \(n\)-fold iteration has not yet been rigorously defined). 類似任意有限個元素的情況,我們目前也無法給出有無限個元素的集合的定義。這需要其他的公理。現在我們先定義什麼是「子集合」。
\end{note}

\begin{definition}[Subsets] \label{def 3.1.15}
Let \(A, B\) be sets. We say that \(A\) is a subset of \(B\), denoted \(A \subseteq B\), if and only if every element of \(A\) is also an element of \(B\), i.e. For any object \(x\), \(x \in A \implies x \in B\). We say that \(A\) is a proper subset of \(B\), denoted \(A \subsetneq B\), if \(A \subseteq B\) and \(A \neq B\).
\end{definition}

\begin{remark} \label{remark 3.1.16}
Because these definition\textbf{s}(both \(\subseteq\) and \(\subsetneq\)) involve only the notions of (set) equality and the “is an element of” relation, both of which already obey the axiom of substitution \AXM{a.7.4}, the notion of subset also automatically obeys the axiom of substitution. Thus for instance if \(A \subseteq B\) and \(A = A'\), then \(A' \subseteq B\).
\end{remark}

\begin{example} \label{example 3.1.17}
前面幹話。Given any set \(A\), we always have \(A \subseteq A\) (why?) and \(\emptyset \subseteq A\) (why?).
\end{example}

\begin{proof}
Let \(x\) be arbitrarily chosen object. Then if \(x \in A\) then \(x \in A\) is trivially true. By \DEF{3.1.15}, \(A \subseteq A\).

Let \(x\) be arbitrarily chosen object. Then if \(x \in \emptyset\) then \(x \in A\) is vacuously true. By \DEF{3.1.15}, \(\emptyset \subseteq A\).
\end{proof}

\begin{note}
下面的\ Proposition 要參考\ \DEF{8.5.1},旨在說明\ "set-inclusion" 這個\ relation 是\ partially ordered。也就是要證明它是\ reflexive(by \EXAMPLE{3.1.17}), anti-symmetric, transitive。
\end{note}

\begin{proposition} [Sets are partially ordered by set inclusion] \label{prop 3.1.18}
Let \(A, B, C\) be sets. If \(A \subseteq B\) and \(B \subsetneq C\) then \(A \subseteq C\). If \(A \subseteq B\) and \(B \subseteq A\), then \(A = B\). Finally, if \(A \subsetneq B\) and \(B \subsetneq C\) then \(A \subsetneq C\).
\end{proposition}

\begin{proof}
Transitive: Let \(A, B, C\) be sets such that \(A \subseteq B\) \MAROON{(1)} and \(B \subseteq C\) \MAROON{(2)}. Suppose object \(x \in A\), wanted: \(x \in C\). Then by \MAROON{(1)}, \(x \in B\), which with \MAROON{(2)} implies \(x \in C\).

Anti-Symmetric: Let \(A, B\) be sets such that \(A \subseteq B \land B \subseteq A\). Wanted: \(A = B\). Then we have
\begin{align*}
     & A \subseteq B \land B \subseteq A \\
\iff & (\forall\ x : x \in A \implies x \in B) \land (\forall\ x : x \in B \implies x \in A) & \text{by \DEF{3.1.15}} \\
\iff & (\forall\ x : x \in A \iff x \in B) & \text{by trivially simplifying logic} \\
\iff & A = B. & \text{by \DEF{3.1.4}}
\end{align*}
\end{proof}

Transitive for proper-inclusion: Let \(A, B, C\) be sets such that \(A \subsetneq B\) \MAROON{(1)} and \(B \subsetneq C\) \MAROON{(2)}. Wanted: \(A \subsetneq C\), that is,
\[
    A \subseteq C \land A \neq C
\] The former is trivially true since \(A \subsetneq B\) by definition implies \(A \subseteq B\) and \(B \subsetneq C\) by definition implies \(B \subseteq C\), and by transitivity of set-inclusion \(A \subseteq B\) and \(B \subseteq C\) implies \(A \subseteq C\). For the latter, suppose for the sake of contradiction that \(A = C\). Then by \MAROON{(1)} and Axiom of Substitution \AXM{a.7.4}, we have \(C \subsetneq B\), but that implies \(C \subseteq B\) and we have known \(B \subseteq C\), together implies \(B = C\) by anti-symmetry of set-inclusion. But this contradicts \(B \subsetneq C\) because that implies \(B \neq C\).

\begin{note}
根據本書的脈絡,我們直到\ \PROP{3.1.18} 才能用\ \(A \subseteq B \land B \subseteq A\) 的方法來證明\ \(A = B\)。
\end{note}

\begin{remark} \label{remark 3.1.19}
There is a relationship between subsets and unions: see for instance \EXEC{3.1.7}. 這就頭腦體操。
\end{remark}
\section{Russel's Paradox}\label{sec 3.2}

\begin{axiom} [Universal specification] \label{axm 3.8} (\RED{Dangerous!})
Suppose for every object \(x\) we have a property \(P(x)\) pertaining to \(x\) (so that for every \(x\), \(P(x)\) is either a true statement or a false statement). Then there exists a set \( \{x : P(x) \text{\ is true} \} \) such that for every object \(y\),
\[
    y \in \{x : P(x) \text{\ is true} \} \iff P(y) \text{\ is true}.
\]
This axiom is also called \emph{axiom of comprehension}. It asserts that \emph{every property corresponds to a set}. This axiom also implies most of the axioms in the previous section (\EXEC{3.2.1}, mind=blown).
\end{axiom}

\begin{note}
這條「公理」跟\ \AXM{3.5} \AXM{3.6} 的差別就是後兩個都是做用在一個\textbf{已知是集合}的東西上的。
\end{note}

\begin{note}
接下來的東西都是在聊羅素悖論,去查各種影片可能會比較有趣。
\end{note}

\begin{note}
We shall simply postulate an axiom which ensures that absurdities such as Russell’s paradox do not occur.
\end{note}

\begin{axiom} \label{axm 3.9}
If \(A\) is a non-empty set, then \textbf{there is} at least one element \(x\) of \(A\) which is either \textit{not a set}, \textit{or} is \textit{disjoint} from \(A\).
\end{axiom}

\begin{note}
BTW, the description of \AXM{3.9} uses the term \emph{disjoint}, which depends on the definition of intersection(\DEF{3.1.23}), which depends on the definition of \AXM{3.5}, so I assume \AXM{3.5} is also asserted when \AXM{3.9} is asserted.
\end{note}

\begin{note}
這個公理可能可以參考一下其他地方(e.g. \href{https://www.wikiwand.com/en/Axiom_of_regularity}{wiki})怎麼描述的,因為這本書是假設有些\ object 不是\ set,但是\ ZFC(或者\ pure set theory?) 實際上所有東西都是\ set。不過看起來其實就是把\ ``\(x\) is either not a set'' 拔掉而已。
\end{note}

\begin{note}
One particular consequence of this axiom is that \textbf{sets are no longer allowed to contain themselves}. (\EXEC{3.2.2}) So if something contains itself, then it is not a set.
\end{note}

\begin{note}
This axiom(\AXM{3.9}) is never needed for the purposes of doing analysis.
\end{note}

\exercisesection

\begin{exercise} \label{exercise 3.2.1}
Show that the universal specification axiom, \AXM{3.8}, if assumed to be true, would imply \AXM{3.2} (empty set), \AXM{3.3} (singleton and pair), \AXM{3.4} (union), \AXM{3.5} (specification), and \AXM{3.6} (replacement). (If we assume that all natural numbers are objects, we also obtain \AXM{3.7}.) Thus, this axiom, if permitted, would simplify the foundations of set theory tremendously (and can be viewed as one basis for an intuitive model of set theory known as ``\emph{naive set theory}''). Unfortunately, as we have seen, \AXM{3.8} is ``too good to be true''!
\end{exercise}

\begin{proof}
First the exercise does not say \AXM{3.8} implies \AXM{3.1}, so it seems that \AXM{3.1} is also needed.

For \AXM{3.2}, we just let \(P(x) := \text{false}\) for any object \(x\). Then by \AXM{3.8} there exists a set \( \emptyset := \{ x : P(x) \} \). Then we must have \(\forall x : x \notin \emptyset\), otherwise if \(x \in \emptyset\) then by definition of \(\emptyset\), \(P(x)\) is true, which contradicts that \(P(x)\) is false.

For \AXM{3.3}, given particular but arbitrary objects \(a, b\), we just let \(P(x) := x = a\) and \(Q(x) := x = a \lor x b\). Then by \AXM{3.8} there exist sets  \(A := \{ x : P(x) \} \) and \(B := \{ x : Q(x) \} \), i.e. \(A := \{ x : x = a \} \) and \(B := \{ x : x = a \lor x \ b \} \). Hence \AXM{3.3} (existence of singleton and pair set) is satisfied.

For \AXM{3.4} Let \(A, B\) be particular but arbitrary sets. And Let \(P(x) := x \in A \lor x \in B\). Then by \AXM{3.8}, there exists a set \(C := \{x : P(x)\} \), that is, \(C := \{x : x \in A \lor x \in B\}\). Thus the union of \(A, B\) exists. Hence \AXM{3.4} is satisfied.

For \AXM{3.5}. Let \(A\) be a particular but arbitrary set and \(P(x)\) be a particular but arbitrary statement that is either true or false for any object \(x \in A\). Then let \(Q(x) := x \in A \land P(x) \text{ is true}\). By \AXM{3.8}, there exists a set \(B := \{ x : Q(x) \} \), that is, \(B := \{x : x \in A \land P(x) \text{ is true}\}\). Thus the specification set \(B\) of \(A\) with statement \(P\) exists. Hence \AXM{3.5} is satisfied.

For \AXM{3.6}. Let \(A\) be a particular but arbitrary set and \(P(x, y)\) satisfy the hypothesis in \AXM{3.6}. By \AXM{3.8}, there exists a set \(B := \{y : P(x, y) \text{ is true}\} \land x \in A\), that is, \(B := \{y : P(x, y) \text{ is true for some \(x \in A\)} \} \). So B is the replacement set of \(A\). Hence \AXM{3.6} is satisfied.

Finally, for \AXM{3.7}, suppose all natural numbers are objects. Then let \(P(x)\) := ``\(x\) is a natural number'' :). Then there exists a set \( \SET{N} := \{ x : P(x) \} \).

\end{proof}

\begin{note}
The proof of implication of \AXM{3.7} is not rigorous(or escape the detail of the statement \(P\), escape what need to be satisfied to be a natural number). We just need to know they are object and hence can be elements of a set.
\end{note}

\begin{exercise} \label{exercise 3.2.2}
Use the axiom of regularity, \AXM{3.9} (and the singleton set axiom, \AXM{3.3}) to show that if \(A\) is a set, then \(A \notin A\). Furthermore, show that if \(A\) and \(B\) are two sets, then either \(A \notin B\) or \(B \notin A\) (or both).
\end{exercise}

\begin{proof}
Again, we use \AXM{3.1}. So if \(A\) is a set, then \(A\) is an object, and by \AXM{3.3}, \(\{A\}\) is a singleton set.

Suppose for the sake of contradiction that there exists an object \(A\) such that \(A \in A\) \MAROON{(1)}. Then because \(A \in \{A\} \)  \MAROON{(2)}, by \MAROON{(1) (2)} we know \(\{A\} \cap A = \{A\} \neq \emptyset\), i.e. not disjoint. But since the singleton set \( \{A\} \) has only one object \(A\), it implies there is no object \(x\) of \(\{A\}\) such that \(x\) and \( \{A\}\) are disjoint, which contradicts \AXM{3.9}. So the supposition is false, so for any object \(A\), \(A \notin A\).

Also, suppose for the sake of contradiction that there exist sets \(A, B\), such that
\begin{center}
    \(A \in B\) \MAROON{(1)} and \(B \in A\) \MAROON{(2)}.
\end{center}
Now consider the set \(\{A, B\}\) (which exists by pair set of \AXM{3.3}), it has two ``elements'' \BLUE{\(A\)} and \GREEN{\(B\)}. For element \BLUE{\(A\)}, since \(B \in \BLUE{A}\) by \MAROON{(2)} and \(B \in \{A, B\}\), \(B \in \BLUE{A} \cap \{A, B\}\), so \BLUE{\(A\)} and \(\{A, B\}\) are not disjoint. For element \GREEN{\(B\)}, since \(A \in \GREEN{B}\) by \MAROON{(1)} and \(A \in \{A, B\}\), \(A \in \GREEN{B} \cap \{A, B\}\), so \GREEN{\(B\)} and \(\{A, B\}\) are not disjoint. So for each element \(x \in \{A, B\}\), \(x\) and \(\{A, B\}\) is not disjoint, which contradicts with \AXM{3.9}.
\end{proof}

\begin{exercise} \label{exercise 3.2.3}
Show (assuming the other axioms of set theory) that the universal specification axiom, \AXM{3.8}, is equivalent to an axiom postulating the existence of a ``universal set'' \(\Omega\) consisting of all objects (i.e., for all objects \(x\), we have \(x \in \Omega\)). In other words, if \AXM{3.8} is true, then a universal set exists, and conversely, if a universal set exists, then \AXM{3.8} is true. (This may explain why \AXM{3.8} is called the axiom of universal specification.) Note that if a universal set \(\Omega\) existed, then we would have \(\Omega \in \Omega\) by \AXM{3.1} (\(\Omega\) is a set by \AXM{3.8} or universal specification, therefore by \AXM{3.1} it is also an object, and any object is \(\in \Omega\), so \(\Omega \in \Omega\)), contradicting \EXEC{3.2.2}. Thus the axiom of foundation specifically rules out the axiom of universal specification.
\end{exercise}

\begin{proof}
Suppose \AXM{3.8} is true. Then let \(P(x) := true\) for any object \(x\). Then by \AXM{3.8}, there exists at set \(\Omega := \{x : P(x) \text{ is true} \}\), and since \(P(x)\) is true for any object \(x\), \(x \in \Omega\).

Suppose the universal set \(\Omega\) exists. Then given an arbitrary property \(P(x)\) satisfiying the hypothesis of \AXM{3.8}, and by \AXM{3.5} (specification), since \(\Omega\) is a set, there exists the set \(A := \{ x \in \Omega : P(x) \text{ is true} \}\). Now we have to show that for every object \(y\),
\[
    y \in \{x \in \Omega : P(x) \text{ is true} \} \iff P(y) \text{ is true \BLUE{(1)}}
\]
to show that \(y \in \{x \in \Omega : P(x) \text{ is true} \}\) is in fact \(y \in \{x : P(x) \text{ is true} \}\) whose existence is guaranteed by \AXM{3.8}.
So let \(y\) be an object.
\begin{itemize}
    \item Suppose \(y \in \{x \in \Omega : P(x) \text{ is true} \). Then trivally \(P(y)\) is true.
    \item Suppose \(P(y)\) is true \MAROON{(1)}. Then since \(y\) is an object, \(y \in \Omega\) \MAROON{(2)}. By \MAROON{(1) (2)}, \(y \in \{x \in \Omega : P(x) \text{ is true} \} \).
\end{itemize}
So \BLUE{(1)} is proved.
\end{proof}


\chapter{Integers and rationals} \label{ch 4}

\newcommand{\M}{\text{---}}
\newcommand{\D}{\text{//}}

\section{The integers} \label{sec 4.1}

\begin{definition} [Integers] \label{def 4.1.1}
An integer is an expression of the form \(a \M b\), where \(a\) and \(b\) are natural numbers.
Two integers are \emph{considered}(defined) to be equal, \(a \M b = c \M d\), if and only if \(a + d = c + b\).
We let \(\SET{Z}\) denote the set of all integers.
\end{definition}

\begin{note}
\DEF{4.1.1} 有一段註解在課本該頁下方,在講述\ \(a \M b\) 這種\ expression 跟集合論還有\ equivalence relation 的關係,可以讀一下。
之後建構有理數跟實數時也會有對應的\ equivalence relation 出現。
\end{note}

Thus for instance \(3 \M 5\) is an integer, and is equal to \(2 \M 4\), because \(3 + 4 = 2 + 5\).

This notation is strange looking, and has a few deficiencies; for instance, \(3\) is \emph{not yet an integer}, because it is not of the form \(a \M b\)! We will rectify these problems later(on page 77 of the textbook).

We have to check that the equality in \DEF{4.1.1} is a legitimate notion of equality.
We need to verify the reflexivity, symmetry, transitivity, and substitution axioms \AXM{a.7.4}.
Now we first prove the equivalence relation.

\begin{additional corollary} \label{ac 4.1.1}
The definition of equality on the integers is reflexive, symmetric and transitive.
\end{additional corollary}

\begin{proof}
Reflexive: Let \(a \M b\) be an integer where \(a, b\) are natural numbers.
Then \(\BLUE{a \M b} = \GREEN{a \M b}\) since \(\BLUE{a} + \GREEN{b} = \GREEN{a} + \BLUE{b}\).

Symmetric: Suppose \(a \M b = c \M d\) where \(a, b, c, d\) are natural numbers, we have to show \(c \M d = a \M b\).
From the supposition \(a \M b = c \M d\) and \DEF{4.1.1}, \(a + d = b + c\), and by the equality and addition of natural numbers, we can get \(b + c = a + d\), and \(c + b = a + d\).
By \DEF{4.1.1}, \(c + b = a + d \iff c \M d = a \M b\), so \(c \M d = a \M b\), as desired.

Transitive: Suppose \(a \M b = c \M d\) \MAROON{(1)} and \(c \M d = e \M f\) \MAROON{(2)} where \(a, b, c, d, e, f\) are natural numbers,
we have to show \(a \M b = e \M f\), or equivalently by \DEF{4.1.1}, we have to show \(a + f = e + b\).
By \DEF{4.1.1}, from \MAROON{(1)} we can get \(a + d = c + b\) \MAROON{(3)}, from \MAROON{(2)} we can get \(c + f = e + d\) \MAROON{(4)}.
Now add both sides of \MAROON{(3) (4)} we can get \((a + d) + (c + f) = (c + b) + (e + d)\), from which we can get \((a + f) + (c + d) = (e + b) + (c + d)\).
By cancellation law(\PROP{2.2.6}), we have \(a + f = e + b\), as desired.
\end{proof}

\begin{note}
The \AXM{a.7.4} will be showed for each basic operation for integers, such as addition, multiplication, and order.
More advanced operations will be defined in terms of these basic operations, so automatically satisfy \AXM{a.7.4}
\end{note}

\begin{definition} [Addition and multiplication] \label{def 4.1.2}
The \emph{sum} of two integers, \((a \M b) + (c \M d)\), is defined by the formula
\[
    (a \M b) + (c \M d) := (a + c) \M (b + d).
\]
The \emph{product} of two integers, \((a \M b) \X (c \M d)\), is defined by
\[
    (a \M b) \X (c \M d) := (ac + bd) \M (ad + bc).
\]
\end{definition}

\begin{note}
Both integer addition and multiplication are defined by that meaningless placeholder \(\M\) and operations of natural numbers, so, the definition is not circular.
\end{note}

Thus for instance, \((3 \M 5) + (1 \M 4) = (3 + 1) \M (5 + 4) =  (4 \M 9)\).

Now we show the addition and multiplication of integers satisfy the \AXM{a.7.4}.

\begin{lemma} [Addition and multiplication are well-defined] \label{lem 4.1.3}
Let \(a, b, a', b', c, d\) be natural numbers.
If \((a \M b) = (a' \M b')\), then \((a \M b) + (c \M d) = (a' \M b') + (c \M d)\) and \((a \M b) \X (c \M d) = (a' \M b') \X (c \M d)\),
and also \((c \M d) + (a \M b) = (c \M d) + (a' \M b')\) and \((c \M d) \X (a \M b) = (c \M d) \X (a' \M b')\).
Thus addition and multiplication are well-defined operations (equal inputs give equal outputs).
\end{lemma}

\begin{note}
看起來是要先證明\ operator \textbf{兩邊}都符合替換公理,才可以推導交換律結合律那些性質(列在\ \PROP{4.1.6})。
第二章自然數的時候沒有這樣推導是因為"所有"的自然數的\ operator 的性質基本上都是從數學歸納法開始推的。
\end{note}

\begin{proof}
Suppose \((a \M b) = (a' \M b')\).

\MAROON{(1)}: \((a \M b) + (c \M d) = (a' \M b') + (c \M d)\)
\begin{align*}
         & (a \M b) + (c \M d) = (a' \M b') + (c \M d) \\
    \iff & (a + c) \M (b + d) = (a' + c) \M (b' + d) & \text{by \DEF{4.1.2}} \\
    \iff & (a + c) + (b' + d) = (a' + c) + (b + d) & \text{by \DEF{4.1.1}} \\
    \iff & (a + b') + (c + d) = (a' + b) + (c + d) \GREEN{\ (1)} & \text{by properties of natural number} \\
    \iff & a + b' = a' + b & \text{\(\Rightarrow\) by \PROP{2.2.6}, \(\Leftarrow\) by \AXM{a.7.4}} \\
    \iff & a \M b = a' \M b' & \text{by \DEF{4.1.1}}
\end{align*}
which is true by supposition, so \((a \M b) + (c \M d) = (a' \M b') + (c \M d)\), as desired.
    
\MAROON{(2)}: \((a \M b) \X (c \M d) = (a' \M b') \X (c \M d)\)
\begin{align*}
         & (a \M b) \X (c \M d) = (a' \M b') \X (c \M d) \\
    \iff & (ac + bd) \M (ad + bc) = (a'c + b'd) \M (a'd + b'c) & \text{by \DEF{4.1.2}} \\
    \iff & (ac + bd) + (a'd + b'c) = (a'c + b'd) + (ad + bc) & \text{by \DEF{4.1.1}} \\
    \iff & ac + b'c + a'd + bd = a'c + bc + ad + b'd & \text{by properties of natural numbers} \\
    \iff & c(a + b') + d(a' + b) = c(a' + b) + d(a + b') \GREEN{\ (2)} & \text{by \PROP{2.3.4}, distributive}
\end{align*}
which is true since \(a + b' = a' + b\), or equivalently by \DEF{4.1.1}, \(a \M b = a' \M b'\), so \((a \M b) \X (c \M d) = (a' \M b') \X (c \M d)\), as desired.

\MAROON{(3)}: \((c \M d) + (a \M b) = (c \M d) + (a' \M b')\)
\begin{align*}
         & (c \M d) + (a \M b) = (c \M d) + (a' \M b') \\
    \iff & (c + a) \M (d + b) = (c + a') \M (d + b') & \text{by \DEF{4.1.2}} \\
    \iff & (c + a) + (d + b') = (c + a') + (d + b) & \text{by \DEF{4.1.1}} \\
    \iff & (a + b') + (c + d) = (a' + b) + (c + d) & \text{by properties of natural number}
\end{align*}
which is already shown by \GREEN{(1)}, so \((c \M d) + (a \M b) = (c \M d) + (a' \M b')\), as desired.

\MAROON{(4)}: \((c \M d) \X (a \M b) = (c \M d) \X (a' \M b')\)
\begin{align*}
         & (c \M d) \X (a \M b) = (c \M d) \X (a' \M b')\\
    \iff & (ca + db) \M (cb + da) = (ca' + db') \M (cb' + da') & \text{by \DEF{4.1.2}} \\
    \iff & (ca + db) + (cb' + da') = (ca' + db') + (cb + da) & \text{by \DEF{4.1.1}} \\
    \iff & ca + cb' + da' + db = ca' + cb + da + db' & \text{by properties of natural numbers} \\
    \iff & c(a + b') + d(a' + b) = c(a' + b) + d(a + b') & \text{by \PROP{2.3.4}, distributive}
\end{align*}
which is already shown by \GREEN{(2)}, so \((c \M d) \X (a \M b) = (c \M d) \X (a' \M b')\), as desired.
\end{proof}

\begin{note}
The \emph{integer} \(n \M 0\) \textbf{behave} in the same way as the \emph{natural number} \(n\);
indeed one can check that \((n \M 0) + (m \M 0) = (n + m) \M 0\), and \( (n \M 0) \X (m \M 0) = nm \M0\).
\end{note}

\begin{note}
Furthermore, \((n \M 0)\) is equal to \((m \M 0)\) (if and only if \(n + 0 = m + 0\) by \DEF{4.1.1},) if and only if \(n = m\).
(The mathematical term for this is that there is an \emph{isomorphism} between the \emph{natural} numbers \(n\) and those \emph{integers of the form} \(n \M 0\).
Thus we may \emph{identify} the \emph{natural} numbers with \emph{integers} by \textbf{setting \(n \equiv n \M 0\)}; this does not affect our definitions of addition or multiplication or equality since they are consistent with each other(i.e. addition or multiplication \textbf{behave} in the same way).
And the natural number \(n\) will also be equal to any other integer which is equal to \(n \M 0\),
for instance (natural number) \(3\) is equal not only to (integer) \(3 \M 0\), but also to (integer) \(4 \M 1\), (integer) \(5 \M 2\), etc.
\end{note}

\begin{note}
We can now define incrementation on the \emph{integers} by defining \(x\INC := x+1\) for any \emph{integer} \(x\);
this is of course consistent with our definition of the increment operation for \emph{natural} numbers.
However, this is no longer an important operation for us, as it has been now superseded by the more general notion of addition.
\end{note}

\begin{definition} [Negation of integers] \label{def 4.1.4}
If \((a \M b)\) is an integer, we \emph{define} the \emph{negation} \( -(a \M b) \) to be the integer \((b \M a)\).
In particular if \(n = n \M 0\) is a \emph{positive natural} number, we can define its negation \(-n = 0 \M n\).
\end{definition}

For instance \(-(3 \M 5) = (5 \M 3)\).
One can check this definition is well-defined (\EXEC{4.1.2}).

\begin{additional corollary} \label{ac 4.1.2}
The definition of negation on the integers is well-defined.
\end{additional corollary}

\begin{proof}
Let \(a, b, a', b'\) be arbitrary natural numbers such that \(a \M b = a' \M b'\), we have to show \(-(a \M b) = -(a' \M b')\).
Then
\begin{align*}
         & -(a \M b) = -(a' \M b') \\
    \iff & b \M a = b' \M a' & \text{by \DEF{4.1.4}} \\
    \iff & b + a' = b' + a & \text{by \DEF{4.1.1}} \\
    \iff & a' + b = b' + a & \text{by properties of natural numbers} \\
    \iff & a' \M b' = a \M b & \text{by \DEF{4.1.1}} \\
\end{align*}
which is true by supposition, so \(-(a \M b) = -(a' \M b')\), as desired.
Therefore, \DEF{4.1.4} is well-defined.
\end{proof}

\begin{lemma} [Trichotomy of integers] \label{lem 4.1.5}
Let \(x\) be an integer.
Then \emph{exactly one} of the following three statements is true:
(a) \(x\) is zero;
(b) \(x\) is equal to a \emph{positive} natural number \(n\);
or (c) \(x\) is the negation \(-n\) of a \emph{positive} natural number \(n\).
\end{lemma}

\begin{proof}
We first show that at least one of (a), (b), (c) is true.
By definition, \(x = a \M b\) for some natural numbers \(a\), \(b\).
By trichotomy of natural numbers(\PROP{2.2.13}), We have three cases: \(a > b\), \(a = b\), or \(a < b\).

If \(a > b\) then \(a = b + c\) for some positive natural number \(c\)(by \PROP{2.2.12}(f)),
which is trivially equivalent to \(a + 0 = b + c\),
and by \DEF{4.1.1}, is equivalent to \(a \M b = c \M 0\),
which by the isomorphism of \(c\) and \(c \M 0\), is equal to natural number \(c\), which is positive,
so \(x = a \M b\) is equal to a positive natural number \MAROON{(1)}, satisfying (b).

If \(a = b\), then \(a \M b = a \M a\), which is by \DEF{4.1.1} trivially equal to \(0 - 0\),
which by the isomorphism of \(0\) and \(0 \M 0\) is equal to the natural number \(0\).
So \(a \M b\) is equal to the natural number \(0\), satisfying (a).

If \(a < b\) then \(b > a\), and by reasoning in \MAROON{(1)}, \(b \M a = n\) for some positive natural number \(n\),
and thus by \DEF{4.1.4}, \(a \M b = -(b \M a)\), which \(= -n\), satisfying (c).

Now we show that no more than one of (a), (b), (c) can hold at a
time.
That is, (a), (b) cannot be simultaneously true, and (a), (c) cannot be simultaneously true, and (b), (c) cannot be simultaneously true.

By \DEF{2.2.7} (a), (b) cannot be simultaneously true, otherwise \(x\) is equal to the natural number \(0\) and a positive natural number, contradicting \DEF{2.2.7}.

Suppose (a), (c) are both true, then \(x = 0\) by (a) and also \(x = -n\) for some positive natural number \(n\) \MAROON{(2)},
So \(0 = -n\).
By the isomorphism of \(c\) and \(c - 0\) for any natural number \(c\), that can be written as \(0 - 0 = -n\), \(0 - 0 = -(n - 0)\), \(0 - 0 = 0 - n\),
from which by \DEF{4.1.1} we can get \(0 + n = 0 + 0\), that is, \(n = 0\), that is, \(n\) is not positive \MAROON{(3)}, contradicting \MAROON{(2)},
so (a), (c) are not both true.

Suppose (b), (c) are both true, then by (b) \(x\) is equal to a positive natural number \(m\), by (c) \(x = -n\) for some positive natural number \(n\), so \(m = -n\).
By the isomorphism of \(c\) and \(c - 0\) for any natural number \(c\), that can be written as \(m - 0 = -n\), \(m - 0 = -(n - 0)\), \(m - 0 = 0 - n\),
from which by \DEF{4.1.1} we can get \(m + n = 0 + 0 = 0\). But \(m, n\) are both positive integer, so by \PROP{2.2.8} \(m + n\) is positive, a contradiction.
So (b), (c) are not true.
\end{proof}

\begin{note}
If \(n\) is a \emph{positive} \emph{natural} number, we call \(n\) a \emph{positive} integer and \(-n\) a \emph{negative} integer.
Thus every integer is positive, zero, or negative, but not more than one of these at a time.
\end{note}

\begin{note}
One could well ask why we don’t use \LEM{4.1.5} to \emph{define} the integers;
i.e., why didn’t we just say an integer is anything which is either a positive natural number, zero, or the negative of a natural number.
The reason is that if we did so, the rules for adding and multiplying integers would split into many different cases
(e.g., negative times positive equals negative; negative plus positive is either negative, positive, or zero, depending on which term is larger, etc.)
and to verify all the properties would end up being much messier.
\end{note}

We now summarize the algebraic properties of the integers.

\begin{proposition} [Laws of algebra for integers] \label{prop 4.1.6}
Let \(x, y, z\) be integers.
Then we have
\begin{align*}
          x + y & = y + x \MAROON{\ (1)} \\
    (x + y) + z & = x + (y + z) \MAROON{\ (2)} \\
          x + 0 & = 0 + x = x \MAROON{\ (3)} \\
       x + (-x) & = (-x) + x = 0 \MAROON{\ (4)} \\
             xy & = yx \MAROON{\ (5)} \\
          (xy)z & = x(yz) \MAROON{\ (6)} \\
        x1 = 1x & = x \MAROON{\ (7)} \\
       x(y + z) & = xy + xz \MAROON{\ (8)} \\
       (y + z)x & = yx + zx \MAROON{\ (9)}
\end{align*}
\end{proposition}

\begin{remark} \label{remark 4.1.7}
The above set of nine identities have a name; they are asserting that the integers form a \emph{commutative ring}.
(If one deleted the identity \(xy = yx\), then they would only assert that the integers form a \emph{ring}).
Note that some of these identities were already proven for the natural numbers, \emph{but this does not automatically mean that they also hold for the integers} because the integers are a larger set than the natural numbers.
On the other hand, this proposition supersedes many of the propositions derived earlier for natural numbers.
\end{remark}

\begin{proof}
The book says you can use \LEM{4.1.5} to prove these identities, but that will split into many many cases depending on whether \(x, y, z\) are zero, positive, or negative.
This becomes very messy.
A shorter way is to write \(x =(a \M b), y = (c \M d)\), and \(z = (e \M f)\) for some natural numbers \(a, b, c, d, e, f\), and expand these identities in terms of \(a, b, c, d, e, f\) and \emph{use the algebra of the natural numbers}.

So let \(x =(a \M b), y = (c \M d)\), and \(z = (e \M f)\) for some natural numbers \(a, b, c, d, e, f\).

\MAROON{(1)} \(x + y = y + x\):
\begin{align*}
    x + y & = (a \M b) + (c \M d) \\
          & = (a + c) \M (b + d) & \text{by \DEF{4.1.2}} \\
          & = (c + a) \M (d + b) & \text{by \PROP{2.2.4}} \\
          & = (c \M d) + (a \M b) & \text{by \DEF{4.1.2}} \\
          & = y + x
\end{align*}

\MAROON{(2)} \((x + y) + z = x + (y + z)\):
\begin{align*}
    (x + y) + z & = ((a \M b) + (c \M d)) + (e \M f) \\
                & = ((a + c) \M (b + d)) + (e \M f) & \text{by \DEF{4.1.2}} \\
                & = ((a + c) + e) \M ((b + d) + f) & \text{by \DEF{4.1.2}} \\
                & = (a + (c + e)) \M (b + (d + f)) & \text{by \PROP{2.2.5}} \\
                & = (a \M b) + ((c + e) \M (d + f)) & \text{by \DEF{4.1.2}} \\
                & = (a \M b) + ((c \M d) + (e \M f)) & \text{by \DEF{4.1.2}} \\
                & = x + (y + z)
\end{align*}

\MAROON{(3)} \(x + 0 = 0 + x = x\):
By \MAROON{(1)} we have \(x + 0 = 0 + x\), so we only have to show \(x + 0 = x\).
Then
\begin{align*}
    x + 0 & = (a \M b) + 0 \\
          & = (a \M b) + (0 - 0) & \text{since (natural) \(0 \equiv (0 \M 0)\) (integer)} \\
          & = (a + 0) \M (b + 0) & \text{by \DEF{4.1.2}} \\
          & = a \M b & \text{by \LEM{2.2.2}} \\
          & = x
\end{align*}

\MAROON{(4)} \(x + (-x) = (-x) + x = 0\):
By \MAROON{(1)} we have \(x + (-x) = (-x) + x\), so we only have to show \(x + (-x) = 0\).
Then
\begin{align*}
    x + (-x) & = (a \M b) + -(a \M b) \\
             & = (a \M b) + (b \M a) & \text{by \DEF{4.1.4}} \\
             & = (a + b) \M (b + a) & \text{by \DEF{4.1.2}} \\
             & = (a + b) \M (a + b) & \text{by \PROP{2.2.4}, commutative} \\
             & = (a \M a) + (b \M b) & \text{by \DEF{4.1.2}} \\
             & = (0 \M 0) + (0 \M 0) & \text{by \DEF{4.1.1}} \\
             & = 0 + 0 & \text{since (natural) \(0 \equiv (0 \M 0)\) (integer)} \\
             & = 0 & \text{by \DEF{2.2.1}}
\end{align*}

\MAROON{(5)} \(xy = yx\):
\begin{align*}
    xy & = (a \M b)(c \M d) \\
       & = (ac + bd) \M (ad + bc) \text{\ \GREEN{(1)}} & \text{by \DEF{4.1.2}}
\end{align*}
And 
\begin{align*}
    yx & = (c \M d)(a \M b) \\
       & = (ca + db) \M (cb + da) & \text{by \DEF{4.1.2}} \\
       & = (ac + bd) \M (ad + bc) & \text{by \PROP{2.2.4}, \PROP{2.2.5}} \\
       & = \text{\GREEN{(1)}}
\end{align*}
So \(xy = yx\).

\MAROON{(6)} \((xy)z = x(yz)\):
\begin{align*}
    (xy)z & = ((a \M b)(c \M d))(e \M f) \\
          & = ((ac + bd) \M (ad + bc))(e \M f) & \text{by \DEF{4.1.2}} \\
          & = ((ace + bde + adf + bcf) \M (acf + bdf + ade + bce)) \text{\ \GREEN{(2)}} & \text{by \DEF{4.1.2}}
\end{align*}
And
\begin{align*}
    x(yz) & = (a \M b)((c \M d)(e \M f)) \\
          & = (a \M b)((ce + df) \M (cf + de)) & \text{by \DEF{4.1.2}} \\
          & = ((ace + adf + bcf + bde) \M (acf + ade + bce + bdf)) & \text{by \DEF{4.1.2}} \\
          & = ((ace + bde + adf + bcf) \M (acf + bdf + ade + bce)) \\
          & = \text{\GREEN{(2)}}
\end{align*}
so \((xy)z = x(yz)\)

\MAROON{(7)} \(x1 = 1x = x\):
Again by \MAROON{(1)}, we only have to show \(x1 = x\).
\begin{align*}
    x1 & = (a \M b) \X 1 \\
       & = (a \M b) \X (1 - 0) & \text{since (natural) \(1 \equiv 1 - 0\) (integer)} \\
       & = (a1 + b0) \M (a0 + b1) & \text{by \DEF{4.1.2}} \\
       & = a \M b & \text{by properties of natural numbers} \\
       & = x
\end{align*}

\MAROON{(8)} \(x(y + z) = xy + xz\):
\begin{align*}
    x(y + z) & = (a \M b)((c \M d) + (e \M f)) \\
             & = (a \M b)((c + e) \M (d + f)) & \text{by \DEF{4.1.2}} \\
             & = (a(c + e) + b(d + f)) \M (b(c + e) + a(d + f)) & \text{by \DEF{4.1.2}} \\
             & = (ac + ae + bd + bf) \M (bc + be + ad + af)) \text{ \GREEN{(4)}} & \text{by \PROP{2.3.4}}
\end{align*}
And
\begin{align*}
    xy + xz & = (a \M b)(c \M d) + (a \M b)(e \M f) \\
            & = ((ac + bd) \M (ad + bc)) + ((ae + bf) \M (af + be)) & \text{by \DEF{4.1.2}} \\
            & = (ac + bd + ae + bf) \M (ad + bc + af + be) & \text{by \DEF{4.1.2}} \\
            & = \GREEN{(4)}
\end{align*}
So \(x(y + z) = xy + xz\).

\MAROON{(9)} \((y + z)x = yx + zx\):
\begin{align*}
    (y + z)x & = x(y + z) & \text{by \MAROON{(1)}} \\
             & = xy + xz & \text{by \MAROON{(8)}} \\
             & = yx + zx & \text{by \MAROON{(1)}}
\end{align*}
\end{proof}

\begin{note}
仔細觀察\ \PROP{4.1.6},這些都不需要整數減法的定義(只需要那個\  meaningless placeholder 的定義)。
\end{note}

\begin{note}
Below refer to \href{https://github.com/ProFatXuanAll/terence-tao-analysis/blob/33a8568dce2474f82e381dcb1309f9ce685ce2f9/Analysis-I/4-1-the-integers.tex#L320-L358}{ProFatXuanAll's repo}
\end{note}

\begin{additional corollary} \label{ac 4.1.3}
Let \(x\) be an integer.
Then \(-x = (-1)x\).
\end{additional corollary}

\begin{proof}
Let \(x = (a \M b)\) where \(a, b\) are natural numbers.
Then
\begin{align*}
    -x & = -(a \M b) \\
       & = (b \M a) & \text{by \DEF{4.1.4}} \\
       & = (1b \M 1a) & \text{by \AC{2.3.4}} \\
       & = (0 + 1b) \M (0 + 1a) & \text{by \DEF{2.2.1}} \\
       & = (0a + 1b) \M (0b + 1a) & \text{by \LEM{2.3.3}} \\
       & = (0 \M 1)(a \M b) & \text{by \DEF{4.1.2}} \\
       & = (-(1 \M 0))(a \M b) & \text{by \DEF{4.1.4}} \\
       & = (-1)(a \M b) & \text{since (natural number) \(-1 \equiv -1 \M 0\) (integer)} \\
       & = (-1)x.
\end{align*}
\end{proof}

\begin{additional corollary} \label{ac 4.1.4}
Let \(x\) be an integer.
Then \(x = -(-x)\).
\end{additional corollary}

\begin{proof}
Let \(x = (a \M b)\) where \(a, b\) are natural numbers.
Then
\begin{align*}
    -(-x) & = -(-(a \M b)) \\
          & = -(b \M a) & \text{by \DEF{4.1.4}} \\
          & = (a \M b)  & \text{by \DEF{4.1.4}} \\
          & = x.
\end{align*}
\end{proof}

\begin{additional corollary} \label{ac 4.1.5}
Let \(x, y\) be integers.
Then \((-x)(-y) = xy\).
\end{additional corollary}

\begin{proof}
\begin{align*}
    (-x)(-y) & = ((-1)x)((-1)y) & \text{by \AC{4.1.3}} \\
             & = (-1)(x(-1))y & \text{by \PROP{4.1.6}(6))} \\
             & = (-1)((-1)x)y & \text{by \PROP{4.1.6}(5)} \\
             & = ((-1)(-1))(xy) & \text{by \PROP{4.1.6}(6)} \\
             & = (-(-1))(xy) & \text{by \AC{4.1.3}} \\
             & = 1(xy) & \text{by \AC{4.1.4}} \\
             & = xy & \text{by \PROP{4.1.6}(7)}
\end{align*}
\end{proof}

Finally, we now \textbf{define the operation of subtraction}. See \DEF{4.1.12}.

We do not need to verify the substitution axiom \AXM{a.7.4} for this operation, since we have defined subtraction \emph{in terms of two other operations on integers}, namely \emph{addition} and \emph{negation}, and we have already verified that those
operations are well-defined(\LEM{4.1.3} and \AC{4.1.2}).

One can easily check now that if \(a\) and \(b\) are \emph{natural numbers} then
\begin{align*}
      a - b & = a + (-b) & \text{by \DEF{4.1.12}} \\
            & = (a \M 0) + (-b) & \text{since (natural) \(a \equiv a \M 0\) (integer)} \\
            & = (a \M 0) + (-(b \M 0)) & \text{since (natural) \(b \equiv b \M 0\) (integer)} \\
            & = (a \M 0) + (0 \M b) & \text{by \DEF{4.1.4}} \\
            & = (a + 0) \M (0 + b) & \text{by \DEF{4.1.2}} \\
            & = a \M b & \text{trivial}
\end{align*}
so \(a \M b\) is just the same thing as \(a - b\).
Because of this we can now discard the \(\M\) notation, and use the familiar operation of subtraction instead.

\begin{note}
Someone else (and me) do not quite get this argument. Refer to \href{https://math.stackexchange.com/questions/3976309/terence-taos-definition-of-subtraction-operation-to-build-integers/3976320}{this} or \href{https://www.wikiwand.com/en/Integer#/Construction}{wiki}.
\end{note}

We can now generalize \LEM{2.3.3} and \CORO{2.3.7} from the \emph{natural} numbers to the \emph{integers}:

\begin{proposition} [Integers have no zero divisors] \label{prop 4.1.8}
Let \(x\) and \(y\) be integers such that \(xy = 0\).
Then either \(x = 0\) or \(y = 0\) (or both).
\end{proposition}

\begin{proof}
Suppose \(xy = 0\) but both \(x \neq 0\) and \(y \neq 0\).
By \LEM{4.1.5}, there are four cases:

\MAROON{(1)} both \(x = a,  y = b\) for some positive natural numbers \(a, b\).
Then by \PROP{2.2.8}, \(xy = ab \neq 0\), a contradiction.

\MAROON{(2)} \(x = a, y = -b\) for some positive natural numbers \(a, b\)
Then
\begin{align*}
    xy & = a \X -b \\
       & = (a - 0) \X -b & \text{by \(a \equiv a - 0\)} \\
       & = (a - 0) \X -(b - 0) & \text{by \(b \equiv b - 0\)} \\
       & = (a - 0) \X (0 - b) & \text{by \DEF{4.1.4}} \\
       & = (a0 + 0b) - (ab + 00) & \text{by \DEF{4.1.2}} \\
       & = 0 - ab & \text{by properties of natural numbers} \\
       & = -(ab - 0) & \text{by \DEF{4.1.4}} \\
       & = -(ab) & \text{\(ab \equiv ab - 0\)}
\end{align*}
So \(xy = -(ab)\), where \(ab\) by \PROP{2.2.8} is a positive natural number.
So by \LEM{4.1.5}, \(xy \neq 0\), a contradiction.

\MAROON{(3)} \(x = -a, y = b\) for some positive natural numbers \(a, b\).
But since by \PROP{4.1.6}(5), \(xy = yx\), and \(yx\) falls back to case \MAROON{(2)}, so \(xy = yx \neq 0\), a contradiction.

\MAROON{(4)} \(x = -a, y = -b\) for some positive natural numbers \(a, b\).
\begin{align*}
    xy & = -a \X -b \\
       & = -(a - 0) \X -b & \text{\(a \equiv a - 0\)} \\
       & = -(a - 0) \X -(b - 0) & \text{\(b \equiv b - 0\)} \\
       & = (0 - a) \X (0 - b) & \text{by \DEF{4.1.4}} \\
       & = (00 + ab) - (a0 + 0b) & \text{by \DEF{4.1.2}} \\
       & = ab - 0 & \text{by properties of natural numbers} \\
       & = ab & \text{\(ab \equiv ab - 0\)}
\end{align*}
So \(xy = ab \neq 0\), a contradiction.

So all cases lead to contradiction, so at least one of \(x\) or \(y\) must be \(0\).
\end{proof}

\begin{corollary} [Cancellation law for integers] \label{corollary 4.1.9}
If \(a, b, c\) are integers such that \(ac = bc\) and \(c\) is non-zero, then \(a = b\).
\end{corollary}

\begin{proof}
\begin{align*}
             & ac = bc \\
    \implies & ac + -(ac) = bc + -(ac) \\
    \implies & 0 = bc + -(ac) & \text{by \PROP{4.1.6}(4)} \\
    \implies & 0 = bc + (-1)(ac) & \text{by \AC{4.1.3}} \\
    \implies & 0 = bc + ((-1)a)c & \text{by \PROP{4.1.6}(6)} \\
    \implies & 0 = bc + (-a)c & \text{by \AC{4.1.3}} \\
    \implies & 0 = (b + (-a))c & \text{by \PROP{4.1.6}(9)} \\
    \implies & 0 = b + (-a) & \text{by \PROP{4.1.8} and \(c \neq 0\)} \\
    \implies & 0 + a = (b + (-a)) + a \\
    \implies & a = (b + (-a)) + a & \text{by \PROP{4.1.6}(3)} \\
    \implies & a = b + ((-a) + a) & \text{by \PROP{4.1.6}(2)} \\
    \implies & a = b + 0 & \text{by \PROP{4.1.6}(4)} \\
    \implies & a = b & \text{by \PROP{4.1.6}(3)}
\end{align*}
\end{proof}

We now extend the notion of order, which was defined on the natural numbers, to the integers by repeating the definition verbatim:

\begin{definition} [Ordering of the integers] \label{def 4.1.10}
Let \(n\) and \(m\) be integers.
We \emph{say} that \(n\) is greater than or equal to \(m\), and write \(n \ge m\) or \(m \le n\), iff we have \(n = m + a\) for some \textbf{natural} number \(a\).
We say that \(n\) is strictly greater than \(m\), and write \(n > m\) or \(m < n\), iff \(n \ge m\) and \(n \neq m\).
\end{definition}

\begin{note}
Clearly this definition is consistent with the notion of order on the \emph{natural} numbers, since we are using the same definition.
\end{note}

\begin{lemma} [Properties of order] \label{lem 4.1.11}
Let \(a, b, c\) be integers.
\begin{enumerate}
    \item \(a > b\) if and only if \(a - b\) is a positive natural number.
    \item (Addition preserves order) If \(a > b\), then \(a + c > b+ c\).
    \item (\emph{Positive} multiplication preserves order) If \(a > b\) and \(c\) is positive, then \(ac > bc\).
    \item (Negation reverses order) If \(a > b\), then \(-a < -b\).
    \item (Order is transitive) If \(a > b\) and \(b > c\), then \(a > c\).
    \item (Order trichotomy) Exactly one of the statements \(a > b\), \(a < b\), or \(a = b\) is true.
\end{enumerate}
\end{lemma}

\begin{proof}
\begin{enumerate}
\item \MAROON{(1)}
    \begin{align*}
             & a > b \\
        \iff & a \ge b \land a \neq b & \text{by \DEF{4.1.10}} \\
        \iff & \exists c \in \SET{N}\ a = b + c \land a \neq b \land c \neq 0 & \text{by \DEF{4.1.10}, and \(c \neq 0\) otherwise \(a = b\)} \\
        \iff & a - b = b + c - b \\
        \iff & a - b = b + c + (-b) & \text{by \DEF{4.1.12}} \\
        \iff & a - b = b + (-b) + c & \text{by \PROP{4.1.6}(1)} \\
        \iff & a - b = 0 + c = c & \text{by \PROP{4.1.6}(4)(3)} \\
    \end{align*}
    \(\iff a - b\) is equal to a positive natural number.

\item
    \begin{align*}
                 & a > b \\
        \implies & \exists d \in \SET{N}\ a = b + d \land a \neq b \land d \neq 0 & \text{same argument in \MAROON{(1)}} \\
        \implies & a + c = b + d + c \\
        \implies & a + c = b + c + d & \text{by \PROP{4.1.6}(1)} \\
        \implies & a + c > b + c & \text{by \DEF{4.1.10}}
    \end{align*}

\item
    \begin{align*}
                 & a > b \\
        \implies & \exists d \in \SET{N}\ a = b + d \land a \neq b \land d \neq 0 & \text{same argument in \MAROON{(1)}} \\
        \implies & ac = (b + d)c \\
        \implies & ac = bc + dc & \text{by \PROP{4.1.6}(9)} \\
        \implies & ac - bc = bc + dc - bc \\
        \implies & ac - bc = dc & \text{by \PROP{4.1.6}(3)(4), \DEF{4.1.4}}
    \end{align*}
    where \(dc\) is a positive natural number because of \LEM{2.3.3}, and both \(d, c\) are equal to a positive natural number.
    So by \MAROON{(1)}, \(ac > bc\).

\item
    \begin{align*}
                 & a > b \\
        \implies & \exists d \in \SET{N}\ a = b + d \land a \neq b \land d \neq 0 & \text{same argument in \MAROON{(1)}} \\
        \implies & (-a) = -(b + d) \\
        \implies & (-a) = (-1)(b + d) & \text{by \AC{4.1.3}} \\
        \implies & (-a) = (-1)b + (-1)d & \text{by \PROP{4.1.6}(8)} \\
        \implies & (-a) = (-b) + (-d) & \text{by \AC{4.1.3}} \\
        \implies & (-a) + d = (-b) + (-d) + d \\
        \implies & (-a) + d = (-b) + 0 & \text{by \PROP{4.1.6}(4)} \\
        \implies & (-a) + d = (-b) & \text{by \PROP{4.1.6}(3)} \\
        \implies & (-a) + d + (-(-a)) = (-b) + (-(-a)) \\
        \implies & d = (-b) + (-(-a)) & \text{by \PROP{4.1.6}(3)(4)} \\
        \implies & d = (-b) - (-a) & \text{by \DEF{4.1.12}}
    \end{align*}
    So \((-b) - (-a)\) is a positive integer. So by \MAROON{(1)}, \(-b > -a\).

\item (Order is transitive) If \(a > b\) and \(b > c\), then \(a > c\):
    \begin{align*}
                 & a > b \land b > c \\
        \implies & a - b = d_1 \land b - c = d_2 \text{\ where \(d_1, d_2\) is positive} & \text{same argument in \MAROON{(1)}} \\
        \implies & (a - b) + (b - c) = d_1 + d_2 \\
        \implies & a + (-b) + b + (-c) = d_1 + d_2 & \text{by \DEF{4.1.12}} \\
        \implies & a + (-c) = d_1 + d_2 & \text{by \PROP{4.1.6}(3)(4)} \\
        \implies & a - c = d_1 + d_2 & \text{by \DEF{4.1.12}}
    \end{align*}
    where \(d_1 + d_2\) by \PROP{2.2.8} is positive.
    So \(a - c\) is positive, so by \MAROON{(1)}, \(a > c\).

\item
    Given any two natural numbers \(a, b\), by \LEM{4.1.5}, \emph{exactly one of} the three cases below is true:
    \begin{enumerate}
        \item \(a - b = 0\): then
            \begin{align*}
                     & a - b = 0 \\
                \iff & a - b + b = 0 + b \\
                \iff & a + (-b) + b = 0 + b & \text{by \DEF{4.1.12}} \\
                \iff & a = b & \text{by \PROP{4.1.6}(3)(4)}
            \end{align*}
            
            Suppose for the sake of contradiction that \(a > b\) or \(b > a\), then by \MAROON{(1)} \(a - b\) or \(b - a\) is a positive natural number, contradicting \(a - b = 0\).
        \item \(a - b = n\) where \(n\) is a positive natural number:
            Then by \MAROON{(1)}, \(a - b\) is equal to a positive natural number if and only if \(a > b\).
            
            It is trivial that the supposition of \(a = b\) or \(a < b\) lead to contradiction.
        \item \(a - b = -n\) where \(n\) is a positive natural number:
            \begin{align*}
                     & a - b = -n \\
                \iff & -(a - b) = -(-n) \\
                \iff & b - a = -(-n) & \text{by \DEF{4.1.4}} \\
                \iff & b - a = n & \text{by \AC{4.1.4}} \\
                \iff & b > a & \text{by \MAROON{(1)}}
            \end{align*}
            It is trivial that the supposition of \(a = b\) or \(a > b\) lead to contradiction.
    \end{enumerate}
\end{enumerate}
\end{proof}

\begin{note}
The book does not label the definition of subtraction of integers, so I define it below.
\end{note}
\begin{definition} [on page 79] \label{def 4.1.12}
We now define the operation of subtraction \(x - y\) of two integers by the formula
\[
    x - y := x + (-y)
\]
\end{definition}


\exercisesection

\begin{exercise} \label{exercise 4.1.1}
Verify that the definition of equality on the integers is both reflexive and symmetric.
\end{exercise}

\begin{proof}
See \AC{4.1.1}.
\end{proof}

\begin{exercise} \label{exercise 4.1.2}
Show that the definition of negation on the integers is well-defined in the sense that if \((a \M b) = (a' \M b')\), then \(-(a \M b) = -(a' \M b')\)
(so equal integers have equal negations).
\end{exercise}

\begin{proof}
See \AC{4.1.2}.
\end{proof}

\begin{exercise} \label{exercise 4.1.3}
Show that \((-1) \X a = -a\) for every integer \(a\).
\end{exercise}

\begin{proof}
See \AC{4.1.3}.
\end{proof}

\begin{exercise} \label{exercise 4.1.4}
Prove the remaining identities in \PROP{4.1.6}.
\end{exercise}

\begin{proof}
See \PROP{4.1.6}.
\end{proof}

\begin{exercise} \label{exercise 4.1.5}
Prove \PROP{4.1.8}.
\end{exercise}

\begin{proof}
See \PROP{4.1.8}.
\end{proof}

\begin{exercise} \label{exercise 4.1.6}
Prove \CORO{4.1.9}.
\end{exercise}

\begin{proof}
See \CORO{4.1.9}.
\end{proof}

\begin{exercise} \label{exercise 4.1.7}
Prove \LEM{4.1.11}.
\end{exercise}

\begin{proof}
See \LEM{4.1.11}.
\end{proof}

\begin{exercise} \label{exercise 4.1.8}
Show that the principle of induction (\AXM{2.5}) does not apply directly to the integers.
More precisely, give an example of a property \(P(n)\) pertaining to an \emph{integer} \(n\) such that \(P(0)\) is true, and that \(P(n)\) implies \(P(n\INC)\) for all integers \(n\), but that \(P(n)\) is not true for all \emph{integers} \(n\).
Thus induction is not as useful a tool for dealing with the \emph{integers} as it is with the \emph{natural} numbers.
(The situation becomes even worse with the rational and real numbers, which we shall define shortly.)
\end{exercise}

\begin{proof}
Let \(P(n) := n \text{ is equal to a positive natural number or \(0\)}\).
    
Then \(P(0)\) is true.
    
Suppose \(P(n)\) is true, then \(n\) is positive or \(0\); if \(n = 0\), and \(n\INC = 1\), which is positive, if \(n\) is positive, then \(n\INC = n + 1\) is positive by \PROP{2.2.8}. So in both cases \(n\INC\) is positive, so \(P(n\INC)\) is true.
    
But \(P(-1)\) is false because if \(n = -1\), then by \LEM{4.1.5} \(n\) cannot be equal to \(0\) or a positive natural number.
\end{proof}
\section{The rationals} \label{sec 4.2}

Of course, just as two differences \(a - b\) and \(c - d\) can be equal if \(a+d = c+b\), we know (from more advanced knowledge) that two quotients \(a / b\) and \(c / d\) can be equal if \(ad = bc\). (or \(ad = cb\), to be consistent with the definition below.)
Thus, in analogy with the integers, we create a new meaningless symbol \(//\) (which will eventually be superseded by division), and define:

\begin{definition} \label{def 4.2.1}
A \emph{rational number} is an expression of the form \(a \D b\), where \(a\) and \(b\) are \emph{integers} and \(b\) is \emph{non-zero}; \(a \D 0\) is not considered to be a rational number.
Two rational numbers are considered to be equal, \(a \D b = c \D d\), if and only if \(ad = cb\).
The set of all rational numbers is denoted \(\SET{Q}\).
\end{definition}

\begin{note}
Like integers, you can refer to \href{https://www.wikiwand.com/en/Rational_number#/Formal_construction}{wiki}
\end{note}

This is a valid definition of equality (\EXEC{4.2.1}).

\begin{additional corollary} \label{ac 4.2.1}
The definition of equality for the rational numbers is reflexive, symmetric and transitive.
\end{additional corollary}

\begin{proof}
Reflexive: Suppose \(a, b\) are integers where \(b \neq 0\), we have to show \(a \D b = a \D b\).
But \(\GREEN{a}\BLUE{b} = \BLUE{a}\GREEN{b}\) by reflexivity of integers. So by \DEF{4.2.1}, \(\GREEN{a \D b} = \BLUE{a \D b}\).

Symmetric: Suppose \(a, b, c, d\) are integers such that \(b, d \neq 0\) and \(a \D b = c \D d\), we have to show \(c \D d = a \D b\). Then
\begin{align*}
             & a \D b = c \D d \\
    \implies & ad = cb & \text{by \DEF{4.2.1}} \\
    \implies & cb = ad & \text{since integers are symmetric} \\
    \implies & c \D d = a \D b & \text{by \DEF{4.2.1}}
\end{align*}

Transitive: Suppose \(a, b, c, d, e, f\) are integers such that \(b, d, f \neq 0\) and \(a \D b = c \D d\) and \(c \D d = e \D f\), we have to show \(a \D b = e \D f\). Then
\begin{align*}
             & a \D b = c \D d \land c \D d = e \D f \\
    \implies & ad = cb \land cf = ed & \text{by \DEF{4.2.1}} \\
    \implies & adf = cbf \land cfb = edb & \text{multiply first equation by \(f\), second by \(b\)}\\
    \implies & adf = cbf \land cbf = edb & \text{by \PROP{4.1.6}(5)} \\
    \implies & adf = edb & \text{since integers are transitive} \\
    \implies & afd = ebd & \text{by \PROP{4.1.6}(5)} \\
    \implies & af = eb & \text{by \PROP{4.1.8} and \(d \neq 0\)} \\
    \implies & a \D b = e \D f & \text{by \DEF{4.2.1}}
\end{align*}
\end{proof}

\begin{definition} \label{def 4.2.2}
If \(a \D b\) and \(c \D d\) are rational numbers, we \emph{define} their sum
\[
    (a \D b) + (c \D d) := (ad + bc) \D (bd)
\]
their product
\[
    (a \D b) \X (c \D d) := (ac) \D (bd)
\]
and the negation
\[
    -(a \D b) := (-a) \D b.
\]
\end{definition}

\begin{note}
If \(b\) and \(d\) are non-zero, then \(bd\) is also non-zero, by \PROP{4.1.8}, so the sum or product of two rational numbers remains a
rational number.
\end{note}

\begin{note}
我覺得\ negation 是用定義的,挺神奇的。另外這個定義是把有理數的\ negation reduce 給整數的\ negation。
\end{note}

\begin{lemma} \label{lem 4.2.3}
The sum, product, and negation operations on rational numbers are well-defined,
in the sense that if one replaces \(a \D b\) with another rational number \(a' \D b'\) which \emph{is equal to} \(a \D b\), then the output
of the above operations remains unchanged, and similarly for \(c \D d\) replaced by \(c' \D d'\) which \emph{is equal to} \(c \D d\).
\end{lemma}

\begin{proof} 
Let \(a, b, a', b', c, d, c' d'\) be integers s.t. \(b, b', d, d' \neq 0\) and \(a \D b = a' \D b'\) and \(c \D d = c' \D d'\).
Note that by \DEF{4.2.1} we have \(ab' = a'b\) \MAROON{(1)} and \(cd' = c'd\) \MAROON{(2)}.

\begin{align*}
         & (a \D b) + (c \D d) = (a' \D b') + (c \D d) \\
    \iff & (ad + bc) \D (bd) = (a'd + b'c) \D (b'd) & \text{by \DEF{4.2.2}} \\
    \iff & (ad + bc)(b'd) = (a'd + b'c)(bd) & \text{by \DEF{4.2.1}} \\
    \iff & adb'd + bcb'd = a'dbd + b'cbd & \text{by \PROP{4.1.6}(9)} \\
    \iff & ab'dd + bb'cd = a'bdd + bb'cd & \text{by \PROP{4.1.6}(5)(6)} \\
    \iff & ab'dd + bb'cd - bb'cd = a'bdd + bb'cd - bb'cd \\
    \iff & ab'dd = a'bdd & \text{by \PROP{4.1.6}(3)(4)} \\
    \iff & ab' = a'b & \text{by \CORO{4.1.9}, cancel \(d\) twice} \\
\end{align*}
which is true by \MAROON{(1)}.

\begin{align*}
         & (a \D b) + (c \D d) = (a \D b) + (c' \D d') \\
    \iff & (ad + bc) \D (bd) = (ad' + bc') \D (bd') & \text{by \DEF{4.2.2}} \\
    \iff & (ad + bc)(bd') = (ad' + bc')(bd) & \text{by \DEF{4.2.1}} \\
    \iff & adbd' + bcbd' = ad'bd + bc'bd & \text{by \PROP{4.1.6}(9)} \\
    \iff & abdd' + bbcd' = abdd' + bbc'd & \text{by \PROP{4.1.6}(5)(6)} \\
    \iff & (-(abdd')) + abdd' + bbcd' = (-(abdd')) + abdd' + bbc'd \\
    \iff & bbcd' = bbc'd & \text{by \PROP{4.1.6}(3)(4)} \\
    \iff & cd'bb = c'dbb & \text{by \PROP{4.1.6}(5)} \\
    \iff & cd' = c'd & \text{by \CORO{4.1.9}, cancel \(b\) twice} \\
\end{align*}
which is true by \MAROON{(2)}.

\begin{align*}
         & (a \D b) \X (c \D d) = (a' \D b') \X (c \D d) \\
    \iff & (ac) \D (bd) = (a'c) \D (b'd) & \text{by \DEF{4.2.2}} \\
    \iff & acb'd = a'cbd & \text{by \DEF{4.2.1}} \\
    \iff & ab'cd = a'bcd & \text{by \PROP{4.1.6}(5)(6)} \\
    \iff & ab' = a'b & \text{by \CORO{4.1.9}, cancel \(c\) and \(d\)}
\end{align*}
which is true by \MAROON{(1)}.

\begin{align*}
         & (a \D b) \X (c \D d) = (a \D b) \X (c' \D d') \\
    \iff & (ac) \D (bd) = (ac') \D (bd') & \text{by \DEF{4.2.2}} \\
    \iff & acbd' = ac'bd & \text{by \DEF{4.2.1}} \\
    \iff & cd'ab = c'dab & \text{by \PROP{4.1.6}(5)(6)} \\
    \iff & cd' = c'd & \text{by \CORO{4.1.9}, cancel \(a\) and \(b\)}
\end{align*}
which is true by \MAROON{(2)}.

\begin{align*}
         & -(a \D b) = -(a' \D b') \\
    \iff & (-a) \D b = (-a') \D b' & \text{by \DEF{4.2.2}} \\
    \iff & (-a)b' = (-a')b & \text{by \DEF{4.2.1}} \\
    \iff & (-1)ab' = (-1)a'b & \text{by \AC{4.1.3}} \\
    \iff & ab' = a'b & \text{by \CORO{4.1.9}, cancel \(-1\)}
\end{align*}
which is true by \MAROON{(1)}
\end{proof}

We note that the \emph{rational} numbers \(a \D 1\) \textbf{behave} in a manner identical to the \emph{integers} \(a\):
\[
    (a \D 1) + (b \D 1) = (a \X 1 + 1 \X b) \D (1 \X 1) = (a + b) \D 1
\]
\[
    (a \D 1) \X (b \D 1) = (a \X b) \D (1 \X 1) = (ab \D 1)
\]
\[
    -(a \D 1) = (-a) \D 1.
\]
Also, \(a \D 1\) and \(b \D 1\) are only equal when \(a\) and \(b\) are equal.
Because of this, we will identify a with \(a \D 1\) for each integer \(a\):
\[
    a \equiv a \D 1;
\]
the above identities then guarantee that \emph{the arithmetic of the integers is consistent with the arithmetic of the rationals}.
Thus just as we embedded the \emph{natural} numbers inside the \emph{integers}, we embed the \emph{integers} inside the \emph{rational} numbers.
In particular, all \emph{natural} numbers are \emph{rational} numbers, for instance \(0\) is equal to \(0 \D 1\) and \(1\) is equal to \(1 \D 1\).

\begin{additional corollary} \label{ac 4.2.2}
Observe that
\begin{align*}
         & a \D b = 0 \\
    \iff & a \D b = 0 \D 1 & \text{since \(0 \equiv 0 \D 1\)} \\
    \iff & a \X 1 = 0 \X b & \text{by \DEF{4.2.1}} \\
    \iff & a = 0 & \text{simplify}
\end{align*}
Thus if \(a\) and \(b\) are non-zero (note that \(b\) must be non-zero) then so is \(a \D b\). This is used in \DEF{4.1.11}.
\end{additional corollary}

We now define a new operation on the rationals: reciprocal. See \DEF{4.1.11} (just give the definition a label to make it convenient to refer to).

\begin{proposition} [Laws of algebra for rationals] \label{prop 4.2.4}
Let \(x, y, z\) be rationals.
Then the following laws of algebra hold:
\begin{align*}
          x + y & = y + x \MAROON{\ (1)} \\
    (x + y) + z & = x + (y + z) \MAROON{\ (2)} \\
          x + 0 & = 0 + x = x \MAROON{\ (3)} \\
       x + (-x) & = (-x) + x = 0 \MAROON{\ (4)} \\
             xy & = yx \MAROON{\ (5)} \\
          (xy)z & = x(yz) \MAROON{\ (6)} \\
        x1 = 1x & = x \MAROON{\ (7)} \\
       x(y + z) & = xy + xz \MAROON{\ (8)} \\
       (y + z)x & = yx + zx \MAROON{\ (9)}
\end{align*}
If \(x\) is non-zero, we also have
\begin{align*}
    xx^{-1} = x^{-1}x = 1. \MAROON{\ (10)}
\end{align*}
\end{proposition}

\begin{proof}
Let \(x = (a \D b), y = (c \D d)\), and \(z = (e \D f)\) for some integers \(a, b, c, d, e, f\) where \(b, d, f \neq 0\).

\MAROON{(1)} \(x + y = y + x\):
\begin{align*}
    x + y & = (a \D b) + (c \D d) \\
          & = (ad + bc) \D (bd) & \text{by \DEF{4.2.2}} \\
          & = (cb + da) \D (db) & \text{by properties of integers \PROP{4.1.6}} \\
          & = (c \D d) + (a \D b) & \text{by \DEF{4.2.2}} \\
          & = y + x
\end{align*}

\MAROON{(2)} \((x + y) + z = x + (y + z)\):
\begin{align*}
    (x + y) + z & = ((a \D b) + (c \D d)) + (e \D f) \\
                & = ((ad + bc) \D (bd)) + (e \D f) & \text{by \DEF{4.2.2}} \\
                & = ((ad + bc)f + (bd)e) \D ((bd)f) & \text{by \DEF{4.2.2}} \\
                & = (adf + bcf + bde) \D (bdf) \GREEN{\ (1)} & \text{by properties of integers \PROP{4.1.6}}
\end{align*}
\begin{align*}
    x + (y + z) & = (a \D b) + ((c \D d) + (e \D f)) \\
                & = (a \D b) + ((cf + de) \D (df) & \text{by \DEF{4.2.2}} \\
                & = (a(df) + b(cf + de)) \D (b(df)) & \text{by \DEF{4.2.2}} \\
                & = (adf + bcf + bde) \D (bdf) & \text{by properties of integers \PROP{4.1.6}} \\
                & = \GREEN{(1)}
\end{align*}
So \((x + y) + z = x + (y + z)\).

\MAROON{(3)} \(x + 0 = 0 + x = x\):
By \MAROON{(1)} we have \(x + 0 = 0 + x\), so we only have to show \(x + 0 = x\).
Then
\begin{align*}
    x + 0 & = (a \D b) + 0 \\
          & = (a \D b) + (0 \D 1) & \text{since (integer) \(0 \equiv (0 \D 1)\) (rational)} \\
          & = (a \X 1 + b \X 0) \D (b \X 1) & \text{by \DEF{4.2.2}} \\
          & = a \D b & \text{by properties of integers \PROP{4.1.6}} \\
          & = x
\end{align*}

\MAROON{(4)} \(x + (-x) = (-x) + x = 0\):
By \MAROON{(1)} we have \(x + (-x) = (-x) + x\), so we only have to show \(x + (-x) = 0\).
Then
\begin{align*}
    x + (-x) & = (a \D b) + -(a \D b) \\
             & = (a \D b) + ((-a) \D b) & \text{by \DEF{4.2.2}} \\
             & = (ab + b(-a)) \D (bb) & \text{by \DEF{4.2.2}} \\
             & = (0) \D (bb) & \text{by properties of integers \PROP{4.1.6}} \\
             & = 0 & \text{since (integer) \( 0 \equiv (0) \D (bb)\) (rational)}
\end{align*}

\MAROON{(5)} \(xy = yx\):
\begin{align*}
    xy & = (a \D b)(c \D d) \\
       & = (ac) \D (bd) \text{\ \GREEN{(2)}} & \text{by \DEF{4.2.2}}
\end{align*}
And 
\begin{align*}
    yx & = (c \D d)(a \D b) \\
       & = (ca) \D (db) & \text{by \DEF{4.2.2}} \\
       & = (ac) \D (bd) & \text{by properties of integers \PROP{4.1.6}} \\
       & = \text{\GREEN{(2)}}
\end{align*}
So \(xy = yx\).

\MAROON{(6)} \((xy)z = x(yz)\):
\begin{align*}
    (xy)z & = ((a \D b)(c \D d))(e \D f) \\
          & = ((ac) \D (bd))(e \D f) & \text{by \DEF{4.2.2}} \\
          & = ((ac)e) \D ((bd)f) & \text{by \DEF{4.2.2}} \\
          & = (a(ce)) \D (b(df)) & \text{by \PROP{4.1.6}(6)} \\
          & = (a \D b)((ce) \D (df)) & \text{by \DEF{4.2.2}} \\
          & = (a \D b)((c \D d)(e \D f)) & \text{by \DEF{4.2.2}} \\
          & = x(yz)
\end{align*}

\MAROON{(7)} \(x1 = 1x = x\):
Again by \MAROON{(1)}, we only have to show \(x1 = x\).
\begin{align*}
    x1 & = (a \D b) \X 1 \\
       & = (a \D b) \X (1 \D 1) & \text{since (integer) \(1 \equiv 1 \D 1\) (rational)} \\
       & = (a \X 1) \D (b \X 1) & \text{by \DEF{4.2.2}} \\
       & = a \D b & \text{by properties of integers \PROP{4.1.6}} \\
       & = x
\end{align*}

\MAROON{(8)} \(x(y + z) = xy + xz\):
\begin{align*}
    x(y + z) & = (a \D b)((c \D d) + (e \D f)) \\
             & = (a \D b)((cf + de) \D (df)) & \text{by \DEF{4.2.2}} \\
             & = (a(cf + de)) \D (b(df)) & \text{by \DEF{4.2.2}} \\
             & = (acf + ade) \D (bdf) \GREEN{\ (3)} & \text{by properties of integers \PROP{4.1.6}}
\end{align*}
And
\begin{align*}
    xy + xz & = (a \D b)(c \D d) + (a \D b)(e \D f) \\
            & = ((ac) \D (bd)) + ((ae) \D (bf)) & \text{by \DEF{4.2.2}} \\
            & = ((ac)(bf) + (bd)(ae)) \D ((bd)(bf)) & \text{by \DEF{4.2.2}} \\
            & = (acbf + abde) \D (bdbf) & \text{by properties of integers \PROP{4.1.6}} \\
            & = (b(acf + ade)) \D (b(bdf)) & \text{by \PROP{4.1.6}(8)} \\
            & = (b \D b)((acf + ade) \D (bdf)) & \text{by \DEF{4.2.2}} \\
            & = (1 \D 1)((acf + ade) \D (bdf)) & \text{by \DEF{4.2.1}, \(\GREEN{1 \D 1} = \BLUE{b \D b} \iff \GREEN{1} \X \BLUE{b} = \BLUE{b} \X \GREEN{1} \)} \\
            & = 1((acf + ade) \D (bdf)) & \text{since (integer) \(1 \equiv 1 \D 1\) (rational)} \\
            & = (acf + ade) \D (bdf) & \text{by \MAROON{(7)}} \\
            & = \GREEN{(3)}
\end{align*}
So \(x(y + z) = xy + xz\).

\MAROON{(9)} \((y + z)x = yx + zx\):
\begin{align*}
    (y + z)x & = x(y + z) & \text{by \MAROON{(1)}} \\
             & = xy + xz & \text{by \MAROON{(8)}} \\
             & = yx + zx & \text{by \MAROON{(1)}}
\end{align*}

\MAROON{(10)} \(xx^{-1} = x^{-1}x = 1\):
Suppose \(x^{-1}\) is well-defined, i.e. \(x = a \D b\) where both \(a, b \neq 0\).
Again by \MAROON{(1)} we only have to show \(xx^{-1} = 1\).
Then
\begin{align*}
    xx^-1 & = (a \D b)(b \D a) \\
          & = (ab) \D (ba) & \text{by \DEF{4.2.2}} \\
          & = (ab) \D (ab) & \text{by \PROP{4.1.6}} \\
          & = 1 \D 1 & \text{by \DEF{4.2.1}, \(\GREEN{1 \D 1} = \BLUE{ab \D ab} \iff \GREEN{1} \X \BLUE{ab} = \BLUE{ab} \X \GREEN{1} \)} \\
          & = 1 & \text{since (integer) \(1 \equiv 1 \D 1\) (rational)}
\end{align*}
\end{proof}

\begin{remark} \label{remark 4.2.5}
The above set of ten identities have a name; they are asserting that the rationals \(\SET{Q}\) form a \emph{field}.
This is \emph{better} than being a \emph{commutative ring} because of \emph{the tenth identity} \(xx^{-1} = x^{-1}x = 1\).
Note that this proposition \emph{supersedes} \PROP{4.1.6}.
\end{remark}

We can now define the \emph{quotient} \(x / y\) of two rational numbers: see \DEF{4.2.12}. And we also define the \emph{subtraction} of twp rational numbers; see \DEF{4.2.13}.

\begin{note}
\PROP{4.2.4} allows us to use all the normal rules of algebra; we will now proceed to do so without further comment.
\end{note}

\begin{definition} \label{def 4.2.6}
A rational number \(x\) is said to be \emph{positive} iff we have \(x = a/b\) for some positive integers \(a\) and \(b\).
It is said to be \emph{negative} iff we have \(x = -y\) for some positive \emph{rational} \(y\) (i.e., \(x =(-a)/b\) for some \emph{positive} integers \(a\) and \(b\)).
\end{definition}

\begin{note}
From \DEF{4.2.6}, note that every positive \emph{integer} is a positive \emph{rational} number,
and every negative \emph{integer} is a negative \emph{rational} number,
so our new definition is consistent with our old one.
\end{note}

\begin{additional corollary} \label{ac 4.2.3}
Let \(x = a / b\) be a rational number where \(a, b\) are integers and \(b \neq 0\).
Then
\begin{align*}
    -x & = (-a) / b & \text{by \DEF{4.2.2}} \\
       & = a / (-b) \MAROON{\ (1)} \\
       & = (-1)(a / b) = (-1)x \MAROON{\ (2)}.
\end{align*}
And \(-(-x) = x\) \MAROON{(3)}.
\end{additional corollary}

\begin{proof}
\begin{align*}
    -x & = -(a / b) \\
       & = (-a) / b & \text{by \DEF{4.2.2}} \\
       & = ((-1) \X a) / b & \text{by \AC{4.1.3}} \\
       & = ((-1) \X a) / (1 \X b) & \text{by \PROP{4.1.6}(7)} \\
       & = (-1 / 1) \X (a / b) & \text{by \DEF{4.2.2}} \\
       & = (-1) \X (a / b) & \text{since (integer) \(-1 \equiv -1 / 1\) (rational)} \\
       & = (-1)x & \text{proved \MAROON{(2)}}
\end{align*}
Now, since \(\GREEN{-1} \X \BLUE{-1} = \BLUE{1} \X \GREEN{1}\) by the properties of integers, by \DEF{4.2.1}, \(\GREEN{(-1)} / \GREEN{1} = \BLUE{1} / \BLUE{(-1)}\).
So from \MAROON{(2)}, we have
\begin{align*}
        (-1)x
    & = ((-1) / 1)x & \text{since (integer) \(-1 \equiv -1 / 1\) (rational)} \\
    & = (1 / (-1))x & \text{we have shown \((-1)/1 = 1/(-1)\)} \\
    & = (1 / (-1))(a/b) \\
    & = (1 \X a)/((-1) \X b) & \text{by \DEF{4.2.2}} \\
    & = a/(-b) & \text{by properties of integer}
\end{align*}
So we have proved \MAROON{(1)}

Finally,
\begin{align*}
    -(-x) = -(-(a / b))
        & = -((-a) / b) & \text{by \DEF{4.2.2}} \\
        & = (-(-a)) / b & \text{by \DEF{4.2.2}} \\
        & = a / b & \text{by \AC{4.1.4}} \\
        & = x
\end{align*}
So we proved \MAROON{(3)}.
\end{proof}

\begin{lemma} [Trichotomy of rationals] \label{lem 4.2.7}
Let \(x\) be a rational number.
Then \emph{exactly one} of the following three statements is true:
\begin{enumerate}
    \item \(x\) is equal to \(0\).
    \item \(x\) is a positive rational number.
    \item \(x\) is a negative rational number.
\end{enumerate}
number.
\end{lemma}

\begin{proof}
We first show at least one of these cases are true.

Let \(x\) be a rational number.
By \DEF{4.2.1}, \(x = a / b\) for some integers \(a, b\) and \(b \neq 0\).
By \LEM{4.1.11}(f), exactly one of \(a = 0, a < 0, a > 0\) is true, and similarly since \(b \neq 0\), exactly one of \(b > 0, b < 0\) is true.
\begin{itemize}
    \item \(a = 0\):
        Then by \AC{4.2.2}, we have \(a / b = 0\).
    \item \(a < 0\):
        \begin{itemize}
            \item
                If \(b < 0\), then by \LEM{4.1.5}, \(a = -c\), \(b = -d\) for some positive natural numbers(i.e. integers) \(c, d\).
                So
                \begin{align*}
                    x & = a / b \\
                      & = (-c)/(-d) \\
                      & = ((-1)(c))/((-1)(d)) & \text{by \AC{4.1.3}} \\
                      & = ((-1)/(-1))(c/d)) & \text{by \DEF{4.2.2}} \\
                      & = (1/1)(c/d) & \text{similar cases showed many times} \\
                      & = 1(c/d) & \text{\(1 \equiv 1 / 1\)} \\
                      & = c/d. & \text{by \PROP{4.2.4}(7)} \\
                \end{align*}
                So \(x = a/ b = c/d\) for some positive integers \(c, d\).
                By \DEF{4.2.6}, \(x\) is positive.
            \item 
                If \(b > 0\), then by \LEM{4.1.5}, only \(a = -c\) for some positive natural number(i.e. integer).
                So \(x = a / b = (-c) / b\) for some positive integers \(c, b\).
                By \DEF{4.2.6}, \(x\) is negative.
        \end{itemize}
    \item \(a > 0\):
        \begin{itemize}
            \item 
                If \(b < 0\), then by \LEM{4.1.5}, only \(b = -c\) for some positive natural number(i.e. integer).
                So
                \begin{align*}
                    x & = a / b \\
                      & = a / (-c) \\
                      & = (-a)/c & \text{by \AC{4.2.3}} \\
                \end{align*} for some positive integers \(a, c\).
                By \DEF{4.2.6}, \(x\) is negative.
            \item 
                If \(b > 0\), then \(x = a/b\) for some positive integers \(a, b\). By \DEF{4.2.6}, \(x\) is positive.
        \end{itemize}
\end{itemize}

Now we show at most one of these cases is true, that is, (a)(b) can not be both true and (a)(c) can not be both true and (b)(c) can not be both true.
For the sake of contradiction, suppose:
\begin{itemize}
    \item
        If (a)(b) are both true, i.e. \(x = 0\) and \(x\) is positive, then we have \(x = 0/1\) and by \DEF{4.2.6} we have \(x = a / b\) for some positive integers \(a, b\).
        So \(0/1 = a/b\), by \DEF{4.2.1}, \(0 \X b = a \X 1\), so \(0 = a\), contradicting that \(a\) is positive.
    \item
        If (a)(c) are both true, i.e. \(x = 0\) and \(x\) is negative, then we have \(x = 0/1\) and by \DEF{4.2.6} we have \(x = (-a) / b\) for some positive integers \(a, b\).
        So \(0/1 = (-a)/b\), by \DEF{4.2.1}, \(0 \X b = (-a) \X 1\), so \(0 = -a\), contradicting that \(a\) is positive.
    \item
        If (b)(c) are both true, i.e. \(x\) is positive and \(x\) is negative, and by \DEF{4.2.6} we have \(x = a/b\) for some positive integers \(a, b\) and \(x = (-c)/d\) for some positive integers \(c, d\).
        So \(a/d = (-c)/d\), by \DEF{4.2.1}, \(ad = (-c)d\), and by \CORO{4.1.9}, since \(d \neq 0\), we have \(a = -c\), contradicting that \(a\) is positive.
\end{itemize}
\end{proof}

\begin{definition} [Ordering of the rationals] \label{def 4.2.8}
Let \(x\) and \(y\) be rational numbers.
We say that \(x > y\) iff \(x - y\) is a positive rational number,
and \(x < y\) iff \(x - y\) is a negative rational number.
We write \(x \ge y\) iff either \(x > y\) or \(x = y\), and similarly define \(x \le y\).
\end{definition}

\begin{note}
From \DEF{4.2.8}, \(x < y\) and \(y > x\) is \emph{not defined} to be equivalent, We need to \emph{prove} it(see \PROP{4.2.9}).
\end{note}
\begin{note}
It is trivial that \(x > 0\) and \(-x < 0\) for some positive rational number \(x\).
\end{note}

\begin{note}
In the proofs of \AC{4.2.4} - \AC{4.2.7}, when we have positive integers, because of the isomorphism between positive integer and natural number, we can use properties of natural numbers(properties of \CH{2}) on these positive integers.
\end{note}

\begin{additional corollary} \label{ac 4.2.4}
If \(x\) and \(y\) are two positive rationals, then \(x + y\) is also a positive rational number.
If \(x\) and \(y\) are two negative rationals, then \(x + y\) is also a negative rational number.
\end{additional corollary}

\begin{proof}
Suppose \(x\) and \(y\) are two positive rationals, we have to show \(x + y\) is positive.
By \DEF{4.2.6} \(x = a / b\) and \(y = c / d\) where \(a, b, c, d\) are \emph{positive} integers.
Then by \DEF{4.2.2} we have \(x + y = (ad + bc) / bd\).
By isomorphism and \LEM{2.3.3}, \(ad, bc, bd\) are also positive.
Since \(ad, bc\) are positive, by isomorphism and \PROP{2.2.8}, \(ad + bc\) and \(bd\) are positive.
So by \DEF{4.2.6}, \(x + y = (ad + bc) / bd\) is a positive rational number.

Now suppose \(x\) and \(y\) are two negative rationals, we have to show \(x + y\) is negative.
By \DEF{4.2.6} \(x = (-a) / b\) and \(y = (-c) / d\) where \(a, b, c, d\) are \emph{positive} integers.
By \DEF{4.2.2} \(x + y = ((-a)d + b(-c)) / bd\).
And
\begin{align*}
      (-a)d + b(-c) & = (-a)d + (-c)b & \text{by \PROP{4.2.4}(5)} \\
                    & = ((-1)a)d + ((-1)c)b & \text{by \AC{4.2.3}} \\
                    & = (-1)(ad) + (-1)(cb) & \text{by \PROP{4.2.4}(6)} \\
                    & = (-1)(ad + cb) & \text{by \PROP{4.2.4}(8)} \\
                    & = -(ad + cb) & \text{by \AC{4.2.3}} \\
                    & = -(ad + bc) & \text{by \PROP{4.2.4}(5)}
\end{align*}
So we have we have \(x + y = ((-a)d + b(-c)) / bd = (-(ad + bc)) / bd\).
By the same argument in the previous case, \(ad + bc\) and \(bd\) are positive.
So by \DEF{4.2.6}, \(x + y = (-(ad + bc)) / bd\) is a negative rational number.
\end{proof}

\begin{additional corollary} \label{ac 4.2.5}
Let \(x\) and \(y\) be two rationals.
If \(x\) and \(y\) are positive, then \(xy\) is positive.
If \(x\) and \(y\) are negative, then \(xy\) is positive.
\end{additional corollary}

\begin{proof}
Suppose \(x\) and \(y\) are two positive rationals, we have to show \(xy\) is a positive rational number.
By \DEF{4.2.6} \(x = a / b\) and \(y = c / d\) where \(a, b, c, d\) are positive integers.
By \DEF{4.2.2} \(xy = ac / bd\).
By isomorphism and \LEM{2.3.2}, \(ac, bd\) are both positive, so by \DEF{4.2.6} \(xy = ac / bd\) is a positive rational number.

Now suppose \(x\) and \(y\) are two negative rationals, we have to show \(xy\) is a positive rational number.
By \DEF{4.2.6} \(x = (-a) / b\) and \(y = (-c) / d\) where \(a, b, c, d \) are positive integers.
By \DEF{4.2.2} \(xy = (-a)(-c) / bd\).
By \AC{4.1.5} \((-a)(-c) = ac\), so \(xy = ac / bd\).
Again by isomorphism and \LEM{2.3.2}, \(ac, bd\) are both positive, so by \DEF{4.2.6} \(xy = ac / bd\) is a positive rational number.
\end{proof}

\begin{additional corollary} \label{ac 4.2.6}
Let \(x\) and \(y\) be two rationals.
If \(x\) is negative and \(y\) is positive, then \(xy\) is negative.
If \(x\) is positive and \(y\) is negative, then \(xy\) is negative.
\end{additional corollary}

\begin{proof}
By \PROP{4.2.4}(1) we know that \(xy = yx\), thus we only need to show that if \(x\) is negative and \(y\) is positive, then \(xy\) is negative.
By \DEF{4.2.6} we know that \(x = (-a) / b\) and \(y = c / d\) where \(a, b, c, d\) are positive integers.
By \DEF{4.2.2}, \(xy = ((-a)c) / bd\).
By \AC{4.1.3} and \PROP{4.1.6}(6), \((-a)c = ((-1)a)(c) = (-1)(ac) = (ac)\), so \(xy = (-(ac)) / bd\).
By isomorphism and \LEM{2.3.2}, \(ac, bd\) are both positive, thus by \DEF{4.2.6} \(xy = (-(ac)) / bd\) is a negative rational number.
\end{proof}

\begin{additional corollary} \label{ac 4.2.7}
\(x \X 0 = 0\) for any rational number \(x\).
\end{additional corollary}

\begin{proof}
\(x = a / b\), where \(a, b\) are integers and \(b \neq 0\).
Then
\begin{align*}
    x \X 0 & = (a / b) \X 0 \\
           & = (a / b) \X (0 / 1) & \text{since (integer) \(0 \equiv 0 / 1\)} \\
           & = (a \X 0 / b \X 1) & \text{by \DEF{4.2.2}} \\
           & = 0 / b & \text{by properties of integer} \\
           & = 0 & \text{by \AC{4.2.2}} \\
\end{align*}
\end{proof}

\begin{additional corollary} \label{ac 4.2.8}
\(x\) is a positive rational number if and only if \(x > 0\).
\(x\) is a negative rational number if and only if \(x < 0\).
\end{additional corollary}

\begin{proof}
We first show that \(x\) is a positive rational number if and only if \(x > 0\).
By \DEF{4.2.6} \(x = a / b\) where \(a, b\) are positive integers.
And
\begin{align*}
         & x = a / b \\
    \iff & x - 0 = a / b - 0 \\
    \iff & x - 0 = a / b + (-0) & \text{by \DEF{4.2.13}} \\
    \iff & x - 0 = a / b + ((-1)0) & \text{by \AC{4.2.3}} \\
    \iff & x - 0 = a / b + 0 & \text{by \AC{4.2.7}} \\
    \iff & x - 0 = a / b & \text{by \PROP{4.2.4}(3)}
\end{align*}
\(\iff x - 0\) is a positive rational.
By \DEF{4.2.8}, \(x > 0\).

Now we show that \(x\) is a negative rational number if and only if \(x < 0\).
By \DEF{4.2.6} we know that \(x = (-a) / b\) where \(a, b\) are positive integers.
Thus
\begin{align*}
         & x = (-a) / b \\
    \iff & x - 0 = (-a) / b - 0 \\
    \iff & x - 0 = (-a) / b + (-0) & \text{by \DEF{4.2.13}} \\
    \iff & x - 0 = (-a) / b + ((-1)0) & \text{by \AC{4.2.3}} \\
    \iff & x - 0 = (-a) / b + 0 & \text{by \AC{4.2.7}} \\
    \iff & x - 0 = (-a) / b & \text{by \PROP{4.2.4}(3)}
\end{align*}
\(\iff x - 0\) is a negative rational.
By \DEF{4.2.8}, \(x < 0\).
\end{proof}

\begin{note}
From now on, since \AC{4.2.8}, I may use ``positive'' and \(> 0\) interchangeably, similarly for ``negative'' and \(< 0\).
\end{note}

\begin{proposition} [Basic properties of order on the rationals] \label{prop 4.2.9}
Let \(x, y, z\) be rational numbers.
Then the following properties hold.
\begin{enumerate}
    \item \MAROON{(1)}
        (Order trichotomy) Exactly one of the three statements \(x = y\), \(x < y\), or \(x > y\) is true.
    \item \MAROON{(2)}
        (Order is \emph{anti-symmetric}) One has \(x < y\) if and only if \(y > x\).
    \item \MAROON{(3)}
        (Order is transitive) If \(x < y\) and \(y < z\), then \(x < z\).
    \item \MAROON{(4)}
        (Addition preserves order) If \(x < y\), then \(x + z < y + z\).
    \item \MAROON{(5)}
        (Positive multiplication preserves order) If \(x<y\) and \(z\) is positive, then \(xz < yz\).
\end{enumerate}
\end{proposition}

\begin{proof}
\begin{enumerate}
    \item \MAROON{(1)}
        Given rational numbers \(x, y\), let \(a = x - y\).
        Then by \LEM{4.2.7}, exactly one of \(a = 0\), \(a\) is positive, \(a\) is negative, is true. But by \AC{4.2.8}, \(a\) is positive iff \(a > 0\), and \(a\) is negative iff \(a < 0\).
        So we can conclude exactly one of \(a = 0\), \(a > 0\), \(a < 0\) is true.
        \begin{itemize}
            \item \(a = 0\): then
                \begin{align*}
                         & a = 0 \\
                    \iff & x - y = 0 \\
                    \iff & x - y + y = 0 + y = y \\
                    \iff & x = y
                \end{align*}
                Suppose \(x > y\), then by \DEF{4.2.8}, \(x - y\) is positive, i.e. \(a\) is positive, contradicting \(a = 0\) and \LEM{4.2.7}.
                Suppose \(x < y\), then by \DEF{4.2.8}, \(x - y\) is negative, i.e. \(a\) is negative, contradicting \(a = 0\) and \LEM{4.2.7}.
            \item \(a < 0\): then
                \begin{align*}
                         & a < 0 \\
                    \iff & x - y < 0 \\
                    \iff & x - y + y < 0 + y = y \\
                    \iff & x < y
                \end{align*}
                Suppose \(x = y\) then trivially we can get \(a = 0\), contradicting \(a < 0\) and \LEM{4.2.7}.
                Suppose \(x > y\) then trivially we can get \(a > 0\), contradicting \(a < 0\) and \LEM{4.2.7}.
            \item \(a > 0\): then
                \begin{align*}
                         & a > 0 \\
                    \iff & x - y > 0 \\
                    \iff & x - y + y > 0 + y = y \\
                    \iff & x > y
                \end{align*}
                Similar with the previous case, we will get contradiction if \(a = 0\) or \(a < 0\).
        \end{itemize}
        So in all cases, exactly one of \(x = y, x > y, x < y\) is true.
    \item \MAROON{(2)}
        \begin{align*}
                 & x < y \\
            \iff & x - y = -a \text{\ for some positive rational \(a\)} & \text{by \DEF{4.2.6}, \DEF{4.2.8}} \\
            \iff & (-1)(x - y) = (-1)(-a) \\
            \iff & (-1)(x + (-y)) = (-1)(-a) & \text{by \DEF{4.2.13}} \\
            \iff & (-1)x + (-1)(-y) = (-1)(-a) & \text{by \PROP{4.2.4}(8)} \\
            \iff & (-x) + (-(-y)) = (-(-a)) & \text{by \AC{4.2.3}} \\
            \iff & (-x) + y = a & \text{by \AC{4.2.3}} \\
            \iff & y + (-x) = a & \text{by \PROP{4.2.4}(1)} \\
            \iff & y - x = a & \text{by \DEF{4.2.13}} \\
            \iff & y > x & \text{by \DEF{4.2.8}}
        \end{align*}
    \item \MAROON{(3)}
        \begin{align*}
                     & x < y \land y < z \\
            \implies & y > x \land z > y & \text{by \MAROON{(2)}} \\
            \implies & y - x = a \land z - y = b \land a, b \text{ is positive} & \text{by \DEF{4.2.8}} \\
            \implies & y + (-x) = a \land z + (-y) = b & \text{by \DEF{4.2.13}} \\
            \implies & y + (-x) + z + (-y) = a + b & \text{add both side} \\
            \implies & z + (-x) + y + (-y) = a + b & \text{by \PROP{4.2.4}(1)(2)} \\
            \implies & z + (-x) = a + b & \text{by \PROP{4.2.4}(3)(4)} \\
            \implies & z - x = a + b & \text{by \DEF{4.2.13}} \\
            \implies & z - x \text{\ is positive} & \text{by \AC{4.2.4}} \\
            \implies & z > x & \text{by \DEF{4.2.8}} \\
            \implies & x < z & \text{by \MAROON{(2)}}
        \end{align*}
    \item \MAROON{(4)}
        \begin{align*}
                     & x < y \\
            \implies & x - y = -a \land a \text{ is positive} & \text{by \DEF{4.2.8}} \\
            \implies & x - y + 0 = -a & \text{by \PROP{4.2.4}(3)} \\
            \implies & x - y + z + (-z) = -a & \text{by \PROP{4.2.4}(4)} \\
            \implies & x + (-y) + z + (-z) = -a & \text{by \DEF{4.2.13}} \\
            \implies & x + z + (-y) + (-z) = -a & \text{by \PROP{4.2.4}(1)(2)} \\
            \implies & (x + z) + (-1)(y) + (-1)z = -a & \text{by \AC{4.2.3}} \\
            \implies & (x + z) + (-1)(y + z) = -a & \text{by \PROP{4.2.4}(8)} \\
            \implies & (x + z) + (-(y + z)) = -a & \text{by \AC{4.2.3}} \\
            \implies & (x + z) - (y + z) = -a & \text{by \DEF{4.2.13}} \\
            \implies & x + z < y + z & \text{by \DEF{4.2.8}}
        \end{align*}
    \item \MAROON{(5)}
        \begin{align*}
                     & x < y \land z \text{ is positive} \\
            \implies & x - y = -a, a \text{ is positive} \\
            \implies & (x - y)z = (-a)z \\
            \implies & (x + (-y))z = (-a)z & \text{by \DEF{4.2.13}} \\
            \implies & xz + (-y)z = (-a)z & \text{by \PROP{4.2.4}(9)} \\
            \implies & xz + ((-1)y)z) = ((-1)a)z & \text{by \AC{4.2.3}} \\
            \implies & xz + (-1)(yz) = (-1)az & \text{by \PROP{4.2.4}(6)} \\
            \implies & xz + (-(yz)) = -(az) & \text{by \AC{4.2.3}} \\
            \implies & xz - yz = -(az) & \text{by \DEF{4.2.13}} \\
            \implies & xz < yz & \text{by \DEF{4.2.8}} \\
                     &         & \text{where \(az\) is positive by \AC{4.2.5}}
        \end{align*}
\end{enumerate}
\end{proof}

\begin{remark} \label{remark 4.2.10}
The above five properties in \PROP{4.2.9}, combined with the field axioms in \PROP{4.2.4}, have a name:
they assert that the rationals \(\SET{Q}\) form an \emph{ordered field}.
It is important to keep in mind that \PROP{4.2.9}(e) only works when \(z\) is \emph{positive}, see \EXEC{4.2.6}.
\end{remark}

\begin{definition} [Reciprocal] \label{def 4.2.11}
If \(x = a \D b\) is a \emph{non-zero} rational (so that \(a, b \neq 0\)) then we define the reciprocal \(x^{-1}\) of \(x\) to be the rational number
\[
    x^{-1} := b \D a.
\]
\end{definition}

It is easy to check that this operation is consistent with our notion of equality: if two rational numbers \(a \D b\), \(a' \D b'\) are equal, then their reciprocals are also equal.
\begin{additional corollary} \label{ac 4.2.9}
The reciprocal operation on rational numbers is well-defined:
if two rational numbers \(a // b\), \(a' // b'\) are equal, then their reciprocals are also equal.
We however leave the reciprocal of \(0\) undefined, i.e. when \(a = 0\), we leave the reciprocal of \(a \D b\) undefined.
\end{additional corollary}

\begin{definition} [Quotient] \label{def 4.2.12}
The \emph{quotient} \(x / y\) of two \emph{rational} numbers \(x\) and \(y\), provided that \(y\) is \emph{non-zero}, is defined by the formula
\[
    x / y := x \X y^{-1}.
\]
\end{definition}
Thus, for instance,
\[
    (3 \D 4) / (5 \D 6) = (3 \D 4) \X (6 \D 5) = (18 \D 20) = (9 \D 10).
\]

Using this formula, it is easy to see that \(a / b = a \D b\) for every \textbf{integer} \(a\) and every non-zero \textbf{integer} \(b\).
Thus we can now discard the \(\D\) notation, and use the more customary \(a / b\) instead of \(a \D b\).

\begin{definition} [Subtraction] \label{def 4.2.13}
In a similar spirit, we define \emph{subtraction} on the rationals by the formula
\[
    x - y := x + (-y),
\]
just as we did with the integers.
\end{definition}

\exercisesection

\begin{exercise} \label{exercise 4.2.1}
Show that the definition of equality for the rational numbers is reflexive, symmetric, and transitive.
\end{exercise}
\begin{proof}
See \AC{4.2.1}.
\end{proof}

\begin{exercise} \label{exercise 4.2.2}
Prove the remaining components of \LEM{4.2.3}.
\end{exercise}
\begin{proof}
See \LEM{4.2.3}.
\end{proof}

\begin{exercise} \label{exercise 4.2.3}
Prove the remaining components of \PROP{4.2.4}.
\end{exercise}
\begin{proof}
See \PROP{4.2.4}.
\end{proof}

\begin{exercise} \label{exercise 4.2.4}
Prove \LEM{4.2.7}.
\end{exercise}
\begin{proof}
See \LEM{4.2.7}.
\end{proof}

\begin{exercise}\label{exercise 4.2.5}
Prove \PROP{4.2.9}.
\end{exercise}
\begin{proof}
See \PROP{4.2.9}.
\end{proof}

\begin{exercise}\label{exercise 4.2.6}
Show that if \(x, y, z\) are rational numbers such that \(x < y\) and \(z\) is negative, then \(xz > yz\).
\end{exercise}
\begin{proof}
\begin{align*}
                 & x < y \land z \text{ is negative} \\
        \implies & x < y \land z < 0 & \text{by \AC{4.2.8}} \\
        \implies & x - y = -a \land z - 0 = z = -b \land a, b \text{ are positive} & \text{by \DEF{4.2.8}}\\
        \implies & (x - y)z = (-a)z \\
        \implies & (x - y)z = (-a)(-b) \\
        \implies & (x + (-y))z = (-a)(-b) & \text{by \DEF{4.2.13}} \\
        \implies & xz + (-y)z = (-a)(-b) & \text{by \PROP{4.2.4}(9)} \\
        \implies & xz + ((-1)y)z) = ((-1)a)((-1)b) & \text{by \AC{4.2.3}} \\
        \implies & xz + (-1)(yz) = (-1)(-1)ab & \text{by \PROP{4.2.4}(5)(6)} \\
        \implies & xz + (-(yz)) = ab & \text{by \AC{4.2.3}} \\
        \implies & xz - yz = ab & \text{by \DEF{4.2.13}} \\
        \implies & xz > yz & \text{by \DEF{4.2.8}, where \(ab\) is positive} \\
                 &         & \text{by \AC{4.2.5}}
    \end{align*}
\end{proof}
\chapter{The real numbers} \label{ch 5}

\begin{note}
We defined the natural numbers using the five Peano axioms, and postulated that such a number system existed;
this is plausible, since the natural numbers correspond to the very intuitive and fundamental notion of \emph{sequential counting}.
\end{note}

\begin{note}
The symbols \(\SET{N}\), \(\SET{Q}\), and \(\SET{R}\) stand for ``natural'', ``quotient'', and ``real'' respectively.
\(\SET{Z}\) stands for ``Zahlen'', the German word for numbers.
There is also the \emph{complex numbers} \(\SET{C}\), which obviously stands for ``complex''.
\end{note}

\begin{note}
\emph{Formal} means ``having the form of'';
at the beginning of our construction the expression \(a \M b\) did not actually \emph{mean} the difference \(a - b\), since the symbol \(\M\) was meaningless.
It only had the \emph{form} of a difference.
Later on we defined subtraction and verified that the formal difference was equal to the actual difference, so this eventually became a non-issue, and our symbol for formal differencing was discarded.
Somewhat confusingly, this use of the term ``formal'' is unrelated to the notions of a formal argument and an informal argument.
\end{note}

\begin{note}
There is a fundamental area of mathematics where the rational number system \emph{does not suffice} - that of \emph{geometry} (the study of lengths, areas, etc.).
For instance, a right-angled triangle with both sides equal to \(1\) gives a hypotenuse of \(\sqrt{2}\), which is an \emph{irrational} number, i.e., not a rational number; see \PROP{4.4.4}.
Things get even worse when one starts to deal with the sub-field of geometry known as \emph{trigonometry}, when one sees numbers such as \(\pi\) or \(\cos(1)\), which turn out to be in some sense ``even more'' irrational than \(\sqrt{2}\).
(These numbers are known as \emph{transcendental numbers}, but to discuss this further would be far beyond the scope of this text.)
Thus, in order to have a number system which can adequately describe geometry
- or even something as simple as measuring lengths on a line
- one needs to replace the rational number system with the real number system.
\end{note}

\begin{note}
In the constructions of integers and rationals, the task was to introduce one more \emph{algebraic} operation to the number system
- e.g., one can get integers from naturals by introducing subtraction, and get the rationals from the integers by introducing division.
But to get the reals from the rationals is to pass \emph{from a ``discrete'' system to a ``continuous'' one}, and requires the introduction of a somewhat different notion
- that of a \emph{limit}.
\end{note}

\begin{note}
The limit is a concept which on one level is quite intuitive, but to pin down rigorously turns out to be quite difficult.
(Even such great mathematicians as Euler and Newton had difficulty with this concept.
It was only in the nineteenth century that mathematicians such as Cauchy and Dedekind figured out how to deal with limits rigorously.)
\end{note}

\begin{note}
In \SEC{4.4} we explored the ``gaps'' in the rational numbers;
now we shall fill in these gaps using limits to create the real numbers.
The real number system will end up being a lot like the rational numbers, but will have some new operations
- notably that of \emph{supremum}, which can then be used to define limits and thence to everything else that calculus needs.
\end{note}

\begin{note}
The procedure we give here of obtaining the real numbers as the limit of sequences of rational numbers may seem rather complicated.
However, it is in fact an instance of a very general and useful procedure, that of \emph{\href{https://www.wikiwand.com/en/Complete_metric_space}{completing} one metric space to form another}.
\end{note}

\newcommand{\LIM}{\text{LIM}}
\newcommand{\varE}{\varepsilon}
\newcommand{\toINF}{\to \infty}

\section{Cauchy sequences}

\begin{definition} [Sequences]  \label{def 5.1.1}
Let \(m\) be an integer.
A sequence \((a_n)_{n = m}^{\infty}\) of \emph{rational} numbers is any function from the set \( \{n \in \SET{Z} : n \ge m\} \) to \(\SET{Q}\),
i.e., a mapping which assigns to each integer \(n\) greater than or equal to \(m\), a rational number \(a_n\).
More informally, a sequence \((a_n)_{n = m}^{\infty}\)
of rational numbers is a collection of rationals \(a_m, a_{m + 1}, a_{m + 2},...\)
\end{definition}

\begin{example}
Simple example.
\end{example}

We want to define the \emph{real} numbers as the \emph{``limits''} of sequences of \emph{rational} numbers.
To do so, we have to distinguish which sequences of rationals are \emph{convergent} and which ones are not.
To do this we use the definition of \DEF{4.3.4} \(\varepsilon\)-closeness.

\begin{definition} [\(\varepsilon\)-steadiness] \label{def 5.1.3}
Let \(\varepsilon > 0\).
A sequence \((a_n)_{n = 0}^{\infty}\) is said to be \emph{\(\varepsilon\)-steady} iff each pair \(a_j, a_k\) of sequence elements is \(\varepsilon\)-close \emph{for every} natural number \(j, k\).
In other words, the sequence \(a_0, a_1, a_2,...\) is \(\varepsilon\)-steady iff \(d(a_j, a_k) \le \varepsilon\) for all \(j, k\).
\end{definition}

\begin{remark} \label{remark 5.1.4}
This definition is not standard in the literature;
we will not need it outside of this section;
similarly for the concept of ``eventual \(\varepsilon\)-steadiness'' below.
We have defined \(\varepsilon\)-steadiness for sequences whose index starts at \(0\), but clearly we can make a similar notion for sequences whose indices start from any other number:
a sequence \(a_N, a_{N + 1}, ...\) is \(\varepsilon\)-steady if one has \(d(a_j, a_k) \le \varepsilon\) for all \(j, k \ge N\).
\end{remark}

\begin{example} \label{example 5.1.5}
The sequence \(1, 0, 1, 0, 1,...\) is \(1\)-steady, but is not \(1/2\)-steady.
The sequence \(0.1, 0.01, 0.001, 0.0001,...\) is \(0.1\)-steady, but is not \(0.01\)-steady (why? Because the distance of the first two elements \(d(0.1, 0.01) = 0.09 > 0.01\)).
The sequence 1, 2, 4, 8, 16,... is not \(\varepsilon\)-steady for any \(\varepsilon\) (why? \MAROON{(1)})
The sequence 2, 2, 2, 2,... is \(\varepsilon\)-steady for every \(\varepsilon > 0\).
\end{example}

\begin{proof}
\MAROON{(1)} Given arbitrary \(\varepsilon\), by \PROP{4.4.1} there exists a natural number \(N\) s.t. \(N > \varepsilon\).
And by definition of the sequence, \(a_{N + 1} = 2^{N + 1}\), which \(\ge 2(N + 1)\), since \(N + 1\) must be positive and that satisfies \EXEC{4.3.5}.
And \(2(N + 1) - a_0 = 2N + 2 - a_0 = 2N + 2 - 1 = 2N + 1 > N\), which implies \(2^{N + 1} - a_0 > N\), that is, \(a_{N + 1} - a_0 > N\).
So we can find the pair \(a_{N + 1}, a_1\) s.t. \(d(a_{N + 1}, a_1) > N > \varepsilon\), so by \DEF{5.1.3}, the sequence is not \(\varepsilon\)-steadiness.
Since \(\varepsilon\) is arbitrary, for all \(\varepsilon\) the sequence is not \(\varepsilon\)-steady.
\end{proof}

\begin{note}
The notion of \(\varepsilon\)-steadiness of a sequence is simple, but does not really capture the \emph{limiting} behavior of a sequence, because it is \emph{too sensitive to the initial members} of the sequence.
So we need a more robust notion of steadiness that does not care about the initial members of a sequence.
\end{note}

\begin{definition} [Eventual \(\varepsilon\)-steadiness] \label{def 5.1.6}
Let \(\varepsilon > 0\).
A sequence \((a_n)_{n = 0}^{\infty}\) is said to be \emph{eventually \(\varepsilon\)-steady} iff the sequence \(a_N, a_{N + 1}, a_{N + 2},...\) is \(\varepsilon\)-steady \emph{for some} natural number \(N \ge 0\).
In other words, the sequence \(a_0, a_1, a_2, ...\) is eventually \(\varepsilon\)-steady iff there exists an \(N \ge 0\) such that \(d(a_j, a_k) \le \varepsilon\) for all \(j, k \ge N\).
\end{definition}

\begin{example} \label{example 5.1.7}
\raggedright The sequence \(a_1, a_2, ...\) defined by \(a_n := 1/n\), (that is, the sequence \(1, 1/2, 1/3, 1/4,...\)) is not \(0.1\)-steady, but is eventually \(0.1\)-steady, because the sequence \(a_{10}, a_{11}, a_{12}, ...\) (that is, \(1/10, 1/11, 1/12,...\)) is 0.1-steady.
The sequence \(10, 0, 0, 0, 0, ...\) is not \(\varepsilon\)-steady for any \(\varepsilon\) less than \(10\), but it is eventually \(\varepsilon\)-steady for every \(\varepsilon > 0\) (why? \MAROON{(1)}).
\end{example}

\begin{proof}
\MAROON{(1)} The distance of the first two elements of the sequence is \(d(10, 0) = 10\), so every \(\varepsilon < 10\) is less than \(d(10, 0)\).
But given any \(\varepsilon > 0\), let \(N = 1\), then for every \(j, k \ge N\), \(d(a_i, a_j) = d(0, 0) = 0 < \varepsilon\).
By \DEF{5.1.6}, the sequence is eventually \(\varepsilon\)-steady.
Since \(\varepsilon\) is arbitrary, for all \(\varepsilon > 0\) the sequence is eventual \(\varepsilon\)-steady.
\end{proof}

Now we can finally define the correct notion of what it means for a sequence of rationals to ``want'' to converge.

\begin{definition} [Cauchy sequences] \label{def 5.1.8}
A sequence \((a_n)_{n = 0}^{\infty}\) of \emph{rational} numbers is said to be a \emph{Cauchy sequence} iff for every rational \(\varepsilon > 0\), the sequence \((a_n)_{n = 0}^{\infty}\) is eventually \(\varepsilon\)-steady.
In other words, the sequence \(a_0, a_1, a_2,...\) is a Cauchy sequence iff for every \(\varepsilon > 0\), there exists an \(N \ge 0\) such that \(d(a_j, a_k) \le \varepsilon\) for all \(j, k \ge N\).
\end{definition}

\begin{remark} \label{remark 5.1.9}
At present, the parameter \(\varepsilon\) is restricted to be a positive \emph{rational};
we cannot take \(\varepsilon\) to be an arbitrary positive \emph{real} number, because the real numbers have not yet been constructed.
However, once we do construct the real numbers, we shall see that \DEF{5.1.8} will not change if we require \(\varepsilon\) to be real instead of rational(\PROP{6.1.4}).
In other words, we will eventually prove that
\begin{center}
    (a sequence is eventually \(\varepsilon\)-steady for every \emph{rational} \(\varepsilon > 0\)) if and only if (it is eventually \(\varepsilon'\)-steady for every \emph{real} \(\varepsilon' > 0\)).
\end{center}
(See \PROP{6.1.4}.)
This rather subtle distinction between a rational \(\varepsilon\) and a real \(\varepsilon'\) turns out \emph{not} to be very important in the long run, and the reader is advised not to pay too much attention as to what type of number \(\varepsilon\) should be.
\end{remark}

\begin{example} [\emph{Informal}] \label{example 5.1.10}
Consider the sequence \(1.4, 1.41, 1.414, 1.4142,...\) mentioned earlier.
This sequence is already \(0.1\)-steady.
If one discards the first element \(1.4\), then the remaining sequence \(1.41, 1.414, 1.4142,...\) is now \(0.01\)-steady, which means that the original sequence was eventually \(0.01\)-steady. 
Discarding the next element gives the \(0.001\)-steady sequence \(1.414, 1.4142,...\);
thus the original sequence was eventually \(0.001\)-steady. 
Continuing in this way it seems plausible that this sequence is in fact \(\varepsilon\)-steady for every \(\varepsilon > 0\), which seems to suggest that this is a Cauchy sequence.
However, this discussion is not rigorous for several reasons, for instance we have not precisely defined what this sequence \(1.4, 1.41, 1.414, \textbf{...}\) really is.
An example of a rigorous treatment follows next.
\end{example}

\begin{note}
我不確定課本說我們還沒定義\ \(1.4, 1.41, 1.414, ...\) 的意思是指什麼。也許可能是我們目前有的工具只有數學歸納法? 目前只能用數學歸納法(或遞迴,或者公式,嚴格來說還是遞迴)來定義\ sequence,但\ \(1.4, 1.41, 1.414, ...\) 其實跟數學歸納法無關,所以嚴格來說我們根本沒說清楚他是什麼?
\end{note}

\begin{proposition} \label{prop 5.1.11}
\sloppy The sequence \(a_1, a_2, a_3,...\) defined by \(a_n := 1/n\) (i.e., the sequence \(1, 1/2, 1/3,...\)) is a Cauchy sequence.
\end{proposition}

\begin{note}
Minor: the index is start from \(1\), not from \(0\).
\end{note}

\begin{proof}
We have to show that for every \(\varepsilon > 0\), the sequence \(a_1, a_2,...\) is eventually \(\varepsilon\)-steady.
So let \(\varepsilon > 0\) be \emph{arbitrary}.
We now have to find a number \(N \ge 1\) such that the sequence \(a_N, a_{N + 1},...\) is \(\varepsilon\)-steady.
This means that \(d(a_j, a_k) \le \varepsilon\) for every \(j, k \ge N\), i.e.
\begin{center}
    \(\abs{1 / j - 1 / k} \le \varepsilon\) for every \(j, k \ge N\).
\end{center}
Now since \(j, k \ge N\), we know that
\begin{align*}
             & 0 < 1 / j \le 1 / N \land 0 < 1 / k \le 1 / N \\
    \implies & 0 \le 1 / j \le 1 / N \land 0 \le 1 / k \le 1 / N & \text{in particular by \DEF{4.2.8}} \\
    \implies & 0 \le 1 / j \le 1 / N \land -(1 / N) \le - (1 / k) \le 0 & \text{by \EXEC{4.2.6}} \\
    \implies & -(1 / N) \le 1 / j - 1 / k \le 1 / N & \text{by \PROP{4.2.9}(e)} \\
    \implies & \abs{1 / j - 1 / k} \le 1 / N & \text{by \PROP{4.3.3}(b)}.
\end{align*}
So in order to force \(\abs{1 / j - 1 / k}\) to be less than or equal to \(\varepsilon\), it would be sufficient for \(1/N\) to be less than (or equal to) \(\varepsilon\).
So all we need to do is choose an \(N\) such that \(1 / N\) is less than \(\varepsilon\), or in other words that \(N\) is
greater than \(1/\varepsilon\).
But this can be done thanks to \PROP{4.4.1}.
(\(\varepsilon\) is rational, so \(1/\varepsilon\) is also rational, by \PROP{4.4.1} we can find a natural number \(N > 1/\varepsilon\).)
\end{proof}

\begin{note}
As you can see, verifying from first principles (i.e., without using any of the machinery of limits, etc.) that a sequence is a Cauchy sequence requires some effort, even for a sequence as simple as \(1/n\).
\emph{The part about selecting an \(N\) can be particularly difficult for beginners} - one has to \emph{think in reverse}, working out what conditions on \(N\) would suffice to force the sequence \(a_N, a_{N + 1}, a_{N + 2}, ...\) to be \(\varepsilon\)-steady, and then finding an \(N\) which obeys those conditions.
Later we will develop some limit laws which allow us to determine when a sequence is Cauchy more easily.
\end{note}

\begin{note}
先去想當\ \(N\) 滿足什麼條件時會導致整個\ sequence 是\ \(\varepsilon\)-steady,然後再找出這個\ \(N\)。
\end{note}

We now relate the notion of a Cauchy sequence to another basic notion, that of a \emph{bounded} sequence.

\begin{definition} [Bounded sequences] \label{def 5.1.12}
Let \(M \ge 0\) be rational.
A \emph{finite} sequence \(a_1, a_2, ..., a_n\) is \emph{bounded by} \(M\) iff \(\abs{a_i} \le M\) for all \(1 \le i \le n\).
An infinite sequence \((a_n)_{n = 1}^{\infty}\) is \emph{bounded by} \(M\) iff \(\abs{a_i} \le M\) for all \(i \ge 1\).
A sequence is said to be \emph{bounded} iff it is bounded by \(M\) \emph{for some} rational \(M \ge 0\).
\end{definition}

\begin{example} \label{example 5.1.13}
The finite sequence \(1, -2, 3, -4\) is bounded (in this case, it is bounded by \(4\), or indeed by any \(M\) greater than or equal to 4).
But the infinite sequence \(1, -2, 3, -4, 5, -6,...\) is unbounded. (Can you prove this? Use \PROP{4.4.1}.)
The sequence \(1, -1, 1, -1,...\) is bounded (e.g., by \(1\)), but is \emph{not} a Cauchy sequence(because in particular it's not eventual \(2.1\)-steady).
\end{example}

\begin{proof}
For the sake of contradiction, suppose the infinite sequence \(1, -2, 3, -4, 5, -6,...\) is bounded, i.e. bounded by some rational \(M\).
By \PROP{4.4.1}, we can find a natural number \(N\) s.t. \(N > M\).
WLOG, if we find \(N = 0\) then we redefine \(N := 1\) to make it positive.
Then it's clear that \(\abs{a_N} = N > M\), so by \DEF{5.1.12}, the sequence is not bounded by \(M\), a contradiction.
\end{proof}

\begin{lemma} [Finite sequences are bounded] \label{lem 5.1.14}
Every finite sequence \(a_1, a_2, ..., a_n\) is bounded.
\end{lemma}

\begin{proof}
We prove this by induction on \(n\).
When \(n = 1\) the arbitrary sequence \(a_1\) is clearly bounded, for if we choose \(M := \abs{a_1}\) then clearly we have \(\abs{a_i} \le M\) for all \(1 \le i \le n\).
Now suppose that we have already proved the lemma for some \(n \ge 1\);
that is, any finite sequence with \(n\) elements is bounded;
we now prove it for \(n + 1\), i.e., we prove every sequence with \(n + 1\) elements \(a_1, a_2, ..., a_{n + 1}\) is bounded.
So given arbitrary sequence of \(n + 1\) elements \(a_1, a_2, ..., a_n, a_{n + 1}\).
By the induction hypothesis we know that the first \(n\) elements \(a_1, a_2, ..., a_n\) is bounded by some \(M \le 0\);
in particular, it must be also bounded by \(M + \abs{a_{n + 1}}\) since \(M + \abs{a_{n + 1}} \ge M\).
On the other hand, \(a_{n + 1}\) is also bounded by \(\abs{a_{n + 1}}\), so is also bounded by \(M + \abs{a_{n + 1}}\) because the latter \(\ge\) the former.
Thus \(a_1, a_2, ... , a_n, a_{n + 1}\) is bounded by \(M +  \abs{a_{n + 1}}\), and is hence bounded.
This closes the induction.
\end{proof}

\begin{note}
Note that while this argument shows that every \textbf{finite} sequence is bounded, no matter how long the finite sequence is, it \textbf{does not} say anything about whether an \textbf{infinite} sequence is bounded or not;
infinity is not a natural number.
\end{note}

However, we have
\begin{lemma} [Cauchy sequences are bounded] \label{lem 5.1.15}
Every Cauchy sequence \((a_n)_{n = 1}^{\infty}\) is bounded.
\end{lemma}

\begin{proof}
Let \((a_n)_{n = 1}^{\infty}\) be an arbitrary Cauchy sequence, we have to show it is bounded by some rational \(M \ge 0\).

Now by \DEF{5.1.8}, given any \(\varepsilon > 0\), the sequence is eventually \(\varepsilon\)-steady.
By \DEF{5.1.6}, that means we can find an index \(N \ge 0\) s.t. \(d(a_j, a_k) \le \varepsilon\) for all \(j, k \ge N\) \MAROON{(1)}.
Now we split the sequence as the first \(a_1, ..., a_{N}\) and \(a_{N + 1}, ...\).

By \LEM{5.1.14}, the first part is finite, so it's bounded.

For the second part and \MAROON{(1)}, and in particular for the index \(j\) fixed as \(N + 1\), \(d(a_{N + 1}, a_k) \le \varepsilon\) for all \(k \ge N\) \MAROON{(2)}.
Now let \(M := \abs{a_{N + 1}} + \varepsilon\), we will show that every element \(a_k\) for all \(k \ge N\) is bounded by \(M\), that is, \(\abs{a_k} \le M\).
Note that \(M = \abs{a_{N + 1}} + \abs{\varepsilon}\) \MAROON{(3)} since \(\varepsilon > 0\).
So let \(k\) be arbitrary integer s.t. \(k \ge N\).
Then
\begin{align*}
             & d(a_{N + 1}, a_k) \le \varepsilon & \text{by \MAROON{(2)}} \\
    \implies & \abs{a_{N + 1} - a_k} \le \varepsilon & \text{by \DEF{4.3.2}} \\
    \implies & \abs{a_{N + 1} - a_k} \le \abs{\varepsilon} & \text{since \(\varepsilon > 0\)} \\
    \implies & \abs{a_{N + 1}} + \abs{a_{N + 1} - a_k} \le \abs{a_{N + 1}} + \abs{\varepsilon} & \text{by \PROP{4.2.9}(d)} \\
    \implies & \abs{a_{N + 1}} + \abs{a_{N + 1} - a_k} \le M & \text{by \MAROON{(3)}} \\
    \implies & \abs{-a_{N + 1}} + \abs{a_{N + 1} - a_k} \le M & \text{by \PROP{4.3.3}(d)} \\
    \implies & \abs{(-a_{N + 1}) + (a_{N + 1} - a_k)} \le \abs{-a_{N + 1}} + \abs{a_{N + 1} - a_k} \le M & \text{by \PROP{4.3.3}(b)} \\
    \implies & \abs{(-a_{N + 1}) + (a_{N + 1} - a_k)} \le M & \text{by \PROP{4.2.9}(c), transitive} \\
    \implies & \abs{-a_k} & \text{by algebra}. \\
    \implies & \abs{a_k} \le M & \text{by \PROP{4.3.3}(d)}
\end{align*}
Since \(k\) is arbitrary, \(\abs{a_k} \le M\) for all \(k \ge N\).
So the second part of the sequence is bounded by \(M\) and hence is bounded.

So the whole sequence is bounded.
\end{proof}

\begin{exercise} \label{exercise 5.1.1}
Prove \LEM{5.1.15}.
\end{exercise}

\begin{proof}
See \LEM{5.1.15}.
\end{proof}
\section{Equivalent Cauchy sequence} \label{sec 5.2}

Consider the two Cauchy sequences of rational numbers:
\[
    1.4, 1.41, 1.414, 1.4142, 1.41421,...
\]
and
\[
    1.5, 1.42, 1.415, 1.4143, 1.41422,...
\]
\emph{Informally}, both of these sequences seem to be converging to the same number, the square root
\(\sqrt{2} = 1.41421...\)
(though this statement is \emph{not yet rigorous} because we have not defined real numbers yet).
If we are to define the real numbers from the rationals as \emph{limits} of Cauchy sequences, \emph{we have to know when two Cauchy sequences of rationals give the same limit}, \textbf{without} first defining a real number
(since that would be circular).

\begin{note}
若你希望用柯西序列的極限來定義實數,那柯西序列的相等的定義就不能跟實數有任何關係,這樣會\ circular。
\end{note}

\begin{definition} [\(\varepsilon\)-close sequences] \label{def 5.2.1}
Let \((a_n)_{n = 0}^{\infty}\) and \((b_n)_{n = 0}^{\infty}\) be two sequences,
and let \(\varepsilon > 0\).
We say that the sequence \((a_n)_{n = 0}^{\infty}\) is \emph{\(\varepsilon\)-close} to \((b_n)_{n = 0}^{\infty}\) iff \(a_n\) is \(\varepsilon\)-close to \(b_n\) for each \(n \in \SET{N}\).
In other words, the sequence \(a_0, a_1, a_2,...\) is \(\varepsilon\)-close to the sequence \(b_0, b_1, b_2,...\) 
iff \(\abs{a_n - b_n} \le \varepsilon\) for all \(n = 0, 1, 2,...\).
\end{definition}

\begin{example} \label{example 5.2.2}
The two sequences
\[
    1, -1, 1, -1, 1,...
\]
and
\[
    1.1, -1.1, 1.1, -1.1, 1.1,...
\]
are \(0.1\)-close to each other.
(Note however that neither of them are \(0.1\)-steady).
\end{example}

\begin{definition} [Eventually \(\varepsilon\)-close sequences] \label{def 5.2.3}
Let \((a_n)_{n = 0}^{\infty}\) and \((b_n)_{n = 0}^{\infty}\) be two sequences, and let \(\varepsilon > 0\).
We say that the sequence \((a_n)_{n = 0}^{\infty}\) is \emph{eventually \(\varepsilon\)-close to} \((b_n)_{n = 0}^{\infty}\)
iff there exists an \(N \ge 0\) such that the sequences \((a_n)_{n = N}^{\infty}\) and \((b_n)_{n = N}^{\infty}\) are \(\varepsilon\)-close.
In other words, \(a_0, a_1, a_2,...\) is eventually \(\varepsilon\)-close to \(b_0, b_1, b_2,...\)
iff there exists an \(N \ge 0\) such that \(\abs{a_n - b_n} \le \varepsilon\) for all \(n \ge N\).
\end{definition}

\begin{remark} \label{remark 5.2.4}
Again, the notations for \(\varepsilon\)-close sequences and eventually \(\varepsilon\)-close sequences are not standard in the literature, and we will not use them outside of this section.
\end{remark}

\begin{example} \label{example 5.2.5}
Simple example that informally shows
\[
    1.1, 1.01, 1.001, 1.0001,...
\]
and
\[
    0.9, 0.99, 0.999, 0.9999,...
\]
are equivalent(see definition below).
\end{example}

\begin{definition} [Equivalent sequences] \label{def 5.2.6}
Two sequences \((a_n)_{n = 0}^{\infty}\) and \((b_n)_{n = 0}^{\infty}\) are \emph{equivalent}
iff for each \emph{rational} \(\varepsilon > 0\), the sequences \((a_n)_{n = 0}^{\infty}\) and \((b_n)_{n = 0}^{\infty}\) are eventually \(\varepsilon\)-close.
In other words, \(a_0, a_1, a_2,...\) and \(b_0, b_1, b_2,...\) are equivalent
iff for every rational \(\varepsilon > 0\), there exists an \(N \ge 0\) such that \(\abs{a_n - b_n} \le \varepsilon\) for all \(n \ge N\).
\end{definition}

\begin{remark} \label{remark 5.2.7}
As with \DEF{5.1.8}, the quantity \(\varepsilon > 0\) is currently restricted to be a positive \emph{rational}, rather than a positive \emph{real}.
However, we shall eventually see that it makes no difference whether \(\varepsilon\) ranges over the positive rationals or positive reals;
see \EXEC{6.1.10}.
\end{remark}

We now prove the sequences in \EXAMPLE{5.2.5} are equivalent rigorously.

\begin{proposition} \label{prop 5.2.8}.
Let \((a_n)_{n = 1}^{\infty}\) and \((b_n)_{n = 1}^{\infty}\) be the sequences
\(a_n := 1 + 10^{-n}\) and \(b_n = 1 - 10^{-n}\).
Then the sequences \(a_n, b_n\) are equivalent.
\end{proposition}

\begin{remark} \label{remark 5.2.9}
This Proposition, in \emph{decimal notation}, asserts that \(1.0000... = 0.9999...\); see \PROP{B.2.3}.
\end{remark}

\begin{proof}
We need to prove that for every \(\varepsilon > 0\), the two sequences \((a_n)_{n = 1}^{\infty}\) and \((b_n)_{n = 1}^{\infty}\) are eventually \(\varepsilon\)-close to each other.
So we let \(\varepsilon\) be an arbitrary rational such that \(\varepsilon > 0\).
We need to find an integer \(N \ge 1\) (where \(1\) is the starting index of the two sequences) such that \((a_n)_{n = N}^{\infty}\) and \((b_n)_{n = N}^{\infty}\) are \(\varepsilon\)-close;
in other words, we need to find an \(N \ge 1\) such that
\[
    \abs{a_n - b_n} \le \varepsilon \text{ for all } n \ge N.
\]
However, we have
\[
    \abs{a_n - b_n} = \abs{(1 + 10^{-n}) - (1 - 10^{-n})} = 2 \X 10^{-n}.
\]
Since \(10^{-n}\) is a \emph{decreasing} function of \(n\) (i.e., \(10^{-m} < 10^{-n}\) whenever \(m > n\);
this is easily proven by induction),
and since \(n \ge N\), we have \(2 \X 10^{-n} \le 2 \X 10^{-N}\).
Thus we have
\[
    \abs{a_n - b_n} \le 2 \X 10^{-N} \text{ for all } n \ge N.
\]
Thus in order to obtain \(\abs{a_n - b_n} \le \varepsilon\) for all \(n \ge N\), it will be sufficient to choose \(N\) so that \(2 \X 10^{-N} \le \varepsilon\).
This is easy to do using logarithms, but we have not yet developed logarithms yet, so we will use a cruder
method.
First, we observe \(10^N\) is always greater than \(N\) for any \(N \ge 1\) (see \EXEC{4.3.5}).
Thus \(10^{-N} \le 1/N\), and \(2 \X 10^{-N} \le 2/N\).

Thus to get \(2 \X 10^{-N} \le \varepsilon\), it will suffice to choose \(N\) so that \(2/N \le \varepsilon\), or equivalently that \(N \ge 2/\varepsilon\).
But by \PROP{4.4.1} since \(2/\varepsilon\) is rational, we can find \(N\) s.t. \(N > 2/\varepsilon\) and in particular \(N \ge 2/\varepsilon\).
So the two sequences are eventually \(\varepsilon\)-close.
Since \(\varepsilon\) are arbitrary, for all \(\varepsilon > 0\) the two sequences are eventually \(\varepsilon\)-close.
By \DEF{5.2.6}, the two sequences are equivalent.
\end{proof}


\exercisesection

\begin{exercise} \label{exercise 5.2.1}
Show that if \((a_n)_{n = 1}^{\infty}\) and \((b_n)_{n = 1}^{\infty}\) are equivalent sequences of rationals,
then \((a_n)_{n = 1}^{\infty}\) is a Cauchy sequence if and only if \((b_n)_{n = 1}^{\infty}\) is a Cauchy sequence.
\end{exercise}
\begin{proof}
Suppose \((a_n)_{n = 1}^{\infty}\) and \((b_n)_{n = 1}^{\infty}\) are equivalent sequences of rationals.
\begin{itemize}
    \item[\(\Longrightarrow\)]:
        Suppose \((a_n)_{n = 1}^{\infty}\) is Cauchy, we have to show \((b_n)_{n = 1}^{\infty}\) is Cauchy.
        So, let \(\varepsilon\) be arbitrary rational s.t. \(\varepsilon > 0\), we have to show \((b_n)_{n = 1}^{\infty}\) is eventual \(\varepsilon\)-steady.
        That is, we have to find a integer \(N \ge 1\), which is the starting index of the two sequences, s.t. \(d(b_j, b_k) \le \varepsilon\) for all \(j, k \ge N\).
        
        But by supposition \((a_n)_{n = 1}^{\infty}\) is Cauchy, so we can find a integer \(N_1 \ge 1\) s.t. \(d(a_j, a_k) \le \varepsilon\) for all \(j, k \ge N_1\) \GREEN{(1)}.
        And also since the two sequences are equivalent, we can find a integer \(N_2 \ge 1\) s.t. \(d(a_i, b_i) \le \varepsilon\) for all \(i \ge N_2\) \GREEN{(2)}.
        
        Now let \(N_3 := max(N_1, N_2)\), for all \(i, j \ge N_3\), by \GREEN{(1)} we have \(d(a_i, a_j) \le \varepsilon\) \MAROON{(1)};
        and by \GREEN{(2)} we have \(d(a_i, b_i) \le \varepsilon\) \MAROON{(2)} and \(d(a_j, b_j) \le \varepsilon)\) \MAROON{(3)}.
        And by \PROP{4.3.7}(f) and \MAROON{(2)}, we have \(d(a_i, b_i) = d(b_i, a_i)\) \MAROON{(4)}.
        So by adding both sides of the three inequalities \MAROON{(1) (3) (4)}, we have \(d(b_i, a_i) + d(a_i, a_j) + d(a_j, b_j) \le 3\varepsilon\).
        But by applying \PROP{4.3.7}(g) twice, we have \(d(b_i, b_j) \le d(b_i, a_i) + d(a_i, a_j) + d(a_j, b_j)\).
        So by transitivity, we have \(d(b_i, b_j) \le 3\varepsilon\) for all \(i, j \le N_3\).
        
        (I do this with purpose and first conclude that the sequence \(b_n\) is \(3\varepsilon\)-steady, because finding \(N\) for \(\varepsilon/3\) in the \emph{beginning} of the proof is somewhat anti-human.)
        
        Now let \(\varepsilon' = \varepsilon/3\).
        Then with the same argument above, we can conclude that we can find \(N_4 \ge 1\) s.t. \(d(b_i, b_j) \le 3\varepsilon'\) for all \(i, j \le N_4\).
        But \(3\varepsilon' = 3(\varepsilon/3) = \varepsilon\), so we have \(d(b_i, b_j) \le \varepsilon\) for all \(i, j \le N_4\).
        So by \DEF{5.1.6}, \((b_n)_{n = 1}^{\infty}\) is \(\varepsilon\)-steady.
        
        Since \(\varepsilon > 0\) is arbitrary, \((b_n)_{n = 1}^{\infty}\) is eventual \(\varepsilon\)-steady for all \(\varepsilon > 0\).
        By \DEF{5.1.8}, \((b_n)_{n = 1}^{\infty}\) is Cauchy!
    \item[\(\Longleftarrow\)]:
        The proof is similar with previous case, with \(a_{blabla}\) and \(b_{blabla}\) swapped.
\end{itemize}
\end{proof}

\begin{exercise} \label{exercise 5.2.2}
Let \(\varepsilon > 0\).
Show that if \((a_n)_{n = 1}^{\infty}\) and \((b_n)_{n = 1}^{\infty}\) are eventually \(\varepsilon\)-close,
then \((a_n)_{n = 1}^{\infty}\) is bounded if and only if \((b_n)_{n = 1}^{\infty}\) is bounded.
\end{exercise}
\begin{proof}
Let \(\varepsilon > 0\) and suppose \((a_n)_{n = 1}^{\infty}\) and \((b_n)_{n = 1}^{\infty}\) are eventually \(\varepsilon\)-close.
\begin{itemize}
    \item[\(\Longrightarrow\)]
        Suppose \((a_n)_{n = 1}^{\infty}\) is bounded, we have to show \((b_n)_{n = 1}^{\infty}\) is also bounded.

        Since \((a_n)_{n = 1}^{\infty}\) is bounded, we can find a rational \(M_1 \ge 0\) s.t. \(\abs{a_i} \le M_1\) for all \(i \ge 1\). And since the two sequence are eventually \(\varepsilon\)-close, we can find an integer \(N_1 \ge 1\) s.t. \(d(a_i, b_i) \le \varepsilon\) for all \(i \ge N_1\).
        
        Now again like the proof in \LEM{5.1.15}, we split \((b_n)_{n = 1}^{\infty}\) into \((b_n)_{n = 1}^{N_1}\) and \((b_n)_{N_1 + 1}^{\infty}\).
        By \LEM{5.1.14} the first finite part is bounded.
        
        Now let \(M_2 := M_1 + \varepsilon\), we will show that the second part of the sequence, \((b_n)_{n = N_1 + 1}^{\infty}\), is bounded by \(M_2\) and hence is bounded.
        So given arbitrary index \(i \ge N_1 + 1\):
        \begin{align*}
                     & d(a_i, b_i) \le \varepsilon & \text{since the two seq. are even. \(\varepsilon\)-close} \\
            \implies & d(b_i, a_i) \le \varepsilon & \text{by \PROP{4.3.3}(f)} \\
            \implies & \abs{b_i - a_i} \le \varepsilon & \text{by \DEF{4.3.2}} \\
            \implies & M_1 + \abs{b_i - a_i} \le M_1 + \varepsilon & \text{by \PROP{4.2.9}(d)} \\
            \implies & M_1 + \abs{b_i - a_i} \le M_2 & \text{since \(M_2 = M_1 + \varepsilon\)}\\
            \implies & \abs{a_i} + \abs{b_i - a_i} \le M_1 + \abs{b_i - a_i} \le M_2 & \text{since \(\abs{a_i} \le M_1\)} \\
            \implies & \abs{a_i} + \abs{b_i - a_i} \le M_2 & \text{transitive} \\
            \implies & \abs{a_i + (b_i - a_i)} \le \abs{a_i} + \abs{b_i - a_i} \le M_2 & \text{by \PROP{4.3.3}(b)} \\
            \implies & \abs{a_i + (b_i - a_i)} \le M_2 & \text{transitive}  \\
            \implies & \abs{b_i} \le M_2 & \text{trivial} \\
        \end{align*}
        Since \(i \ge N_1 + 1\) is arbitrary, so for all \(i \ge N_1 + 1\), \(\abs{b_i} \le M_2\).
        By \DEF{5.1.12}, \((b_n)_{n = N_1 + 1}^{\infty}\) is bounded by \(M_2\) and hence is bounded.

        So the whole sequence is bounded.
    \item[\(\Longleftarrow\)]:
        The proof is similar with previous case, with \(a_{blabla}\) and \(b_{blabla}\) swapped.
\end{itemize}
\end{proof}
\section{The construction of the real numbers} \label{sec 5.3}

\begin{definition} [Real numbers] \label{def 5.3.1}
A \emph{real number} is defined to be an \emph{object} of the \emph{form} \(\LIM_{n \toINF} a_n\),
where \((a_n)_{n = 1}^{\infty}\) is a Cauchy sequence of \emph{rational} numbers.
Two real numbers \(\LIM_{n \toINF} a_n\) and \(\LIM_{n \toINF} b_n\) are said to be \emph{equal} iff \((a_n)_{n = 1}^{\infty}\) and \((b_n)_{n = 1}^{\infty}\) are equivalent Cauchy sequences.
The set of all real numbers is denoted \(\SET{R}\).
\end{definition}

\begin{example} [Informal] \label{example 5.3.2}
Let \(a_1, a_2, a_3, ...\) denote the sequence
\[
    1.4, 1.41, 1.414, 1.4142, 1.41421,...
\]
and let \(b_1, b_2, b_3,...\) denote the sequence
\[
    1.5, 1.42, 1.415, 1.4143, 1.41422,...
\]
then \(\LIM_{n \toINF} a_n\) is a real number, and is \emph{the same} real number as \(\LIM_{n \toINF} b_n\),
because \((a_n)_{n = 1}^{\infty}\) and \((b_n)_{n = 1}^{\infty}\) are equivalent Cauchy sequences:
\(\LIM_{n \toINF} a_n = \LIM_{n \toINF} b_n\).
\end{example}
\begin{note}
到目前為止,我們還沒有證明:
\begin{itemize}
    \item 這兩個\ sequence 都是\ Cauchy sequences。
    \item 進一步來說,也無法知道他們對應的\ \LIM\ object 是\ real number。
    \item 所以也無法知道討論他們是否相等。
\end{itemize}
而且開天眼,這整節看完之後我們好像還是不能知道他們是\ Cauchy...。
\end{note}

\begin{note}
We will refer to \(\LIM_{n \toINF} a_n\) as the \emph{formal limit} of the sequence \((a_n)_{n = 1}^{\infty}\).
Later on we will define a genuine notion of limit, and show that the formal limit of a Cauchy sequence is the same as the limit of that sequence;
after that, we will not need formal limits ever again.
(The situation is much like what we did with \emph{formal} subtraction \(\M\) and \emph{formal} division \(\D\).)
\end{note}

\begin{note}
we need to check that the notion of equality in the definition obeys the first three laws of
equality:
\end{note}

\begin{proposition} [Formal limits are well-defined] \label{prop 5.3.3}
Let \(x = \LIM_{n \toINF} a_n,\ y = \LIM_{n \toINF} b_n\), and \(z = \LIM_{n \toINF} c_n\) be real numbers.
(This implies the corresponding sequences are Cauchy.)
Then, with the above \DEF{5.3.1} of equality for real numbers, we have \(x = x\).
Also, if \(x = y\), then \(y = x\).
Finally, if \(x = y\) and \(y = z\), then \(x = z\).
\end{proposition}

\begin{proof}
Reflexive: Since given \(\varE > 0\), \(\varE > 0 = d(a_n, a_n)\) for all \(n \ge 1\), by \DEF{5.2.6}, \((a_n)_{n = 1}^{\infty}\) and \((a_n)_{n = 1}^{\infty}\) are equivalent.
And by \DEF{5.3.1}, \(\LIM_{n \toINF} a_n = \LIM_{n \toINF} a_n\), that is, \(x = x\).

Symmetric: Suppose \(x = y\).
By \DEF{5.3.1}, \((a_n)_{n = 1}^{\infty}\) and \((b_n)_{n = 1}^{\infty}\) are equivalent.
And by \DEF{5.2.6}, For all \(\varE > 0\), there exists \(N \ge 1\) s.t. for all \(n \ge N\), \(d(a_n, b_n) \le \varE\) \MAROON{(1)}.
But by \PROP{4.3.3}(f), \(d(a_n, b_n) = d(b_n, a_n)\).
So with \MAROON{(1)}, we have For all \(\varE > 0\), there exists \(N \ge 1\) s.t. for all \(n \ge N\), \(\MAROON{d(b_n, a_n)} \le \varE\).
By \DEF{5.2.6}, \((b_n)_{n = 1}^{\infty}\) and \((a_n)_{n = 1}^{\infty}\) are equivalent.
By \DEF{5.3.1}, \(\LIM_{n \toINF} b_n = \LIM_{n \toINF} b_n\), that is, \(y = x\).

Transitive: Suppose \(x = y\) and \(y = z\).
By \DEF{5.3.1}, \((a_n)_{n = 1}^{\infty}, (b_n)_{n = 1}^{\infty}\) are equivalent and \((b_n)_{n = 1}^{\infty}, (c_n)_{n = 1}^{\infty}\) are equivalent.
Now given arbitrary \(\varE > 0 \), so \emph{in particular \(\varE/2 > 0\)}, we have
\begin{itemize}
    \item There exists \(N_1 \ge 1\) s.t. for all \(n \ge N_1\), \(d(a_n, b_n) \le \varE/2\) \MAROON{(2)}.
    \item There exists \(N_2 \ge 1\) s.t. for all \(n \ge N_2\), \(d(b_n, c_n) \le \varE/2\) \MAROON{(3)}.
\end{itemize}
Let \(N_3 := max(N_1, N_2)\), then by \MAROON{(2) (3)}, we have for all \(n \ge N_3\), \(d(a_n, b_n) \le \varE/2\) and \(d(b_n, c_n) \le \varE/2\).
But by \DEF{4.3.4}, that implies \(a_n, b_n\) and \(b_n, c_n\) are \(\varE/2\)-close.
By \PROP{4.3.7}, we have \(a_n, c_n\) are \(\varE/2 + \varE/2 = \varE\)-close.
By \DEF{4.3.4}, \(d(a_n, c_n) \le \varE\).
So for all \(n \ge N_3\), \(d(a_n, c_n) \le \varE\).
By \DEF{5.2.6}, \((a_n)_{n = 1}^{\infty}, (c_n)_{n = 1}^{\infty}\) are equivalent.
By \DEF{5.3.1}, \(\LIM_{n \toINF} a_n = \LIM_{n \toINF} c_n\), that is, \(x = z\).
\end{proof}

\begin{note}
By \PROP{5.3.3}, we now have well-defined equality between two real numbers.
Of course, when we define other operations on the reals, \emph{we have to check that they obey the law of substitution} \AXM{a.7.4}:
two real number inputs which are \emph{equal} should give equal outputs when applied to any operation on the real numbers.

Now we want to define all the usual arithmetic operations on the real numbers, such as addition and multiplication. 
\end{note}

\begin{definition} [Addition of reals] \label{def 5.3.4} 
Let \(x = \LIM_{n \toINF} a_n\) and \(y = \LIM_{n \toINF} b_n\) be real numbers.
Then we \emph{define} the \emph{sum} \(x + y\) to be \(x + y := \LIM_{n \toINF} (a_n + b_n)\).
\end{definition}

\begin{example} \label{example 5.3.5}
The sum of \(\LIM_{n \toINF} 1 + 1/n\) and \(\LIM_{n \toINF} 2 + 3 / n\) is \(\LIM_{n \toINF} 3 + 4/n\).
\end{example}

We now check that \DEF{5.3.4} is valid.
The first thing we need to do is to confirm that the sum of two real numbers is in fact a real number:

\begin{lemma} [Sum of Cauchy sequences is Cauchy] \label{lem 5.3.6}
Let \(x = \LIM_{n \toINF} a_n\) and \(y = \LIM_{n \toINF} b_n\) be real numbers.
Then \(x + y\) is also a real number. (i.e., \((a_n + b_n)_{n = 1}^{\infty}\) is a Cauchy sequence of rationals.)
\end{lemma}


\begin{proof}
We need to show that for every \(\varE > 0\), the sequence \((a_n + b_n)_{n = 1}^{\infty}\) is eventually \(\varE\)-steady.
Now from hypothesis and \DEF{5.3.1} we know that \((a_n)_{n = 1}^{\infty}\) is Cauchy and hence by \DEF{5.1.8} is eventually \(\varE\)-steady, and similarly \((b_n)_{n = 1}^{\infty}\) is eventually \(\varE\)-steady;
but it turns out that this is not quite enough
(this can be used to imply that \((a_n + b_n)_{n = 1}^{\infty}\) is eventually \(2\varE\)-steady, but that's not what we want).
So we need to do a little trick, which is to play with the value of \(\varE\).

We know that \((a_n)_{n = 1}^{\infty}\) is eventually \(\delta\)-steady for every value of \(\delta > 0\).
This implies not only that \((a_n)_{n = 1}^{\infty}\) is eventually \(\varE\)-steady, but it is also eventually \(\varE / 2\)-steady(just in particular let \(\delta = \varE/2\)).
Similarly, the sequence \((b_n)_{n = 1}^{\infty}\) is also eventually \(\varE / 2\)-steady.
This will turn out to be enough to conclude that \((a_n + b_n)_{n = 1}^{\infty}\) is eventually \(\varE\)-steady.

Since \((a_n)_{n = 1}^{\infty}\) is eventually \(\varE / 2\)-steady, we know that there exists an \(N \geq 1\) such that \((a_n)_{n = N}^{\infty}\) is \(\varE / 2\)-steady, i.e., \(a_n\) and \(a_m\) are \(\varE / 2\)-close for every \(n, m \geq N\).
Similarly there exists an \(M \geq 1\) such that \((b_n)_{n = M}^{\infty}\) is \(\varE / 2\)-steady, i.e., \(b_n\) and \(b_m\) are \(\varE / 2\)-close for every \(n, m \geq M\).

Let \(\max(N, M)\) be the larger of \(N\) and \(M\)
(we know from \PROP{2.2.13} that one has to be greater than or equal to the other).
If \(n, m \geq \max(N, M)\), then we know that \(a_n\) and \(a_m\) are \(\varE / 2\)-close, and \(b_n\) and \(b_m\) are \(\varE / 2\)-close, and so by \PROP{4.3.7}(d) we see that \(a_n + b_n\) and \(a_m + b_m\) are \(\varE/2 + \varE/2 = \varE\)-close for every \(n, m \geq \max(N, M)\).
This implies that the sequence \((a_n + b_n)_{n = 1}^{\infty}\) is eventually \(\varE\)-steady, as desired.
\end{proof}

The other thing we need to check is the axiom of substitution \AXM{a.7.4}:
if we replace a real number \(x\) by another number \(x'\) \emph{equal} to \(x\),  then we must have \(x' + y = x + y\)
(and similarly if we substitute \(y\) by another number \(y'\) equal to \(y\)).

\begin{lemma} [Sums of equivalent Cauchy sequences are equivalent]\label{lem 5.3.7}.
Let \(x = \LIM_{n \toINF} a_n, y = \LIM_{n \toINF} b_n\), and \(x' = \LIM_{n \toINF} a'_n\) be real numbers.
Suppose that \(x = x'\).
Then we have \(x + y = x' + y\).
\end{lemma}

\begin{proof}
We need to show \(x + y = x' + y\), 
that is by \DEF{5.3.4} \(\LIM_{n \toINF} (a_n + b_n) = \LIM_{n \toINF} (a'_n + b_n)\), 
that is by \DEF{5.3.1}, the sequences \((a_n + b_n)_{n = 1}^{\infty}\) and \((a'_n + b_n)_{n = 1}^{\infty}\) are eventually \(\varE\)-close for each \(\varE > 0\) \MAROON{(1)}. 
But since \(x = x'\), we know that the Cauchy sequences \((a_n)_{n = 1}^{\infty}\) and \((a'_n)_{n = 1}^{\infty}\) are equivalent, by similar definition derivation we know that there is an \(N \ge 1\) such that \((a_n)_{n = N}^{\infty}\) and \((a'_n)_{n = N}^{\infty}\) are \(\varE\)-close,
i.e., that \(a_n\) and \(a'_n\) are \(\varE\)-close for each \(n \ge N\).
Since \(b_n\) is of course \(0\)-close to \(b_n\)
(where we extend the notion of \(\varE\)-closeness(The author seems to mean \DEF{4.3.4}, \(\varE\)-closeness for two numbers) to include \(\varE=0\) in the obvious fashion),
we thus see from \PROP{4.3.7}(d)
(extended in the obvious manner to the \(\delta =  0\) case, or extended to cover the \(0\)-close case)
that \(a_n + b_n\) and \(a'_n + b_n\) are \(\varE + 0 = \varE\)-close for each \(n \ge N\).
This implies \MAROON{(1)} is true, and we are done.
\end{proof}

\begin{remark} \label{remark 5.3.8}
The above lemma verifies the axiom of substitution \AXM{a.7.4} for the ``\(x\)'' variable in \(x + y\),
but one can similarly prove the \AXM{a.7.4} for the ``\(y\)'' variable.
(A quick way is to observe from the definition of \(x + y\) that we certainly have \(x + y = y + x\), since \(a_n + b_n\) = \(b_n + a_n\).)
\end{remark}

We can define multiplication of real numbers in a manner similar to that of addition:

\begin{definition} [Multiplication of reals] \label{def 5.3.9}
Let \(x = \LIM_{n \toINF} a_n\) and \(y = \LIM_{n \toINF} b_n\) be real numbers.
Then we \emph{define} the product \(xy\) to be \(xy :=  \LIM_{n \toINF} a_n b_n\).
\end{definition}

\begin{proposition} [Multiplication is well defined] \label{prop 5.3.10}
Let \(x = \LIM_{n \toINF} a_n, y = \LIM_{n \toINF} b_n\), and \(x' = \LIM_{n \toINF} a'_n\) be real numbers.
Then \(xy\) is also a real number.
Furthermore, if \(x = x'\), then \(xy = x'y\). 
\end{proposition}

\begin{proof}
\begin{itemize}
    \item
        First we show \(xy\) is real;
        that is, \((a_n b_n)_{n = 1}^{\infty}\) is Cauchy;
        that is, for all \(\varE > 0\), there exists \(N \ge 1\) s.t. \(d(a_i b_i, a_j b_j) \le \varE\) for all \(i, j \ge N\).
        
        So let \(3 > \varE > 0\). \BLUE{(trick 1)}
        (The purpose of the upper bound is to make some trick in the latter proof).
        It's trivial that if \((a_n b_n)_{n = 1}^{\infty}\) is eventually \(\varE\)-close, then given any \(\delta \ge 3\), \((a_n b_n)_{n = 1}^{\infty}\) is also \(\delta\)-close.
        (By \PROP{4.3.7}(e) and the definition of closeness of the sequences and rational numbers.)
        So together we can conclude \((a_n b_n)_{n = 1}^{\infty}\) is eventually \(\varE\)-close for all \(\varE > 0\).
        
        Since \((a_n)_{n = 1}^{\infty}\) and \((b_n)_{n = 1}^{\infty}\) are Cauchy, by \LEM{5.1.15} they are bounded by some rational \(M_1, M_2 \ge 0\),
        that is, \(\abs{a_n} \le M_1\) and \(\abs{b_n} \le M_2\) for all \(n \ge 1\).
        Now let \(M = max(M_1, M_2, 1)\), and that trivially implies \(\abs{a_n} \le M\) \MAROON{(1)} and \(\abs{b_n} \le M\) \MAROON{(2)} for all \(n \ge 1\).
        (Note that \(M \ge 1\), this is also used as a trick in the latter proof \BLUE{(trick 2)}).
        
        And since \((a_n)_{n = 1}^{\infty}\) and \((b_n)_{n = 1}^{\infty}\) are Cauchy, there exists \(N_1 \ge 1\) s.t. \(d(a_i , a_j) \le \varE\) for all \(i, j \ge N_1\),
        and there exists \(N_2 \ge 1\) s.t. \(d(b_i , b_j) \le \varE\) for all \(i, j \ge N_2\).
        Let \(N = max(N_1, N_2)\), then \(d(a_i , a_j) \le \varE\) and \(d(b_i, b_j)\le \varE\) for all \(i, j \ge N\).
        And by \PROP{4.3.7}(h), we have \(d(a_i b_i, a_j b_j) \le \varE\abs{b_i} + \varE\abs{a_i} + \varE^2\).
        And by \MAROON{(1) (2)}
        \begin{align*}
                & \varE\abs{b_i} + \varE\abs{a_i} + \varE^2 \\
            \le &  \varE M + \varE M + \varE^2 \\
              = & \varE(2M + \varE),
        \end{align*}
        so we have \(d(a_i b_i, a_j b_j) \le \varE(2M + \varE)\) for all \(i, j \ge N\) \MAROON{(3)}.
        
        \sloppy Now let \(\varE' = \frac{\varE}{3M}\), then by similar argument until \MAROON{(3)},
        there exists \(N_3 \ge 1\) s.t. for all \(i, j \ge N_3\).
        \begin{align*}
            d(a_i b_i, a_j b_j) & \le \varE'(2M + \varE') \\
                                & = \frac{\varE}{3M}(2M + \frac{\varE}{3M}) \\
                                & = \frac{2\varE}{3} + \frac{\varE^2}{9M^2} \\
                                & \le \frac{2\varE}{3} + \frac{\varE^2}{9\X1^2} & \text{since \(M \ge 1\), trick \BLUE{(2)}} \\
                                & = \frac{2\varE}{3} + \frac{\varE^2}{9} \\
                                & = \frac{2\varE}{3} + \varE \X \frac{\varE}{9} \\
                                & \le \frac{2\varE}{3} + 3 \X \frac{\varE}{9} & \text{since \(\varE < 3\), trick \BLUE{(1)}} \\
                                & = \frac{2\varE}{3} + \frac{\varE}{3} \\
                                & = \varE
        \end{align*}
        By \DEF{5.1.6}, \((a_n b_n)_{n = 1}^{\infty}\) is eventually \(\varE\)-close.
        Since \(\varE\) is arbitrary between \(0\) and \(3\), \((a_n b_n)_{n = 1}^{\infty}\) is eventually \(\varE\)-close for all \(0 < \varE < 3\),
        and by the previous discussion it's trivial that \((a_n b_n)_{n = 1}^{\infty}\) is eventually \(\varE\)-close for all \(\varE > 0\).
        So by \DEF{5.1.8}, \((a_n b_n)_{n = 1}^{\infty}\) is Cauchy.
    \item
        Now we prove if \(x = x'\), then \(xy = x'y\).
        Since \(x = x'\), \((a_n)_{n = 1}^\infty\) and \((a'_n)_{n = 1}^\infty\) are equivalent sequences.
        Since \((b_n)_{n = 1}^\infty\) is Cauchy, there is some number \(M' \ge 0\) which bounds it.
        Now let \(M = max(M', 1)\), then it's trivial that \(M\) also bounds \((b_n)_{n = 1}^\infty\) and \(M\) is positive.

        Now let \(\varE > 0\), so in particular \(\varE/M > 0\).
        Then since \((a_n)_{n=1}^\infty\) and \((a'_n)_{n=1}^\infty\) are equivalent sequences, they are eventually \(\varE/M\)-close.
        So there exists \(N \ge 1\) s.t. \(a_i, a'_i\) are \(\varE/M\)-close for all \(i \ge N\).
        And by \PROP{4.3.7}(g), \(a_i b_i, a'_i b_i\) are \((\varE/M)\abs{b_i}\)-close for all \(i \ge N\).
        But since \(M\) bounds \((b_n)_{n = 1}^\infty\), we have \(\abs{b_i} \le M\), so \((\varE/M)\abs{b_i} \le (\varE/M)M = \varE\).
        So \(a_i b_i, a'_i b_i\) are \(\varE\)-close for all \(i \ge N\).
        So \((a_n b_n)_{n = 1}^\infty\) and \((a'_n b_n)_{n = 1}^\infty\) is eventually \(\varE\)-close.
        Since \(\varE > 0\) is arbitrary, by \DEF{5.2.6} \((a_n b_n)_{n = 1}^\infty\) and \((a'_n b_n)_{n = 1}^\infty\) are equivalent, that is, \(xy = x'y\).
\end{itemize}

\end{proof}

Of course we can prove a similar substitution rule when \(y\) is replaced by a real number \(y'\) which is equal to \(y\).

\begin{note}
At this point we \emph{embed the rationals back into the reals}, by equating every \emph{rational} number \(q\) with the \emph{real} number \(\LIM_{n \toINF} q\).
(That is, \(q \equiv \LIM_{n \toINF} q\).)
For instance, if \(a_1, a_2, a_3, ...\) is the sequence
\[
	0.5, 0.5, 0.5, 0.5, 0.5,...
\]
Then we let \(\LIM_{n \toINF} a_n\) equal to \(0.5\).

This embedding is consistent with our definitions of addition and multiplication, since for any rational numbers \(a, b\) we have
\begin{align*}
	a + b & = \LIM_{n \toINF} a + \LIM_{n \toINF} b & \text{by equating real and rational} \\
          & = \LIM_{n \toINF}(a + b) & \text{by \DEF{5.3.4}} \\
          & = a + b & \text{by equating again}
\end{align*}
And
\begin{align*}
	ab & = \LIM_{n \toINF} a \X \LIM_{n \toINF} b & \text{by equating} \\
          & = \LIM_{n \toINF}(ab) & \text{by \DEF{5.3.9}} \\
          & = ab & \text{by equating}
\end{align*}
(Note that the element \(a\) or \(b\) of the formula \(\LIM_{n \toINF} a\) or \(\LIM_{n \toINF} b\) is independent of the index \(n\).)

This means that when one wants to add or multiply two rational numbers \(a, b\) it does \emph{not} matter whether one thinks of these numbers as \emph{rationals} or as the \emph{real} numbers \(\LIM_{n \toINF} a, \LIM_{n \toINF} b\).
Also, this identification of rational numbers and real numbers is consistent with our definitions of equality(\EXEC{5.3.3}).
\end{note}

We can now easily define negation \(-x\) for real numbers \(x\) by the formula
\[
    -x := (-1) \X x,
\]
since \(-1\) is a rational number and is hence real.
(This definition is dependent on \DEF{5.3.9}, and we give it a label: \DEF{5.3.18}.)
Note that this is clearly consistent with our \emph{negation for rational} numbers since we have \(-q = (-1) \X q\) (by \AC{4.2.3}) for all rational numbers \(q\).
Also, from our definitions it is clear (why? see below) that
\begin{align*}
      & -(\LIM_{n \toINF} a_n) \\
    = & (-1) \X (\LIM_{n \toINF} a_n) & \text{by \DEF{5.3.18}, negation} \\
    = & \LIM_{n \toINF} (-1) \X \LIM_{n \toINF} a_n & \text{by equating rational and real} \\
    = & \LIM_{n \toINF} (-1) a_n & \text{by \DEF{5.3.9}} \\
    = & \LIM_{n \toINF} (-a_n)  & \text{by \AC{4.2.3}}
\end{align*}

Once we have addition and negation, we can define subtraction as usual by
\[
    x - y = x + (-y),
\]
(We label it as \DEF{5.3.19}.) Note that
\begin{align*}
      & \LIM_{n \toINF} a_n - \LIM_{n \toINF} b_n \\
    = & \LIM_{n \toINF} a_n + (-\LIM_{n \toINF} b_n) & \text{by \DEF{5.3.19}} \\
    = & \LIM_{n \toINF} a_n + \LIM_{n \toINF} (-b_n) & \text{shown above} \\
    = & \LIM_{n \toINF} (a_n + (-b_n)) & \text{by \DEF{5.3.4}} \\
    = & \LIM_{n \toINF} (a_n - b_n) & \text{by \DEF{4.2.13}, subtraction of rationals}
\end{align*}

We can now easily show that the \emph{real} numbers obey all the usual rules of algebra (\emph{except} perhaps for the laws involving \emph{division}
(since we currently have not defined the \emph{reciprocal} of the \emph{reals}), which we shall address shortly):

\begin{proposition} \label{prop 5.3.11}
All the laws of algebra from \PROP{4.1.6} (not \PROP{4.2.4}, not defining division yet) hold not only for the integers, but for the reals as well.
\end{proposition}

\begin{proof}
(The proof use the algebra laws of \emph{rationals}.)
We illustrate this with one such rule: \(x(y + z) = xy + xz\).
Let \(x = \LIM_{n \toINF} a_n\), \(y = \LIM_{n \toINF} b_n\),
and \(z = \LIM_{n \toINF} c_n\) be real numbers.
Then by \DEF{5.3.9}, \(xy = \LIM_{n \toINF} a_n b_n\) and \(xz = \LIM_{n \toINF} a_n c_n\),
and so by \DEF{5.3.4} \(xy + xz = \LIM_{n \toINF} (a_n b_n + a_n c_n)\).
A similar line of reasoning shows that \(x(y + z) = \LIM_{n \toINF} a_n (b_n + c_n)\).
But we already know that \(a_n (b_n + c_n)\) is equal to \(a_n b_n + a_n c_n\) for the \emph{rational} numbers \(a_n, b_n, c_n\), and the claim follows.
The other laws of algebra are proven similarly.
\end{proof}

The last basic arithmetic operation we need to define is reciprocation: \(x \to x^{-1}\).
This one is a little \emph{more subtle}.
On obvious first guess for how to proceed would be define
\[
    (\LIM_{n \toINF} a_n)^{-1} := \LIM_{n \toINF} a^{-1},
\]
but there are \emph{a few problems} with this.
For instance, let \(a_1, a_2, a_3,...\) be the \emph Cauchy sequence
\[
    0.1, 0.01, 0.001, 0.0001,...,
\]
and let \(x := \LIM_{n \toINF} a_n\).
Then by this definition, \(x^{-1}\) would be \(\LIM_{n \toINF} b_n\), where \(b_1, b_2, b_3,...\) is the sequence
\[
    10, 100, 1000, 10000,...
\]
but this is \emph{not} a Cauchy sequence (it isn't even bounded).
Of course, the problem here is that our original Cauchy sequence \((a_n)_{n = 1}^{\infty}\) was equivalent to the \emph{zero sequence} \((0)_{n = 1}^{\infty}\) (why? \MAROON{(1)}),
and hence that our real number \(x\) was in fact equal to \(0\).
So we should only allow the operation of reciprocal \emph{when \(x\) is non-zero}.

\begin{proof}
\MAROON{(1)} Suppose for the sake of contradiction, \((a_n)_{n = 1}^{\infty}\) was \emph{not} equivalent to \((0)_{n = 1}^{\infty}\).
Then by \DEF{5.2.6}, there exists \(\varE > 0\) s.t. \textbf{for all \(N \ge 1\)} there exists \(i \ge N\) s.t. \(\abs{a_i - 0} = \abs{a_i} = \abs{10^{-i}} > \varE\), or \(10^{-i} > \varE\), or \(10^i < 1/\varE\).
But since \(10^x\) an increasing function, and \(i \ge N\), we have \(10^N \le 10^i\), so we have \(10^N \le 1/\varE\).
And by \EXEC{4.3.5}, we have \(N \le 10^N\), wo we have \(N \le 1/\varE\).
So we conclude for all \(N \ge 1\), \(N \le 1/\varE\), which contradicts \PROP{4.4.1} that we can always find a natural number \(N\) s.t. \(N \ge 1/\varE\).
So the two sequences must be equivalent.
\end{proof}

However, even when we restrict ourselves to non-zero real numbers, we have a slight problem, because a non-zero real number might be the formal limit of a Cauchy sequence which \emph{contains zero elements}.
For instance, the number \(1\), which is rational and hence real, is the formal limit \(1 = \LIM_{n \toINF} a_n\) of the Cauchy sequence
\[
	0, 0.9, 0.99, 0.999, 0.9999,...
\]
(It can be proved that \(0, 0.9, 0.99,...\) and \(1, 1, 1, ...\) is equivalent.) but using our naive definition of reciprocal, we \emph{cannot} invert the real number \(1\), because we can’t invert the first element \(0\) of this Cauchy sequence!

To get around these problems we need to \emph{keep our Cauchy sequence away from zero}.
To do this we first need a definition.

\begin{definition} [Sequences bounded away from zero]  \label{def 5.3.12}
A sequence \((a_n)_{n = 1}^{\infty}\) of rational numbers is said to be \emph{bounded away from zero} iff there exists a \emph{rational} number \(c > 0\) such that \(\abs{a_n} \ge c\) for all \(n \ge 1\).
\end{definition}

\begin{example} \label{example 5.3.13}
The sequence \(1, -1, 1, -1, 1, -1, 1,...\) is bounded away from zero (all the coefficients have absolute value at least 1). 
But the sequence \(0.1, 0.01, 0.001,...\) is not bounded away from zero(proved by contradiction),
and neither is \(\textbf{0}, 0.9, 0.99, 0.999, 0.9999,....\)
The sequence \(10, 100, 1000,...\) is bounded away from zero, but is not bounded.
\end{example}

We now show that every \emph{non-zero} real number is the formal limit of a Cauchy sequence \emph{bounded away from zero}:
\begin{lemma} \label{lem 5.3.14}
Let \(x\) be a \emph{non-zero} real number.
Then \(x = \LIM_{n \toINF} a_n\) for some Cauchy sequence \((a_n)_{n = 1}^{\infty}\) which is \emph{bounded away from zero}.
\end{lemma}

\begin{proof}
Since \(x\) is real, we know that \(x = \LIM_{n \toINF} b_n\) for some Cauchy sequence \((b_n)_{n = 1}^{\infty}\).
But we are not yet done, because we do not know that \((b_n)_{n = 1}^{\infty}\) is bounded away from zero.

On the other hand, we are given that \(x \ne 0 = \LIM_{n \toINF} 0\), which means that the sequence \((b_n)_{n = 1}^{\infty}\) is \textbf{not} equivalent to \((0)_{n = 1}^{\infty}\).
Thus by \DEF{5.2.6} the sequence \((b_n)_{n = 1}^{\infty}\) \emph{cannot} be eventually \(\varE\)-close to \((0)_{n = 1}^{\infty}\) for every \(\varE > 0\).
Therefore we \emph{can find} an \(\varE >  0\) such that \((b_n)_{n = 1}^{\infty}\) is not eventually \(\varE\)-close to \((0)_{n = 1}^{\infty}\).

So let arbitrary \(\varE > 0\).
Since \((b_n)_{n = 1}^{\infty}\) is Cauchy, it is eventually \(\varE\)-steady.
Moreover, since \(\varE/2\) also \(> 0\) so \((b_n)_{n = 1}^{\infty}\) is eventually \(\varE/2\)-steady.
Thus there is an \(N \ge 1\) such that \(\abs{b_n - b_m} \le \varE/2\) for all \(n, m \ge N\) \MAROON{(1)}.
On the other hand, we cannot have \(\abs{b_n - 0} = \abs{b_n} \le \varE\) for all \(n \ge N\), 
since this would imply that \((b_n)_{n = 1}^{\infty}\) is eventually \(\varE\)-close to \((0)_{n = 1}^{\infty}\).
Thus there must be some \(n_0 \ge N\) for which \(\abs{b_{n_0} - 0} = \abs{b_{n_0}} > \varE\) \MAROON{(2)}.
Now from \MAROON{(1)}, in particular we have \(\abs{b_{n_0} - b_n} \le \varE/2\) for all \(n \ge N\),
and from the triangle inequality:
\begin{align*}
	         & \abs{b_{n_0} - b_n} \le \varE/2 \\
    \implies & -\abs{b_{n_0} - b_n} \ge -\varE/2 \\
    \implies & \abs{b_{n_0}} - \abs{b_{n_0} - b_n} \ge \abs{b_{n_0}} - \varE/2 \\
    \implies & \abs{b_{n_0}} - \abs{b_{n_0} - b_n} \ge \abs{b_{n_0}} - \varE/2 \ge \varE - \varE/2 = \varE/2 & \text{by \MAROON{(2)}}\\
    \implies & \abs{b_{n_0}} - \abs{b_{n_0} - b_n} \ge \varE/2 \\
    \implies & \abs{b_{n_0}} - \abs{b_n - b_{n_0}} \ge \varE/2 \\
    \implies & \abs{b_{n_0} + (b_n - b_{n_0})} \ge \abs{b_{n_0}} - \abs{b_n - b_{n_0}} \ge \varE/2 & \text{by \AC{4.3.1}} \\
    \implies & \abs{b_{n_0} + (b_n - b_{n_0})} \ge \varE/2 \\
    \implies & \abs{b_n} \ge \varE/2
\end{align*}
So \(\abs{b_n} \ge \varE/2\) for all \(n \ge N\).

This almost proves that \((b_n)_{n = 1}^{\infty}\) is bounded away from zero.
Actually, what it does is show that \((b_n)_{n = 1}^{\infty}\) is \emph{eventually} bounded away from zero.
But this is easily fixed, by \emph{defining a new sequence} \(a_n\), by setting \(a_n := \varE/2\) if \(n < N\) and \(a_n := b_n\) if \(n \ge N\).
Since \(b_n\) is a Cauchy sequence, it is not hard to verify that \(a_n\) is also a Cauchy sequence which is \emph{equivalent} to \(b_n\) (because the two sequences are eventually the same),
and so \(x = \LIM_{n \toINF} a_n\).
And since \(\abs{b_n} \ge \varE/2\) for all \(n \ge \textbf{N}\), we know that \(\abs{a_n} \ge \varE/2\) for all \(n \ge \textbf{1}\)
(splitting into the two cases \(n \ge N\) and \(n < N\) separately).
Thus we have a Cauchy sequence \(a_n\) which is bounded away from zero (by \(\varE/2\) instead of \(\varE\), but that’s still OK since \(\varE/2 > 0\)) and which has \(x\) as a formal limit, and so we are done.
\end{proof}

Once a sequence is bounded away from zero, we can take its \emph{reciprocal} without any difficulty:

\begin{lemma} \label{lem 5.3.15}
Suppose that \((a_n)_{n = 1}^{\infty}\) is a Cauchy sequence which is bounded away from zero.
Then the sequence \((a_n^{-1})_{n = 1}^{\infty}\) is also a Cauchy sequence.
\end{lemma}

\begin{proof}
Since \((a_n)_{n = 1}^{\infty}\) is bounded away from zero, by \DEF{5.3.12} we know that there is a \(c > 0\) such that \(\abs{a_n} \ge c\) for all \(n \ge 1\).

Now we need to show that \((a_n^{-1})_{n = 1}^{\infty}\) is eventually \(\varE\)-steady for each \(\varE > 0\).
Thus let arbitrary \(\varE > 0\);
our task is now to find an \(N \ge 1\) such that \(\abs{a_n^{-1} - a_m^{-1}} \le \varE\) for all \(n, m \ge N\).
But
\begin{align*}
    \abs{a_n^{-1} - a_m^{-1}} & = \abs{\frac1{a_n} - \frac1{a_m}} \\
                              & = \abs{\frac{a_m - a_n}{a_n a_m}} \\
                              & = \frac{\abs{a_m - a_n}}{\abs{a_n a_m}} \\
                              & = \frac{\abs{a_m - a_n}}{\abs{a_n}\abs{a_m}} \\
                              & \le \frac{\abs{a_m - a_n}}{c^2} & \text{since \(\abs{a_m}, \abs{a_n} \ge c\)}
\end{align*}
and so to make \(\abs{a_n^{-1} - a_m^{-1}}\) less than or equal to
\(\varE\), it will suffice to make \(\abs{a_m - a_n}\) less than or equal to \(c^2\varE\).

But since \((a_n)_{n = 1}^{\infty}\) is a Cauchy sequence, and \(c^2\varE > 0\), we can certainly find an \(N\) such that the sequence \((a_n)_{n = N}^{\infty}\) is \(c^2\varE\)-steady, 
i.e., \(\abs{a_m - a_n} \le c^2\varE\) for all \(m, n \ge N\).
By what we have said above, this shows that \(\abs{a_n^{-1} - a_m^{-1}} \le \varE\) for all \(m, n \ge N\), and hence the sequence \((a_n^{-1})_{n = 1}^{\infty}\) is eventually \(\varE\)-steady.
Since we have proven this for arbitrary \(\varE > 0\), we have that \((a_n^{-1})_{n = 1}^{\infty}\) is a Cauchy sequence, as desired.
\end{proof}

We are now ready to define reciprocation:
\begin{definition} [Reciprocals of \emph{real} numbers] \label{def 5.3.16}
Let \(x\) be a \emph{non-zero} real number.
Let \((a_n)_{n = 1}^{\infty}\) be a Cauchy sequence bounded away from zero such that \(x = \LIM_{n \toINF} a_n\)
(such a sequence exists by \LEM{5.3.14}).
Then we define the reciprocal \(x^{-1}\) by the formula \(x^{-1} := \LIM_{n \toINF} a_n^{-1}\).
(From \LEM{5.3.15} we know that \(x^{-1}\) is still a real number.)
\end{definition}

And we need to check that the reciprocals of the two different but \emph{equivalent}(in the sense of \DEF{5.2.6}) Cauchy sequences are still equivalent.

\begin{lemma} [Reciprocation is well defined] \label{lem 5.3.17}
Let \((a_n)_{n = 1}^{\infty}\) and \((b_n)_{n = 1}^{\infty}\) be two Cauchy sequences bounded away from zero such that \(\LIM_{n \toINF} a_n = \LIM_{n \toINF} b_n\)
(i.e., the two sequences are \emph{equivalent}).
Then \(\LIM_{n \toINF} a_n^{-1} = \LIM_{n \toINF} b_n^{-1}\).
\end{lemma}

\begin{proof}
Consider the following product \(P\) of three real numbers:
\[
    P := (\LIM_{n \toINF} a_n^{-1}) \X \BLUE{(\LIM_{n \toINF} a_n)} \X (\LIM_{n \toINF} b_n^{-1}).
\]
If we multiply (by \DEF{5.3.9}) this out, we obtain
\[
    P = \LIM_{n \toINF} a_n^{-1} a_n b_n^{-1} = \LIM_{n \toINF} b_n^{-1}.
\]
On the other hand, since \BLUE{\(\LIM_{n \toINF} a_n = \LIM_{n \toINF} b_n\)}, by \PROP{5.3.10}(multiplication is well-defined), we can write \(P\) in another way as
\[
     P = (\LIM_{n \toINF} a_n^{-1}) \X \BLUE{(\LIM_{n \toINF} b_n)} \X (\LIM_{n \toINF} b_n^{-1}).
\]
Multiplying things out again, we get
\[
    P = \LIM_{n \toINF} a_n^{-1} b_n b_n^{-1} = \LIM_{n \toINF} a_n^{-1}.
\]
Comparing our different formulae for \(P\) we see that \(\LIM_{n \toINF} a_n^{-1} = \LIM_{n \toINF} b_n^{-1}\), as desired.
\end{proof}

Thus reciprocal is well-defined (for each non-zero real number \(x\), we have \emph{exactly one definition of} the reciprocal \(x^{-1}\)).

\begin{note}
Note it is clear from the definition that \(xx^{-1} = x^{-1}x = 1\) (from the commutative law in \PROP{5.3.11} we only have to show \(xx^{-1} = 1\)):
\begin{align*}
    xx^{-1} & = \LIM_{n \toINF} a_n \LIM{n \toINF} a_n^{-1} \\
            & = \LIM_{n \toINF} a_n a_n^{-1} & \text{by \DEF{5.3.9}} \\
            & = \LIM_{n \toINF} 1 & \text{by \PROP{4.2.4}(10)} \\
            & = 1 & \text{by equating real to rational}
\end{align*}
Thus with this property, all the \emph{field} axioms (\PROP{4.2.4}) apply to the \emph{reals} as well as to the rationals.
\end{note}

\begin{note}
We of course cannot give \(0\) a reciprocal, since \(0\) multiplied by anything gives \(0\), not \(1\). 
\end{note}

\begin{note}
Also note that if \(q\) is a non-zero \emph{rational}, and hence equal to the real number \(\LIM_{n \toINF} q\),
then the reciprocal of \(\LIM_{n \toINF} q\) is \(\LIM_{n \toINF} q^{-1} = q^{-1}\);
That is,
\begin{align*}
    q^{-1} & = (\LIM_{n \toINF} q)^{-1} & \text{by equating rational and real} \\
           & = \LIM_{n \toINF} q^{-1} & \text{by \DEF{5.3.16}} \\
           & = q^{-1} & \text{by equating rational and real}
\end{align*}
thus the operation of \emph{reciprocal on real} numbers is consistent with the operation of \emph{reciprocal on rational} numbers.
\end{note}

Once one has reciprocal, one can define \emph{division} \(x/y\) of two real numbers \(x, y\), provided \(y\) is non-zero, by the formula
\[
    x/y := x \X y^{-1},
\]
just as we did with the rationals. (We label it as \DEF{5.3.20}.)
In particular, we have the cancellation law (label as \LEM{5.3.21}): if \(x, y, z\) are real numbers such that \(xz = yz\), and \(z\) is non-zero, then by dividing by \(z\) we conclude that \(x = y\).
Note that this cancellation law does not work when \(z\) is zero.

We now have all four of the basic arithmetic operations on the reals: addition, subtraction, multiplication, and division, with all the usual
rules of algebra.

\begin{definition} \label{def 5.3.18}
We define negation \(-x\) for real numbers \(x\) by the formula
\[
    -x := (-1) \X x,
\]
\end{definition}

\begin{definition} \label{def 5.3.19}
We define subtraction for real numbers \(x - y\) by the formula
\[
    x - y := x + (-y).
\]
\end{definition}

\begin{definition} \label{def 5.3.20}
We define division for real numbers \(x/y\), provided \(y \ne 0\), by the formula
\[
    x/y := x \X y^{-1}.
\]
\end{definition}

\begin{lemma} [Cancellation law for real number] \label{lem 5.3.21}
If \(x, y, z\) are real numbers such that \(xz = yz\), and \(z\) is non-zero, then by dividing by \(z\) we conclude that \(x = y\).
\end{lemma}

\exercisesection

\begin{exercise} \label{exercise 5.3.1}
Prove \PROP{5.3.3}.
\end{exercise}

\begin{proof}
See \PROP{5.3.3}.
\end{proof}

\begin{exercise} \label{exercise 5.3.2}
Prove \PROP{5.3.10}.
\end{exercise}

\begin{proof}
See \PROP{5.3.10}.
\end{proof}

\begin{exercise} \label{exercise 5.3.3}
Let \(a, b\) be \emph{rational} numbers.
Show that \(a = b\) if and only if \(\LIM_{n \toINF} a = \LIM_{n \toINF} b\)
(i.e., the Cauchy sequences \(a, a, a, a, ...\) and \(b, b, b, b, ...\) are equivalent if and only if \(a = b\)).
This allows us to \emph{embed the rational numbers inside the real numbers in a well-defined manner}.
\end{exercise}

\begin{proof}
\begin{align*}
         & a = b \\
    \iff & a - b = 0 \\
    \iff & \abs{a - b} = 0 \\
    \iff & \abs{a - b} < \varE\ \forall \varE > 0 \\
    \iff & \abs{a - b} < \varE\ \forall \varE > 0, \forall i \ge 1 & \text{just give a dummy index} \\
    \iff & (a)_{n = 1}^{\infty} = (b)_{n = 1}^{\infty} & \text{by \DEF{5.2.6}} \\
    \iff & \LIM_{n \toINF} a_n = \LIM_{n \toINF} b_n & \text{by \DEF{5.3.1}}
\end{align*}
\end{proof}

\begin{exercise} \label{exercise 5.3.4}
Let \((a_n)_{n = 0}^{\infty}\) be a sequence of rational numbers which is bounded.
Let \((b_n)_{n = 0}^{\infty}\) be another sequence of rational numbers which is \emph{equivalent} to \((a_n)_{n = 0}^{\infty}\).
Show that \((b_n)_{n = 0}^{\infty}\) is also bounded.
(Hint: use \EXEC{5.2.2}.)
\end{exercise}

\begin{proof}
Since \((a_n)_{n = 0}^{\infty}\) and \((b_n)_{n = 0}^{\infty}\) are equivalent, they are eventually \(\varE\)-closes for all \(\varE > 0\).
And since \((a_n)_{n = 0}^{\infty}\) is bounded, the conditions in the \EXEC{5.2.2} are satisfied, and by \EXEC{5.2.2}, \((b_n)_{n = 0}^{\infty}\) is bounded.
\end{proof}

\begin{note}
若\ seq A 是\ bounded,且\ seq B 等價於\ seq A,則\ seq B 也是\ bounded。
\end{note}

\begin{exercise} \label{exercise 5.3.5}
Show that \(\LIM_{n \toINF} 1 / n = 0\).
\end{exercise}

\begin{proof}
We have to show \(\LIM_{n \toINF} 1 / n = 0 = \LIM_{n \toINF} 0\).
Suppose for the sake of contradiction that \(\LIM_{n \toINF} 1 / n \ne \LIM_{n \toINF} 0\).
Then by \DEF{5.2.6}, There exists \(\varE > 0\), such that \textbf{for all \(N \ge 1\)}, there exists \(i \ge N\) s.t. \(\abs{1/i - 0} > \varE\).
And \(\abs{1/i - 0} = \abs{1/i}\) is positive, so we have \(1/i > \varE\), or \(i < 1/\varE\).
But since \(i \ge N\), we have \(N < 1/\varE\).
So we have for all \(N \ge 1\), \(N < 1/\varE\), which contradicts \PROP{4.4.1} that we can always find a natural number \(N\) s.t. \(N > 1/\varE\).
So \(\LIM_{n \toINF} 1 / n = 0\).
\end{proof}
\chapter{Limits of sequences} \label{ch 6}

\section{Convergence and limit laws}

In the previous chapter, we defined the real numbers as \emph{formal limits} of \emph{rational} (Cauchy) sequences, and we then defined various operations on the real numbers.
However, \emph{unlike} our work in constructing the integers (where we eventually replaced formal differences with actual differences) and rationals (where we eventually replaced formal quotients with actual quotients),
we never really finished the job of constructing the real numbers, because we never got around to replacing formal limits \(\LIM_{n \toINF} a_n\) with actual limits \(\lim_{n \toINF} a_n\).
In fact, we \emph{haven’t defined} limits at all yet.
This will now be rectified.

We begin by repeating much of the machinery of \(\varE\)-close sequences, etc. again
- but this time, we do it for sequences of \emph{real} numbers, not rational numbers.
Thus this discussion will \emph{supersede} what we did in the previous chapter.

\begin{definition} [Distance between two real numbers] \label{def 6.1.1}
Given two real numbers \(x\) and \(y\), we define their distance \(d(x, y)\) to be \(d(x, y):= \abs{x - y}\).
\end{definition}

\begin{note}
Clearly \DEF{6.1.1} is consistent with \DEF{4.3.2}.
Further, \PROP{4.3.3} works just as well for real numbers as it does for rationals, because the real numbers obey all the rules of algebra that the rationals do.
(In fact this had been discussed in the middle of page 114; it's somewhat redundant.)
\end{note}

\begin{definition} [\(\varE\)-close real numbers] \label{def 6.1.2}
Let \(\varE > 0\) be a real number.
We say that two real numbers \(x, y\) are \(\varE\)-close iff we have \(d(y, x) \le \varE\).
\end{definition}

\begin{note}
Again, it is clear that this \DEF{6.1.2} of \(\varE\)-close is consistent with \DEF{4.3.4}.
\end{note}

\begin{note}
Now let \((a_n)_{n = m}^\infty\) be a sequence of \emph{real} numbers;
i.e., we assign a real number \(a_n\) for every integer \(n \geq m\).
The starting index \(m\) is some integer;
usually this will be \(1\), but in some cases we will start from some index other than \(1\).
(The choice of \emph{label} used to index this sequence is unimportant; we could use for instance \((a_k)_{k = m}^{\infty}\) and this would represent exactly the same sequence as \((a_n)_{n = m}^{\infty}\).)
\end{note}

We can define the notion of a Cauchy sequence in the same manner as before.

\begin{definition} [Cauchy sequences of reals] \label{def 6.1.3}
(Annoyingly, we actually define three concepts at the same time.)
Let \(\varE > 0\) be a \emph{real} number.
A sequence \((a_n)_{n = N}^{\infty}\) of \emph{real} numbers starting at some integer index \(N\) is said to be \emph{\(\varE\)-steady} iff \(a_j\) and \(a_k\) are \(\varE\)-close for every \(j, k \ge N\).
A sequence \((a_n)_{n = m}^{\infty}\) starting at some integer index \(m\) is said to be \emph{eventually \(\varE\)-steady} iff there exists an \(N \ge m\) such that \((a_n)_{n = N}^{\infty}\) is \(\varE\)-steady.
We say that \((a_n)_{n = m}^{\infty}\) is a \emph{Cauchy sequence} iff it is eventually \(\varE\)-steady \emph{for every} \(\varE > 0\).
\end{definition}

\begin{note}
You can compare \DEF{5.1.3}, \DEF{5.1.6}, \DEF{5.1.8} with \DEF{6.1.3}.
Note that ``\(\varE\)-close'' in \DEF{6.1.3} is depend on \DEF{6.1.2}, \emph{not} \DEF{4.3.4}.

These definitions are consistent with the corresponding definitions for rational numbers, although \emph{verifying consistency} for Cauchy sequences takes a little bit of care:
\end{note}

\begin{proposition} \label{prop 6.1.4}
Let \((a_n)_{n = m}^{\infty}\) be a sequence of \emph{rational} numbers starting at some integer index \(m\).
Then \((a_n)_{n = m}^{\infty}\) is a Cauchy sequence in the sense of \DEF{5.1.8} if and only if it is a Cauchy sequence in the sense of \DEF{6.1.3}.
\end{proposition}

\begin{note}
這邊只要考慮\ sequence of ``rational'' 在新的定義跟舊的定義相容。 我們也不用考慮\ sequence of ``real'' 要\ ``相容'' \DEF{5.1.8},因為\ \DEF{5.1.8} 根本不\ care sequence of real。
\end{note}

\begin{proof}
Suppose first that \((a_n)_{n = m}^{\infty}\) is a Cauchy sequence in the sense of \DEF{6.1.3};
then it is eventually \(\varE\)-steady for every \emph{real} \(\varE > 0\).
In particular, given arbitrary rational \(\varE\), since rational is real, by \DEF{6.1.3} \((a_n)_{n = m}^{\infty}\) is eventually \(\varE\)-steady.
Since rational \(\varE > 0\) is arbitrary, \((a_n)_{n = m}^{\infty}\) is eventually \(\varE\)-steady for every rational \(\varE > 0\).
which makes it a Cauchy sequence in the sense of \DEF{5.1.8}.

Now suppose that \((a_n)_{n = m}^{\infty}\) is a Cauchy sequence in the sense of \DEF{5.1.8};
then it is eventually \(\varE\)-steady (in the sense of \DEF{5.1.6}) for every \emph{rational} \(\varE > 0\).
Now given arbitrary \emph{real} number \(\varE > 0\), then there exists a \emph{rational} \(\varE' > 0\) which is smaller than (or equal to) \(\varE\), by \PROP{5.4.12}.
Since \(\varE'\) is \emph{rational}, we know that \((a_n)_{n = m}^{\infty}\) is eventually \(\varE'\)-steady(since it is supposed to satisfy \DEF{5.1.8});
and since \(\varE' < \varE\), this implies that \((a_n)_{n = m}^{\infty}\) is eventually \(\varE\)-steady.
Since \(\varE\) is an arbitrary positive \emph{real} number, we thus see that \((a_n)_{n = m}^{\infty}\) is a Cauchy sequence in the sense of \DEF{6.1.3}.
\end{proof}

\begin{note}
Because of this proposition, we will no longer care about the distinction between \DEF{5.1.8} and \DEF{6.1.3}, and view the concept of a Cauchy sequence as a single unified concept.
\end{note}

Now we talk about what it means for a sequence of real numbers to \emph{converge} to some limit \(L\).

\begin{definition} [Convergence of sequences] \label{def 6.1.5}
(Again annoyingly, we actually define three concepts at the same time.)
Let \(\varE > 0\) be a \emph{real} number, and let \(L\) be a \emph{real} number.
\MAROON{DEF(1)} A sequence \((a_n)_{n = N}^{\infty}\) of real numbers is said to be ``\(\varE\)-close'' to \(L\) iff \(a_n\) is \(\varE\)-close(in the sense of \DEF{6.1.2}) to \(L\) for every \(n \ge N\),
i.e., we have \(\abs{a_n - L} \le \varE\) for every \(n \ge N\).
\MAROON{DEF(2)} We say that a sequence \((a_n)_{n = m}^{\infty}\) is ``eventually \(\varE\)-close'' to \(L\) iff there exists an \(N \ge m\) such that \((a_n)_{n = N}^{\infty}\) is \(\varE\)-close to \(L\).
\MAROON{DEF(3)} We say that a sequence \((a_n)_{n = m}^{\infty}\) \textbf{converges to} \(L\) iff it is ``eventually \(\varE\)-close'' to \(L\) for every real \(\varE > 0\).
\end{definition}

\begin{note}
One can unwrap all the definitions here and write the concept of convergence more directly; see \EXEC{6.1.2}.
\end{note}

\begin{example} \label{examples 6.1.6}
The sequence
\[
    0.9, 0.99, 0.999, 0.9999,...
\]
is \(0.1\)-close to \(1\), but is not \(0.01\)-close to \(1\), because of the first element of the sequence.
However, it is \emph{eventually} \(0.01\)-close to \(1\).
In fact, for every real \(\varE > 0\), this sequence is \emph{eventually} \(\varE\)-close to \(1\), hence is \emph{convergent} to \(1\).
\end{example}

\begin{proposition} [Uniqueness of limits] \label{prop 6.1.7}
Let \((a_n)_{n = m}^{\infty}\) be a real sequence starting at some integer index \(m\),
and let \(L \neq L'\) be two distinct real numbers.
Then it is \emph{not} possible for \((a_n)_{n = m}^{\infty}\) to converge to \(L\) while also converging to \(L'\).
\end{proposition}

\begin{proof}
Suppose for sake of contradiction that \((a_n)_{n = m}^{\infty}\) was converging to both \(L\) and \(L'\).
Let \(\varE = \abs{L - L'}/3\) \MAROON{(*)};
note that \(\abs{L - L'}\) is positive \MAROON{(**)} since \(L \neq L'\), and hence \(\abs{L - L'}/3\) is positive so \(\varE\) is positive.
Since \((a_n)_{n = m}^{\infty}\) converges to \(L\), we know that \((a_n)_{n = m}^{\infty}\) is eventually \(\varE\)-close to \(L\);
thus there is an \(N \ge m\) such that \(d(a_n, L) \le \varE\) for all \(n \ge N\).
\emph{Similarly}, there is an \(M \ge m\) such that \(d(a_n, L') \le \varE\) for all \(n \ge M\).
In particular, if we set \(n := \max(N, M)\), then we have \(d(a_n, L) \le \varE\) and \(d(a_n, L') \le \varE\), hence by the triangle inequality \(d(L,L') \le d(L, a_n) + d(a_n, L') \le 2\varE\), which by \MAROON{(*)} is equal to \(2\abs{L - L'}/3\).
But then we have
\begin{align*}
             & d(L, L') = \abs{L - L'} \le 2\abs{L - L'}/3 \\
    \implies & \abs{L - L'}/3 \le 0 \\
    \implies & \abs{L - L'} \le 0
\end{align*}
which contradicts \MAROON{(**)}.
Thus it is not possible to converge to both \(L\) and \(L'\).
\end{proof}

Now that we know limits are unique, we can set up notation to specify them:

\begin{definition} [Limits of sequences] \label{def 6.1.8}
If a sequence \((a_n)_{n = m}^{\infty}\) converges to some real number \(L\), we say that \((a_n)_{n = m}^{\infty}\) is \emph{convergent} and that its limit is \(L\);
we write
\[
    L = \lim_{n \toINF} a_n
\]
to denote this fact.
If a sequence \((a_n)_{n = m}^{\infty}\) is \emph{not} converging to any real number \(L\), we say that the sequence \((a_n)_{n = m}^{\infty}\) is \emph{divergent} and we leave \(\lim_{n \toINF} a_n\) \emph{undefined}.
\end{definition}

\begin{note}
\PROP{6.1.7} ensures that a sequence can have \emph{at most one} limit.
Thus, if the limit exists, it is a single real number, otherwise
it is undefined.
\end{note}

\begin{remark} \label{remark 6.1.9}
The notation \(\lim_{n \toINF} a_n\) does \emph{not} give any indication about the \emph{starting index} \(m\) of the sequence,
but the starting index is irrelevant (\EXEC{6.1.3}).
Thus in the rest of this discussion we shall not be too careful as to where these sequences start, as we shall be mostly focused on their limits.
\end{remark}

\begin{note}
We sometimes use the phrase ``\(a_n \to x\) as \(n \to \infty\)'' as an alternate way of writing the statement ``\((a_n)_{n = m}^{\infty}\) converges to \(x\)''.
Bear in mind, though, that the \emph{individual} statements \(a_n \to x\) and \(n \to \infty\) do not have any rigorous meaning;
this phrase is just a convention, though of course a very suggestive one.
\end{note}

\begin{remark} \label{remark 6.1.10}
The exact choice of letter used to denote the index (in this case \(n\)) is \emph{irrelevant}:
the phrase \(lim_{n \toINF} a_n\) has exactly the same meaning as \(\lim_{k \toINF} a_k\), for instance.
Sometimes it will be convenient to change the label of the index to avoid conflicts of notation;
for instance, we might want to change \(n\) to \(k\) because \(n\) is simultaneously being used for some other purpose, and we want to reduce confusion.
\end{remark}

\begin{note}
這邊說要我們去看\ \EXEC{6.1.4} 來比較\ \RMK{6.1.10},但那個練習感覺沒有直接關聯,該練習在講一個\ sequence 拔掉前面有限項,跟它是否收斂無關。
\end{note}

\begin{proposition} \label{prop 6.1.11}
We have \(\lim_{n \toINF} 1/n = 0\).
\end{proposition}

\begin{note}
Compare to \EXEC{5.3.5}.
\end{note}

\begin{proof}
By \DEF{6.1.8}, we have to show that the sequence \((a_n)_{n = 1}^{\infty}\) converges to \(0\), where \(a_n := 1/n\).
In other words, by \DEF{6.1.5}, for every \(\varE > 0\), we need to show that the sequence \((a_n)_{n = 1}^{\infty}\) is eventually \(\varE\)-close to \(0\).

So, let \(\varE > 0\) be an arbitrary real number.
We have to find an \(N\) such that \(\abs{a_n - 0} \le \varE\) for every \(n \ge N\).
But \(\varE > 0\), and by Archimedean property(\CORO{5.4.13}), we can find an integer \(N\) s.t. \(N\varE > 1\) and
\begin{align*}
             & N \varE > 1 \\
    \implies & N > 1/\varE \\
    \implies & 1/N < \varE \MAROON{(*)}.
\end{align*}
But for all \(n \ge N\), \(1/n \le 1/N\).
With \MAROON{(*)}, we have \(1/n \le \varE\), and trivially we have \(\abs{1/n - 0} \le \varE\).
So for all \(n \ge N\), \(\abs{1/n - 0} \le \varE\).
So by \DEF{6.1.5} \((a_n)_{n = 1}^{\infty}\) is eventually \(\varE\)-close to \(0\).
Since \(\varE\) was arbitrary, again by \DEF{6.1.5}, \((a_n)_{n = 1}^{\infty}\) converges to \(0\).
\end{proof}

\begin{proposition} [Convergent sequences are Cauchy] \label{prop 6.1.12}
Suppose that \((a_n)_{n = m}^{\infty}\) is a \emph{convergent} sequence of real numbers.
Then \((a_n)_{n = m}^{\infty}\) is also a \emph{Cauchy} sequence.
\end{proposition}

\begin{proof}
Suppose \((a_n)_{n = m}^{\infty}\) is a convergent sequence of real numbers.
By \DEF{6.1.3}, we have to show that given arbitrary \(\varE > 0\), there exists integer \(N \ge m\) s.t. \(d(a_n, a_{n'}) \le \varE\) for all \(n, n' \ge N\).

So let \(\varE > 0\).
In particular \(\varE/2 > 0\).
And WLOG, we let \((a_n)_{n = m}^{\infty}\) converge to \(L\) (or \(L = \lim_{n \toINF} a_n\).).
And by \DEF{6.1.5}, we can find an integer \(N \ge m\) s.t. \(d(a_n, L) \le \varE/2\) for all \(n \ge N\) \MAROON{(*)}.
So in particular given integer \(n, n' \ge N\):
\begin{align*}
             & d(a_n, L) \le \varE/2 \land d(a_{n'}, L) \le \varE/2 & \text{by \MAROON{(*)}} \\
    \implies & d(a_n, L) + d(a_{n'}, L) \le \varE \\
    \implies & d(a_n, a_{n'}) \le d(a_n, L) + d(a_{n'}, L) \le \varE & \text{by \PROP{4.3.3}(g)} \\
    \implies & d(a_n, a_{n'}) \le \varE & \text{simplify}
\end{align*}
So by \DEF{6.1.3}, \((a_n)_{n = m}^{\infty}\) is eventually \(\varE\)-steady.
Since \(\varE > 0\) is arbitrary, \((a_n)_{n = m}^{\infty}\) is eventually-\(\varE\)-steady for all \(\varE > 0\).
Again by \DEF{6.1.3}, \((a_n)_{n = m}^{\infty}\) is a Cauchy sequence.
\end{proof}

\begin{example} \label{example 6.1.13}
The sequence \(1, -1, 1, -1, 1, -1,...\) is not a Cauchy sequence (because it is not eventually \(1\)-steady), and is hence not a
convergent sequence, by (contrapositive of) \PROP{6.1.12}.
\end{example}

\begin{remark} \label{remark 6.1.14}
For a \emph{converse} to \PROP{6.1.12}, see \THM{6.4.18} below.
\end{remark}

\begin{note}
開天眼,\THM{6.4.18} (in particular, \PROP{6.1.12} 的\ converse) 是想說實數系是個「完備賦距空間」。
實數的柯西序列會收斂到實數,但是有理數的柯西序列會跑出有理數\
(例如, informally, \(1, 1.4, 1.41, 1.414,...\) 會跑去無理數\ \(\sqrt{2}\))。
\end{note}

Now we show that \emph{formal} limits can be superseded by \emph{actual} limits.

\begin{proposition} [Formal limits are genuine limits] \label{prop 6.1.15}
Suppose that \((a_n)_{n = 1}^{\infty}\) is a Cauchy sequence of \emph{rational} numbers.
Then \((a_n)_{n = 1}^{\infty}\) converges to \(\LIM_{n \toINF} a_n\),
i.e.
\[
    \LIM_{n \toINF} a_n = \lim_{n \toINF} a_n.
\]
\end{proposition}

\begin{note}
Similar as \PROP{6.1.4}, we only need to consider the sequence of rationals because the symbol \(\LIM\) in \CH{5} was used only on the sequence of rationals.
\end{note}

\begin{proof}
Let \((a_n)_{n = m}^{\infty}\) be a Cauchy sequence of rationals, and write \(L := \LIM_{n \toINF} a_n\).
We have to show that \((a_n)_{n = m}^{\infty}\) converges to \(L\).
Let \(\varE > 0\).
Assume for sake of contradiction that sequence \(a_n\) is \emph{not} eventually \(\varE\)-close to \(L\) \MAROON{(*)}.
Then by \DEF{6.1.5}, for all \(N \ge m\), there exists an integer \(n \ge N\) s.t. \(d(a_n, L) > \varE\).
Also since \((a_n)_{n = m}^{\infty}\) is Cauchy, and in particular \(\varE/2 > 0\), there exists an integer \(N_1 \ge m\) s.t. \(d(a_n, a_k) \le \varE/2\) for all \(n, k \ge N_1\) \MAROON{(**)}.
We preserve this \(N_1\).
\emph{Again}, since \MAROON{(*)}, in particular for the integer \(N_1 \ge m\), there exists an integer \(N_2 \ge N_1\) s.t. \(d(a_{N_2}, L) > \varE\) \MAROON{(***)}.
We preserve this \(N_2\).
Also note that since \(N_2 \ge N_1\), in particular it can be replaced into the free variable \(k\) in \MAROON{(**)} so that we have:
there exists an integer \(N_1 \ge m\) s.t. \(d(a_n, a_{N_2}) \le \varE/2\) for all \(n \ge N_1\).
And from \EXEC{5.4.6}, we can rewrite it as the in equality \(-\varE/2 \le a_n - a_{N_2} \le \varE/2\), and hence \(-\varE/2 + a_{N_2} \le a_n \le \varE/2 + a_{N_2}\), for all \(n \ge N_1\) \BLUE{(*)}.

Now we consider the value of \(a_{N_2} - L\):
Since from \MAROON{(***)}, we know \(a_{N_2} - L \neq 0\), so there are two cases:
\begin{itemize}
    \item[>>] \(a_{N_2} - L > 0\). Then
        \begin{align*}
            a_{N_2} - L & = \abs{a_{N_2} - L} \\
                        & = d(a_{N_2}, L) \\
                        & > \varE & \text{from \MAROON{(***)}}
        \end{align*}
        So we have \(\varE < a_{N_2} - L\).
        And from \BLUE{(*)}, in particular we have \(-\varE/2 + a_{N_2} \le a_n\), for all \(n \ge N_1\).
        Add both sides of these two inequalities, we have:
        \begin{align*}
                     & -\varE/2 + a_{N_2} \le a_n \land \varE < a_{N_2} - L \\
            \implies & \varE/2 + a_{N_2} < a_n + a_{N_2} - L \\
            \implies & \varE/2 + L < a_n & \text{for all \(n \ge N_1\).}
        \end{align*}
        Now we defined a sequence \((b_n)_{n = 1}^{\infty}\) s.t. \(b_n := a_{n + N_1 - 1}\).
        Then it's trivial that \(\LIM b_n = \LIM a_n\) and \(\varE/2 + L < b_n\) for all \(n \ge 1\). (If we proved informally, we can just use \EXEC{5.4.8} directly on the sequence \(a_n\)).
        Then from \EXEC{5.4.8}, we have \(\varE/2 + L < \LIM_{n \toINF} b_n = L\),
        which implies \(\varE < 0\), a contradiction.
    \item[>>] \(a_{N_2} - L < 0\). Then
        \begin{align*}
            a_{N_2} - L & = -\abs{a_{N_2} - L} \\
                        & = -d(a_{N_2}, L) \\
                        & < -\varE & \text{from \MAROON{(***)}}
        \end{align*}
        So we have \(\varE < L - a_{N_2}\).
        And from \BLUE{(*)}, in particular we have \(a_n \le \varE/2 + a_{N_2}\), for all \(n \ge N_1\).
        Add both sides of these two inequalities, we have:
        \begin{align*}
                     & a_n \le \varE/2 + a_{N_2} \land \varE < L - a_{N_2} \\
            \implies & a_n + \varE < \varE/2 + L \\
            \implies & a_n < L - \varE/2 & \text{for all \(n \ge N_1\).}
        \end{align*}
        Then similarly with the previous case, from \EXEC{5.4.8} we have \(\LIM_{n \toINF} a_n < L - \varE/2\), or \(L < L - \varE/2\), which implies \(\varE < 0\), a contradiction.
\end{itemize}
So in all cases we have contradiction, so there is no \(\varE > 0\) s.t. the sequence \(a_n\) is not \(\varE\)-close to \(L\).
So the sequence \(a_n\) is convergent to \(L\), that is, \(\lim_{n \toINF} a_n = L\), that is, \(\lim_{n \toINF} a_n = \LIM_{n \toINF} a_n\).
\end{proof}

\begin{definition} [Bounded sequences] \label{def 6.1.16}
A sequence \((a_n)_{n = m}^{\infty}\) of \emph{real} numbers is \emph{bounded} by a \emph{real} number \(M\) iff we have \(\abs{a_n} \le M\) for all \(n \ge m\).
We say that \((a_n)_{n = m}^{\infty}\) is bounded iff it is \emph{bounded} by \(M\) for some real number \(M > 0\).
\end{definition}

\begin{note}
This definition is consistent with \DEF{5.1.12}; see \EXEC{6.1.7}.
\end{note}

Recall from \LEM{5.1.15} that every Cauchy sequence of \emph{rational} numbers is bounded.
An inspection of the proof of that Lemma shows that the same argument works for \emph{real} numbers;
every Cauchy sequence of real numbers is bounded.
In particular, from \PROP{6.1.12} we have
\begin{corollary} \label{corollary 6.1.17}
Every convergent sequence of real numbers by \PROP{6.1.12} is Cauchy, and Cauchy by the argument above(``real'' version of \LEM{5.1.15}) is bounded.
So every convergent sequence of real numbers is bounded.
\end{corollary}

\begin{example} \label{example 6.1.18}
The sequence \(1, 2, 3, 4, 5,...\) is not bounded, and hence (by contrapositive of \CORO{6.1.17}) is not convergent.
\end{example}

We can now prove the usual limit laws.
\begin{theorem} [Limit Laws] \label{thm 6.1.19}
Let \((a_n)_{n = m}^\infty\) and \((b_n)_{n = m}^\infty\) be convergent sequences of real numbers, and let \(x, y\) be the real numbers \(x := \lim_{n \to \infty} a_n\) and \(y := \lim_{n \to \infty} b_n\).
\begin{enumerate}
\item
    The sequence \((a_n + b_n)_{n = m}^\infty\) converges to \(x + y\);
    in other words,
    \[
        \lim_{n \to \infty} (a_n + b_n) = \lim_{n \to \infty} a_n + \lim_{n \to \infty} b_n.
    \]
\item
    The sequence \((a_n b_n)_{n = m}^\infty\) converges to \(xy\);
    in other words,
    \[
        \lim_{n \to \infty} (a_n b_n) = (\lim_{n \to \infty} a_n)(\lim_{n \to \infty} b_n).
    \]
\item
    For any real number \(c\), the sequence \((c a_n)_{n = m}^\infty\) converges to \(cx\);
    in other words,
    \[
        \lim_{n \to \infty} (c a_n) = c(\lim_{n \to \infty} a_n).
    \]
\item
    The sequence \((a_n - b_n)_{n = m}^\infty\) converges to \(x - y\);
    in other words,
    \[
        \lim_{n \to \infty} (a_n - b_n) = \lim_{n \to \infty} a_n - \lim_{n \to \infty} b_n.
    \]
\item
    Suppose that \(y \neq 0\), and that \(b_n \neq 0\) for all \(n \geq m\).
    Then the sequence \((b_n^{-1})_{n = m}^\infty\) converges to \(y^{-1}\);
    in other words,
    \[
        \lim_{n \to \infty} b_n^{-1} = (\lim_{n \to \infty} b_n)^{-1}.
    \]
\item
    Suppose that \(y \neq 0\), and that \(b_n \neq 0\) for all \(n \geq m\).
    Then the sequence \((a_n / b_n)_{n = m}^\infty\) converges to \(x / y\);
    in other words,
    \[
        \lim_{n \to \infty} \frac{a_n}{b_n} = \frac{\lim_{n \to \infty} a_n}{\lim_{n \to \infty} b_n}.
    \]
\item
    The sequence \((\max(a_n, b_n))_{n = m}^\infty\) converges to \(\max(x, y)\);
    in other words,
    \[
        \lim_{n \to \infty} \max(a_n, b_n) = \max(\lim_{n \to \infty} a_n, \lim_{n \to \infty} b_n).
    \]
\item
    The sequence \((\min(a_n, b_n))_{n = m}^\infty\) converges to \(\min(x, y)\);
    in other words,
    \[
        \lim_{n \to \infty} \min(a_n, b_n) = \min(\lim_{n \to \infty} a_n, \lim_{n \to \infty} b_n).
    \]
\end{enumerate}
\end{theorem}

\begin{note}
\begin{enumerate}
\item 各項相加而成的\ seq 的極限等於兩個\ seq 的極限相加。 
\item 各項相乘而成的\ seq 的極限等於兩個\ seq 的極限相乘。
\item 逐項乘以常數而成的\ seq 的極限等於\ seq 的極限乘以該常數。
\item 各項相減而成的\ seq 的極限等於兩個\ seq 的極限相減。
\item 若原\ seq 每一項都\ \(!= 0\) 且該\ seq 不會收斂到\ 0,
則逐項取倒數而成的\ seq 的極限等於原本的\ seq 的極限的倒數。
\item 若要被當分母的\ seq 每一項都\ \(!= 0\) 且該\ seq 不會收斂到\ 0,則各項相除而成的\ seq 的極限等於兩個\ seq 的極限相除。
\item 各項取最大值而成的\ seq 的極限等於兩個\ seq 的極限的最大值。
\item 各項取最小值而成的\ seq 的極限等於兩個\ seq 的極限的最小值。
\end{enumerate}
\end{note}

\begin{proof} (a)

Then by \DEF{6.1.5}, we have to show that \(\forall \varE > 0, \exists N \le n\) s.t. \(\abs{(a_i + b_i) - (x + y)} \le \varE\ \forall i \ge N\) \BLUE{(*)}.

So let arbitrary \(\varE > 0\).
In particular \(\varE/2 > 0\).
Since \((a_n)_{n = m}^{\infty}\) converges to \(x\), we can find \(N_1 \ge m\) s.t. \(\abs{a_i - x} \le \varE/2\ \forall i \ge N_1\) \MAROON{(1)}.
Similarly we can find \(N_2 \ge m\) s.t. \(\abs{b_j - y} \le \varE/2\ \forall j \ge N_2\) \MAROON{(2)}.
Now let \(N = \max(N_1, N_2)\).
Then from \MAROON{(1)(2)} we have
\begin{align*}
             & \abs{a_k - x} \le \varE/2 \land \abs{b_k - y} \le \varE/2\ \forall k \ge N \\
    \implies & \abs{a_k - x} + \abs{b_k - y} \le \varE/2 + \varE/2 = \varE\ \forall k \ge N \\
    \implies & \abs{(a_k - x) + (b_k - y)} \le \abs{a_k - x} + \abs{b_k - y} \le \varE\ \forall k \ge N & \text{by \PROP{4.3.3}(b)} \\
    \implies & \abs{(a_k - x) + (b_k - y)} \le \varE\ \forall k \ge N \\
    \implies & \abs{(a_k + b_k) - (x + y)} \le \varE\ \forall k \ge N
\end{align*}
satisfying \BLUE{(*)} as desired.
\end{proof}

\begin{proof} (b)

By \DEF{6.1.5} we have to show that \(\forall \varE > 0, \exists N \ge n\) s.t. \(\abs{a_n b_n - xy} \le \varE\) \BLUE{(1)}.

So let arbitrary \(\varE > 0\).
Since \((a_n)_{n = m}^{\infty}\) converges to \(x\) and \((b_n)_{n = m}^{\infty}\) converges to \(y\), by \CORO{6.1.17}, they are bounded.
So by \DEF{6.1.16} we can find \(A, B > 0\) s.t. \(\abs{a_n} \le A\) \MAROON{(*1)} and \(\abs{b_n} \le B\) \MAROON{(*2)} for all \(n \ge m\).

And in particular, from \MAROON{(*1)} we have \(x \le A\) \MAROON{(*3)}, otherwise if \(x > A\), then \(x - A > 0\), hence for \(\varE = (x - A)/2 > 0\), given any \(N \ge m\), we just have found \(N\) s.t. 
\begin{align*}
    \abs{a_N - x} & = \abs{x - a_N} \\
                  & = \abs{x + (-a_N)} \\
                  & \ge \abs{x} - \abs{-a_N} & \text{by \AC{4.3.1}} \\
                  & \ge \abs{x} - \abs{a_N} \\
                  & \ge x - \abs{a_N} & \text{since \(x > A > 0\)} \\
                  & \ge x - A & \text{since by \MAROON{(*1)} \(\abs{a_N} \le A\)} \\
                  & > (x - A)/2 = \varE
\end{align*}
which implies \((a_n)_{n = m}^{\infty}\) does \emph{not} converge to \(x\), a contradiction.

Now in particular \(\varE/2A > 0\) and \(\varE/2B > 0\).
And since \((a_n)_{n = m}^{\infty}\) converges to \(x\) and \((b_n)_{n = m}^{\infty}\) converges to \(y\),
\(\exists N_1\) s.t. \(\abs{a_i - x} \le \varE/2\RED{B}\) for all \(i \ge N_1\) \MAROON{(**1)}, and \(\exists N_2\) s.t. \(\abs{b_j - y} \le \varE/2\RED{A}\) for all \(j \ge N_2\) \MAROON{(**2)}.

Now let \(N = \max(N_1, N_2)\) then for all \(n \ge N\),
\begin{align*}
    \abs{a_n b_n - xy} & = \abs{a_n b_n - xy + x b_n - x b_n}  \\
                       & = \abs{a_n b_n - x b_n + x b_n - xy} \\
                       & = \abs{b_n(a_n - x) + x(b_n - y)} \\
                       & \le \abs{b_n(a_n - x)} + \abs{x(b_n - y)} & \text{by \PROP{4.3.3}(b)} \\
                       & = \abs{b_n}\abs{a_n - x} + \abs{x}\abs{b_n - y} \\
                       & \le B \X \varE/2\RED{B} + \abs{x}\abs{b_n - y} & \text{by \MAROON{(*2) (**1)}} \\
                       & \le \varE/2 + A \X \varE/2\RED{A} & \text{by \MAROON{(*3) (**2)}} \\
                       & = \varE,
\end{align*}
which satisfies \BLUE{(1)}, as desired.
\end{proof}

\begin{proof} (c)
For a constant \(c\), \(c = \lim_{n \toINF} b_n\) where \(b_n = c\ \forall n \ge m\) \MAROON{(*)}. So
\begin{align*}
    \lim_{n \toINF} c a_n & = \lim_{n \toINF} b_n a_n & \text{by \MAROON{(*)}} \\
                          & = \lim_{n \toINF} b_n \lim_{n \toINF} a_n & \text{by part (b)} \\
                          & = cx
\end{align*}
\end{proof}

\begin{proof} (d)
\begin{align*}
\lim_{n \toINF}(a_n - b_n) & = \lim_{n \toINF}(a_n + (-b_n)) & \text{by algebra} \\
                           & = \lim_{n \toINF}(a_n + (-1) b_n) & \text{by algebra} \\
                           & = \lim_{n \toINF} a_n + \lim_{n \toINF}(-1) b_n & \text{by part(a)} \\
                           & = \lim_{n \toINF} a_n + (-1)\lim_{n \toINF} b_n & \text{by part(c)} \\
                           & = \lim_{n \toINF} a_n - \lim_{n \toINF} b_n & \text {by algebra}
\end{align*}
\end{proof}

\begin{proof} (e)
From the hint, we first show that \((b_n)_{n = m}^{\infty}\) is bounded away from zero.
But that is equivalent to \((b_n^{-1})_{n = m}^{\infty}\) is \emph{bounded}:
Suppose \((b_n)_{n = m}^{\infty}\) is bounded away from zero.
Then \(\exists M > 0\) s.t. \(\abs{b_n} \ge M\ \forall n \ge m\), which implies \(\frac1{\abs{b_n}} \le \frac1{M}\ \forall n \ge m\).
But it's trivial that \(\frac1{\abs{b_n}} = \abs{\frac1{b_n}} = \abs{b_n^{-1}}\).
So we have \(\abs{b_n^{-1}} \le \frac1{M}\ \forall n \ge m\), and by \DEF{6.1.16}, \((b^{-1})_{n = m}^{\infty}\) is bounded.
Now suppose \((b_n^{-1})_{n = m}^{\infty}\) is bounded.
Then from \DEF{6.1.16} \(\exists M > 0\) s.t. \(\abs{b^{-1}} \le M\ \forall n \ge m\), which again implies \(\frac1{\abs{b_n}} \le M\ \forall n \ge m\);
which implies \(\abs{b_n} \ge \frac1{M}\ \forall n \ge m\).
So we have found \(\frac1{M} > 0\) s.t. \(\abs{b_n} \ge \frac1{M}\ \forall n \ge m\). So \((b_n)_{n = m}^{\infty}\) is bounded away from zero.

Now, we first show \((b_n^{-1})_{n = m}^{\infty}\) is bounded.
Since \(y \neq 0\), \(\abs{y} > 0\) and \(\abs{y}/2 > 0\).
Also, since \((b_n)_{n = m}^{\infty}\) converges to \(y\), \(\exists N \ge m\) s.t. \(\abs{b_n - y} \le \abs{y}/2\ \forall n \ge N\).
But since \(\abs{y} - \abs{b_n} \le \abs{b_n - y}\) (by some absolute value algebra and \AC{4.3.1}), we have \(\abs{y} - \abs{b_n} \le \abs{y}/2\),
or \(\abs{y}/2 \le \abs{b_n}\), which implies \(2/\abs{y} \ge 1/\abs{b_n} = \abs{1/b_n}\ \forall n \ge N\), or \(\abs{b_n^{-1}} \le 2/\abs{y}\ \forall n \ge N\).
So currently we have shown \(b_n^{-1}\) is bounded for \(n \ge N\).
But by \LEM{5.1.14}, the \emph{finite} \((b_n^{-1})_{n = m}^{N - 1}\) is also bounded.
Let them bounded by \(M_0\).
Then the whole sequence \((b_n^{-1})_{n = m}^{\infty}\) is bounded by \(\max(\frac{2}{\abs{y}}, M_0)\); label it as \(M\) \MAROON{(*)}.

Now we start the actual proof.
That is, \(\forall \varE > 0, \exists N \ge m\) s.t. \(\abs{b_n^{-1} - y^{-1}} \le \varE\ \forall n \ge N\).
So let arbitrary \(\varE > 0\).
Since \((b_n)_{n = m}^{\infty}\) converges to \(y\), and in particular \(\frac{\varE \abs{y}}{M} > 0\),
\(\exists N_1 \ge m\) s.t. \(\abs{b_n - y} \le \frac{\varE \abs{y}}{M}\ \forall n \ge N_1\) \MAROON{(**)}.
Then, given this \(N_1 \ge m\), we have for all \(n \ge N_1\),
\begin{align*}
    \abs{b_n^{-1} - y^{-1}} & = \abs{\frac{1}{b_n} - \frac{1}{y}} \\
                            & = \abs{\frac{y - b_n}{b_n y}} \\
                            & = \frac{1}{\abs{b_n y}} \abs{y - b_n} & \text{by absolute value algebra} \\
                            & = \frac{1}{\abs{b_n} \abs{y}} \abs{y - b_n} & \text{by absolute value algebra} \\
                            & = \abs{\frac{1}{b_n}} \frac{1}{\abs{y}} \abs{y - b_n} & \text{by absolute value algebra} \\
                            & \le M \frac{1}{\abs{y}} \abs{y - b_n} & \text{by \MAROON{(*)}} \\
                            & = M \frac{1}{\abs{y}} \abs{b_n - y} & \text{by absolute value algebra} \\
                            & \le M \frac{1}{\abs{y}} \frac{\varE \abs{y}}{M} & \text{by \MAROON{(**)}} \\
                            & = \varE,
\end{align*}
as desired.
\end{proof}

\begin{proof}(f)
\begin{align*}
\lim_{n \toINF} \frac{a_n}{b_n} & = \lim_{n \toINF} a_n b_n^{-1} & \text{by algebra} \\
                                & = \lim_{n \toINF} a_n \lim_{n \toINF} b_n^{-1} & \text {by part(b)} \\
                                & = (\lim_{n \toINF} a_n) (\frac{1}{\lim_{n \toINF} b_n}) & \text {by part(e)} \\
                                & = \frac{\lim_{n \toINF} a_n}{\lim_{n \toINF} b_n} & \text{of course}
\end{align*}
\end{proof}

\begin{proof}(g)
We split into two cases: \(x \ge y\) and \(x < y\).
\begin{itemize}
\item[\(x \ge y\)]: Then \(\max(x, y) = x\).
    So we have to show \((\max(a_n, b_n))_{n = m}^{\infty}\) converges to \(x\).
    That is, \(\forall \varE > 0, \exists N \ge m\) s.t. \(\abs{\max(a_n, b_n) - x} \le \varE\ \forall n \ge N\).
    
    Sot let arbitrary \(\varE > 0\).
    Since \((a_n)_{n = m}^{\infty}\) converges to \(x\), \(\exists N_1 \ge m\) s.t. \(\abs{a_n - x} \le \varE\ \forall n \ge N_1\) \MAROON{(*)}.
    Since \((b_n)_{n = m}^{\infty}\) converges to \(y\), \(\exists N_2 \ge m\) s.t. \(\abs{b_n - y} \le \varE\ \forall n \ge N_2\) \MAROON{(**)}.
    Now let \(N = \max(N_1, N_2)\).
    Then with \MAROON{(*)(**)} and \EXEC{5.4.6}, \(-\varE \le a_n - x \le \varE \land -\varE \le b_n - y \le \varE\ \forall n \ge N\),
    or \(x - \varE \le a_n \le x + \varE \land y -\varE \le b_n \le y + \varE\ \forall n \ge N\) \MAROON{(***)}.
    
    But since \(x \ge y\), we have \(x + \varE \ge y + \varE\), so from \MAROON{(***)} we have \(a_n \le x + \varE\) and \(b_n \le y + \varE \le x + \varE\).
    So both \(a_n, b_n \le x + \varE\), so \(\max(a_n, b_n) \le x + \varE\) \BLUE{(*)}.
    Also from \MAROON{(***)}, we know \(x - \varE \le a_n\), which \(\le \max(a_n, b_n)\), so we have \(x - \varE \le \max(a_n, b_n)\) \BLUE{(**)}.
    By \BLUE{(*)(**)} we have \(x - \varE \le \max(a_n, b_n) \le x + \varE\).
    By \EXEC{5.4.6}, we have \(\abs{\max(a_n, b_n) - x} \le \varE\).
    So we have found \(N \ge m\) s.t. \(\abs{\max(a_n, b_n) - x} \le \varE\ \forall n \ge N\), as desired.
\item[\(x < y\)]: Then \(\max(x, y) = y\).
    So similarly we have to show \(\forall \varE > 0, \exists N \ge m\) s.t. \(\abs{\max(a_n, b_n) - y} \le \varE\ \forall n \ge N\).
    
    So let arbitrary \(\varE > 0\).
    Again, with the same argument up to \MAROON{(***)}, we have \(x - \varE \le a_n \le x + \varE \land y -\varE \le b_n \le y + \varE\ \forall n \ge N\) \MAROON{(****)}.
    
    Since \(x < y\), we have \(x + \varE < y + \varE\).
    So from \MAROON{(****)}, we have \(a_n \le x + \varE < y + \varE\) and \(b_n \le y + \varE\).
    So both \(a_n, b_n \le y + \varE\), so \(\max(a_n, b_n) \le y + \varE\) \BLUE{(***)}.
    Also, from \MAROON{(****)}, we have \(y - \varE \le b_n\), which implies \(y - \varE \le \max(a_n, b_n)\) \BLUE{(****)}.
    So from \BLUE{(***)(****)} we have \(y - \varE \le \max(a_n, b_n) \le y + \varE\).
    By \EXEC{5.4.6}, we get \(\abs{\max(a_n, b_n) - y} \le \varE\).
    So we have found \(N \ge m\) s.t. \(\abs{\max(a_n, b_n) - y} \le \varE\ \forall n \ge N\), as desired.
\end{itemize}
\end{proof}

\begin{proof}(h)
First we give a fact that \(\min(a, b) = -\max(-a, -b)\) \BLUE{(*)};
we split into three cases:
\begin{itemize}
    \item[\(a > b\)]: then \(\min(a, b) = b\).
        Also, \(-a < -b\), so \(\max(-a, -b) = -b\), so \(-\max(-a, -b) = -(-b) = b\).
    \item[\(a = b\)]: then \(\min(a, b) = b\).
        Also, \(-a = -b\), so \(\max(-a, -b) = -b\), so \(-\max(-a, -b) = -(-b) = b\).
    \item[\(a < b\)]: then \(\min(a, b) = a\).
        Also, \(-a > -b\), so \(\max(-a, -b) = -a\), so \(-\max(-a, -b) = -(-a) = a\).
\end{itemize}
So in all cases, LHS = RHS. That is, \(\min(a, b) = -\max(-a, -b)\).

So we have
\begin{align*}
    \lim_{n \toINF} \min(a_n, b_n) & = \lim_{n \toINF} -\max(-a_n, -b_n) & \text{by \BLUE{(*)}} \\
                                   & = -\lim_{n \toINF} \max(-a_n, -b_n) & \text{by part(c)} \\
                                   & = -\max(\lim_{n \toINF} -a_n, \lim_{n \toINF} -b_n) & \text{by part(g)} \\
                                   & = -\max(-\lim_{n \toINF} a_n, -\lim_{n \toINF} b_n) & \text{by part(c)} \\
                                   & = \min(\lim_{n \toINF} a_n, \lim_{n \toINF} b_n) & \text{by \BLUE{(*)}}
\end{align*}
\end{proof}

\exercisesection

\begin{exercise} \label{exercise 6.1.1}
Let \((a_n)_{n = 0}^{\infty}\) be a sequence of real numbers, such that \(a_{n + 1} > a_n\) for each natural number \(n\).
Prove that whenever \(n\) and \(m\) are natural numbers such that \(m > n\), then we have \(a_m > a_n\).
(We refer to these sequences as \emph{increasing} sequences.)
\end{exercise}

\begin{note}
每一項的下一項都比自己大,則比自己還要後面的任何項都比自己大。
\end{note}

\begin{proof}
We prove this by induction.

What we want: Given any natural number \(n\), \(a_{n + k} > a_n\) for any natural number \(k \ge 1\). So let \(n\) be arbitrary natural number.

For base case \(k = 1\), \(a_{n + 1} > a_n\) is guaranteed by the property of the sequence.

Suppose for some natural number \(k \ge 1\), \(a_{n + k} > a_n\);
we have to show \(a_{(n + 1) + k} > a_n\).
But again by the property of the sequence we have \(a_{n + k + 1} > a_{n + k}\), or \(a_{(n + 1) + k} > a_{n + k}\).
With the hypothesis \(a_{n + k} > a_n\), by order transitivity  we have \(a_{(n + 1) + k} > a_n\), as desired.
This closes the induction.

So given any two natural number \(m > n\), \(m = n + k\) for some positive natural number, so by the statement we have shown, \(a_{n + k} > a_n\), or \(a_m > a_n\).
\end{proof}

\begin{exercise} \label{exercise 6.1.2}
Let \((a_n)_{n = m}^{\infty}\) be a sequence of real numbers, and let \(L\) be a real number.
Show that \((a_n)_{n = m}^{\infty}\) converges to \(L\) if and only if,
given any real \(\varE > 0\), one can find an \(N \ge m\) such that \(\abs{a_n - L} \le \varE\) for all \(n \ge N\).
\end{exercise}

\begin{align*}
         & (a_n)_{n = m}^{\infty} \text{ converges to } L \\
    \iff & \forall \varE > 0, (a_n)_{n = m}^{\infty} \text{ is eventually } \varE\text{-close to } L & \text{by \DEF{6.1.5} \MAROON{DEF(3)}} \\
    \iff & \forall \varE > 0, \exists N \ge n \text{ s.t. } (a_n)_{n = N}^{\infty} \text{ is } \varE\text{-close to } L & \text{by \DEF{6.1.5} \MAROON{DEF(2)}} \\
    \iff & \forall \varE > 0, \exists N \ge n \text{ s.t. } \abs{a_n - L} \le \varE\ \forall n \ge N & \text{by \DEF{6.1.2}}
\end{align*}

\begin{exercise} \label{exercise 6.1.3}
Let \((a_n)_{n = m}^{\infty}\) be a sequence of real numbers, let \(c\) be a real number, and let \(m' \ge m\) be an integer.
Show that \((a_n)_{n = m}^{\infty}\) converges to \(c\) if and only if \((a_n)_{n = m'}^{\infty}\) converges to \(c\).
\end{exercise}

\begin{proof}
Suppose \((a_n)_{n = m}^{\infty}\) converges to \(c\), we have to show \((a_n)_{n = m'}^{\infty}\) converges to \(c\).

So given arbitrary \(\varE > 0\), we have to find an integer \(N \ge m'\) s.t. \(\abs{a_n - c} \le \varE\) for all \(n \ge N\).
But since \((a_n)_{n = m}^{\infty}\) converges to \(c\), by \DEF{6.1.5}(or unwrapped \EXEC{6.1.2}), we can find an \(N_1 \ge m\) s.t. \(\abs{a_n - c} \le \varE\ \forall n \ge N_1\).
Now there are two cases: \(m' \ge N_1\) or \(m' < N_1\).
\begin{itemize}
    \item[\(m' \ge N_1\)]:
        Then since \(\abs{a_n - c} \le \varE\ \forall n \ge N_1\), we of course have \(\abs{a_n - c} \le \varE\ \forall n \ge m'\).
        So by \DEF{6.1.5} \((a_n)_{n = m'}^{\infty}\) is \(\varE\)-close (not just eventually \(\varE\)-close) to \(c\).
    \item[\(m' < N_1\)]:
        Then again we have found \(N_1 \ge m'\) s.t. \(\abs{a_n - c} \le \varE\ \forall n \ge N_1\).
        So by \DEF{6.1.5} \((a_n)_{n = m'}^{\infty}\) is eventually \(\varE\)-close) to \(c\).
\end{itemize}
So in all cases, \((a_n)_{n = m'}^{\infty}\) is (eventually) \(\varE\)-close to \(c\).
Since \(\varE > 0\) is arbitrary, by \DEF{6.1.5} \((a_n)_{n = m'}^{\infty}\) converges to \(c\).

Now suppose \((a_n)_{n = m'}^{\infty}\) converges to \(c\), we have to show \((a_n)_{n = m}^{\infty}\) converges to \(c\).
So let arbitrary \(\varE > 0\).
Since \((a_n)_{n = m'}^{\infty}\) converges to \(c\), we can find an integer \(N_2 \ge m'\) s.t. \(\abs{a_n - c} \le \varE\) for all \(n \ge N_2\).
But since \(m' \ge m\), we have \(N_2 \ge m\).
So we just also find an integer \(N_2 \ge m\) s.t. \(\abs{a_n - c} \le \varE\) for all \(n \ge N_2\).
So by \DEF{6.1.5} \((a_n)_{n = m}^{\infty}\) is eventually \(\varE\)-close to \(c\).
Since \(\varE > 0\) is arbitrary, by \DEF{6.1.5}, \((a_n)_{n = m}^{\infty}\) converges to \(c\).
\end{proof}

\begin{exercise} \label{exercise 6.1.4}
Let \((a_n)_{n = m}^{\infty}\) be a sequence of real numbers, let \(c\) be a real number, and let \(k \ge 0\) be a non-negative integer.
Show that \((a_n)_{n = m}^{\infty}\) converges to \(c\) if and only if \((a_{n + k})_{n = m}^{\infty}\) converges to \(c\).
\end{exercise}

\begin{proof}
With some reindexing, we have \BLUE{\((a_{n + k})_{n = m}^{\infty} = (a_n)_{n = m + k}^{\infty}\)}.
Also since \(m + k \geq m\), by \EXEC{6.1.3} \((a_n)_{n = m}^{\infty}\) converges to \(c\) if and only if \BLUE{\((a_n)_{n = m + k}^{\infty}\)} converges to \(c\).
Thus \((a_n)_{n = m}^{\infty}\) converges to \(c\) if and only if \BLUE{\((a_{n + k})_{n = m}^{\infty}\)} converges to \(c\).
\end{proof}

\begin{exercise} \label{exercise 6.1.5}
Prove \PROP{6.1.12}.
(Hint: use the triangle inequality, or \PROP{4.3.7}.)
\end{exercise}

\begin{proof}
See \PROP{6.1.12}.
\end{proof}

\begin{exercise} \label{exercise 6.1.6}
Prove \PROP{6.1.15}, using the following outline.
Let \((a_n)_{n = m}^{\infty}\) be a Cauchy sequence of rationals, and write \(L := \LIM_{n \toINF} a_n\).
We have to show that \((a_n)_{n = m}^{\infty}\) converges to \(L\).
Let \(\varE > 0\).
Assume for sake of contradiction that sequence an is not eventually \(\varE\)-close to \(L\).
Use this, and the fact that \((a_n)_{n = m}^{\infty}\) is Cauchy, to show that there is an \(N \ge m\) such that either \(a_n > L + \varE/2\) \emph{for all} \(n \ge N\), or \(a_n < L - \varE/2\) for all \(n \ge N\).
Then use \EXEC{5.4.8}.
\end{exercise}

\begin{proof}
See \PROP{6.1.15}.
\end{proof}

\begin{exercise} \label{exercise 6.1.7}
Show that \DEF{6.1.16} is consistent with \DEF{5.1.12}
(i.e., prove an analogue of \PROP{6.1.4} for bounded sequences instead of Cauchy sequences).
\end{exercise}

\begin{proof}
I think the point is, the sequence of \emph{rationals} is bounded by a \emph{real} if and only if the sequence is bounded by a \emph{rational}.

So let \((a_n)_{n = m}^{\infty}\) be a sequence of \emph{rationals}.

Suppose the sequence is bounded by a \emph{real} number \(M\).
Then by \CORO{5.4.13} for \(\varE = 1\) we can find a positive integer \(N\) s.t. \(N \varE > M\), or \(N \X 1 > M\), or \(N > M\);
and of course the sequence is bounded by \(N\).
But \(N\) is integer, so it's \emph{rational}, so we have found a rational by which the sequence if bounded.

Now suppose the sequence is bounded by a \emph{rational} number \(M\).
Then \(M\) is just also a \emph{real} number, so the sequence is bounded by a real number.
\end{proof}

\begin{exercise} \label{exercise 6.1.8}
Prove \THM{6.1.19}.
(Hint: you can use some parts of the theorem to prove others, e.g., (b) can be used to prove (c);
(a), (c) can be used to prove (d);
and (b), (e) can be used to prove (f).
The proofs are similar to those of \LEM{5.3.6}, \PROP{5.3.10}, and \LEM{5.3.15}.
For (e), you may need to first prove the \emph{auxiliary result} that any sequence whose elements are non-zero, and which converges to a non-zero limit, is \emph{bounded away from zero}.)
\end{exercise}

\begin{proof}
See \THM{6.1.19}.
\end{proof}

\begin{exercise} \label{exercise 6.1.9}
Explain why \THM{6.1.19}(f) fails when the limit of the denominator is \(0\).
(To repair that problem requires L’\^{H}opital’s rule, see \SEC{10.5}.)
\end{exercise}

\begin{proof}
For the sake of contradiction suppose \THM{6.1.19}(f) also works when \((b_n)_{n = m}^{\infty}\) converges to \(0\).
Then in particular, let \(a_n = b_n := 1/n\) for all \(n \ge m\); we have shown in \EXEC{5.3.5} that both \((a_n)_{n = m}^{\infty}\) and \((b_n)_{n = m}^{\infty}\) converge to \(0\).
And \(\lim_{n \toINF} \frac{a_n}{b_n} = 1\) since \(\frac{a_n}{b_n} = \frac{\frac{1}{n}}{\frac{1}{n}} = 1\) for all \(n \ge m\).
But \(\frac{\lim_{n \toINF} a_n}{\lim_n \toINF b_n} = \frac{0}{0}\), which is undefined.
So the two sides of equation is \emph{not} consistent(and the RHS is even undefined).
So \THM{6.1.19}(f) cannot be applied when the sequence in the denominator converges to \(0\).
\end{proof}

\begin{exercise} \label{exercise 6.1.10}
Show that the concept of equivalent Cauchy sequence, as defined in \DEF{5.2.6}, does not change if \(\varE\) is required to be positive \emph{real} instead of positive \emph{rational}.
More precisely, if \((a_n)_{n = 0}^{\infty}\) and \((b_n)_{n = 0}^{\infty}\) are sequences of reals, show that \((a_n)_{n = 0}^{\infty}\) and \((b_n)_{n = 0}^{\infty}\) are eventually \(\varE\)-close for every \emph{rational} \(\varE > 0\) if and only if they are eventually \(\varE\)-close for every \emph{real} \(\varE > 0\).
(Hint: modify the proof of \PROP{6.1.4}.)
\end{exercise}

\begin{note}
It's mind=blowing that we have not rigorously defined the equivalence between to Cauchy sequences of \emph{real}s.
What we have defined is that they can converge to some real numbers, and the real number may be the same.
But with \EXEC{6.1.10}, the operator in \THM{6.1.19} becomes \emph{well-defined}(i.e. satisfying \AXM{a.7.4}), just replace the \(\LIM\) with \(\lim\) (it's also valid by \PROP{6.1.15}) in the definition of these operator in \CH{5}, i.e. \DEF{5.3.4}, \DEF{5.3.9}, \DEF{5.3.16}.
\end{note}

\begin{proof}
Let \((a_n)_{n = 0}^{\infty}\) and \((b_n)_{n = 0}^{\infty}\) be sequences of reals.

Suppose \((a_n)_{n = 0}^{\infty}\) and \((b_n)_{n = 0}^{\infty}\) are eventually \(\varE\)-close for every \emph{rational} \(\varE > 0\).
Now let \(\varE'\) be arbitrary \emph{real} s.t. \(\varE' > 0\).
Then by \PROP{5.4.12} we can find a \emph{rational} \(\varE\) s.t. \(0 < \varE < \varE'\).
And by supposition, \((a_n)_{n = 0}^{\infty}\) and \((b_n)_{n = 0}^{\infty}\) are eventually \(\varE\)-close for this \emph{rational}.
But since \(\varE < \varE'\), \((a_n)_{n = 0}^{\infty}\) and \((b_n)_{n = 0}^{\infty}\) are of course eventually \(\varE'\)-close for the real \(\varE'\).
Since \(\varE' > 0\) is arbitrary, \((a_n)_{n = 0}^{\infty}\) and \((b_n)_{n = 0}^{\infty}\) are eventually \(\varE\)-close for every \(real > 0\).

Now suppose \((a_n)_{n = 0}^{\infty}\) and \((b_n)_{n = 0}^{\infty}\) are eventually \(\varE\)-close for every \emph{real} \(\varE > 0\).
Now given any \emph{rational} \(\varE' > 0\), since \(\varE'\) are also real, by supposition \((a_n)_{n = 0}^{\infty}\) and \((b_n)_{n = 0}^{\infty}\) are eventually \(\varE'\)-close.
Since \(\varE' > 0\) is arbitrary, \((a_n)_{n = 0}^{\infty}\) and \((b_n)_{n = 0}^{\infty}\) are eventually \(\varE\)-close for every \emph{rational} \(\varE > 0\).
\end{proof}
\section{The Extended real number system} \label{sec 6.2}

To deal with the sequences that does \emph{not} converge, e.g.
\[
    1, 2, 3, 4, 5,...
\]
\[
    -1, -2, -3, -4, -5,...
\]
\[
    1, -1, 1, -1, 1, -1,...
\]
(, which does not converge to anything but has \(-1\) and \(1\) as ``limit point'')
\[
    1, -2, 3, -4, 5, -6,...
\]
we need to talk about something called the \emph{extended real number system}.

\begin{definition} [Extended real number system] \label{def 6.2.1}
The \emph{extended real number system} \(\SET{R}^*\) is the real line \(\SET{R}\) with two additional elements attached, called \(+\infty\) and \(-\infty\).
These elements are \emph{distinct} from each other and \emph{also distinct} from every real number.
An extended real number \(x\) is called \emph{finite} iff it is a real number,
and \emph{infinite} iff it is equal to \(+\infty\) or \(-\infty\).
(This definition is \emph{not} directly related to the notion of finite and infinite sets in \SEC{3.6}, though it is of course similar in spirit.)
\end{definition}

\begin{note}
注意,\(+\infty\) or \(-\infty\) 各自都是「一個」符號,而不是\ \(+\) 作用在\ \(\infty\) 之類的;在\ \DEF{6.2.1} 之下,\(\infty\) 是無意義的東西。這件事情不搞清楚的話,\DEF{6.2.2} 就會感到問號。
\end{note}

These new symbols, \(+\infty\) and \(-\infty\), at present do not have much meaning, since we have no operations to manipulate them
(other than equality \(=\) and inequality \(\neq\)).
Now we place a few operations on the extended real number system.

\begin{definition} [Negation of extended reals] \label{def 6.2.2}
The operation of negation \(x \to -x\) on \(\SET{R}\), we now extend to \(\SET{R}^*\) by defining \(-(+\infty) := -\infty\) and \(-(-\infty) := +\infty\).
\end{definition}

Thus every \emph{extended} real number \(x\) has a negation, and \(-(-x)\) is always equal to \(x\) \MAROON{(*)}.

\begin{note}
\MAROON{(*)} is tricky.
It does \emph{not} mean \(-(-\infty) = \infty\), in fact \(\infty\) is undefined in \DEF{6.2.1}.
It in fact means \(-(-(\RED{-\infty})) = \RED{-\infty}\) and \(-(-(\RED{+\infty})) = +\infty\), why?
\begin{align*}
    -(-(\RED{-\infty})) & = -(+\infty) & \text{by \DEF{6.2.2}} \\
                        & = -\infty & \text{by \DEF{6.2.2}}
\end{align*}
and
\begin{align*}
    -(-(\RED{+\infty})) & = -(-\infty) & \text{by \DEF{6.2.2}} \\
                        & = +\infty & \text{by \DEF{6.2.2}}
\end{align*}
\end{note}

\begin{definition} [Ordering of extended reals] \label{def 6.2.3}
Let \(x\) and \(y\) be \emph{extended real} numbers.
We say that \(x \le y\), i.e., \(x\) is less than or equal to \(y\), iff \textbf{one of} the following three statements is true:
\begin{enumerate}
    \item \(x\) and \(y\) are \emph{real} numbers, and \(x \le y\) as \emph{real} numbers.
    \item \(y = +\infty\).
    \item \(x = -\infty\).
\end{enumerate}
We say that \(x < y\) if we have \(x \le y\) and \(x \ne y\).
We sometimes write \(x < y\) as \(y > x\), and \(x \le y\) as \(y \ge x\).
\end{definition}

\begin{note}
注意一下最後一句話;對於\ \(\SET{R}^*\),我們「又」把\ \(x < y\) 定義成\ \(x \le y \land x \ne y\) 了。
在這本書裡,自然數跟整數的\ order 也是這樣定義的;但是有理數跟實數就是定義成相減為正數(所以他們的定義相依於正數的定義)。
另外對於自然數、整數、\(\SET{R}^*\),\(x < y\) 跟\ \(y > x\) 我們直接定義成是「一樣的」;
但對於有理數跟實數,\(x < y\) 跟\ \(y > x\) 等價是需要證明的。
\end{note}

\begin{example} \label{example 6.2.4}
\(3 \le 5\) (satisfying \DEF{6.2.3}(a)), \(3 < +\infty\) (satisfying (b)), and \(-\infty < +\infty\) (satisfying (b), (c)), but \(3 \not \le -\infty\) (all (a), (b), (c) are not satisfied).
\end{example}

\begin{proposition} \label{prop 6.2.5}
Let \(x, y, z\) be \emph{extended real} numbers.
Then the following statements are true:
\begin{enumerate}
    \item (Reflexivity) We have \(x \le x\).
    \item (Trichotomy) Exactly one of the statements \(x < y\), \(x = y\), or \(x > y\) is true.
    \item (Transitivity) If \(x \le y\) and \(y \le z\), then \(x \le z\).
    \item (Negation reverses order) If \(x \le y\), then \(-y \le -x\).
\end{enumerate}
\end{proposition}

\begin{proof}
We already know the \emph{real numbers} has these properties, so we only need to prove the cases that one of \(x, y, z\) is \(+\infty\) or \(-\infty\).

Reflexivity: if \(x = +\infty\), then by \DEF{6.2.3}(b) \(x \le x\). if \(x = -\infty\), then by \DEF{6.2.3}(c), \(x \le x\), as desired.

Trichotomy: we prove by cases, and in each case we show ``at least'' and ``at most'' one of the trichotomy is true:
\begin{itemize}
\item \(x \in \SET{R}\) and \(y \in \SET{R}\): This fallbacks to \PROP{4.2.9}(a).
\item \(x \in \SET{R}\) and \(y = +\infty\):
    Then by \DEF{6.2.3}(b), \(x \le y\).
    Also, \(x \ne y\) since by \DEF{6.2.1} \(+\infty\) is distinct from any real number.
    Together, by \DEF{6.2.3}, we have \(x < y\), so the ``at least'' part is shown.
    Now since we have shown \(x \ne y\), we only need to show \(x > y\) is false to show the ``at most'' part.
    For the sake of contradiction, suppose \(x > y\) is true. Then by \DEF{6.2.3} this means \(y < x\), and in particular \(y \le x\).
    But \(y \le x\) is false because it satisfies none of the conditions in \DEF{6.2.3}, so we have contradiction.
\item \(x \in \SET{R}\) and \(y = -\infty\):
    Then by \DEF{6.2.3}(c), \(y \le x\).
    Also, \(y \ne x\) since by \DEF{6.2.1} \(-\infty\) is distinct from any real number.
    Together, by \DEF{6.2.3}, we have \(y < x\), so the ``at least'' part is shown.
    Now since we have shown \(y \ne x\), we only need to show \(y > x\) is false to show the ``at most'' part.
    For the sake of contradiction, suppose \(y > x\) is true. Then by \DEF{6.2.3} this means \(x < y\), and in particular \(x \le y\).
    But \(x \le y\) is false because it satisfies none of the conditions in \DEF{6.2.3}, so we have contradiction.
\item \(x = +\infty\) and \(y \in \SET{R}\):
    Then by \DEF{6.2.3}(b), \(y \le x\).
    Also, \(y \ne x\) since by \DEF{6.2.1} \(+\infty\) is distinct from any real number.
    Together, by \DEF{6.2.3}, we have \(y < x\), so the ``at least'' part is shown.
    Now since we have shown \(y \ne x\), we only need to show \(y > x\) is false to show the ``at most'' part.
    For the sake of contradiction, suppose \(y > x\) is true. Then by \DEF{6.2.3} this means \(x < y\), and in particular \(x \le y\).
    But \(x \le y\) is false because it satisfies none of the conditions in \DEF{6.2.3}, so we have contradiction.
\item \(x = +\infty\) and \(y = +\infty\):
    Then we have \(x = y\), so we cannot have \(x \ne y\) so by \DEF{6.2.3} it is not possible to have \(x < y\) or \(x > y\) since both need \(x \ne y\).
\item \(x = +\infty\) and \(y = -\infty\):
    Then by \DEF{6.2.3}(b) or (c), \(y \le x\).
    Also, \(y \ne x\) since by \DEF{6.2.1} \(+\infty\) and \(-\infty\) are distinct from each other.
    Together, by \DEF{6.2.3}, we have \(y < x\), so the ``at least'' part is shown.
    Now since we have shown \(y \ne x\), we only need to show \(y > x\) is false to show the ``at most'' part.
    For the sake of contradiction, suppose \(y > x\) is true. Then by \DEF{6.2.3} this means \(x < y\), and in particular \(x \le y\).
    But \(x \le y\) is false because it satisfies none of the conditions in \DEF{6.2.3}, so we have contradiction.
\item \(x = -\infty\) and \(y \in \SET{R}\):
    Then by \DEF{6.2.3}(c), \(x \le y\).
    Also, \(x \ne y\) since by \DEF{6.2.1} \(-\infty\) is distinct from any real number.
    Together, by \DEF{6.2.3}, we have \(x < y\), so the ``at least'' part is shown.
    Now since we have shown \(x \ne y\), we only need to show \(x > y\) is false to show the ``at most'' part.
    For the sake of contradiction, suppose \(x > y\) is true. Then by \DEF{6.2.3} this means \(y < x\), and in particular \(y \le x\).
    But \(y \le x\) is false because it satisfies none of the conditions in \DEF{6.2.3}, so we have contradiction.
\item \(x = -\infty\) and \(y = +\infty\):
    Then by \DEF{6.2.3}(b) or (c), \(x \le y\).
    Also, \(x \ne y\) since by \DEF{6.2.1} \(+\infty\) and \(-\infty\) are distinct from each other.
    Together, by \DEF{6.2.3}, we have \(x < y\), so the ``at least'' part is shown.
    Now since we have shown \(x \ne y\), we only need to show \(x > y\) is false to show the ``at most'' part.
    For the sake of contradiction, suppose \(x > y\) is true. Then by \DEF{6.2.3} this means \(y < x\), and in particular \(y \le x\).
    But \(y \le x\) is false because it satisfies none of the conditions in \DEF{6.2.3}, so we have contradiction.
\item \(x = -\infty\) and \(y = -\infty\):
    Then we have \(x = y\), so we cannot have \(x \ne y\) so by \DEF{6.2.3} it is not possible to have \(x < y\) or \(x > y\) since both need \(x \ne y\).
\end{itemize}

Transitivity: Suppose \(x \le y\) and \(y \le z\).
We split into cases by the value of \(z\):
\begin{itemize}
\item \(z = +\infty\):
    Then by \DEF{6.2.3}(b) we just have \(x \le z\).
\item \(z = -\infty\):
    Then since \(y \le z\), and both (a), (b) of \DEF{6.2.3} cannot be satisfied, so (c) must be satisfied, i.e. \(y = -\infty\).
    Similarly since \(x \le y\), we have \(x = -\infty\).
    And by \DEF{6.2.3}(c), we have \(x \le z\).
\item \(z \in \SET{R}\):
    Then since \(y \le z\), by \DEF{6.2.3}, \(y\) can only be real numbers or \(y = -\infty\). Again we split these into cases:
    \begin{itemize}
        \item
            If \(y \in \SET{R}\), then again since \(x \le y\), \(x\) can only be real numbers or \(x = -\infty\).
            If \(x \in \SET{R}\), then now all \(x, y, z\) are real numbers and \(x \le y\) and \(y \le z\), so we fallbacks to the transitivity of real numbers, so \(x \le z\).
            If \(x = -\infty\), then by \DEF{6.2.3}(c), we have \(x \le z\).
        \item
            If \(y = -\infty\), then since \(x \le y\) and both (a), (b) of \DEF{6.2.3} cannot be satisfied, (c) must be satisfied, i.e. \(x = -\infty\).
            So by \DEF{6.2.3}(c), we have \(x \le z\).
    \end{itemize}
\end{itemize}
So in all cases, we have \(x \le z\) as desired.

Negation reverses order: Suppose \(x \le y\), we have to show \(-y \le -x\).
Since \(x \le y\), we can split into cases according to \DEF{6.2.3}:
\begin{itemize}
    \item Both \(x \in \SET{R}\) and \(y \in \SET{R}\):
        Then it fallbacks to the case of real numbers, so clearly we have \(-y \le -x\).
    \item \(y = +\infty\):
        Then by \DEF{6.2.2} \(-y = -\infty\), and by \DEF{6.2.3}(c), \(-y \le -x\).
    \item \(x = -\infty\):
        Then by \DEF{6.2.2} \(-x = +\infty\), and by \DEF{6.2.3}(b), \(-y \le -x\).
\end{itemize}
So in all cases, we have \(-y \le -x\) as desired.
\end{proof}

\begin{note}
One could also introduce other operations on the extended real number system, such as addition, multiplication, etc.
However, this is somewhat dangerous as \emph{these operations will almost certainly fail to obey the familiar rules of algebra}.
For instance, to define addition it seems reasonable (given one’s intuitive notion of infinity) to set \(+\infty + 5 = +\infty\) and \(+\infty + 3 = +\infty\), but then this implies that \(+\infty + 5 = +\infty + 3\), while \(5 \neq 3\).
So things like the cancellation law begin to break down once we try to operate involving infinity.
To avoid these issues \emph{we shall simply not define any arithmetic operations} on the extended real number system other than negation and order.
\end{note}

\begin{definition} [Supremum of sets of \emph{extended reals}] \label{def 6.2.6}
Let \(E\) be a subset of \(\SET{R}^*\).
Then we define the \emph{supremum} \(\sup(E)\) or \emph{least upper bound} of \(E\) by the following rule.
\begin{enumerate}
    \item If \(E\) is \emph{contained} in \(\SET{R}\) (i.e., \(+\infty\) and \(-\infty\) are \emph{not} elements of \(E\)), then we let \(\sup(E)\) be as defined in \DEF{5.5.10}.
    \item If \(E\) contains \(+\infty\), then we set \(\sup(E) := +\infty\).
    \item If \(E\) does not contain \(+\infty\) but does contain \(-\infty\), then we set \(\sup(E) := \sup(E \setminus \{-\infty\})\) (which is a subset of \(\SET{R}\) and thus falls under case (a)).
\end{enumerate}
We also define the \emph{infimum} \(\inf(E)\) of \(E\) (also known as the \emph{greatest lower bound} of \(E\)) by the formula \(\inf(E) := -\sup(-E)\)
where \(-E\) is the set \(-E := \{ -x : x \in E \}\).
\end{definition}

\begin{note}
Just a reminder, although \DEF{6.2.6}(a) use \DEF{5.5.10}, from that definition if \(E\) (is empty and) has no upper bound, then \(\sup(E)\) is again \(+\infty\).
\end{note}

\begin{note}
Note that ``\(-x\)'' in the last sentence used negation of \emph{extended} real numbers, i.e. \DEF{6.2.2}.
\end{note}

\begin{note}
Note that \DEF{6.2.6} is \emph{well-defined}, i.e. the supremum is unique.
For case (a), it fallbacks to \DEF{5.5.10}, which is well defined.
For case (b), \(\sup(E) := +\infty\), which is also unique.
For case (c), it again fallbacks to case (a) so is well-defined.
So it is impossible that \(\sup(E) < \sup(E)\); this fact is used in the proof of \EXAMPLE{6.2.10}.
\end{note}

\begin{example} \label{example 6.2.7}
Let \(E\) be the negative integers, \emph{together with} \(-\infty\):
\[
    E = \{-1, -2, -3, -4,...\} \cup \{-\infty\}.
\]
Then by \DEF{6.2.6}(c), \(\sup(E) = \sup(E \setminus \{-\infty\}) = -1\), while
\begin{align*}
    \inf(E) & = -\sup(-E) & \text{by \DEF{6.2.6}} \\
           & = -(+\infty) & \text{by \DEF{6.2.6}(b)} \\
           & = -\infty & \text{by \DEF{6.2.2}}
\end{align*}
\end{example}

\begin{note}
Although currently we are discussing the \emph{extended} real numbers, the notation \(\{1, 2, 3, ...\}\) still does \emph{not} contain the \(+\infty\) as member.
\end{note}

\begin{example} \label{example 6.2.8}
The set \(\{0.9, 0.99, 0.999, 0.9999,...\}\) has infimum \(0.9\) and supremum \(1\).
(Indeed this case fallbacks to \DEF{5.5.10}, with the set non-empty and having upper bound.)
Note that in this case the supremum does \emph{not} actually
belong to the set, but it is in some sense ``touching the set'' from the right.
\end{example}

\begin{example} \label{example 6.2.9}
The set \(\{1, 2, 3, 4, 5 ...\}\) has infimum \(1\) and supremum \(+\infty\). (Indeed this case fallbacks to \DEF{5.5.10}, with the set non-empty and having \emph{no} upper bound.)
But again, \(+\infty\) does \emph{not} belong to the set.
\end{example}

\begin{example} \label{example 6.2.10}
Let \(E\) be the empty set.
Then \(\sup(E) = -\infty\) and \(\inf(E) = +\infty\) (why?).
This is the \emph{only case} in which the supremum can be \emph{less than} the infimum (why?).
\end{example}

\begin{proof}
Since \(E\) is contained in \(\SET{R}\), by \DEF{6.2.6}, \(\sup(E)\) fallbacks to \DEF{5.5.10}.
By \DEF{5.5.10}, since \(E\) is empty, in that definition we have \(\sup(E) = -\infty\).
Now, by \DEF{6.2.6}, it's clear that \(-E := \{ -x : x \in E \}\) is also empty, so again by \DEF{5.5.10}, \(\sup(-E) = -\infty\).
And by \DEF{6.2.2}, \(-\sup(-E) = +\infty\), that is, by \DEF{6.2.6}, \(\inf(E) = +\infty\).

Now we show that the only case in which the supremum can be less than the infimum is when \(E = \emptyset\).
Suppose for the sake of contradiction that there exists an non-empty set \(E'\) s.t. \(\sup(E') < \inf(E')\) \MAROON{(1)}.

Since \(E'\) is non-empty, (by \LEM{3.1.6}) we can find element \(x \in E'\).
And by \THM{6.2.11}, we have \(x \le \sup(E')\) \MAROON{(2)} and \(\inf(E') \le x\) \MAROON{(3)}.
Then by \MAROON{(1)(2)(3)} we have \(\sup(E') < \inf(E') \le x \le \sup(E')\), in particular \(\sup(E') < \sup(E')\), which is impossible(see the third note of \DEF{6.2.6}).
\end{proof}

\begin{note}
The proof above used \THM{6.2.11}, but it's ok because the theorem does not depend on this example.
\end{note}

\begin{note}
One can intuitively think of the supremum of \(E\) as follows.
Imagine the real line with \(+\infty\) ``somehow'' on the far right, and \(-\infty\) on the far left.
Imagine a \emph{piston} at \(+\infty\) moving leftward until it is stopped by the presence of a set \(E\);
\emph{the location where it stops is the supremum of \(E\)}.
Similarly if one imagines a piston at \(-\infty\) moving rightward until it is stopped by the presence of \(E\), the location where it stops is the infimum of \(E\).
\emph{In the case when \(E\) is the empty set}, \textbf{the pistons pass through each other}, the supremum landing at \(-\infty\) and the infimum landing at \(+\infty\).
\end{note}

\begin{note}
我其實不知道作者為什麼用\ ``piston'' 來形容... 感覺應該比較像``子彈''才對...
\end{note}

The following \emph{theorem} justifies the terminology \sout{``least upper bound''} ``supremum'' and \sout{``greatest lower bound''} ``infimum'':

\begin{note}
In my opinion the theorem ``justifies'' sup and inf, not LUB and GLB.
From \DEF{6.2.6}, sup/inf are always defined, but LUB/GLB may not exist.
\end{note}

\begin{theorem} \label{thm 6.2.11}.
Let \(E\) be a subset of \(\SET{R}^*\).
Then the following statements are true.
\begin{enumerate}
    \item For every \(x \in E\) we have \(x \le \sup(E)\) and \(x \ge \inf(E)\).
    \item Suppose that \(M \in \SET{R}^*\) is an upper bound for \(E\), i.e., \(x \le M\) for all \(x \in E\).
          Then we have \(\sup(E) \le M\).
    \item Suppose that \(M \in \SET{R}^*\) is a lower bound for \(E\), i.e., \(x \ge M\) for all \(x \in E\).
          Then we have \(\inf(E) \ge M\).
\end{enumerate}
\end{theorem}

\begin{note}
會需要證明這個定理,i.e. 要確定任意(廣義)實數子集合內所有\ element 小於等於該集合的\ supremum 的原因,我認為是這本書的 ``最小上界''跟\ ``supremum'' 不是等價名詞(去看\ \DEF{5.5.10} 還有 \DEF{6.2.6});最小上界不一定會存在,但是\ supremum 一定存在(可能是\ \(+\infty\) 或 \(-\infty\))。
可能要去看看其他分析書是怎麼區分這兩者的。
\end{note}

\begin{proof}
\begin{enumerate}
\item
    We first show \(\forall x \in E, x \le \sup(E)\) \BLUE{(1)}.
    There are three cases:
    \begin{itemize}
    \item \(E \subseteq \SET{R}\):
        We further split into two cases: \(E\) is empty or not.
        If \(E\) is empty, then \BLUE{(1)} is vacuously true.
        If \(E\) is non-empty \emph{and has an upper bound}, then by \DEF{5.5.10}, \(\sup(E)\) is \emph{the} least upper bound of \(E\), i.e. an upper bound of \(E\), so by definition of upper bound, \BLUE{(1)} is true.
        If \(E\) is non-empty \emph{but has no upper bound}, then again by \DEF{5.5.10}, \(\sup(E) = +\infty\). And by \DEF{6.2.3}(b), \BLUE{(1)} is true.
    \item \(E\) contains \(+\infty\):
        Then by \DEF{6.2.6}, \(\sup(E) = +\infty\).
        And by \DEF{6.2.3}(b), \BLUE{(1)} is true.
    \item \(E\) does not contain \(+\infty\) but contains \(-\infty\):
        Then we consider the value of \(x\): \(x = -\infty\) or \(x\) is a real number.
        If \(x = -\infty\), then by \DEF{6.2.3}(c), \(x\) is less than any extended real number.
        In particular, \(x \le \sup(E)\).
        Now suppose \(x\) is a real number.
        We first consider \(\sup(E)\).
        By \DEF{6.2.6}(c), \(\sup(E) = \sup(E \setminus \{-\infty\})\), so we only need to show \(x \le \sup(E \setminus \{-\infty\})\).

        Clearly \(x \in E \setminus \{-\infty\}\) because \(x\) is a real number that belongs to \(E\) \MAROON{(1)}.
        Again we have two cases: \(E \setminus \{-\infty\}\) has an upper bound or not.
        If \(E \setminus \{-\infty\}\) has an upper bound, then by \DEF{5.5.10} \(\sup(E \setminus \{-\infty\})\) is the least upper bound, i.e. an upper bound.
        And by definition of upper bound and \MAROON{(1)}, \(x \le \sup(E \setminus \{-\infty\})\).
        If \(E \setminus \{-\infty\}\) has no upper bound, then again by \DEF{5.5.10}, \(\sup(E \setminus \{-\infty\}) = +\infty\), and by \DEF{6.2.3}(b), \(x \le \sup(E \setminus \{-\infty\})\).
        So in all cases, \(x \le \sup(E \setminus \{-\infty\})\), as desired.
    \end{itemize}
    Now we show \(\forall x \in E, x \ge \inf(E)\) \BLUE{(2)}.
    But given any \(x \in E\), by \DEF{6.2.6} we know that \(-x \in -E\), and by the previous case, \(-x \le \sup(-E)\).
    By \PROP{6.2.5}(d) (negation reverses order), \(x \ge -(\sup(-E))\), that is, by \DEF{6.2.6}, \(x \ge \inf(E)\).
\item
    Again we have Three cases:
    \begin{itemize}
    \item \(E \subseteq \SET{R}\):
        There are three cases.

        If \(E\) is empty, then by \DEF{5.5.10}, \(\sup(E) = -\infty\), and by \DEF{6.2.3}(c), \(\sup(E) \le M\).
        
        If \(E\) is non-empty and has an upper bound, then by \DEF{5.5.10} \(\sup(E)\) is the least upper bound of \(E\), so for any upper bound \(M\) of \(E\) we have \(\sup(E) \le M\).
        
        If \(E\) is non-empty and has no upper bound, then by \DEF{5.5.10}, \(\sup(E) = +\infty\), and \(M\) must equal to \(+\infty\) (otherwise trivially we have contradiction that \(M\) is not an upper bound of \(E\).
        So by \DEF{6.2.3}(b), we have \(\sup(E) \le M\).
    \item \(E\) contains \(+\infty\):
        Then we must have \(+\infty \le M\) since \(M\) is an upper bound of \(E\) (otherwise if \(+\infty > M\) we can find element \(+\infty \in E\) s.t. \(M < +\infty\), contradicting \(M\) is an upper bound of \(E\)).
        And by \DEF{6.2.6}(b), \(\sup(E) = +\infty\), so together we have \(\sup(E) = +\infty \le M\).
    \item \(E\) does not contain \(+\infty\) but contains \(-\infty\):
        Then by \DEF{6.2.6}(c), \(\sup(E) = \sup(E \setminus \{-\infty\})\).
        So we only have to show \(\sup(E \setminus \{-\infty\}) \le M\).
        Now we claim that \(M\) is \emph{also} an upper bound of \(E \setminus \{-\infty\})\), since \(E \setminus \{-\infty\} \subseteq E\).
        Also, since \(E \setminus \{-\infty\}\) is contained in \(\SET{R}\), by the first case and the upper bound \(M\) of this set, we have \(\sup(E \setminus \{-\infty\}) \le M\), as desired.
    \end{itemize}
\item
    Let arbitrary \(y \in -E\), then \(y = -x\) for some \(x \in E\).
    Since \(x \in E\) and \(M\) is a lower bound of \(E\), we have \(M \le x\), which by \PROP{6.2.5}(d) implies \(-M \ge -x\), or \(-M \ge y\).
    Since \(y\) is arbitrary, that implies for all \(y \in -E\), \(-M \ge y\), so by definition, \(-M\) is an upper bound of \(-E\).
    Then by part(b), we have \(\sup(-E) \le -M\), which by \PROP{6.2.5}(d) implies \(-\sup(-E) \ge M\), or by \DEF{6.2.6}, \(\inf(E) \ge M\), as desired.
\end{enumerate}
\end{proof}

\exercisesection

\begin{exercise} \label{exercise 6.2.1}
Prove \PROP{6.2.5}.
(Hint: you may need \PROP{5.4.7}.)
\end{exercise}

\begin{proof}
See \PROP{6.2.5}.
\end{proof}

\begin{exercise} \label{exercise 6.2.2}
Prove \THM{6.2.11}.
(Hint: you may need to break into cases depending on whether \(+\infty\) or \(-\infty\) belongs to \(E\).
You can of course use \DEF{5.5.10}, provided that \emph{\(E\) consists only of real numbers}.)
\end{exercise}

\begin{proof}
See \THM{6.2.11}.
\end{proof}
\section{Suprema and Infima of sequences} \label{sec 6.3}

Having defined the notion of a supremum and infimum of sets of reals, we can now also talk about the supremum and infimum \emph{of sequences}.

\begin{definition} [Sup and inf of sequences] \label{def 6.3.1}
Let \((a_n)_{n = m}^{\infty}\) be a sequence of real numbers.
Then we define \(\sup(a_n)_{n = m}^{\infty}\) to be the supremum of the \emph{set} \(\{a_n : n \ge m\}\), and \(\inf(a_n)_{n = m}^{\infty}\) to the infimum of the same set \(\{a_n : n \ge m\}\).
\end{definition}

\begin{remark} \label{remark 6.3.2}
The quantities \(\sup(a_n)_{n = m}^{\infty}\) and \(\inf(a_n)_{n = m}^{\infty}\) are sometimes written as \(\sup_{n \ge m} a_n\) and \(\inf_{n \ge m} a_n\) respectively.
\end{remark}

\begin{example} \label{example 6.3.3}
Let \(a_n := (-1)^n\); thus \((a_n)_{n=1}^{\infty}\) is the sequence
\(-1, 1, -1, 1,...\).
Then the set \(\{a_n : n \ge 1\}\) is just the two-element set \(\{-1, 1\}\), and hence \(\sup(a_n)_{n = m}^{\infty}\) is equal to \(1\).
Similarly \(\inf(a_n)_{n = m}^{\infty}\) is equal to \(-1\).
\end{example}

\begin{example} \label{example 6.3.4}
Let \(a_n := 1/n\); thus \((a_n)_{n = 1}^{\infty}\) is the sequence \(1, 1/2, 1/3,...\).
Then the set \(\{a_n : n \ge 1\}\) is the \emph{countable}(\RED{warning}) set \(\{1, 1/2, 1/3, 1/4,...\}\).
Thus \(\sup(a_n)_{n = m}^{\infty} = 1\) and \(\inf_{n \ge m} a_n\) (\EXEC{6.3.1}).
Notice here that the infimum of the sequence is \emph{not actually a member of} the sequence, though it becomes very close to the sequence \emph{eventually}.
(So it is a little inaccurate to think of the supremum and infimum as the ``largest element \emph{of} the sequence'' and ``smallest element \emph{of} the sequence'' respectively.)
\end{example}

\begin{proof}
The sequence is bounded by \(1\), it's trivial to prove it using induction.
So the corresponding set \(\{a_n : n \ge 1\}\) of the sequence has an upper bound \(1\).
So by \DEF{5.5.10}, the supremum is the least upper bound of the set.
To show that the supremum is equal to \(1\), we can show that any real number \(x < 1\) is not an upper bound of the set.
But it is of course true because \(1\) belongs to the set.

Now we show the infimum of the sequence is \(0\).
Again, it's trivial that each element of the sequence is greater than \(0\) using induction.
So the corresponding set \(\{a_n : n \ge 1\}\) of the sequence has an lower bound \(0\).
To show that the infimum of the set is equal to \(0\), we can show that any real number \(x > 0\) is not a lower bound of the set.
But by \EXEC{5.4.4}, given \(x > 0\), we can find a natural number \(n\) s.t. \(1/n < x\), and that is the value of \(a_n\), so we have found an element \(a_n\) in the set s.t. \(x > a_n\), hence \(x\) is not a lower bound of the set.
\end{proof}

\begin{note}
\RED{Warning}: the word ``countable'' in \EXAMPLE{6.3.4} is nout defined yet.
It's defined in \CH{8}.
\end{note}

\begin{example} \label{example 6.3.5}
Let \(a_n := n\); thus \((a_n)_{n = 1}^{\infty}\) is the sequence \(1, 2, 3, 4,...\).
Then the set \(\{a_n : n \ge 1\}\) is just the positive integers \(\{1, 2, 3, 4,...\}\).
Then \(\sup(a_n)_{n = 1}^{\infty} = +\infty\) and \(\inf(a_n)_{n = 1}^{\infty} = 1\).
\end{example}

As the last example shows, it is possible for the supremum or infimum of a sequence(of \emph{real} numbers, not \emph{extended real} numbers) to be \(+\infty\) or \(-\infty\).

\begin{additional corollary} \label{ac 6.3.1}
However, if a sequence \((a_n)_{n = m}^{\infty}\) is
bounded, say bounded by \(M\), then all the elements \(a_n\) of the sequence lie between \(-M\) and \(M\), so that the set \(\{a_n : n \ge m\}\) has \(M\) as an upper bound and \(-M\) as a lower bound.
Since this set is clearly non-empty, we can thus conclude that the supremum and infimum of a bounded sequence are \emph{real} numbers (i.e., not \(+\infty\) and \(-\infty\)).
\end{additional corollary}

\begin{proposition} [Least upper bound property] \label{prop 6.3.6}
Let \((a_n)_{n = m}^{\infty}\) be a sequence of ``real numbers'', and let \(x\) be the ``extended real number'' \(x := \sup(a_n)_{n = m}^{\infty}\).

\BLUE{(1)} Then we have \(a_n \le x\) for all \(n \ge m\).

\BLUE{(2)} Also, whenever \(M \in \SET{R}^*\) is an upper bound for \(a_n\) (i.e., \(a_n \le M\) for all \(n \ge m\)), we have \(x \le M\).

\BLUE{(3)} Finally, for every ``extended real number'' \(y\) for which \(y < x\), there exists at least one \(n \ge m\) for which \(y < a_n \le x\).
\end{proposition}

\begin{proof}
\BLUE{(1)} Given any integer \(n \ge m\), \(a_n\) is of course belong to the set \(\{a_n : n \ge m\}\), and by \THM{6.2.11}(a), \(a_n\) is less than the supremum of the set, i.e. \(a_n \le \sup(a_n)_{n = m}^{\infty}\), i.e. \(a_n \le x\).

\BLUE{(2)} Now, if \(M \in \SET{R}^*\) is an upper bound of the sequence, then \(M\) is of course an upper bound of the corresponding set \(\{a_n : n \ge m\}\).
And by \THM{6.2.11}(b), the supremum of the set is less than \(M\), i.e. \(\sup(a_n)_{n = m}^{\infty} \le M\), i.e. \(x \le M\).

\BLUE{(3)} For the last statement of the proposition, suppose for the sake of contradiction that there exists an ``extended real number`` \(y\) for which \(y < x\) \MAROON{(1)} and for all integer \(n \ge m\), \(\lnot (y < a_n \le x)\), i.e. \(y \ge a_n \lor a_n > x\).
First \(a_n > x\) is impossible since \(x\) we have shown from \BLUE{(1)} that \(a_n \le x\).
But then we have for all integer \(n \ge m\), \(y \ge a_n\). which implies \(y\) is an upper bound of the sequence.
But by \BLUE{(2)}, that implies \(\sup(a_n)_{n = m}^{\infty} = x \le y\), which contradicts \MAROON{(1)}.
\end{proof}

\begin{remark} \label{remark 6.3.7}
There is a corresponding Proposition for infima, but with all the references to order reversed, e.g., all upper bounds should now be lower bounds, etc.
The proof is exactly the same.

The corresponding Proposition: Let \((a_n)_{n = m}^{\infty}\) be a sequence of ``real numbers'', and let \(x\) be the ``extended real number'' \(x := \inf(a_n)_{n = m}^{\infty}\).

\BLUE{(1)} Then we have \(a_n \ge x\) for all \(n \ge m\).

\BLUE{(2)} Also, whenever \(M \in \SET{R}^*\) is a \emph{lower bound} for \(a_n\) (i.e., \(a_n \ge M\) for all \(n \ge m\)), we have \(x \ge M\).

\BLUE{(3)} Finally, for every ``extended real number`` \(y\) for which \(y > x\), there exists at least one \(n \ge m\) for which \(y > a_n \ge x\).
\end{remark}

\begin{note}
In the previous section we saw that all convergent sequences are bounded(by \CORO{6.1.17}).
It is natural to ask whether the \emph{converse} is true:
are all bounded sequences convergent?
The answer is no;
for instance, the sequence \(1, -1, 1, -1,...\) is bounded, but not Cauchy and hence not convergent(contrapositive of \PROP{6.1.12}).
However, if we make the sequence \textbf{both bounded and \emph{monotone}} (i.e., increasing or decreasing), then it is true that it must converge.
\end{note}

\begin{proposition} [Monotone bounded sequences converge] \label{prop 6.3.8}
Let \((a_n)_{n = m}^{\infty}\) be a sequence of \emph{real} numbers which has some \emph{finite upper bound} \(M \in \SET{R}\), and which is also increasing (i.e., \(a_{n + 1} \ge a_n\) for all \(n \ge m\)).
Then \((a_n)_{n = m}^{\infty}\) is convergent, and in fact
\[
    \lim_{n \toINF} a_n = \sup(a_n)_{n = m}^{\infty} \le M
\]
\end{proposition}

\begin{proof}
First we call the supremum of the sequence \(\alpha := \sup(a_n)_{n = m}^{\infty}\).
Since the sequence has a ``real''(i.e. finite) number \(M\) as upper bound, by \AC{6.3.1}, \(\alpha\) is a real number.
And by \PROP{6.3.6}, \(\alpha \le M\).
Now what we left is to prove the sequence converges to \(\alpha\).
That is, we have to show given arbitrary \(\varE > 0\), there exists an integer \(N \ge m\) s.t. \(\abs{a_n - \alpha} \le \varE\) for all \(n \ge N\), or equivalently, \(-\varE \le a_n - \alpha \le \varE\), or \(\alpha - \varE \le a_n \le \alpha + \varE\), for all \(n \ge N\).

So let arbitrary \(\varE > 0\).
Now of course \(\alpha - \varE < \alpha\).
And by \PROP{6.3.6}, we can find an integer \(N \ge m\) s.t. \(\alpha - \varE < a_N \le \alpha\).
Also \emph{since the sequence is increasing}, for all \(n \ge N\), \(\alpha - \varE < a_N \le a_n\);
in particular \(\alpha - \varE \le a_n\) for all \(n \ge N\) \MAROON{(1)}.
But again by \PROP{6.3.6}, for all \(n \ge N\), \(a_n \le \alpha\) (\(\alpha\) is the supremum of the sequence) \MAROON{(2)}.
So with \MAROON{(1)(2)} we have \(\alpha - \varE \le a_n \le \alpha\) for all \(n \ge N\), which of course implies \(\alpha - \varE \le a_n \le \alpha + \varE\) for all \(n \ge N\), as desired.
So the sequence converges to \(\alpha\).
\end{proof}

\begin{note}
One can similarly prove that if a sequence \((a_n)_{n = m}^{\infty}\) is bounded \emph{below} and \emph{decreasing} (i.e., \(a_{n + 1} \le a_n)\), then it is convergent, and that the limit is equal to the infimum.
\end{note}

\begin{additional corollary} \label{ac 6.3.2}
Let \((a_n)_{n = m}^{\infty}\) be a sequence of \emph{real} numbers which has some \emph{finite lower bound} \(M \in \SET{R}\), and which is also decreasing (i.e., \(a_{n + 1} \leq a_n\) for all \(n \ge m\)).
Then \((a_n)_{n = m}^\infty\) is convergent, and in fact
\[
    \lim_{n \toINF} a_n = \inf(a_n)_{n = m}^\infty \ge M.
\]
\end{additional corollary}

\begin{proof} (This proof is just modified from the proof of \PROP{6.3.8}.)
First we call the infimum of the sequence \(\beta := \inf(a_n)_{n = m}^{\infty}\).
Since the sequence has a ``real''(i.e. finite) number \(M\) as lower bound, by \AC{6.3.1}, \(\beta\) is also a real number.
And by \RMK{6.3.7}, \(\beta \ge M\).
Now what we left is to prove the sequence converges to \(\beta\).
That is, we have to show given arbitrary \(\varE > 0\), there exists an integer \(N \ge m\) s.t. \(\abs{a_n - \beta} \le \varE\) for all \(n \ge N\), or equivalently, \(-\varE \le a_n - \beta \le \varE\), or \(\beta - \varE \le a_n \le \beta + \varE\), for all \(n \ge N\).

So let arbitrary \(\varE > 0\).
Now of course \(\beta + \varE > \beta\).
And by \RMK{6.3.7}, we can find an integer \(N \ge m\) s.t. \(\beta + \varE > a_N \ge \beta\).
Also \emph{since the sequence is decreasing}, for all \(n \ge N\), \(\beta + \varE > a_N \ge a_n\);
in particular \(\beta + \varE \ge a_n\) for all \(n \ge N\) \MAROON{(1)}.
But again by \RMK{6.3.7}, for all \(n \ge N\), \(a_n \ge \beta\) (\(\beta\) is the infimum of the sequence) \MAROON{(2)}.
So with \MAROON{(1)(2)} we have \(\beta + \varE \ge a_n \ge \beta\) for all \(n \ge N\), which of course implies \(\beta + \varE \ge a_n \ge \beta - \varE\) for all \(n \ge N\), as desired.
So the sequence converges to \(\beta\).
\end{proof}

\begin{note}
A sequence is said to be \emph{monotone} if it is either increasing or decreasing.
From \PROP{6.3.8} and \CORO{6.1.17} we see that a
monotone sequence converges if and only if it is bounded.
(Given monotone sequence, if it converges, then by \CORO{6.1.17}, is bounded(in fact that corollary does not care if it's monotone); if it is bounded, then by \PROP{6.3.8}, it converges.)
\end{note}

\begin{example} \label{example 6.3.9}
The sequence \(3, 3.1, 3.14, 3.141, 3.1415,...\) is increasing, and is bounded above by \(4\).
Hence by \PROP{6.3.8} it must have a limit, which is a real number less than or equal to \(4\).
\end{example}

\begin{note}
\PROP{6.3.8} asserts that the limit of a monotone sequence exists, but does not directly say what that limit is.
Nevertheless, with a little extra work one can often find the limit once one is given that the limit does exist.
\end{note}

\begin{proposition} \label{prop 6.3.10}
Let \(0 < x < 1\).
Then we have \(\lim_{n \toINF} x^n = 0\).
\end{proposition}

\begin{proof}
Since \(0 < x < 1\), one can show that the sequence \((x^n)_{n = 1}^{\infty}\) is decreasing (because \(1 < x \implies 1 \X x^n < x \X x^n \implies x^n < x^{n + 1}\)).
On the other hand, the sequence \((x_n)_{n = 1}^{\infty}\) has a lower bound of 0. (trivial by induction)
Thus by \AC{6.3.2} (infima version of \PROP{6.3.8}) the sequence \((x_n)_{n = 1}^{\infty}\) converges to some limit \(L\).
Since \(x^{n + 1} = x \X x^n\), we thus see from the limit laws (\THM{6.1.19}) that
\begin{align*}
             & \lim_{n \toINF} x^n = L \\
    \implies & x \lim_{n \toINF} x^n = x L & \text{by algebra} \\
    \implies & \lim_{n \toINF} x \X x^n = x L & \text{by \THM{6.1.19}(c)} \\
    \implies & \lim_{n \toINF} x^{n + 1} = x L
\end{align*}
So \((x^{n + 1})_{n = 1}^{\infty}\) converges to \(xL\).
But the sequence \((x^{n + 1})_{n = 1}^{\infty}\) is just the sequence \((x^n)_{n = 2}^{\infty}\) shifted by one, and so they must have the same limits
(by \EXEC{6.1.3}, \((x^{n + 1})_{n = 1}^{\infty}\) and \((x^{n + 1})_{n = 2}^{\infty}\) have the same limit;
and by \EXEC{6.1.4}, \((x^{n + 1})_{n = 2}^{\infty}\) and \((x^n)_{n = 2}^{\infty}\) have the same limit).
But \((x^n)_{n = 2}^{\infty}\) and \((x^n)_{n = 1}^{\infty}\) also have the same limit(by \EXEC{6.1.3}).
So all in all, that implies the limit of \((x^{n + 1})_{n = 1}^{\infty}\), which is \(xL\), is equal to the limit of \((x^n)_{n = 1}^{\infty}\), which is \(L\).
So \(xL = L\), or \(L(x - 1) = 0\).
Since \(x \neq 1\), \(x - 1 \ne 0\), so \(L\) must be \(0\).
Thus \((x_n)_{n = 1}^{\infty}\) converges to \(0\).
\end{proof}

\exercisesection

\begin{exercise} \label{exercise 6.3.1}
Verify the claim in \EXAMPLE{6.3.4}.
\end{exercise}

\begin{proof}
See \EXAMPLE{6.3.4}.
\end{proof}

\begin{exercise} \label{exercise 6.3.2}
Prove \PROP{6.3.6}. (Hint: use \THM{6.2.11}.)
\end{exercise}

\begin{proof}
See \PROP{6.3.6}.
\end{proof}

\begin{exercise} \label{exercise 6.3.3}
Prove \PROP{6.3.8}.
(Hint: use Proposition 6.3.6, \emph{together with} the assumption that an is \emph{increasing}, to show that an converges to \(\sup(a_n)_{n = m}^{\infty}\).)
\end{exercise}

\begin{proof}
See \PROP{6.3.8}.
\end{proof}

\begin{exercise} \label{exercise 6.3.4}
Explain why \PROP{6.3.10} fails when \(x > 1\).
In fact, show that the sequence \((x_n)_{n = 1}^{\infty}\) diverges when \(x > 1\).
(Hint: prove by contradiction and use the identity \((1/x)^n x^n = 1\) and the limit laws in \THM{6.1.19}.)
Compare this with the argument in \EXAMPLE{1.2.3}; can you now explain the flaws in the reasoning in that example?
\end{exercise}

\begin{proof}
Suppose \((x^n)_{n = 1}^{\infty}\) converges when \(x > 1\) (and hence \(0 < 1/x < 1\)).
Let it converge to \(L\).
Then
\begin{align*}
    1 & = \lim_{n \toINF} 1 & \text{of course} \\
      & = \lim_{n \toINF} (1/x)^n x^n & \text{by the identity} \\
      & = (\lim_{n \toINF} (1/x)^n) \X (\lim_{n \toINF} x^n) & \text{by \THM{6.1.19}(b)} \\
      & = 0 \X (\lim_{n \toINF} x^n) & \text{since \(0 < 1/x < 1\) and by \PROP{6.3.10}} \\
      & = 0 \X L \\
      & = 0,
\end{align*}
which is impossible.

The \EXAMPLE{1.2.3} cannot assume the equation \emph{for all} real number \(x\).
It can assume the equation for \(0 < x < 1\), by \PROP{6.3.10}.
\end{proof}
\section{Limsup, Liminf, and limit points} \label{sec 6.4}

Consider the sequence
\[
    1.1, -1.01, 1.001, -1.0001, 1.00001,....
\]
If one plots this sequence, then one sees (informally, of course) that this sequence does \emph{not} converge;
\emph{half} the time the sequence is getting close to \(1\), and \emph{half} the time the sequence is getting close to \(-1\), but it is not converging to either of them;
for instance, it never gets eventually \(1/2\)-close to \(1\), and never gets eventually \(1/2\)-close to \(-1\).
However, even though \(-1\) and \(+1\) are not quite limits of this sequence, it does seem that \emph{in some vague way they “want” to be limits}.
To make this notion precise we introduce the notion of a \emph{limit point}.

\begin{definition} [Limit points] \label{def 6.4.1}
Let \((a_n)_{n = m}^{\infty}\) be a sequence of real numbers, let \(x\) be a real number, and let \(\varE > 0\) be a real number.

\BLUE{(1)} We say that \(x\) is \emph{\(\varE\)-adherent to} \((a_n)_{n = m}^{\infty}\) iff there \emph{exists} an \(n \ge m\) such that \(a_n\) is \(\varE\)-close to \(x\).

\BLUE{(2)} We say that \(x\) is \emph{continually \(\varE\)-adherent} to \((a_n)_{n = m}^{\infty}\) iff it is \(\varE\)-adherent to \((a_n)_{n = \textbf{N}}^{\infty}\) \emph{for every} \(N \ge m\).

\BLUE{(3)} We say that \(x\) is a(i.e. can be many) \emph{limit point} or \emph{adherent point} of \((a_n)_{n = m}^{\infty}\) iff it is continually \(\varE\)-adherent to \((a_n)_{n = m}^{\infty}\) \emph{for every} \(\varE > 0\).
\end{definition}

\begin{note}
\(\varE\)-adherent 只要\ sequence "存在"一個\ element 跟該點距離\ \(\varE\) 即可。
continually \(\varE\)-adherent 則是不管拿掉多少\ sequence 前面的\ elements,剩下的\ sequence 都還是存在\ element 跟該點距離\ \(\varE\)。感覺有點永不放棄。
limit point 則是不管給我多小的\ \(\varE\),sequence 都跟該點是\ continually \(\varE\)-adherent。
\end{note}

\begin{remark} \label{remark 6.4.2}
The verb ``to adhere'' means much the same as ``to stick to'';
hence the term ``adhesive''.
\end{remark}

\begin{additional corollary} \label{ac 6.4.1}
\emph{Unwrapping} all the definitions, we see that \(x\) is a limit point of \((a_n)_{n = m}^{\infty}\) if, for every \(\varE > 0\) and every \(N \ge m\), there exists an \(n' \ge N\) such that \(\abs{a_{n'} - x} \le \varE\).
Why?
\begin{align*}
             & x \text{ is a limit point of } (a_n)_{n = m}^{\infty} \\
    \implies & \forall \varE > 0, (a_n)_{n = m}^{\infty} \text{ is continually } \varE \text{-adherent to } (a_n)_{n = m}^{\infty} \text{, by \DEF{6.4.1}(3)} \\
    \implies & \forall \varE > 0, \forall N \ge m, x \text{ is \(\varE\)-adherent to } (a_n)_{n = \textbf{N}}^{\infty} \text{, by \DEF{6.4.1}(2)} \\
    \implies & \forall \varE > 0, \forall N \ge m, \exists n' \ge N \text{ s.t. } x \text{ is \(\varE\)-close to \(a_{n'}\), by \DEF{6.4.1}(1)}
\end{align*}
\end{additional corollary}

\begin{note}
Note the difference between a sequence being \(\varE\)-close to \(L\) (which means that \emph{all the elements} of the sequence stay within a distance \(\varE\) of \(L\))
and \(L\) being \(\varE\)-adherent to the sequence (which \emph{only needs a single element} of the sequence to stay within a distance \(\varE\) of \(L\)).
Also, for \(L\) to be continually \(\varE\)-adherent to \((a_n)_{n = m}^{\infty}\), it has to be \(\varE\)-adherent to \((a_n)_{n = N}^{\infty}\) \emph{for all} \(N \ge m\),
whereas for \((a_n)_{n = m}^{\infty}\) to be eventually \(\varE\)-close to \(L\), we only need \((a_n)_{n = N}^{\infty}\) to be \(\varE\)-close to \(L\) \emph{for some} \(N \ge m\).
Thus there are some subtle differences in \emph{quantifiers} between limits and limit points.
\end{note}

\begin{note}
Currently limit points are only defined (in \DEF{6.4.1}) for \emph{finite} real numbers.
It is also possible to rigorously define the concept of \(+\infty\) or \(-\infty\) being a limit point;
see \EXEC{6.4.8}.
\end{note}

\begin{example} \label{example 6.4.3}
Let \((a_n)_{n = 1}^{\infty}\) denote the sequence
\[
    0.9, 0.99, 0.999, 0.9999, 0.99999,....
\]
The number \(0.8\) is \(0.1\)-adherent to this sequence, since \(0.8\) is \(0.1\)-close to 0.9, which is a member (the first member) of this sequence.
However, \(0.8\) is not \emph{continually} \(0.1\)-adherent to this sequence, since once one discards the first element of this sequence there is no member of the sequence to be \(0.1\)-close to \(0.8\).
In particular, \(0.8\) is not a limit point of this sequence.
On the other hand, the number \(1\) is \(0.1\)-adherent to this sequence, and in fact \(0.1\) is \emph{continually} \(0.1\)-adherent to this sequence,
since no matter how many \emph{initial members} of the sequence one discards, there is still \emph{something}(i.e. ``some'') for \(1\) to be \(0.1\)-close to.
In fact, it is continually \(\varE\)-adherent for every \(\varE > 0\), and is hence a limit point of this sequence.
\end{example}

\begin{example} \label{example 6.4.4}
Now consider the sequence
\[
    1.1, -1.01, 1.001, -1.0001, 1.00001,....
\]
The number \(1\) is \(0.1\)-adherent to this sequence;
in fact \(1\) is continually \(0.1\)-adherent to this sequence,
because no matter how many elements of the sequence one discards, there are \emph{some} elements of the sequence that \(1\) is \(0.1\)-close to.
(As discussed earlier, one does \emph{not} need \emph{all} the elements to be \(0.1\)-close to 1, just \emph{some}; 
thus \(0.1\)-adherent is weaker than \(0.1\)-close, and continually \(0.1\)-adherent is a different notion from eventually \(0.1\)-close.)
In fact, for every \(\varE > 0\), the number \(1\) is continually \(\varE\)-adherent to this sequence, and is thus a limit point of this sequence.
Similarly \(-1\) is a limit point of this sequence;
however \(0\) is not a limit point of this sequence, since the sequence is not continually \(0.1\)-adherent to \(0\).
\end{example}

Limits are \emph{of course a special case} of limit points:

\begin{proposition} [Limits are limit points] \label{prop 6.4.5}
Let \((a_n)_{n = m}^{\infty}\) be a sequence which \emph{converges} to a real number \(c\).
Then \(c\) is a limit point of \((a_n)_{n = m}^{\infty}\), and in fact it is \emph{the only limit point} of \((a_n)_{n = m}^{\infty}\).
\end{proposition}

\begin{proof}
To show \(c\) is a limit point of \((a_n)_{n = m}^{\infty}\), by \AC{6.4.1} we have to show:
for every \(\varE > 0\) and every \(N \ge m\), there exists \(n' \ge N\) such that \(\abs{a_{n'} - c} \le \varE\).

So let arbitrary \(\varE > 0\) and arbitrary \(N \ge m\).
We have to find \(n' \ge N\) such that \(\abs{a_{n'} - c} \le \varE\) \BLUE{(1)}.

Since the sequence converges to \(c\), by \DEF{6.1.5} (or \EXEC{6.1.2}),
there exists \(N' \ge m\) s.t. \(\abs{a_{n'} - c} \le \varE\) \emph{for all} \(n' \ge N'\) \MAROON{(1)}.
Now there are two cases: \(N' < N\) or \(N' \ge N\).

If \(N' < N\), then in particular we can replace \(n' = N\) into \MAROON{(1)} s.t. \(\abs{a_{N} - c} \le \varE\),
so we have found \(n' = N \ge N\) s.t. \(\abs{a_{n'} - c} \le \varE\).

If \(N' \ge N\), then in particular we just use \(n' = N'\) into \MAROON{(1)} s.t. \(\abs{a_{N'} - c} \le \varE\),
so we have found \(n' = N' \ge N\) s.t. \(\abs{a_{n'} - c} \le \varE\).

In all cases, \BLUE{(1)} is satisfied, as desired.

Now we show that \(c\) is the \emph{only} limit point of the sequence.
For the sake of contradiction suppose \(c' \ne c\) is also a limit point of the sequence.

Then let \(\varE = \abs{c - c'}/2 > 0\).
\emph{In particular}, \(\abs{c - c'}/\BLUE{3} > 0\).
Since the sequence converges to \(c\), there exists \(N \ge m\) s.t. \(\abs{a_n - c} \le \abs{c - c'}/\BLUE{3}\) \MAROON{(2)} for all \(n \ge N\).
And we have for all \(n \ge N\),
\begin{align*}
    \abs{c - c'} & = \abs{c - c' + a_n - a_n} \\
                 & = \abs{(c - a_n) + (a_n - c')} \\
                 & \le \abs{c - a_n} + \abs{a_n - c'} \\
                 & = \abs{a_n - c} + \abs{a_n - c'},
\end{align*}
which implies \(\abs{a_n - c'} \ge \abs{c - c'} - \abs{a_n - c}\) for all \(n \ge N\).
And for all \(n \ge N\),
\begin{align*}
    \abs{a_n - c'} & \ge \abs{c - c'} - \abs{a_n - c} \\
                   & \ge \abs{c - c'} - \abs{c - c'}/\BLUE{3} & \text{by \MAROON{(2)}} \\
                   & = 2\abs{c - c'}/3 \\
                   & \RED{>} \abs{c - c'}/2 \\
                   & = \varE.
\end{align*}
So for all \(n \ge N\), \(a_n\) is not \(\varE\)-close to \(c'\),
which implies the sequence is not continually \(\varE\)-adherent to \(c'\),
which implies \(c'\) is not a limit point of the sequence, a contradiction.
\end{proof}

Now we will look at two special types of limit points: the limit superior (\(\limsup\)) and limit inferior (\(\liminf\)).
(\RED{warning}: I think it's too early to say they are limit points;
by \PROP{6.4.12}(e) they are limit points only if they are finite;
by \EXEC{6.4.8} they are limit points only if we allow limit points to be infinite.)

\begin{definition} [Limit superior and limit inferior]  \label{def 6.4.6}
Suppose that \((a_n)_{n = m}^{\infty}\) is a sequence.

\BLUE{(1)}  We define a \emph{new} sequence \((a_N^+)_{N = m}^{\infty}\) by the formula
\[
    a_{\textbf{N}}^+ := \sup(a_n)_{n = \textbf{N}}^{\infty}.
\]
More informally, \(a_N^+\) is \emph{the supremum of all the elements} in the sequence \textbf{from \(a_N\) onwards}.

\BLUE{(2)} We then define the \emph{limit superior} of the sequence \((a_n)_{n = m}^{\infty}\), denoted \(\limsup_{n \toINF} a_n\), by the formula
\[
    \limsup_{n \toINF} a_n := \inf(a_N^+)_{N = m}^{\infty}.
\]

\BLUE{(3)} Similarly, we can define \(a_N^- := \inf(a_n)_{n = N}^{\infty}\) and \BLUE{(4)} define the \emph{limit inferior} of the sequence \((a_n)_{n = m}^{\infty}\), denoted \(\liminf_{n \toINF} a_n\), by the formula
\[
    \liminf_{n \toINF} a_n := \sup(a_N^-)_{N = m}^{\infty}.
\]

Note that the definition of new sequence in \BLUE{(1)(2)} are \emph{well-defined}, since supremum and infimum are well-defined(i.e. unique; see \DEF{6.2.6}).
\end{definition}

\begin{note}
``From \(a_N\) onwards'': 從\ \(a_N\) 開始繼續往下數。

不精確的講,\(\limsup\) 是所有(給定任意\ \(N \ge m\),從第\ \(N\) 個\ element 開始算的\ sequence 所成的集合的) supremum 所成的集合的\ infimum(不正確但好記的說法: 所有\ sup 裡面最小的那個)。

而\ \(\liminf\) 是所有(給定任意\ \(N \ge m\),從第\ \(N\) 個\ element 開始算的\ sequence 所成的集合的) infimum 所成的集合的\ supremum(不正確但好記的說法: 所有\ inf 裡面最大的那個)。
\end{note}

\begin{example} \label{example 6.4.7}
Let \(a_1, a_2, a_3,...\) denote the sequence
\[
    1.1, -1.01, 1.001, -1.0001, 1.00001,....
\]
Then \(a_1^+, a_2^+, a_3^+, ...\) is the sequence
\[
    1.1, 1.001, 1.001, 1.00001, 1.00001,...
\]
Why?

For \(n = 1\), \(1.1\) is \(\sup(a_n)_{n = 1}^{\infty}\), or \(\sup \{1.1, -1.01, 1.001, -1.0001,...\} \).

For \(n = 2\), \(1.001\) is \(\sup(a_n)_{n = 2}^{\infty}\), or \(\sup \{-1.01, 1.001, -1.0001,... \}\).

For \(n = 3\), \(1.001\) is \(\sup(a_n)_{n = 3}^{\infty}\), or \(\sup \{1.001, -1.0001, 1.00001,...\}\).

For \(n = 4\), \(1.00001\) is \(\sup(a_n)_{n=4}^{\infty}\), or \(\sup \{-1.0001, 1.00001, ...\}\).

For \(n = 5\), \(1.00001\) is \(\sup(a_n)_{n = 5}^{\infty}\), or \(\sup \{1.00001, ...\}\).

And the infimum of the sequence \(a_1^+, a_2^+, a_3^+, ...\) is \(1\).
Hen the \emph{limit superior} of \(a_1, a_2, a_3,...\) is \(1\).

Similarly, \(a_1^-, a_2^-, a_3^- ,...\) is the sequence
\[
    -1.01, -1.01, -1.0001, -1.0001, -1.000001,...
\]
and the supremum of the sequence \(a_1^-, a_2^-, a_3^- ,...\) is \(-1\).
Hence the \emph{limit inferior} of \(a_1, a_2, a_3,...\) is \(-1\).

One should compare this with the \emph{supremum} and \emph{infimum} of the sequence, which are \(1.1\) and \(-1.01\) respectively.
\end{example}

\begin{example} \label{example 6.4.8}
Trivial.
\end{example}

\begin{example} \label{example 6.4.9}
Trivial.
\end{example}

\begin{example} \label{example 6.4.10}
Let \(a_1, a_2, a_3,...\) denote the sequence
\[
    1, 2, 3, 4, 5, 6,...
\]
Then \(a_1^+, a_2^+ ,...\) is the sequence
\[
    +\infty, +\infty, +\infty,...
\]
so the limit superior (of \(1, 2, 3, 4, 5, 6,...\)) is \(+\infty\).
Similarly, \(a_1^-, a_2^-, ...\) is the sequence
\[
    1, 2, 3, 4, 5,...
\]
which has a \emph{supremum} of \(+\infty\).
So the limit inferior (of \(1, 2, 3, 4, 5, 6,...\)) is also \(+\infty\).
\end{example}

\begin{remark} \label{remark 6.4.11}
\sloppy Some authors use the notation \(\overline{\lim}_{n \toINF} a_n\) instead of \(\limsup_{n \toINF} a_n\), and \(\underline{\lim}_{n \toINF} a_n\) instead of \(\liminf_{n \toINF} a_n\).
Note that the starting index \(m\) of the sequence is irrelevant (see \EXEC{6.4.2}).
\end{remark}

\begin{note}
Returning to the piston analogy, imagine a piston at \(+\infty\) moving leftward until it is stopped by the presence of the sequence \(a_1, a_2, \dots\).
The place it will stop is the \emph{supremum} of \(a_1, a_2, a_3, \dots\), which in our new notation is \(a_1^+\).
Now let us \emph{remove the first element} \(a_1\) from the sequence;
this may cause our piston to slip leftward, to a new point \(a_2^+\)
(though in many cases the piston will not move and \(a_2^+\) will just be the same as \(a_1^+\)).
Then we remove the second element \(a_2\), causing the piston to slip a little more.
If we keep doing this the piston will keep slipping, but \emph{there will be some point where it cannot go any further}, and \emph{this is the limit superior of the sequence}.
A similar analogy can describe the limit inferior of the sequence.
\end{note}

\begin{additional corollary} [Rigorous proof for piston analogy] \label{ac 6.4.2}
Given sequence \((a_n)_{n = m}^{\infty}\), and integer \(m_1, m_2\) s.t. \(m_1 \ge m_2 \ge m\).
Then \(\sup(a_n)_{n = m_1}^{\infty} \le \sup(a_n)_{n = m_2}^{\infty}\).
Similarly, \(\inf(a_n)_{n = m_1}^{\infty} \ge \inf(a_n)_{n = m_2}^{\infty}\).
\end{additional corollary}

\begin{proof}
For the sake of contradiction, suppose \(\sup(a_n)_{n = m_1}^{\infty} > \sup(a_n)_{n = m_2}^{\infty}\) for some \(m_1 \ge m_2 \ge m\).
Then by \PROP{6.3.6}(1), for all \(n \ge m_2\), \(a_n \le \sup(a_n)_{n = m_2}^{\infty}\);
in particular, for all \(n \ge m_1\) where \(m_1 \ge m_2\), \(a_n \le \sup(a_n)_{n = m_2}^{\infty}\).
That is, \(\sup(a_n)_{n = m_2}^{\infty}\) is an upper bound of \(\{a_n : n \ge m_1\}\),
and \(\sup(a_n)_{n = m_2}^{\infty} < \sup(a_n)_{n = m_1}^{\infty}\),
which contradicts that \(\sup(a_n)_{n = m_1}^{\infty}\) is the supremum of \(\{a_n : n \ge m_1\}\).

Now for the sake of contradiction, suppose \(\inf(a_n)_{n = m_1}^{\infty} < \inf(a_n)_{n = m_2}^{\infty}\) for some \(m_1 \ge m_2 \ge m\).
Then by \RMK{6.3.7}(1), for all \(n \ge m_2\), \(a_n \ge \inf(a_n)_{n = m_2}^{\infty}\);
in particular, for all \(n \ge m_1\) where \(m_1 \ge m_2\), \(a_n \ge \inf(a_n)_{n = m_2}^{\infty}\).
That is, \(\inf(a_n)_{n = m_2}^{\infty}\) is a lower bound of \(\{a_n : n \ge m_1\}\),
and \(\inf(a_n)_{n = m_1}^{\infty} < \inf(a_n)_{n = m_2}^{\infty}\),
which contradicts that \(\inf(a_n)_{n = m_1}^{\infty}\) is the infimum of \(\{a_n : n \ge m_1\}\).
\end{proof}

We now describe some basic properties of limit superior and limit inferior.

\begin{proposition} \label{prop 6.4.12}
Let \((a_n)_{n = m}^{\infty}\) be a sequence of real numbers, let \(L^+\) be the \emph{limit superior} of this sequence, and let \(L^-\) be the \emph{limit inferior} of
this sequence (thus both \(L^+\) and \(L^-\) are \emph{extended} real numbers, because supremum and infimum can be extended real numbers, by \DEF{6.2.6}).
\begin{enumerate}
\item For every \(x > L^+\), there exists an \(N \ge m\) such that \(a_n < x\) \emph{for all} \(n \ge N\).
      (In other words, for every \(x > L^+\), the elements of the sequence \((a_n)_{n = m}^{\infty}\) are \emph{eventually less than} \(x\).)
      Similarly, for every \(y < L^-\) there exists an \(N \ge m\) such that \(a_n > y\) \emph{for all} \(n \ge N\).
\item For every \(x < L^+\), and every \(N \ge m\), there exists an \(n \ge N\) such that \(a_n > x\).
      (In other words, for every \(x < L^+\), the elements of the sequence \((a_n)_{n = m}^{\infty}\) exceed \(x\) \emph{infinitely often}.)
      Similarly, for every \(y > L^-\) and every \(N \ge m\), there exists an \(n \ge N\) such that \(a_n < y\).
\item We have \(\inf(a_n)_{n = m}^{\infty} \le L^- \le L^+ \le \sup(a_n)_{n = m}^{\infty}\).
\item If \(c\) is any \emph{limit point} of \((a_n)_{n = m}^{\infty}\), then we have \(L^- \le c \le L^+\).
\item If \(L^+\) is finite, then it \emph{is a limit point} of \((a_n)_{n = m}^{\infty}\).
      Similarly, if \(L^-\) is finite, then it is a limit point of \((a_n)_{n = m}^{\infty}\).
\item Let \(c\) be a real number.
      If \((a_n)_{n = m}^{\infty}\) converges to \(c\), then we must have \(L^+ = L^- = c\).
      Conversely, if \(L^+ = L^- = c\), then \((a_n)_{n = m}^{\infty}\) converges to \(c\).
\end{enumerate}
\end{proposition}

\begin{note}
\begin{enumerate}
\item 若某個數字比上極限大,則序列最終會永遠比該數字小;若某個數字比下極限小,則序列最終會永遠比該數字大。
\item 若某數字比上極限小,則不管怎麼丟掉序列前面的項,剩下的序列都還是存在一個項比該數字大(這叫做\ ``infinitely often``);若某個數字比下極限大,則不管怎麼丟掉序列前面的數字,剩下的序列都還是存在一個項比該數字小。
\item 序列的最大下界,小於等於下極限,小於等於上極限,小於等於序列的最小上界。
\item 極限點在下極限跟上極限之間
\item 若上下極限是有限的,則他們也是極限點。
\item 數列收斂到某實數,若且唯若上極限跟下極限都等於該實數。
\end{enumerate}
\end{note}

\begin{proof}
\begin{enumerate}
\item Suppose first that \(x > L^+\).
      Then by definition(\DEF{6.4.6}(2)) of \(L^+\), we have \(x > \inf(a_N^+)_{N = m}^{\infty}\).
      By \PROP{6.3.6}(actually, infimum version, \RMK{6.3.7}(3)), there must then exist an integer \(N' \ge m\) such that \(x > a_{N'}^+\).
      And by \DEF{6.4.6}(1), this means \(x > \sup(a_n)_{n = N'}^{\infty}\).
      Finally, by \PROP{6.3.6}(2) again, we have \(x > \sup(a_n)_{n = N'}^{\infty} \ge a_n\) for all \(n \ge N'\).
      So all in all, we have found \(N' \ge m\) s.t. \(x > a_n\), for all \(n \ge N'\), as desired.
      
      Now suppose first that \(y < L^-\).
      Then by definition(\DEF{6.4.6}(4)) of \(L^-\), we have \(y < \sup(a_N^-)_{N = m}^{\infty}\).
      By \PROP{6.3.6}(3), there must then exist an integer \(N' \ge m\) such that \(y < a_{N'}^-\).
      And by \DEF{6.4.6}(3), this means \(y < \inf(a_n)_{n = N'}^{\infty}\).
      Finally, by \RMK{6.3.7}(2) again, we have \(y < \inf(a_n)_{n = N'}^{\infty} \le a_n\) for all \(n \ge N'\).
      So all in all, we have found \(N' \ge m\) s.t. \(y < a_n\), for all \(n \ge N'\), as desired.
\item Suppose that \(x < L^+\).
      Then again by \DEF{6.4.6}(2) we have \(x < \inf(a_N^+)_{N = m}^{\infty}\) \MAROON{(1)}.
      Then first, by \RMK{6.3.7}(1), we have \(a_N^+ \ge \inf(a_N^+)_{N = m}^{\infty}\) for all \(N \ge m\).
      \GREEN{Let \(N'\) be arbitrary integer} s.t. \(N' \ge m\).
      Then in particular \(a_{N'}^+ \ge \inf(a_N^+)_{N = m}^{\infty}\) \MAROON{(2)}.
      With \MAROON{(1)(2)} we have \(x < a_{N'}^+\).
      And by \DEF{6.4.6}(1), this means \(x < \sup(a_n)_{n = N'}^{\infty}\).
      Then by \PROP{6.3.6}(3) again, \GREEN{there must exists} \(n' \ge N'\) s.t. \(a_{n'} > x\), \GREEN{as desired}.
      
      Now suppose that \(y > L^-\).
      Then again by \DEF{6.4.6}(4) we have \(y > \sup(a_N^-)_{N = m}^{\infty}\) \MAROON{(3)}.
      Then first, by \PROP{6.3.6}(1), we have \(a_N^- \le \sup(a_N^-)_{N = m}^{\infty}\) for all \(N \ge m\).
      \GREEN{Let \(N'\) be arbitrary integer} s.t. \(N' \ge m\).
      Then in particular \(a_{N'}^- \le \inf(a_N^+)_{N = m}^{\infty}\) \MAROON{(4)}.
      With \MAROON{(3)(4)} we have \(y > a_{N'}^-\).
      And by \DEF{6.4.6}(3), this means \(y > \inf(a_n)_{n = N'}^{\infty}\).
      Then by \RMK{6.3.7}(3) again, \GREEN{there must exists} \(n' \ge N'\) s.t. \(a_{n'} < y\), \GREEN{as desired}.
\item We first show that \(\inf(a_n)_{n = m}^{\infty} \le L^-\).
      Then
      \begin{align*}
        L^- & = \sup(a_N^-)_{N = m}^{\infty} & \text{by \DEF{6.4.6}(4)} \\
            & \ge a_m^- & \text{in particular for \(N = m\), and by \PROP{6.3.6}(1)} \\
            & = \inf(a_n)_{n = m}^{\infty} & \text{by \DEF{6.4.6}(3) and \(a_m^-\)}
      \end{align*}
      
      Now we show that \(L^+ \le \sup(a_n)_{n = m}^{\infty}\).
      Then
      \begin{align*}
        L^+ & = \inf(a_N^+)_{N = m}^{\infty} & \text{by \DEF{6.4.6}(2)} \\
            & \ge a_m^+ & \text{in particular for \(N = m\), and by \RMK{6.3.7}(1)} \\
            & = \sup(a_n)_{n = m}^{\infty} & \text{by \DEF{6.4.6}(1) and \(a_m^+\)}
      \end{align*}
      
      Finally we show that \(L^- \le L^+\).
      For the sake of contradiction, suppose \(L^+ < L^-\).
      Then
      \begin{align*}
                 & L^+ < L^- \\
        \implies & \inf(a_N^+)_{N = m}^{\infty} < \sup(a_N^-)_{N = m}^{\infty} & \text{by \DEF{6.4.6}(2)(4)} \\
        \implies & \exists N_1 \ge m \text{ s.t. } \inf(a_N^+)_{N = m}^{\infty} \le a_{N_1}^+ < \sup(a_N^-)_{N = m}^{\infty} & \text{by \RMK{6.3.7}(3),} \\
                 & & \text{by replacing \(x = \inf(a_N^+)_{N = m}^{\infty}\),} \\
                 & & \text{\(y = \sup(a_N^-)_{N = m}^{\infty}\)} \\
        \implies & a_{N_1}^+ < \sup(a_N^-)_{N = m}^{\infty} & \text{in particular} \\
        \implies & \exists N_2 \ge m \text{ s.t. } a_{N_1}^+ < a_{N_2}^- \le \sup(a_N^-)_{N = m}^{\infty} & \text{by \PROP{6.3.6}(3),} \\
                 & & \text{by replacing \(x = \sup(a_N^-)_{N = m}^{\infty}\),} \\
                 & & \text{\(y = a_{N_1}^+\)} \\
        \implies & a_{N_1}^+ < a_{N_2}^- & \text{in particular} \\
        \implies & \sup(a_n)_{n = N_1}^{\infty} < \inf(a_n)_{n = N_2}^{\infty} & \text{by \DEF{6.4.6}(1)(3)} \\
      \end{align*}
      Now, \AC{6.4.2}(rigorous piston analogy), let \(N = max(N_1, N_2)\), then we have
      \[
        \sup(a_n)_{n = N}^{\infty} \le \sup(a_n)_{n = N_1}^{\infty} < \inf(a_n)_{n = N_2}^{\infty} \le \inf(a_n)_{n = N}^{\infty},
      \]
      or \(\sup(a_n)_{n = N}^{\infty} < \inf(a_n)_{n = N}^{\infty}\).
      But that is impossible! (Since that implies \(a_N \ge \inf(a_n)_{n = N}^{\infty} > \sup(a_n)_{n = N}^{\infty}\), so \(a_N\) is greater than supremum, impossible.)
      So \(L^- \le L^+\).

      So all in all we have \(\inf(a_n)_{n = m}^{\infty} \le L^- \le L^+ \le \sup(a_n)_{n = m}^{\infty}\).
\item Suppose \(c\) is any limit point of \((a_n)_{n = m}^{\infty}\).

      We first show that \(L^- \le c\).
      For the sake of contradiction, suppose \(c < L^-\).
      So \(L^- - c > 0\), in particular we let \(\varE = (L^- - c)/2 > 0\).
      Since \(c\) is a limit point, by \DEF{6.4.1}(3) (or unwrapping, \AC{6.4.1}), \(\forall N \ge m, \exists n \ge N\) s.t. \(\abs{a_n - c} \le \varE\),
      or \(-\varE \le a_n - c \le \varE\), and thus \(a_n \le c + \varE\) \GREEN{(1)}.
      And since \(L^- - \varE < L^-\), by part(a) there exists \(N' \ge m\) s.t. \(a_{n'} > L^- - \varE\) for all \(n' \ge N'\) \GREEN{(2)}.
      In particular, given this \(N'\), by \GREEN{(1)} we can find \(n_2 \ge N'\) s.t. \(a_{n_2} \le c + \varE\), and use that \(n_2\) in \GREEN{(2)} we have \(a_{n_2} > L^- - \varE\).
      Together, we have \(L^- - \varE < a_{n_2} \le c + \varE\), or \(L^- - \varE < c + \varE\).
      But substitute \(\varE\) back to \((L^- - c)/2\), we then have \(L^- - (L^- - c)/2 < c + (L^- - c)/2\),
      or \(L^-/2 + 2/c < 2/c + L^-/2\), which is impossible.
      So it must be true that \(L^- \le c\).
      
      Now we show that \(c \le L^+\).
      For the sake of contradiction, suppose \(c > L^+\).
      So \(c - L^+ > 0\), in particular we let \(\varE = (c - L^+)/2 > 0\).
      Since \(c\) is a limit point, by \DEF{6.4.1}(3) (or unwrapping, \AC{6.4.1}), \(\forall N \ge m, \exists n \ge N\) s.t. \(\abs{a_n - c} \le \varE\),
      or \(-\varE \le a_n - c \le \varE\), and thus \(a_n \ge c - \varE\) \GREEN{(3)}.
      And since \(L^+ + \varE > L^+\), by part(a) there exists \(N' \ge m\) s.t. \(a_{n'} < L^+ + \varE\) for all \(n' \ge N'\) \GREEN{(4)}.
      In particular, given this \(N'\), by \GREEN{(3)} we can find \(n_2 \ge N'\) s.t. \(a_{n_2} \ge c - \varE\), and use that \(n_2\) in \GREEN{(4)} we have \(a_{n_2} < L^+ + \varE\).
      Together, we have \(c - \varE \le a_{n_2} < L^+ + \varE\), or \(c - \varE < L^+ + \varE\).
      But substitute \(\varE\) back to \((c - L^+)/2\), we then have \(c - (c - L^+)/2 < L^+ + (c - L^+)/2\),
      or \(2/c + L^+/2 < L^+/2 + 2/c\), which is impossible.
      So it must be true that \(c \le L^+\).

      So together, we have \(L^- \le c \le L^+\).
\item We first prove the case for \(L^+\).
      Suppose \(L^+\) is finite.
      We have to show that \(L^+\) is a limit point of the sequence.

      So by \DEF{6.4.1}(3)(or unwrapping, \AC{6.4.1}), we have to show for any \(\varE > 0\), and for any \(N \ge m\), there exists \(n \ge N\) s.t. \(\abs{a_n - L^+} \le \varE\).
      So let arbitrary \(\varE > 0\).
      And let arbitrary \(N \ge m\).
      Then since \(L^+\) is finite, \(L^+ + \varE\) is also finite so of course \(L^+ + \varE > L^+\).
      And by part(a), we can find \(N' \ge m\) s.t. \(L^+ + \varE > a_n\) \emph{for all} \(n \ge N'\).
      And in particular, given \(\max(N, N')\), \(L^+ + \varE > a_n\) \emph{for all} \(n \ge \max(N, N')\) \GREEN{(5)}.
      Also, \(L^+ - \varE\) is finite so of course \(L^+ - \varE < L^+\).
      By part(b), given that \(\max(N, N')\), there exists \(n' \ge \max(N, N')\) s.t. \(a_{n'} > L^+ - \varE\). \GREEN{(6)}.
      So by \GREEN{(5)(6)}, we have found \(n' \ge \max(N, N')\), which of course \(\ge N\), s.t. \(L^+ - \varE < a_{n'} < L^+ + \varE\),
      or \(-\varE < a_{n'} - L^+ < \varE\), in particular \(\abs{a_n - L^+} \le \varE\), as desired.
      
      Now we prove the case for \(L^-\).
      Suppose \(L^-\) is finite.
      We have to show that \(L^-\) is a limit point of the sequence.

      So by \DEF{6.4.1}(3)(or unwrapping, \AC{6.4.1}), we have to show for any \(\varE > 0\), and for any \(N \ge m\), there exists \(n \ge N\) s.t. \(\abs{a_n - L^-} \le \varE\).
      So let arbitrary \(\varE > 0\).
      And let arbitrary \(N \ge m\).
      Then since \(L^-\) is finite, \(L^- - \varE\) is also finite so of course \(L^- - \varE < L^-\).
      And by part(a), we can find \(N' \ge m\) s.t. \(L^- - \varE < a_n\) \emph{for all} \(n \ge N'\).
      And in particular, given \(\max(N, N')\), \(L^- - \varE < a_n\) \emph{for all} \(n \ge \max(N, N')\) \GREEN{(7)}.
      Also, \(L^- + \varE\) is finite so of course \(L^- + \varE > L^-\).
      By part(b), given that \(\max(N, N')\), there exists \(n' \ge \max(N, N')\) s.t. \(a_{n'} < L^- + \varE\). \GREEN{(8)}.
      So by \GREEN{(7)(8)}, we have found \(n' \ge \max(N, N')\), which of course \(\ge N\), s.t. \(L^- - \varE < a_{n'} < L^- + \varE\),
      or \(-\varE < a_{n'} - L^- < \varE\), in particular \(\abs{a_n - L^-} \le \varE\), as desired.
\item Suppose \((a_n)_{n = m}^{\infty}\) converges to \(c\), we have to show \(L^- = c = L^+\).
      Now since the sequence converges to \(c\), by \PROP{6.4.5} \(c\) is also a limit point, and \emph{the only limit point}.
      And by \CORO{6.1.17}, the sequence are bounded, and by \AC{6.3.1}, the supremum/infimum of the sequence are finite.
      And again, by \AC{6.4.2}(rigorous piston analogy), \emph{\(L^-\) and \(L^+\) are also finite}.
      So by part(e), \(L^-\) and \(L^+\) are also limit points, together that implies \(L^- = c = L^+\), otherwise the converged sequence has multiple limit points, contradicting \PROP{6.4.5}.

      Now suppose \(L^- = c = L^+\), we have to show the \((a_n)_{n = m}^{\infty}\) converges to \(c\).
      So let arbitrary \(\varE > 0\).
      We have to show there exists \(N \ge m\) s.t. \(\abs{a_n - c} \le \varE\) for all \(n \ge N\).
      Then of course \(c - \varE < c < c + \varE\).
      And of course \(c < c + \varE\), that is, \(L^+ < L^+ + \varE\).
      And by part(a), there exists \(N_1 \ge m\) s.t. \(a_n < L^+ + \varE\) for all \(n \ge N_1\) \GREEN{(9)}.
      And again of course \(c - \varE < c\), that is, \(L^- - \varE < L^-\).
      And by part(a) again, there exists \(N_2 \ge m\) s.t. \(L^- - \varE < a_n\) for all \(n \ge N_2\) \GREEN{(10)}.
      Then by \GREEN{(9)(10)}, we have found \(N = \max(N_1, N_2)\) s.t. \(L^- - \varE < a_n < L^+ + \varE\) for all \(n \ge N\).
      That is, \(c - \varE < a_n < c + \varE\) for all \(n \ge N\).
      In particular, \(\abs{a_n - c} \le \varE\) for all \(n \ge N\), as desired.
\end{enumerate}
\end{proof}

\begin{note}
Parts (c) and (d) of \PROP{6.4.12} say, in particular, that \(L^+\) is the \emph{largest} limit point of \((a_n)_{n = m}^{\infty}\), and \(L^-\) is the \emph{smallest} limit point
(providing that \(L^+\) and \(L^-\) are \emph{finite}).
Part (f) then says that if \(L^+\) and \(L-\) coincide (so there is only \emph{one} limit point), then the sequence in fact \emph{converges}.
\emph{This gives a way to test if a sequence converges}:
compute its limit superior and limit inferior, and see if they are equal.
\end{note}

We now give a basic comparison property of limit superior and limit inferior.

\begin{lemma} [Comparison principle] \label{lem 6.4.13}
Suppose that \((a_n)_{n = m}^{\infty}\) and \((b_n)_{n = m}^{\infty}\) are two sequences of real numbers such that \(a_n \le b_n\) for all
\(n \ge m\).
Then we have the inequalities
\begin{enumerate}
    \item \(\sup(a_n)_{n = m}^{\infty} \le \sup(b_n)_{n = m}^{\infty}\)
    \item \(\inf(a_n)_{n = m}^{\infty} \le \inf(b_n)_{n = m}^{\infty}\)
    \item \(\limsup_{n \toINF} a_n \le \limsup_{n \toINF} b_n\)
    \item \(\liminf_{n \toINF} a_n \le \liminf_{n \toINF} b_n\)
\end{enumerate}
\end{lemma}

\begin{proof}
We first show \BLUE{(a)}.
Suppose for the sake of contradiction that \(\sup(b_n)_{n = m}^{\infty} < \sup(a_n)_{n = m}^{\infty}\).
Then by \PROP{6.3.6}(3), there exists integer \(n' \ge m\) s.t. \(\sup(b_n)_{n = m}^{\infty} < a_{n'} \le \sup(a_n)_{n = m}^{\infty}\) \MAROON{(1)}.
But again by \PROP{6.3.6}(1), we have \(b_n \le \sup(b_n)_{n = m}^{\infty}\) for all \(n \ge m\).
In particular \(b_{n'} \le \sup(b_n)_{n = m}^{\infty}\) \MAROON{(2)}.
By \MAROON{(1)(2)}, we have \(b_{n'} \le \sup(b_n)_{n = m}^{\infty} < a_{n'}\), in particular \(b_{n'} < a_{n'}\), which contradicts \(a_n \le b_n\) for all \(n \ge m\).

For \BLUE{(b)}, suppose for the sake of contradiction that \(\inf(a_n)_{n = m}^{\infty} > \inf(b_n)_{n = m}^{\infty} \).
Then by \RMK{6.3.7}(3), there exists integer \(n' \ge m\) s.t. \(\inf(a_n)_{n = m}^{\infty} > b_{n'} > \inf(b_n)_{n = m}^{\infty} \) \MAROON{(3)}.
But again by \RMK{6.3.7}(1), we have \(a_n \ge \inf(a_n)_{n = m}^{\infty}\) for all \(n \ge m\).
In particular \(a_{n'} \ge \inf(a_n)_{n = m}^{\infty}\) \MAROON{(4)}.
By \MAROON{(3)(4)}, we have \(a_{n'} \ge \inf(a_n)_{n = m}^{\infty} > b_{n'}\), in particular \(a_{n'} > b_{n'}\), which contradicts \(a_n \le b_n\) for all \(n \ge m\).

For \BLUE{(c)}, suppose for the sake of contradiction that \(\limsup_{n \toINF} b_n < \limsup_{n \toINF} a_n\).
Then by \DEF{6.4.6}(2), we have \(\inf (b_N^+)_{N = m}^{\infty} < \inf (a_N^+)_{N = m}^{\infty}\).
By \RMK{6.3.7}(3), (with \(\inf (b_N^+)_{N = m}^{\infty}\) as \(x\), \(\inf (a_N^+)_{N = m}^{\infty}\) as \(y\),) there exists \(n' \ge m\) s.t. \(\inf (a_N^+)_{N = m}^{\infty} > b_{n'}^+ \ge \inf (b_N^+)_{N = m}^{\infty}\) \MAROON{(5)}.
But given that \(n' \ge m\), by \RMK{6.3.7}(1) we have \(a_{n'}^+ \ge \inf (a_N^+)_{N = m}^{\infty}\) \MAROON{(6)}.
So by \MAROON{(5)(6)} we have \(a_{n'}^+ \ge \inf (a_N^+)_{N = m}^{\infty} > b_{n'}^+\), in particular \(a_{n'}^+ > b_{n'}^+\).
Finally, by \DEF{6.4.6}(1), that implies \(\sup(a_n)_{n = n'}^{\infty} > \sup(b_n)_{n = n'}^{\infty}\).
But again, since \(n' \ge m\), in particular from the supposition of this lemma, \(a_n \le b_n\) for all \(n \ge n'\).
And by part(a) we have \(\sup(a_n)_{n = n'}^{\infty} \le \sup(b_n)_{n = n'}^{\infty}\), so we get a contradiction.

For \BLUE{(d)}, suppose for the sake of contradiction that \(\liminf_{n \toINF} b_n < \liminf_{n \toINF} a_n\).
Then by \DEF{6.4.6}(4), we have \(\sup (b_N^-)_{N = m}^{\infty} < \sup (a_N^-)_{N = m}^{\infty}\).
By \PROP{6.3.6}(3), (with \(\sup (a_N^-)_{N = m}^{\infty}\) as \(x\), \(\sup (b_N^-)_{N = m}^{\infty}\) as \(y\),) there exists \(n' \ge m\) s.t. \(\sup (b_N^-)_{N = m}^{\infty} < a_{n'}^- \le \sup (a_N^-)_{N = m}^{\infty}\) \MAROON{(7)}.
But given that \(n' \ge m\), by \PROP{6.3.6}(1) we have \(b_{n'}^- \le \sup (b_N^-)_{N = m}^{\infty}\) \MAROON{(8)}.
So by \MAROON{(7)(8)} we have \(b_{n'}^- \le \sup (b_N^-)_{N = m}^{\infty} < a_{n'}^-\), in particular \(b_{n'}^- < a_{n'}^-\).
Finally, by \DEF{6.4.6}(3), that implies \(\inf(b_n)_{n = n'}^{\infty} < \inf(a_n)_{n = n'}^{\infty}\).
But again, since \(n' \ge m\), in particular from the supposition of this lemma, \(a_n \le b_n\) for all \(n \ge n'\).
And by part(b) we have \(\inf(a_n)_{n = n'}^{\infty} \le \inf(b_n)_{n = n'}^{\infty}\), so we get a contradiction.
\end{proof}

\begin{corollary} [Squeeze test] \label{corollary 6.4.14}
Let \((a_n)_{n = m}^{\infty}\), \((b_n)_{n = m}^{\infty}\), and \((c_n)_{n = m}^{\infty}\) be sequences of real numbers such that
\[
    a_n \le b_n \le c_n
\]
for all \(n \ge m\).
Suppose also that \((a_n)_{n = m}^{\infty}\) and \((c_n)_{n = m}^{\infty}\) both converge to the same limit \(L\).
Then \((b_n)_{n = m}^{\infty}\) is also convergent to \(L\).
\end{corollary}

\begin{proof}
\sloppy First, since both \((a_n)_{n = m}^{\infty}\) and \((c_n)_{n = m}^{\infty}\) converge to \(L\), by \PROP{6.4.12},
we have \(\limsup_{n \toINF} a_n = \liminf_{n \toINF} a_n = L\) and \(\limsup_{n \toINF} c_n = \liminf_{n \toINF} c_n = L\).
That is, \(\limsup_{n \toINF} a_n = \liminf_{n \toINF} a_n = \limsup_{n \toINF} c_n = \liminf_{n \toINF} c_n = L\).

Then since \(b_n \le c_n\) for all \(n \ge m\), by \LEM{6.4.13}(c) we have \(\limsup_{n \toINF} b_n \le \limsup_{n \toINF} c_n = L\) \MAROON{(1)}.
And since \(a_n \le b_n\) for all \(n \ge m\), by \LEM{6.4.13}(d) we have \(L = \liminf_{n \toINF} a_n \le \liminf_{n \toINF} b_n\) \MAROON{(2)}.

Also, by \PROP{6.4.12}(C), we have \(\liminf_{n \toINF} b_n \le \limsup_{n \toINF} b_n\) \MAROON{(3)}.

So by \MAROON{(1)(2)(3)}, we have \(L \le \liminf_{n \toINF} b_n \le \limsup_{n \toINF} b_n \le L\),
which, of course, implies \(\liminf_{n \toINF} b_n = \limsup_{n \toINF} b_n = L\),
which by \PROP{6.4.12}(f) implies \((b_n)_{n = m}^{\infty}\) converges to \(L\).
\end{proof}

\begin{example} \label{example 6.4.15}
We already know (see \PROP{6.1.11}) that \(\lim_{n \toINF} 1/n = 0\).
By the limit laws (\THM{6.1.19}), this also implies that \(\lim_{n \toINF} 2/n = 0\) and \(\lim_{n \toINF} -2/n = 0\). 
The \emph{squeeze test then shows that} any sequence \((b_n)_{n = 1}^{\infty}\) for which \(-2/n \le b_n \le 2/n\) for all \(n \ge 1\) is convergent to \(0\).
For instance, we can use this to show that the sequence \((-1)^n/n + 1/n^2\) converges to zero,
or that \(2^{-n}\) converges to zero.
Note one can use induction to show that \(0 \le 2^{-n} \le 1/n\) for all \(n \ge 1\).
\end{example}

\begin{remark} \label{remark 6.4.16}
The squeeze test, combined with the limit laws and the principle that monotone bounded sequences always have limits(\PROP{6.3.8}),
allows one to compute a large number of limits.
We give some examples in the next chapter.
\end{remark}

\begin{corollary} [Zero test for sequences] \label{corollary 6.4.17}
Let \((a_n)_{n = m}^{\infty}\) be a sequence
of real numbers.
Then the limit \(\lim_{n \toINF} a_n\) exists and is equal to zero if and only if the limit \(\lim_{n \toINF} \abs{a_n}\) exists and is equal to zero.
\end{corollary}

\begin{proof}
Suppose \(\lim_{n \toINF} a_n = 0\). Then
\begin{align*}
             & \lim_{n \toINF} a_n = 0 \\
    \implies & \forall \varE > 0, \exists N \ge m \text{ s.t. } \abs{a_n - 0} \le \varE\ \forall n \ge N \\
    \implies & \forall \varE > 0, \exists N \ge m \text{ s.t. } \abs{a_n} \le \varE\ \forall n \ge N \\
    \implies & \forall \varE > 0, \exists N \ge m \text{ s.t. } \abs{\abs{a_n}} \le \varE\ \forall n \ge N & \text{nasty but of course} \\
    \implies & \forall \varE > 0, \exists N \ge m \text{ s.t. } \abs{\abs{a_n} - 0} \le \varE\ \forall n \ge N \\
    \implies & \lim_{n \toINF} \abs{a_n} = 0
\end{align*}
Now suppose \(\lim_{n \toINF} \abs{a_n} = 0\). Then
\begin{align*}
             & \lim_{n \toINF} \abs{a_n} = 0 \\
    \implies & -1 \X \lim_{n \toINF} \abs{a_n} = -1 \X 0 = 0 \\
    \implies & \lim_{n \toINF} -\abs{a_n} = 0 & \text{by \THM{6.1.19}(c)}
\end{align*}
So we have \(\lim_{n \toINF} \abs{a_n} = \lim_{n \toINF} -\abs{a_n} = 0\). \\
And (trivially) \(-\abs{a_n} \le a_n \le \abs{a_n}\) for all \(n \ge m\).
So by squeeze test \CORO{6.4.14}, \(\lim_{n \toINF} a_n = 0\).
\end{proof}

We close this section with the following \emph{improvement} to \PROP{6.1.12}.

\begin{theorem} [Completeness of the reals] \label{thm 6.4.18}
A sequence \((a_n)_{n = 1}^{\infty}\) of \textbf{real} numbers is a Cauchy sequence \textbf{if and only if} it is convergent.
\end{theorem}

\begin{remark} \label{remark 6.4.19}
Note that while this is very similar in spirit to \PROP{6.1.15}, it is a bit more general, since \PROP{6.1.15} refers to Cauchy sequences of \emph{rationals} instead of real numbers.

(Self note), \PROP{6.1.12} says convergent sequences of \emph{real} are Cauchy, \PROP{6.1.15} says Cauchy sequences of \emph{rational} are convergent.
So the last piece of the puzzle is ``Cauchy sequences of \emph{real} are also convergent''.
\end{remark}

\begin{proof}
\PROP{6.1.12} already tells us that every convergent sequence(of \emph{real}) is Cauchy, so it suffices to show that every Cauchy sequence(of \emph{real}) is convergent.

Let \((a_n)_{n = 1}^{\infty}\) be any Cauchy sequence(of \emph{real} numbers).
We know from \LEM{5.1.15} (or more precisely, from the extension of this lemma to the \emph{real} numbers, which is proven in exactly the same fashion) that the sequence \((a_n)_{n = 1}^{\infty}\) is bounded;
So by \AC{6.3.1}, the supremum and infimum of the sequence are finite numbers.
And by \PROP{6.4.12}(c), this implies that \(L^- := \liminf_{n \toINF} a_n\) and \(L^+ := \limsup_{n \toINF} a_n\) of the sequence are both finite.
To show that the sequence converges, it will suffice by \PROP{6.4.12}(f) to show that \(L^- = L^+\).

Now let \(\varE > 0\) be any real number.
Since \((a_n)_{n = 1}^{\infty}\) is a Cauchy
sequence, it must be eventually \(\varE\)-steady(by \DEF{6.1.3}), so in particular there exists an \(N \ge 1\) such that the sequence \((a_n)_{n = N}^{\infty}\) is \(\varE\)-steady.
That is(again by \DEF{6.1.3}), \(\abs{a_n - a_{n'}} \le \varE\) for all \(n, n' \ge N\).
In particular, fix \(n' = N\), we have \(\abs{a_n - a_N} \le \varE\), or \(-\varE \le a_n - a_N \le \varE\), or
\begin{center}
    \(a_N - \varE \le a_n \le a_N + \varE\), \emph{for all} \(i \ge N\).
\end{center}
Then by \PROP{6.3.6}(2), we have \(\sup(a_n)_{n = N}^{\infty} \le a_N + \varE\), and by \RMK{6.3.7}(2), we have \(a_N - \varE \le \inf(a_n)_{n = N}^{\infty}\).
But by \PROP{6.4.12}(c), \(\inf(a_n)_{n = N}^{\infty} \le L^- \le L^+ \le \sup(a_n)_{n = N}^{\infty}\), together we have
\[
    a_N - \varE \le \inf(a_n)_{n = N}^{\infty} \le L^- \le L^+ \le \sup(a_n)_{n = N}^{\infty} \le a_N + \varE.
\]
In particular,
\[
    a_N - \varE \le L^- \le L^+ \le a_N + \varE.
\]
And that implies \(0 \le L^- - (a_N - \varE) \le L^+ - (a_N - \varE) \le 2\varE\).
Just for readability, in particular: \(L^+ - (a_N - \varE) \le 2\varE\) and \(L^- - (a_N - \varE) \ge 0\).
Together that implies \(L^+ - (a_N - \varE) - (L^- - (a_N - \varE)) \le 2\varE - (L^- - (a_N - \varE)) \le 2\varE\).
Simplifying, we have \(L^+ - L^- \le 2\varE\).
And by \PROP{6.4.12}(c), \(L^+ \ge L^-\), or \(0 \le L^+ - L^-\), so together we have
\[
    0 \le L^+ - L^- \le 2\varE.
\]
Since \(\varE > 0\) is arbitrary, this is true for all \(\varE > 0\), and \(L^+\) and \(L^-\) do not depend on \(\varE\);
so we must therefore have \(L^+ = L^-\).
(If \(L^+ > L^-\) then we could set \(\varE := (L^+ - L^-)/3\) such that,
\begin{align*}
             & L^+ - L^- \le 2\varE \\
    \implies & L^+ - L^- \le 2(L^+ - L^-)/3 \\
    \implies & L^+/3 \le L^-/3 \\
    \implies & L^+ \le L^-
\end{align*}
and obtain a contradiction.)
By \PROP{6.4.12}(f) we thus see that the sequence converges.
\end{proof}

\begin{remark} \label{remark 6.4.20}
In the \emph{language of metric spaces} (see \CH{B.2}), \THM{6.4.18} asserts that the real numbers are a \textbf{complete metric space}
- that they do not contain ``holes'' the same way the rationals do.
(Certainly the rationals have lots of \emph{Cauchy} sequences \textbf{which do not converge to other rationals};
take for instance the sequence \(1, 1.4, 1.41, 1.414, 1.4142,...\) which converges to the irrational \(\sqrt{2}\).) This property is \textbf{closely related to the least upper bound property} (\THM{5.5.9}),
and is one of the principal characteristics which make the real numbers superior to the rational numbers for the purposes of doing analysis
(taking limits, taking derivatives and integrals, finding zeroes of functions, that kind of thing),
as we shall see in later chapters.
\end{remark}

\exercisesection

\begin{exercise} \label{exercise 6.4.1}
Prove \PROP{6.4.5}.
\end{exercise}

\begin{proof}
See \PROP{6.4.5}.
\end{proof}

\begin{exercise} \label{exercise 6.4.2}
State and prove analogues of \EXEC{6.1.3} and \EXEC{6.1.4} for limit points, limit superior, and limit inferior.

That is, Let \((a_n)_{n=m}^{\infty}\) be a sequence of real numbers, let \(c\) be a real number, and \(m' \ge m\) be an integer, and \(k \ge 0\) be an non-negative integer. 
\begin{enumerate}
\item \(c\) is a limit point of \((a_n)_{n=m}^{\infty}\) if and only if \(c\) is a limit point of \((a_n)_{n=m'}^{\infty}\).
\item \(\limsup_{n \toINF} a_n = c\) for \((a_n)_{n = m}^{\infty}\) if and only if \(\limsup_{n \toINF} a_n = c\) for \((a_n)_{n = m'}^{\infty}\).
\item \(\liminf_{n \toINF} a_n = c\) for \((a_n)_{n = m}^{\infty}\) if and only if \(\liminf_{n \toINF} a_n = c\) for \((a_n)_{n = m'}^{\infty}\).
\item \(c\) is a limit point of \((a_n)_{n = m}^{\infty}\) if and only if \(c\) is a limit point of \((a_{n + k})_{n = m}^{\infty}\).
\item \(\limsup_{n \toINF} a_n = c\) for \((a_n)_{n = m}^{\infty}\) if and only if \(\limsup_{n \toINF} a_n = c\) for \((a_{n + k})_{n = m}^{\infty}\).
\item \(\liminf_{n \toINF} a_n = c\) for \((a_n)_{n = m}^{\infty}\) if and only if \(\liminf_{n \toINF} a_n = c\) for \((a_{n + k})_{n = m}^{\infty}\).
\end{enumerate}
\end{exercise}

\begin{proof}
\begin{enumerate}
\item
Suppose \(c\) is a limit point of \((a_n)_{n = m}^{\infty}\).
By \DEF{6.4.1}(3), (or unwrapping, by \AC{6.4.1}), \(\forall \varE > 0, \forall N \ge m, \exists n \ge N\) s.t. \(\abs{a_n - c} \le \varE\).
In particular, since \(m' \ge m\),\(\forall \varE > 0, \forall N \ge m'\), where \(m' \ge m\), \(\exists n \ge N\) s.t. \(\abs{a_n - c} \le \varE\).
Then again by \DEF{6.4.1}(3), \(c\) is a limit point of \((a_n)_{n = m'}^{\infty}\).

Now suppose \(c\) is a limit point of \((a_n)_{n = m'}^{\infty}\).
So similarly by \DEF{6.4.1}(3), \(\forall \varE > 0, \forall N \ge m', \exists n \ge N\) s.t. \(\abs{a_n - c} \le \varE\) \MAROON{(1)}.
And since \(m' \ge m\), what we left is showing \(\forall \varE > 0, \forall N'\) s.t. \(m' > N' \ge m, \exists n' \ge N'\) s.t. \(\abs{a_{n'} - c} \le \varE\) \MAROON{(2)}.
But from \MAROON{(1)}, in particular for \(N = m'\) we can find \(n' \ge N = m'\) s.t. \(\abs{a_{n'} - c} \le \varE\), and that \(n' \ge N'\);
so we can always find such \(n' \ge N'\) for \MAROON{(2)}, so together we have \(\forall \varE > 0, \forall N \ge m, \exists n \ge N\) s.t. \(\abs{a_n - c} \le \varE\), i.e. \(c\) is a limit point of \((a_n)_{n = m}^{\infty}\).

\item
We have for all \(m' \ge m\):
\begin{align*}
             & m \le m' \\
    \implies & \sup(a_n)_{n = m}^{\infty} \ge \sup(a_n)_{n = m'}^{\infty} & \text{by \AC{6.4.2}} \\
    \implies & a_m^+ \ge a_{m'}^+ \MAROON{(3)} & \text{by \DEF{6.4.6}(1)}
\end{align*}
And for all \(m' \ge m\):
\begin{align*}
             & m \le m' \land a_m^+ \ge a_{m'}^+ \\
    \implies & \inf(a_N^+)_{N = m}^{\infty} \le \inf(a_N^+)_{N = m'}^{\infty} \land a_m^+ \ge a_{m'}^+ & \text{by \AC{6.4.2}}
\end{align*}
Now we claim that \(\inf(a_N^+)_{N = m}^{\infty} = \inf(a_N^+)_{N = m'}^{\infty}\),
for if \(\inf(a_N^+)_{N = m}^{\infty} < \inf(a_N^+)_{N = m'}^{\infty}\), then there exists \(M\) s.t. \(m \le M < m'\) and \(a_M^+ < a_{m'}^+\).
(Otherwise \(\inf(a_N^+)_{N = m}^{\infty} < \inf(a_N^+)_{N = m'}^{\infty}\) cannot be true.)
But by \MAROON{(3)} we have \(a_M^+ \ge a_{m'}^+\), so we have a contradiction.
So \(\inf(a_N^+)_{N = m}^{\infty} = \inf(a_N^+)_{N = m'}^{\infty}\);
that is, by \DEF{6.4.6}(2), \(\limsup\) for \((a_n)_{n = m}^{\infty}\) is equal to \(\limsup\) for \((a_n)_{n = m'}^{\infty}\).

\item
We have for all \(m' \ge m\):
\begin{align*}
             & m \le m' \\
    \implies & \inf(a_n)_{n = m}^{\infty} \le \inf(a_n)_{n = m'}^{\infty} & \text{by \AC{6.4.2}} \\
    \implies & a_m^- \le a_{m'}^- \MAROON{(4)} & \text{by \DEF{6.4.6}(3)}
\end{align*}
And for all \(m' \ge m\):
\begin{align*}
             & m \le m' \land a_m^- \le a_{m'}^- \\
    \implies & \sup(a_N^-)_{N = m}^{\infty} \ge \sup(a_N^-)_{N = m'}^{\infty} \land a_m^- \le a_{m'}^- & \text{by \AC{6.4.2}}
\end{align*}
Now we claim that \(\sup(a_N^-)_{N = m}^{\infty} = \sup(a_N^-)_{N = m'}^{\infty}\),
for if \(\sup(a_N^-)_{N = m}^{\infty} > \sup(a_N^-)_{N = m'}^{\infty}\), then there exists \(M\) s.t. \(m \le M < m'\) and \(a_M^- > a_{m'}^-\).
(Otherwise \(\sup(a_N^-)_{N = m}^{\infty} > \sup(a_N^-)_{N = m'}^{\infty}\) cannot be true.)
But by \MAROON{(4)} we have \(a_M^- \le a_{m'}^-\), so we have a contradiction.
So \(\sup(a_N^-)_{N = m}^{\infty} = \sup(a_N^-)_{N = m'}^{\infty}\);
that is, by \DEF{6.4.6}(4), \(\liminf\) for \((a_n)_{n = m}^{\infty}\) is equal to \(\liminf\) for \((a_n)_{n = m'}^{\infty}\).

\item
Since \((a_{n + k})_{n = m}^{\infty} = (a_n)_{n = m + k}^{\infty}\) \MAROON{(5)} so for \(m' = m + k\),
by part (a) we conclude that \(c\) is a limit point of \((a_n)_{n = m}^{\infty}\)
(iff \(c\) is a limit point of \((a_n)_{n = m + k}^{\infty}\),)
iff \(c\) is a limit point of \((a_{n + k})_{n = m}^{\infty}\).

\item
Again by \MAROON{(5)}, for \(m' = m + k\),
by part (b) we conclude that \(c\) is the \(\limsup\) for \((a_n)_{n = m}^{\infty}\)
(iff \(c\) is the \(\limsup\) for\((a_n)_{n = m + k}^{\infty}\),)
iff \(c\) is the \(\limsup\) for \((a_{n + k})_{n = m}^{\infty}\).

\item
Again by \MAROON{(5)}, for \(m' = m + k\),
by part (c) we conclude that \(c\) is the \(\liminf\) for \((a_n)_{n = m}^{\infty}\)
(iff \(c\) is the \(\liminf\) for\((a_n)_{n = m + k}^{\infty}\),)
iff \(c\) is the \(\liminf\) for \((a_{n + k})_{n = m}^{\infty}\).
\end{enumerate}
\end{proof}

\begin{exercise} \label{exercise 6.4.3}
Prove parts (c), (d), (e), (f) of \PROP{6.4.12}.
(Hint: you can use earlier parts of the proposition to prove later ones.)
\end{exercise}

\begin{proof}
See \PROP{6.4.12}.
\end{proof}

\begin{exercise} \label{exercise 6.4.4}
Prove \LEM{6.4.13}.
\end{exercise}

\begin{proof}
See \LEM{6.4.13}.
\end{proof}

\begin{exercise} \label{exercise 6.4.5}
Use \LEM{6.4.13} to prove \CORO{6.4.14}.
\end{exercise}

\begin{proof}
See \CORO{6.4.14}.
\end{proof}

\begin{exercise} \label{exercise 6.4.6}
Give an example of two \emph{bounded} sequences \((a_n)_{n = 1}^{\infty}\) and \((b_n)_{n = 1}^{\infty}\) such that \(a_n < b_n\) for all \(n \ge 1\),
but that \(\sup(a_n)_{n = 1}^{\infty} \not < \sup(b_n)_{n = 1}^{\infty}\).
Explain why this does not contradict \LEM{6.4.13}.
\end{exercise}

\begin{proof}
Let \(a_n := 1/n\), \(b_n := 2/n\) for all \(n \ge 1\).
Then \(a_n < b_n\) for all \(n \ge 1\).
But by what have shown in \EXAMPLE{6.4.15}, \(\lim_{n = 1}^{\infty} (a_n) = \lim_{n = 1}^{\infty} = 0\) so \(\lim_{n = 1}^{\infty} (a_n) \not < \lim_{n = 1}^{\infty}\).

But this does not contradict \LEM{6.4.13}, since the lemma does not say anything for ``strict less than'' relation.
\end{proof}

\begin{exercise} \label{exercise 6.4.7}
Prove \CORO{6.4.17}. Is the corollary still true if we replace zero in the statement of this Corollary by some other number?
\end{exercise}

\begin{proof}
See \CORO{6.4.17}.
Also, zero cannot be replaced by another real number \(c\).
because given that \(c\), we have a counter example:
\(\lim_{n \toINF} \abs{c \X (-1)^n)} = \lim_{n \toINF} (\abs{c} \X \abs{(-1)^n}) =  \abs{c}\),
but \(\lim_{n \toINF} c \X (-1)^n = c \X \lim_{n \toINF} (-1)^n\), which is undefined.
\end{proof}

\begin{exercise} \label{exercise 6.4.8}
Let us say that a sequence \((a_n)_{n = m}^{\infty}\) of real numbers has \(+\infty\) \emph{as a limit point} iff it has no finite upper bound,
and that it has \(-\infty\) as a limit point iff it has no finite lower bound.
With this definition, show that \(\limsup_{n \toINF} a_n\) is a limit point of \((a_n)_{n = m}^{\infty}\), and furthermore that it is larger than all the other limit points of \((a_n)_{n = m}^{\infty}\);
in other words, the limit superior is the largest limit point of a sequence.
Similarly, show that the limit inferior is the smallest limit point of a sequence.
(One can use \PROP{6.4.12} in the course of the proof.)
(\RED{warning}: To prove this exercise, any statements that have been proved \RED{but depend on the limit point in terms of \DEF{6.4.1}(3)} can not be used, or should be used with limitation,
because the ``limit points'' in these statements are only finite.)
\end{exercise}

\begin{proof}
For the case that the sequence is bounded above(below), by \AC{6.3.1}, the supremum(infimum) is finite, so by \PROP{6.4.12}(c), \(L^+\)(\(L^-\)) is finite,
and by \PROP{6.4.12}(e) we know they are limit points, and by \PROP{6.4.12}(d) we know they are largest and smallest limit points, respectively.

So we only need to show the case when the sequence is not bounded above/below.

Suppose \((a_n)_{n = m}^{\infty}\) has no upper bound, then by definition of this exercise, it has \(+\infty\) as a limit point.
By \DEF{6.2.3}(b), any limit point, which is a \emph{extended} real number, is less than or equal to \(+\infty\).
Now what we left is showing \(L^+\) is also \(\ge\) any limit point, including \(+\infty\);
that is, \(L^+ = +\infty\).
For the sake of contradiction, suppose \(L^+\) is finite.
Then \(L^+ + 1\) is also finite, and \(L^+ < L^+ + 1\).
So, by \PROP{6.4.12}(a), there exists \(N \ge m\) s.t. \(a_n < L^+ + 1\) \emph{for all} \(n \ge N\),
which implies \(a_n\) \emph{eventually} cannot exceed a finite number \(L^+ + 1\),
which contradicts that \((a_n)_{n = m}^{\infty}\) has no upper bound!.
So \(L^+ = +\infty\), so \(L^+ \ge\) any limit points.

Now suppose \((a_n)_{n = m}^{\infty}\) has no lower bound, then by definition of this exercise, it has \(-\infty\) as a limit point.
By \DEF{6.2.3}(c), any limit point, which is a \emph{extended} real number, is greater than or equal to \(-\infty\).
Now what we left is showing \(L^-\) is also \(\le\) any limit point, including \(-\infty\);
that is, \(L^- = -\infty\).
For the sake of contradiction, suppose \(L^-\) is finite.
Then \(L^+ - 1\) is also finite, and \(L^- - 1 < L^-\).
So, by \PROP{6.4.12}(a), there exists \(N \ge m\) s.t. \(L^- - 1 < a_n\) \emph{for all} \(n \ge N\),
which implies \(a_n\) \emph{eventually} cannot less than a finite number \(L^- - 1\), which contradicts that \((a_n)_{n = m}^{\infty}\) has no lower bound!.
So \(L^- = -\infty\), so \(L^- \le\) any limit points.
\end{proof}

\begin{exercise} \label{exercise 6.4.9}
Using the definition in \EXEC{6.4.8}, construct a sequence \((a_n)_{n = 1}^{\infty}\) which has exactly three limit points, at \(-\infty\), \(0\), and \(+\infty\).
\end{exercise}

\begin{proof}
Let
\begin{equation}
    a_n =
    \begin{cases}
      n, & \text{ if } n = 3k + 1 \text{ for some integer } k \\
      1/n, & \text{ if } n = 3k + 2 \text{ for some integer } k \\
      -n, & \text{ if } n = 3k \text{ for some integer } k
    \end{cases}
\end{equation}
Then by basic number theory, the definition is well-defined.

Furthermore, it is trivial to show that \(-\infty, 0, +\infty\) are limit points of this sequence by \DEF{6.4.1} and definition of \EXEC{6.4.8}.
\end{proof}

\begin{exercise}
\label{exercise 6.4.10}
Let \((a_n)_{n = N}^{\infty}\) be a sequence of real numbers, and let \((b_m)_{m = M}^{\infty}\) be another sequence of real numbers such that \emph{each \(b_m\) (may repeat) is a limit point of} \((a_n)_{n = N}^{\infty}\).
Let \(c\) be a limit point of \((b_m)_{m = M}^{\infty}\).
Prove that \(c\) is also a limit point of \((a_n)_{n = N}^{\infty}\).
(In other words, limit points of sequence of limit points of the original sequence are themselves limit points of the original sequence.)
\end{exercise}

\begin{proof}
Suppose \(c\) is a limit point of \((b_m)_{m = M}^{\infty}\), we have to show that \(c\) is also a limit point of \((a_n)_{n = N}^{\infty}\).
That is, by \DEF{6.4.1}, we have to show that \(\forall \varE > 0, \forall N' \ge N, \exists n \ge N'\) s.t. \(\abs{a_n - c} \le \varE\).

So let arbitrary \(\varE > 0\) and arbitrary \(N' \ge N\) \BLUE{(1)}.
In particular, \(\varE/2 > 0\).
And since \(c\) is a limit point of \((b_m)_{m = M}^{\infty}\), so again by \DEF{6.4.1}, \(\forall M' \ge M, \exists m \ge M'\) s.t. \(\abs{b_m - c} \le \varE/2\) \MAROON{(1)}.
We preserve that \(m \ge M'\) for later use.
In particular \(b_m\) is a limit point of \((a_n)_{n = N}^{\infty}\).
So again by \DEF{6.4.1}, given the particular \(N' \ge N\) from \BLUE{(1)}, \(\exists n \ge N'\) s.t. \(\abs{a_n - b_m} \le \varE/2\). \MAROON{(2)}
Again we preserve that \(n \ge N'\) for later use.
Then we have
\begin{align*}
    \abs{a_n - c} & = \abs{a_n - c + b_m - b_m} \\
                  & = \abs{(a_n - b_m) + (b_m - c)} \\
                  & \le \abs{a_n - b_m} + \abs{b_m - c} \\
                  & \le \varE/2 + \varE/2 & \text{by \MAROON{(1)(2)}} \\
                  & = \varE
\end{align*}
So all in all, from the given \(\varE > 0\) and \(N' \ge N\), we have found \(n \ge N'\) s.t. \(\abs{a_n - c} \le \varE\), so by \DEF{6.4.1}, \(c\) is a limit point of \((a_n)_{n = N}^{\infty}\).
\end{proof}

\section{Some standard limits} \label{sec 6.5}

\begin{note}
\RED{WARNING}: TODO: some steps need to refer to lemmas in \SEC{5.6}.
\end{note}

Armed now with the limit laws(\THM{6.1.19}) and the squeeze test(\CORO{6.4.14}), we can now compute a large number of limits.

A particularly simple limit is that
\begin{additional corollary} \label{ac 6.5.1}
The constant sequence \(c, c, c, c, ...\) have
\begin{center}
    \(\lim_{n \toINF} c = c\) for any constant \(c\).
\end{center}
(But this statement does not require limit law and squeeze test :).
In fact I have already somewhat used this fact before this section.)
\end{additional corollary}

\begin{proof}
Let \((a_n)_{n = 1}^{\infty}\) be a constant sequence where \(a_n = c\) for all \(n \ge 1\).
Then given arbitrary \(\varE > 0\), we can find \(N = 1\) s.t. \(\abs{a_n - c} = \abs{c - c} = \abs{0} = 0 < \varE\) for all \(n \ge N\).
So by \DEF{6.1.5}, the sequence converges to \(c\).
\end{proof}

In \PROP{6.1.11}, we proved that \(\lim_{n \toINF} 1/n = 0\).
This now implies

\begin{corollary} \label{corollary 6.5.1}
We have \(\lim_{n \toINF} 1/n^{1/k} = 0\) for every integer \(k \ge 1\).
\end{corollary}

\begin{proof}
From \LEM{5.6.6}(e) we know that \(1/n^{1/k}\) is a \emph{decreasing} function of \(n\), while being \emph{bounded below} by \(0\) (by \LEM{5.6.6}(c)).
By \PROP{6.3.8} (for decreasing sequences instead of increasing sequences) we thus know that this sequence converges to some limit \(L \ge 0\):
\[
    L = \lim_{n \toINF} 1/n^{1/k}.
\]
Raising this to the \(k^{th}\) power and using the limit laws (or more precisely, \THM{6.1.19}(b) and induction), we obtain
\[
    L^k = \lim_{n \toINF} 1/n.
\]
By \PROP{6.1.11} we thus have \(L^k = 0\);
but this means that \(L\) cannot be positive (else \(L^k\) would be positive, by \LEM{5.6.6}(c)), so \(L = 0\), and we are done.
\end{proof}

Some other basic limits:
\begin{lemma} \label{lem 6.5.2}
Let \(x\) be a real number.
Then the limit \(\lim_{n \toINF} x^n\) exists and is equal to zero when \(\abs{x} < 1\),
exists and is equal to \(1\) when \(x = 1\),
and diverges when \(x = -1\) or when \(\abs{x} > 1\).
\end{lemma}

\begin{proof}
First we have \(-\abs{x}^n \le x^n \le \abs{x}^n\) for all positive integer \(n\) \MAROON{(1)}.
This can be proved by \(-\abs{x} \le x \le \abs{x}\) and induction.

Suppose \(\abs{x} < 1\) (that is, \(-1 < x < 1\)); in particular \(0 \le \abs{x} < 1\).
Then by \PROP{6.3.10}, \(\lim_{n \toINF} \abs{x}^n = 0\).
Also, by \THM{6.1.19}(c), \(\lim_{n \toINF} -\abs{x}^n = -1 \X \lim_{n \toINF} \abs{x}^n = -1 \X 0 = 0\).
So by \CORO{6.4.14}, squeeze test, and \MAROON{(1)}, we have \(\lim_{n \toINF} x^n = 0\).

Now suppose \(x = 1\).
Then we have \(x^n = 1\), a \emph{constant}, for all \(n \ge 1\), by induction.
So by \AC{6.5.1}, \(\lim_{n \toINF} x^n = 1\).

Now suppose \(x = -1\) or \(\abs{x} > 1\) (that is, \(x \le -1\) or \(x > 1\)).
If \(x = -1\), then by \EXAMPLE{6.1.13}, \(\lim_{n \toINF} x^n\) does not exist.
If \(\abs{x} > 1\), we split into two cases:
\begin{itemize}
    \item \(x > 1\): Then by \EXEC{6.3.4}, \((x^n)_{n = 1}^{\infty}\) diverges, so \(\lim_{n \toINF} x^n\) does not exist.
    \item \(x < -1\): Then trivially the sequence is not eventually \(2\)-steady, so is not Cauchy, hence by \THM{6.4.18} \(\lim_{n \toINF} x^n\) does not exist.
\end{itemize}
\end{proof}

\begin{lemma} \label{lem 6.5.3}
For any \(x > 0\), we have \(\lim_{n \toINF} x^{1/n} = 1\).
\end{lemma}

\begin{proof}
We first show that for every \(\varE > 0\) and every real number \(M > 0\), there exists an \(n\) such that \(M^{1/n} \le 1 + \varE\) \BLUE{(1)}.
Since \(1 < 1 + \varE\), we have \(\frac1{1 + \varE} < 1\), and by \LEM{6.5.2}, \(\lim_{n \toINF} (\frac1{1 + \varE})^n = 0\).
But since \((\frac1{1 + \varE})^n = \frac1{(1 + \varE)^n}\), we have \(\lim_{n \toINF} \frac1{(1 + \varE)^n} = 0\).

Now we defined sequence \((a_n)_{n = 1}^{\infty}\) s.t. \(a_n = \frac1{(1 + \varE)^n}\).
Then it's trivial that \(\inf(a_n)_{n = 1}^{\infty} = 0\).
Since \(M > 0\), \(1/M > 0 = \inf(a_n)_{n = 1}^{\infty}\), and by \RMK{6.3.7}(3), there exists \(n \ge 1\) s.t. \(1/M > a_n\).
That is \(1/M > \frac1{(1 + \varE)^n}\).
And that implies \(M < (1 + \varE)^n\).
And by \LEM{5.6.9}(d), we have \(M^{1/n} < ((1 + \varE)^n)^{1/n}\), which by \LEM{5.6.6}(b) is equal to \(1 + \varE\), as desired.

Now we prove the lemma by splitting into cases:
\begin{itemize}
\item \(x \ge 1\):
    Let arbitrary \(\varE > 0\).
    We have to find \(n \ge 1\) s.t. \(\abs{x^{1/n} - 1} \le \varE\), or \(1 - \varE \le x^{1/n} \le 1 + \varE\).
    But by \BLUE{(1)}, we can find \(n \ge 1\) s.t. \(x^{1/n} \le 1 + \varE\).
    Also, since \(x \ge 1\),
    \begin{align*}
                 & x \ge 1 \\
        \implies & x^{1/n} \ge 1^{1/n} = 1 & \text{by \LEM{5.6.6}(d)(e)} \\
        \implies & x^{1/n} \ge 1 - \varE.
    \end{align*}
    So we have found \(n \ge 1\) s.t. \(1 - \varE \le x^{1/n} \le 1 + \varE\), as desired.
\item \(0 < x < 1\):
    Then \(1/x > 1\), so by the previous case we have \(\lim_{n \toINF} (1/x)^{1/n} = 1\),
    which implies \(\lim_{n \toINF} (x^{-1})^{1/n} = 1\), which by \LEM{5.6.9}(b) implies \(\lim_{n \toINF} x^{-1/n} = 1\).
    But by \THM{6.1.19}(e), we have \(\lim_{n \toINF} (x^{-1/n})^{-1} = 1^{-1} = 1\), which by \LEM{5.6.9}(b) implies \(\lim_{n \toINF} x^{1/n} = 1\), as desired.
\end{itemize}
\end{proof}

\begin{note}
We will derive a few more standard limits later on, once we develop the root and ratio tests(\SEC{7.5}) for series and for sequences.
\end{note}

\exercisesection

\begin{exercise} \label{exercise 6.5.1}
\BLUE{(1)} Show that \(\lim_{n \toINF} 1/n^q = 0\) for any \emph{rational} \(q > 0\).
(Hint: use \CORO{6.5.1} and the limit laws, \THM{6.1.19}.)

\BLUE{(2)} Conclude that the limit \(\lim_{n \toINF} n^q\) does not exist.
(Hint: argue by contradiction using \THM{6.1.19}(e).)
\end{exercise}

\begin{proof}
First we show \BLUE{(1)}.
Let arbitrary rational \(q = a/b > 0\), where \(a, b\) are positive integers.
Then by \CORO{6.5.1} we have \(\lim_{n \toINF} 1/n^{1/b} = 0\).
And with limit laws \THM{6.1.19}(b) and induction, we have \(\lim_{n \toINF} (\frac1{n^{1/b}})^a = 0^a = 0\),
which implies \(\lim_{n \toINF} \frac{1^a}{(n^{1/b})^a} = 0\),
which implies \(\lim_{n \toINF} \frac1{(n^{1/b})^a} = 0\),
which by \DEF{5.6.7} implies \(\lim_{n \toINF} \frac1{n^q} = 0\), as desired.

Now we show \BLUE{(2)}.
Suppose for sake of contradiction that \(\lim_{n \toINF} n^q\) exists and equals to \(L\).
Then
\begin{align*}
    0 & = L \X 0 \\
      & = (\lim_{n \toINF} n^q) \X 0 \\
      & = (\lim_{n \toINF} n^q) \X (\lim_{n \toINF} 1/n^q) & \text{by \BLUE{(1)}} \\
      & = \lim_{n \toINF} n^q \X 1/n^q & \text{by \THM{6.1.19}(b)} \\
      & = \lim_{n \toINF} n^q \X (1/n)^q & \text{of course...} \\
      & = \lim_{n \toINF} (n(1/n))^q & \text{by \LEM{5.6.9}(f)} \\
      & = \lim_{n \toINF} 1^q \\
      & = 1^q & \text{by \AC{6.5.1}} \\
      & = 1,
\end{align*}
which is impossible.
\end{proof}

\begin{exercise} \label{exercise 6.5.2}
Prove \LEM{6.5.2}.
(Hint: use \PROP{6.3.10}, \EXEC{6.3.4}, and the squeeze test, \CORO{6.4.14}.)
\end{exercise}

\begin{proof}
See \LEM{6.5.2}.
\end{proof}

\begin{exercise} \label{exercise 6.5.3}
Prove \LEM{6.5.3}.
(Hint: you may need to treat the cases \(x \ge 1\) and \(x < 1\) separately.
You might wish to first use \LEM{6.5.2} to prove the preliminary result that for every \(\varE > 0\) and every real number \(M > 0\), there exists an \(n\) such that \(M^{1/n} \le 1 + \varE\).)
\end{exercise}

\begin{proof}
See \LEM{6.5.3}.
\end{proof}
\section{Subsequences} \label{sec 6.6}

Some sequences were convergent to a single limit, while others had multiple limit points.
For instance, the sequence
\[
    1.1, 0.1, 1.01, 0.01, 1.001, 0.001, 1.0001,...
\]
has two limit points at \(0\) and \(1\) (which are incidentally also the \(\liminf\) and \(\limsup\) respectively),
but is not actually convergent (since \(0\) and \(1\) are not equal, and by (contrapositive of) \PROP{6.4.5}).
However, while this sequence is not convergent, it does appear to contain \emph{convergent components};
it seems to be \emph{a mixture of} two convergent subsequences, namely
\[
    1.1, 1.01, 1.001,...
\]
and
\[
    0.1, 0.01, 0.001,....
\]
To make this notion more precise, we need a notion of `\emph{subsequence}.

\begin{definition} [Subsequences] \label{def 6.6.1}
Let \((a_n)_{n = 0}^{\infty}\) and \((b_n)_{n = 0}^{\infty}\) be sequences of real numbers.
We say that \((b_n)_{n = 0}^{\infty}\) is a \emph{subsequence} of \((a_n)_{n = 0}^{\infty}\)
iff there exists a function \(f : \SET{N} \rightarrow \SET{N}\) which is \emph{strictly increasing}
(i.e., \(f(n +1 ) > f(n)\) for all \(n \in \SET{N}\)) such that
\begin{center}
    \(b_n = a_{f(n)}\) for all \(n \in \SET{N}\).
\end{center}

More generally, we say that \((b_n)_{n=m'}^{\infty}\) is a subsequence of \((a_n)_{n=m}^{\infty}\) if there exists a strictly increasing function \(f: \{ n \in \SET{N}: n \ge m'\} \to \{ n \in \SET{N}: n \ge m\}\) such that \(b_n = a_{f(n)}\) for all \(n \ge m'\).
\end{definition}

\begin{note}
最後一個\ general case 是從\ Errata 補的;
目的是定義從任何\ index 起始的兩個\ sequence 之間的\ subsequence 關係。
\end{note}

\begin{example} \label{example 6.6.2}
If \((a_n)_{n = 0}^{\infty}\) is a sequence, then \((a_{2n})_{n = 0}^{\infty}\) is a subsequence of \((a_n)_{n = 0}^{\infty}\),
since the function \(f : \SET{N} \rightarrow \SET{N}\) defined by \(f(n) := 2n\) is a strictly increasing function from \(\SET{N}\) to \(\SET{N}\).
Note that we do not assume \(f\) to be bijective, although it is necessarily injective.
Why? Because ``strictly increasing'' implies injective.
Recall that injectivity: \(n \ne m \implies f(n) \ne f(m)\).
So let natural numbers \(n \ne m\).
Then by trichotomy, either \(n < m\) or \(n > m\).
If \(n < m\) then since \(f\) is strictly increasing, \(f(n) < f(m)\);
If \(n > m\) then since \(f\) is strictly increasing, \(f(m) < f(n)\);
so in all cases \(f(n) \ne f(m)\), as desired.

More informally, the sequence
\[
    a_0, a_2, a_4, a_6,...
\]
is a subsequence of
\[
    a_0, a_1, a_2, a_3, a_4,....
\]
\end{example}

\begin{example} \label{example 6.6.3}
The two sequences
\[
    1.1, 1.01, 1.001,...
\]
and
\[
    0.1, 0.01, 0.001,...
\]
mentioned earlier are both subsequences of
\[
    1.1, 0.1, 1.01, 0.01, 1.001, 1.0001,...
\]
\end{example}

The property of being a subsequence is reflexive and transitive, though not symmetric:
\begin{lemma} \label{lem 6.6.4}
Let \((a_n)_{n = 0}^{\infty}\), \((b_n)_{n = 0}^{\infty}\), and \((c_n)_{n = 0}^{\infty}\) be sequences of real numbers.
Then \((a_n)_{n = 0}^{\infty}\) is a subsequence of \((a_n)_{n = 0}^{\infty}\).
Furthermore, if \((b_n)_{n = 0}^{\infty}\) is a subsequence of \((a_n)_{n = 0}^{\infty}\), and \((c_n)_{n = 0}^{\infty}\) is a subsequence of \((b_n)_{n = 0}^{\infty}\),
then \((c_n)_{n = 0}^{\infty}\) is a subsequence of \((a_n)_{n = 0}^{\infty}\).
\end{lemma}

\begin{proof}
For reflexivity, just let \(f(n) = n\).
For transitivity, suppose \((b_n)_{n = 0}^{\infty}\) is a subsequence of \((a_n)_{n = 0}^{\infty}\), and \((c_n)_{n = 0}^{\infty}\) is a subsequence of \((b_n)_{n = 0}^{\infty}\).
Then there exists
\begin{itemize}
    \item[\MAROON{(1)}] \(f : \SET{N} \rightarrow \SET{N}\) s.t. \(f\) is strictly increasing and \(b_n = a_{f(n)}\) for all \(n \in \SET{N}\).
    \item[\MAROON{(2)}] \(g : \SET{N} \rightarrow \SET{N}\) s.t. \(g\) is strictly increasing and \(c_n = b_{g(n)}\) for all \(n \in \SET{N}\).
\end{itemize}

Now we claim that the composition function \(f \circ g\) is what we need: \(f \circ g\) is strictly increasing, and \(c_n = a_{(f \circ g)(n)}\).

Now suppose natural numbers \(n > m\).
Then
\begin{align*}
             & n > m \\
    \implies & g(n) > g(m) & \text{since \(g\) is strictly increasing} \\
    \implies & f(g(n)) > f(g(m)) & \text{since \(f\) is strictly increasing} \\
    \implies & (f \circ g)(n) > (f \circ g)(m) & \text{by \DEF{3.3.10}}
\end{align*}

Also, given any \(n \in \SET{n}\),
\begin{align*}
    c_n & = b_{g(n)} & \text{by \MAROON{(2)}} \\
        & = a_{f(g(n))} & \text{by \MAROON{(1)}} \\
        & = a_{(f \circ g)(n)} & \text{by \DEF{3.3.10}}
\end{align*}
\end{proof}

We now relate the concept of subsequences to the concept of limits and limit points.
\begin{proposition} [Subsequences related to \textbf{limits}] \label{prop 6.6.5}
Let \((a_n)_{n = 0}^{\infty}\) be a sequence of real numbers, and let \(L\) be a real number.
Then the following two statements are \emph{logically equivalent} (each one implies the other):
\begin{enumerate}
    \item The sequence \((a_n)_{n = 0}^{\infty}\) \textbf{converges to} \(L\)
    \item \textbf{Every} subsequence of \((a_n)_{n = 0}^{\infty}\) converges to \(L\).
\end{enumerate}
\end{proposition}

\begin{proof}
Suppose part(a) is true.
And suppose \((b_n)_{n = 0}^{\infty}\) is an arbitrary subsequence of \((a_n)_{n = 0}^{\infty}\), where \(b_n = a_{f(n)}\) with \(f\) being a strictly increasing function from \(\SET{N}\) to \(\SET{N}\) \MAROON{(1)}.
Let arbitrary \(\varE > 0\), we have to show that there exists \(N \ge 0\) s.t. \(\abs{b_n - L} \le \varE\) for all \(n \ge N\).

Now since \((a_n)_{n = 0}^{\infty}\) converges to \(L\), we can found that \(N_a \ge 0\) s.t. \(\abs{a_n - L} \le \varE\) for all \(n \ge N_a\) \MAROON{(2)}.
Also, there must exist \(N_b \ge 0\) s.t. \(f(n) \ge N_a\) for all \(n \ge N_b\) \MAROON{(3)}, otherwise \(f(n) < N_a\) for all \(n \in \SET{N}\), which, with the similar reason in \RMK{6.6.9}, implies \(f\) is not a strictly increasing function from \(\SET{N}\) to \(\SET{N}\).
So we have found \(N_b \ge 0\) s.t. for all \(n \ge N_b\), we have
\begin{align*}
             & f(n) \ge N_a & \text{by \MAROON{(3)}} \\
    \implies & \abs{a_{f(n)} - L} \le \varE & \text{by \MAROON{(2)}} \\
    \implies & \abs{b_n - L} \le \varE & \text{by \MAROON{(1)}},
\end{align*}
as desired.

Now suppose part(b) is true.
Then in particular by reflexivity \LEM{6.6.4}, \((a_n)_{n = 0}^{\infty}\) is a subsequence of itself, so \((a_n)_{n = 0}^{\infty}\) converges to \(L\).
\end{proof}

\begin{proposition}  [Subsequences related to \textbf{limit points}] \label{prop 6.6.6}
Let \((a_n)_{n = 0}^{\infty}\) be a sequence of real numbers, and let \(L\) be a real number.
Then the following two statements are logically equivalent.
\begin{enumerate}
    \item \(L\) is a \textbf{limit point} of \((a_n)_{n = 0}^{\infty}\).
    \item \textbf{There exists} a subsequence of \((a_n)_{n = 0}^{\infty}\) which converges to \(L\).
\end{enumerate}
\end{proposition}

\begin{proof}
See \EXEC{6.6.5}.
\end{proof}

\begin{remark} \label{remark 6.6.7}
The above two propositions give a \emph{sharp contrast} between the notion of a limit, and that of a limit point.
When a sequence has a limit \(L\), then \textbf{all subsequences} also converge to \(L\).
But when a sequence has \(L\) as a limit point, then only \textbf{some subsequences} converge to \(L\).
\end{remark}

We can now prove an important \emph{theorem} in real analysis, due to Bernard Bolzano (1781 -- 1848) and Karl Weierstrass (1815 --1897):
every \emph{bounded} sequence has a convergent \emph{sub}sequence.

\begin{theorem} [Bolzano-Weierstrass theorem] \label{thm 6.6.8}
Let \((a_n)_{n = 0}^{\infty}\) be a \emph{bounded} sequence
(i.e., there exists a real number \(M > 0\) such that \(\abs{a_n} \le M\) for all \(n \in \SET{N}\)).
Then there is at least one subsequence of \((a_n)_{n = 0}^{\infty}\) which converges.
\end{theorem}

\begin{proof}
Let \(L\) be the \emph{limit superior} of the sequence \((a_n)_{n = 0}^{\infty}\).
Since we have \(-M \le a_n \le M\) for all natural numbers \(n\), it follows from the comparison principle (\LEM{6.4.13}) that \(-M \le L \le M\).
(\(-M = \limsup_{n \toINF} -M \le \limsup_{n \toINF} a_n\) and \( \limsup_{n \toINF} a_n \le \limsup_{n \toINF} M = M\).)
In particular, \(L\) is a real number (not \(+\infty\) or \(-\infty\)).
By \PROP{6.4.12}(e), \(L\) is thus a limit point of \((a_n)_{n = 0}^{\infty}\).
Thus by \PROP{6.6.6}, there exists a subsequence of \((a_n)_{n = 0}^{\infty}\) which converges (to \(L\)).
\end{proof}

\begin{remark} \label{remark 6.6.9}
The Bolzano-Weierstrass theorem says that if a sequence is \emph{bounded}, then eventually it has no choice but to ``converge'' in some places(limit points);
it has ``no room'' to spread out and stop itself from acquiring limit points.
It is not true for unbounded sequences; for instance, the sequence \(1, 2, 3,...\) has no convergent subsequences whatsoever.
Why? Suppose there is a convergent subsequence \((a_n)_{n = 1}^{\infty}\) of \(1, 2, 3,...\).
First, any subsequence of \(1, 2, 3,...\) is also strictly increasing(trivial) \MAROON{(1)}.
And since \(1, 2, 3,...\) is a sequence of positive integers, \((a_n)_{n = 1}^{\infty}\) is also a sequence of positive integers \MAROON{(2)}.
And since \((a_n)_{n = 1}^{\infty}\) is convergent, it is \(0.5\)-steady; so there exists \(N \ge 1\) s.t. \(\abs{a_i - a_j} \le 0.5\) for all \(i, j \ge N\).
But by \MAROON{(2)} \(a_i, a_j\) in particular are integers, that must implies \(a_i - a_j = 0\), that is, \(a_i = a_j\), for all \(i, j \ge 1\), otherwise their difference is greater than or equal to \(1\).
But that implies \((a_n)_{n = 1}^{\infty}\) is in fact not a subsequence because it violates \MAROON{(1)}, so we have a contradiction.
So the sequence \(1, 2, 3,...\) has no convergent subsequences.

In the language of topology, this means that the interval \(\{ x \in \SET{R} : -M \le x \le M \}\) is \emph{compact}, whereas an unbounded set such as the real line \(\SET{R}\) is not compact.
The distinction between compact sets and non-compact sets will be very important in later chapters - of similar importance to the distinction between finite sets and infinite sets.
\end{remark}

\exercisesection

\begin{exercise} \label{exercise 6.6.1}
Prove \LEM{6.6.4}.
\end{exercise}

\begin{proof}
See \LEM{6.6.4}.
\end{proof}

\begin{exercise} \label{exercise 6.6.2}
Can you find two sequences \((a_n)_{n = 0}^{\infty}\) and \((b_n)_{n = 0}^{\infty}\) which are \emph{not} the same sequence, but such that each is a subsequence of the other?
\end{exercise}

\begin{proof}
Just let \((a_n)_{n = 0}^{\infty} = 1, -1, 1, -1, 1, -1,...\) and \((b_n)_{n = 0}^{\infty} = -1, 1, -1, 1, -1, 1,...\).
Then first, \(f(n) = n + 1\) is a strictly increasing function.
And both \(b_n = a_{f(n)} = a_{n + 1}\), and \(a_n = b_{f(n)} = b_{n + 1}\), for all \(n\).
So by \DEF{6.6.1}, the two sequences are subsequences of each other.

Also refer to \href{https://math.stackexchange.com/questions/3677059}{this}
\end{proof}

\begin{exercise} \label{exercise 6.6.3}
Let \((a_n)_{n = 0}^{\infty}\) be a sequence which is \emph{not} bounded.
Show that there exists a subsequence \((b_n)_{n = 0}^{\infty}\) of \((a_n)_{n = 0}^{\infty}\) such that \(\lim_{n \toINF} 1/b_n\) exists and is equal to zero.
(Hint: for each natural number \(j\), \emph{recursively} introduce the quantity \(n_j := \min\{n \in \SET{N} : \abs{a_n} \ge j; n > n_{j - 1}\}\)
(omitting the condition \(n > n_{j - 1}\) when \(j = 0\)), 
first explaining why the set \(\{n \in \SET{N} : \abs{a_n} \ge j; n > n_{j - 1}\}\) is non-empty.
Then set \(b_j := a_{n_j}\).)
\end{exercise}

\begin{proof}
Let \(f: \SET{N} \rightarrow \SET{N}\) be a function such that \(f(j) = n_j\) where
\begin{equation} \label{eq 6.2}
    n_j = 
    \begin{cases}
        \min\{n \in \SET{N} : \abs{a_n}\ \RED{>}\ 0\} \text{, if } j = 0 \\
        \min\{n \in \SET{N} : (\abs{a_n}\ \RED{>}\ j) \land (n > n_{j - 1})\} \text{, if } j \ge 1
    \end{cases}
\end{equation}
We have to show \(f\) is well-defined, and show that the subsequence defined by \(f\) meets the requirement.

Since \((a_n)_{n = 0}^{\infty}\) is unbounded, for any \(j \in \SET{N}\) (and any integer \(n_{j - 1}\), when \(j \ge 1\)) we can find another integer \(n\) s.t. \(\abs{a_n} > j\) and \(n > n_{j - 1}\);
for if we cannot find such integer, then it implies that for every \(n > n_{j - 1}\), \(\abs{a_n} \le j\), which implies \((a_n)_{n = 0}^{\infty}\) is bounded!
So in particular, for each \(j \in \SET{N}\), the introduced set applied by \(\min\) operator in \EQUATION{6.2} is \textbf{non-empty}.
Also, since the non-empty set is \textbf{the set of natural numbers}, by well-ordering-principle,
(well, although it is introduced in \PROP{8.1.4}, everyone struggling until now can find a wiki or something similar and know the principle is equivalent to mathematical induction) the minimum element exists.
(Note that if the non-empty set is a set of real numbers, then the ``minimum'' may not exist, although by \THM{5.5.9} the supremum/infimum exist.)
So given any \(j \in \SET{N}\), the corresponding minimum of the set in \EQUATION{6.2} exists, so \(f\) is well-defined.

Now again, from the definition of \EQUATION{6.2}, we have \(j_1 > j_2 \implies n_{j_1} > n_{j_2}\), so \(f\) is strictly increasing, so we can use \(f\) to define a subsequence.
We define \((b_n)_{n = 0}^{\infty}\) where \(b_n = a_{f(n)}\).
Then by \EQUATION{6.2}, we have \(\abs{a_{f(j)}} = \abs{b_j} > j\), which implies \(0 \le 1/\abs{b_j} < 1/j\), which trivially implies \(0 \le \abs{1/b_j} < 1/j\).
Since \(\lim_{j \toINF} 0 = 0\) and \(\lim_{j \toINF} 1/j = 0\), by squeeze test, \CORO{6.4.14}, we have \(\lim_{j \toINF} \abs{1/b_j} = 0\).
And by zero test, \CORO{6.4.17}, we have \(\lim_{n \toINF} 1/b_n = 0\), as desired.
\end{proof}

\begin{exercise} \label{exercise 6.6.4}
Prove \PROP{6.6.5}.
(Note that one of the two implications has a very short proof.)
\end{exercise}

\begin{proof}
See \PROP{6.6.5}.
\end{proof}

\begin{exercise} \label{exercise 6.6.5}
Prove \PROP{6.6.6}.
(Hint: to show that (a) implies (b), define the numbers \(n_j\) for each \emph{natural} numbers \(j\) by the formula \(n_j := \min \{ n > n_{j - 1} : \abs{a_n - L} \le 1/j \} \), with the convention \(n_0 := 0\),
explaining why the set \(\{ n > n_{j - 1} : \abs{a_n - L} \le 1/j \}\) is non-empty.
Then consider the sequence \(a_{n_j}\).)
\end{exercise}

\begin{proof}
Suppose \(L\) is a limit point of \((a_n)_{n = 0}^{\infty}\).
We have to find a subsequence \((b_n)_{n = 0}^{\infty}\) of \((a_n)_{n = 0}^{\infty}\) s.t. \((b_n)_{n = 0}^{\infty}\) converges to \(L\).
Now let \(f: \SET{N} \rightarrow \SET{N}\) be a function such that \(f(j) = n_j\) where
\begin{equation} \label{eq 6.3}
    n_j = 
    \begin{cases}
        0 \text{, if } j = 0 \\
        \min\{n > n_{j - 1} \land \abs{a_n - L } \le 1/j\} \text{, if } j \ge 1
    \end{cases}
\end{equation}
First, given arbitrary \(j \in \SET{N}\) such that \(j \ge 1\), we have \(1/j > 0\).
And since \(L\) is a limit point of \((a_n)_{n = 0}^{\infty}\),
by \DEF{6.4.1}(3), given \(n_{j - 1} \in \SET{N}\), we can find an integer \(n \ge n_{j - 1} + 1\), which \(> n_{j - 1}\), s.t. \(\abs{a_n - L} \le 1/j\).
And that implies the set of the \(\min\) operator in \EQUATION{6.3} is non-empty;
And similarly with \EXEC{6.6.4}, since the non-empty set is a set of natural numbers, by well-ordering principle, the minimum of the set exists.
Hence the \EQUATION{6.3} is well-defined.
Now again, from the definition of \EQUATION{6.3}, we have \(j_1 > j_2 \implies n_{j_1} > n_{j_2}\), so \(f\) is strictly increasing, so we can use \(f\) to define a subsequence.
We define \((b_n)_{n = 0}^{\infty}\) where \(b_n = a_{f(n)}\).
Now we show that \((b_n)_{n = 0}^{\infty}\) converges to \(L\).
So let arbitrary \(\varE > 0\).
In particular by \EXEC{5.4.4}, we can find \(j \in \SET{N}\) s.t. \(1/j < \varE\).
And from \EQUATION{6.3}, we have \(\abs{b_j - L} = \abs{a_{n_j} - L} \le 1/j\).
Also for any \(j' \ge j\), from \EQUATION{6.3} we have \(\abs{b_{j'} - L} = \abs{a_{n_{j'}} - L} \le 1/{j'} \le 1/{j}\).
All in all, we have found \(j \in \SET{N}\) s.t. \(\abs{b_{j'} - L} \le 1/j < \varE\) for all \(j' \ge j\).
So \((b_n)_{n = 0}^{\infty}\) converges to \(L\).

Now suppose there exists a subsequence \((b_n)_{n = 0}^{\infty}\) of \((a_n)_{n = 0}^{\infty}\) which converges to \(L\).
We have to show that \(L\) is a limit point of \((a_n)_{n = 0}^{\infty}\).
Let \(f\) be the strictly increasing function that is used to define \((b_n)_{n = 0}^{\infty}\).
And let arbitrary real number \(\varE > 0\), arbitrary natural number \(N \ge 0\).
We have to show that there exists \(n \ge N\) s.t. \(\abs{a_n - L} \le \varE\) \MAROON{(1)}, to show \(L\) is a limit point of \((a_n)_{n = 0}^{\infty}\).
Since \((b_n)_{n = 0}^{\infty}\) converges to \(L\), we can find an integer \(n_b \ge 0\) s.t. \(\abs{b_n - L} \le \varE\) for all \(n \ge n_b\) \MAROON{(2)}.
In particular, \(\abs{a_{f(n)} - L} \le \varE\) for all \(n \ge n_b\).
But since \(f\) is strictly increasing function from \(\SET{N}\) to \(\SET{N}\), we can find \(n_a \ge n_b\) s.t. \(f(n_a) \ge N\) (otherwise we will get a contradiction that \(f\) is not strictly increasing).
So we have found \(f(n_a) \ge N\), (where \(n_a \ge n_b\)) s.t.
\begin{align*}
             & f(n_a) \ge N \land n_a \ge n_b \\
    \implies & f(n_a) \ge N \land \abs{b_{n_a} - L} \le \varE & \text{by \MAROON{(2)}} \\
    \implies & f(n_a) \ge N \land \abs{a_{f(n_a)} - L} \le \varE
\end{align*}
So \MAROON{(1)} is satisfied, as desired.
\end{proof}
\section{Real exponentiation, part II} \label{sec 6.7}

In \SEC{5.6} we defined \(x^q\) for all \emph{rational} \(q\) and positive real numbers \(x\), but we have not yet defined \(x^{\alpha}\) when \(\alpha\) is \emph{real}.
We now rectify this situation \emph{using limits} (in a similar way as to how we defined all the other standard operations on the real numbers).
First, we need a lemma:
\begin{lemma} [\emph{Continuity} of exponentiation] \label{lem 6.7.1}
Let \(x > 0\), and let \(\alpha\) be a real number.
Let \((q_n)_{n = 1}^{\infty}\) be any sequence of \emph{rational} numbers converging to \(\alpha\).
Then \((x^{q_n})_{n = 1}^{\infty}\) is also a convergent sequence.
Furthermore, if \((q'_n)_{n = 1}^{\infty}\) is any other sequence of \emph{rational} numbers converging to \(\alpha\), then \((x^{q'_n})_{n = 1}^{\infty}\) has the same limit as \((x^{q_n})_{n = 1}^{\infty}\)
\[
    \lim_{n \toINF} x^{q_n} = \lim_{n \toINF} x^{q'_n}.
\]
\end{lemma}

\begin{note}
若一個\ sequence of rationals 收斂,則給定任一正實數\ \(x\),以\ \(x\) 為底、且以這個\ sequence 為指數的新的\ sequence 也收斂。
\end{note}

\begin{proof}
\RED{WARNING}: Some steps implicitly use \LEM{5.6.9}.

There are three cases: \(x < 1\), \(x = 1\), and \(x > 1\).
The case \(x = 1\) is rather easy (because then \(x^q = 1\) for all rational \(q\)).
We first show the case \(x > 1\).

Let us first prove that \((x^{q_n})_{n = 1}^{\infty}\) converges.
By \THM{6.4.18} it is enough to show that \((x^{q_n})_{n = 1}^{\infty}\) is a Cauchy sequence.
That is, by \DEF{6.1.3}(3), we have to show that \((x^{q_n})_{n = 1}^{\infty}\) is eventually \(\varE\)-steady for all \(\varE > 0\).

To do this, we need to \emph{estimate the distance} between \(x^{q_n}\) and \(x^{q_m}\) (given any integers \(n, m \ge 1\));
let us say for the time being that \(q^n \ge q^m\), so that \(x^{q_n} \ge x^{q_m}\) \MAROON{(1)} (since \(x > 1\)).
We have
\begin{align*}
    d(x^{q_n}, x^{q_m}) & = \abs{x^{q_n} - x^{q_m}} \\
                        & = x^{q_n} - x^{q_m} & \text{by \MAROON{(1)}} \\
                        & = x^{q_m}(x^{q_n - q_m} - 1) \MAROON{(2)}.
\end{align*}
Since \((q_n)_{n = 1}^{\infty}\) is a convergent sequence, (by \CORO{6.1.17}) it has some upper bound \(M\), so we have \(q_m \le M\);
since \(x > 1\), we have \(x^{q_m} \le x^M\) \MAROON{(3)}.
Thus continue from \MAROON{(2)}, we have
\begin{align*}
    d(x^{q_n}, x^{q_m}) & = x^{q_m}(x^{q_n - q_m} - 1) & \text{by \MAROON{(2)}} \\
                        & \le x^M(x^{q_n - q_m} - 1). \MAROON{(4)} & \text{by \MAROON{(3)}}
\end{align*}

Now let arbitrary \(\varE > 0\).
We know by \LEM{6.5.3} that since the sequence \((x^{1/k})_{k=1}^{\infty}\) converges to \(1\), and \(\varE x^{-M} > 0\), it is eventually \(\varE x^{-M}\)-close to \(1\).
Thus there exists some \(K \ge 1\) s.t.
\[
    \abs{x^{1/K} - 1} \le \varE x^{-M}. \MAROON{(5)}
\]
Now since \((q_n)_{n = 1}^{\infty}\) is convergent, it is a Cauchy sequence so is \(1/K\)-steady,
hence there is an \(N \ge 1\) such that \(q_n\) and \(q_m\) are \(1/K\)-close for all \(n, m \ge N\) \MAROON{(6)}.
Thus we have for every \(n, m \ge M\), such that \(q^n \ge q^m\) \MAROON{(7)},
\begin{align*}
    d(x^{q_n}, x^{q_m}) & \le x^M(x^{q_n - q_m} - 1) & \text{by \MAROON{(4)}} \\
                        & \le x^M(x^{1/K} - 1) & \text{by \MAROON{(6)(7)}} \\
                        & \le x^M \varE x^{-M} & \text{by \MAROON{(5)}} \\
                        & = \varE.
\end{align*}
By symmetry we also have this bound when \(n, m \ge N\) \emph{but} \(q_n \le q_m\).
Since \(\varE > 0\) is arbitrary, the sequence \((x^{q_n})_{n = N}^{\infty}\) is eventually \(\varE\)-steady for all \(\varE > 0\), and is thus a Cauchy sequence as desired.
This proves the convergence of \((x^{q_n})_{n = 1}^{\infty}\).

Now we prove the second claim.
It will suffice to show that
\[
    \lim_{n \toINF} x^{q_n - q'_n} = 1 \MAROON{(8)},
\]
since
\begin{align*}
    \lim_{n \toINF} x^{q_n} & = \lim_{n \toINF} x^{q_n - q'_n} x^{q'_n} & \text{by \SEC{5.6}} \\
                            & = \lim_{n \toINF} x^{q_n - q'_n} \X \lim_{n \toINF} x^{q'_n} \MAROON{(9)}, & \text{by \THM{6.1.19}(c)}.
\end{align*}
So So if \(\lim_{n \toINF} x^{q_n - q'_n} = 1\),
then from \MAROON{(9)} we have \(\lim_{n \toINF} x^{q_n} = \lim_{n \toINF} x^{q'_n}\).
Define \((r_n)_{n = 1}^{\infty}\) s.t. \(r_n := q_n - q'_n\).
Then
\begin{align*}
    \lim_{n \toINF} r_n & = \lim_{n \toINF} q_n - q'_n \\
                        & = \lim_{n \toINF} q_n - \lim_{n \toINF} q'_n & \text{by \THM{6.1.19}(d)} \\
                        & = \alpha - \alpha \\
                        & = 0 \MAROON{(10)}.
\end{align*}
To prove \MAROON{(8)} we have to show that for every \(\varE > 0\), the sequence \((x^{r_n})_{n = 1}^{\infty}\) is eventually \(\varE\)-close to \(1\).
But again from \LEM{6.5.3} we know that the sequence \((x^{1/k})_{k = 1}^{\infty}\) is eventually \(\varE\)-close to \(1\).
Also, since \(x > 0\), \(1/x > 0\), so again by \LEM{6.5.3}, \(\lim_{k \toINF} (1/x)^{1/k} = 1\), that is, \(\lim_{k \toINF} x^{-1/k} = 1\).
So, we know that \((x^{-1/k})_{k = 1}^{\infty}\) is also eventually \(\varE\)-close to \(1\).
Thus we can find a \(K\) such that \(x^{1/K}\) and \(x^{-1/K}\) are \emph{both} \(\varE\)-close to \(1\).
But given that \(K\), since by \MAROON{(10)} \((r_n)_{n = 1}^{\infty}\) is convergent to \(0\), it is eventually \(1/K\)-close to 0,
so we can find \(N \ge 1\) s.t. for all \(n \ge N\), \(-1/K \le r_n \le 1/K\), and thus \(x^{-1/K} \le x^{r_n} \le x^{1/K}\).
In particular since both \(x^{1/K}\) and \(x^{-1/K}\) are \(\varE\)-close to \(1\), by \PROP{4.3.7}(f) (precisely, real number version, \PROP{5.4.16}) \(x^{r_n}\) is also \(\varE\)-close to \(1\) for all \(n \ge N\), as desired.

Next we show when \(x < 1\) then \((x^{q_n})_{n = 1}^{\infty}\) also converges.
That is, \((x^{q_n})_{n = 1}^\infty\) is a Cauchy sequence.

To do this, we need to \emph{estimate the distance} between \(x^{q_n}\) and \(x^{q_m}\);
let us say for the time being that \(q_n \le q_m\), so that \(x^{q_n} \ge x^{q_m}\) (since \(x < 1\)).
We have
\[
    d(x^{q_n}, x^{q_m}) = x^{q_n} - x^{q_m} = x^{q_m} (x^{q_n - q_m} - 1).
\]
Since \((q_n)_{n = 1}^{\infty}\) is a convergent sequence, it has some \emph{lower bound} \(M\);
since \(x < 1\), we have \(x^{q_m} \le x^M\).
Thus
\[
    d(x^{q_n}, x^{q_m}) = \abs{x^{q_n} - x^{q_m}} \le x^M (x^{q_n - q_m} - 1).
\]
Now let arbitrary \(\varE > 0\).
We know by \LEM{6.5.3} that the sequence \((x^{1/K})_{k = 1}^{\infty}\) is eventually \(\varE x^{-M}\)-close to \(1\).
Thus there exists some \(K \ge 1\) such that
\[
    \abs{x^{1/K} - 1} \le \varE x^{-M}.
\]
Now since \((q_n)_{n = 1}^{\infty}\) is convergent, it is a Cauchy sequence, and so there is an \(N \ge 1\) such that \(q_n\) and \(q_m\) are \(1/K\)-close for all \(n, m \ge N\).
Thus we have
\[
    d(x^{q_n}, x^{q_m}) \le x^M (x^{q_n - q_m} - 1) \le x^M (x^{1/K} - 1) \le x^M \varE x^{-M} = \varE.
\]
for every \(n, m \ge N\) such that \(q_n \le q_m\).
By symmetry we also have this bound when \(n, m \ge N\) \emph{and} \(q_n \ge q_m\).
Thus the sequence \((x^{q_n})_{n = 1}^{\infty}\) is \(\varE\)-steady.
Since \(\varE > 0\) is arbitrary, the sequence \((x^{q_n})_{n = 1}^{\infty}\) is eventually \(\varE\)-steady for all \(\varE > 0\), and is thus a Cauchy sequence as desired.
This proves the convergence of \((x^{q_n})_{n = 1}^\infty\) when \(x < 1\).

Now we prove the second claim.
It will suffice to show that
\[
    \lim_{n \toINF} x^{q_n - q_n'} = 1,
\]
since the claim would then follow from limit laws
(since \(x^{q_n} = x^{q_n - q_n'} x^{q_n'}\)).

Write \(r_n := q_n - q_n'\);
by limit laws we know that \((r_n)_{n = 1}^{\infty}\) converges to \(0\).
We have to show that for every \(\varE > 0\), the sequence \((x^{r_n})_{n = 1}^{\infty}\) is eventually \(\varE\)-close to \(1\).
But from \LEM{6.5.3} we know that the sequence \((x^{1/K})_{k = 1}^\infty\) is eventually \(\varE\)-close to \(1\).
Since \(\lim_{k \to \infty} x^{-1/K}\) is also equal to \(1\) by \LEM{6.5.3}, we know that \((x^{-1/K})_{k = 1}^{\infty}\) is also eventually \(\varE\)-close to \(1\).
Thus we can find a \(K\) such that \(x^{1/K}\) and \(x^{-1/K}\) are \emph{both} \(\varepsilon\)-close to \(1\).
But given that \(K\), since \((r_n)_{n = 1}^\infty\) is convergent to \(0\), it is eventually \(1/K\)-close to \(0\), so that eventually \(-1/K \le r_n \le 1/K\), and thus since \(x < 1\) we have \(x^{1/K} \le x^{r_n} \le x^{-1/K}\).
In particular \(x^{r_n}\) is also eventually \(\varE\)-close to \(1\) (by \PROP{4.3.7}(f), or real version, \PROP{5.4.16}), as desired.
\end{proof}

We may now make the following (well-defined) definition.
\begin{definition} [Exponentiation to a real exponent] \label{def 6.7.2}
Let \(x > 0\) be real, and let \(\alpha\) be a \emph{real} number.
We define the quantity \(x^{\alpha}\) by the formula \(x^{\alpha} = \lim_{n \toINF} x^{q_n}\),
where \((q_n)_{n = 1}^{\infty}\) is \emph{any} sequence of rational numbers converging to \(\alpha\).
\end{definition}

\begin{note}
Let us check that this definition is \emph{well-defined}.
First of all, given any real number \(\alpha\) we always have at least one sequence \((q_n)_{n = 1}^{\infty}\) of \emph{rational} numbers converging to \(\alpha\),
by the definition of real numbers (by \DEF{5.3.1}, \(\LIM\), and \PROP{6.1.15}, \(\lim\) can supersede \(\LIM\)).
Secondly, given \emph{any} such sequence \((q_n)_{n = 1}^{\infty}\), the limit \(\lim_{n \toINF} x^{q_n}\) exists by \LEM{6.7.1}.
Finally, even though there can be \emph{multiple choices} for the sequence \((q_n)_{n = 1}^{\infty}\), \emph{they all give the same limit} by \LEM{6.7.1}.
Thus this definition is well-defined.
\end{note}

\begin{note}
If \(\alpha\) is not just real but \emph{rational}, i.e., \(\alpha = q\) for some rational \(q\),
then this definition could in principle be inconsistent with our earlier definition of exponentiation in \SEC{5.6}.
But in this case \(\alpha\) is clearly the limit of the sequence \((q)_{n = 1}^{\infty}\) (\(\alpha = q = \lim_{n \toINF} q\)), so by \DEF{6.7.2} \(x^{\alpha} = \lim_{n \toINF} x^q\), which is a sequence of constant, so by \AC{6.5.1}, \(= x^q\).
Thus the new definition of exponentiation is consistent with the old one.
\end{note}

\begin{proposition} \label{prop 6.7.3}
All the results of \LEM{5.6.9}, which held for \emph{rational} numbers (exponents), continue to hold for real numbers (exponents).
\end{proposition}

\begin{proof}
The idea is to start with \LEM{5.6.9} for \emph{rationals} and then \emph{take limits} to obtain the corresponding results for reals.

So let \(x, y > 0\) be positive reals, and \(q, r\) be \emph{reals}.
\begin{enumerate}
\item
    We have to show \(x^r\) is positive.
    First (by \DEF{5.3.1} and \PROP{6.1.15}), we can write \(r = \lim_{n \toINF} r_n\) for some Cauchy sequence of rationals \((r_n)_{n = 1}^{\infty}\).
    Since \((r_n)_{n = m}^{\infty}\) is Cauchy, by \CORO{6.1.17} it's bounded by some \emph{rationals} \(M \ge 0\) (by \LEM{5.1.14}, \DEF{5.1.12}).
    That is, \(-M \le r_n \le M\) for all \(n \ge 1\) \MAROON{(a1)}.
    And since \(M\) is rational, by \LEM{5.6.9}(a), \(x^{-M}\) and \(x^{M}\) is positive.
    Also by \LEM{5.6.9}(e), if \(0 < x < 1\), from \MAROON{(a1)} we have \(0 < x^M \le x^{r_n} \le x^{-M}\) for all \(n \ge 1\);
    if \(x \ge 1\), from \MAROON{(a1)} we have \(0 < x^{-M} \le x^{r_n} \le x^M\) for all \(n \ge 1\).
    In all cases, we have \(x^{r_n} > \min(x^M, x^{-M} > 0)\) for all \(n \ge 1\).
    That is, \((x^{r_n})_{n = 1}^{\infty}\) is positively bounded away from zero, so by \DEF{5.4.1} and \DEF{5.4.3}, \(x^r = \lim_{n \toINF} x^{r_n}\) is positive.

\item
    First we show \(x^{q + r} = x^q x^r\).
    Again, by the definition of real numbers (\DEF{5.3.1}, \(\LIM\), and \PROP{6.1.15}, \(\LIM = \lim\)), we can write \(q = \lim_{n \toINF} q_n\) and \(r = \lim_{n \toINF} r_n\) for some sequences \((q_n)_{n = 1}^{\infty}\) and \((r_n)_{n = 1}^{\infty}\) of \emph{rationals}.
    Then by the limit laws,
    \begin{align*}
        q + r & = \lim_{n \toINF} q_n + \lim_{n \toINF} r_n \\
              & = \lim_{n \toINF} q_n + r_n & \text{by \THM{6.1.19}(a)}
    \end{align*}
    And by \DEF{6.7.2}, we have
    \begin{align*}
        x^{q + r} & = \lim_{n \toINF} x^{q_n + r_n} & \MAROON{(b1)}; \\
              x^q & = \lim_{n \toINF} x^{q_n} & \MAROON{(b2)}; \\
              x^r & = \lim_{n \toINF} x^{r_n} & \MAROON{(b3)}.
    \end{align*}
    But by \LEM{5.6.9}(b) (applied to \emph{rational} exponents) we have \(x^{q_n + r_n} = x^{q_n} x^{r_n}\) for all \(n \ge 1\) \MAROON{(b4)}.
    Thus we have
    \begin{align*}
        x^{q + r} & = \lim_{n \toINF} x^{q_n + r_n} & \text{by \MAROON{(b1)}} \\
                  & = \lim_{n \toINF} x^{q_n} x^{r_n} & \text{by \MAROON{(b4)}} \\
                  & = \lim_{n \toINF} x^{q_n} \lim_{n \toINF} x^{r_n} & \text{by \THM{6.1.19}(b)} \\
                  & = x^q \lim_{n \toINF} x^{r_n} & \text{by \MAROON{(b2)}} \\
                  & = x^q x^r & \text{by \MAROON{(b3)}},
    \end{align*}
    as desired.
    
    \RED{TODO}: show \((x^q)^r = x^{qr}\).

\item
    We have to show that \(x^{-r} = 1/x^r\).
    Again we can let \(r = \lim_{n \toINF} r_n\) \MAROON{(c1)} for some sequence of rationals \((r_n)_{n = 1}^{\infty}\).
    And by \THM{6.1.19}(c), \(-r = -\lim_{n \toINF} r_n = \lim_{n \toINF} -r_n\) \MAROON{(c2)}.
    And by part(a) we know that \(x^{-r} > 0\).
    Also by \LEM{5.6.9}(a) (since \(r_n\) is rational) \(x^{r_n} > 0\) for all \(n \ge 1\), we have \(1/x^{r_n} > 0\) for all \(n \ge 1\) \MAROON{(c3)}.
    Thus we have
    \begin{align*}
        x^{-r} & = \lim_{n \toINF} x^{-r_n} & \text{by \DEF{6.7.2} and \MAROON{(c2)}} \\
               & = \lim_{n \toINF} 1/x^{r_n} & \text{by \LEM{5.6.9}(c)} \\
               & = (\lim_{n \toINF} x^{r_n})^{-1} & \text{by \MAROON{(c3)} and \THM{6.1.19}(e)} \\
               & = (x^r)^{-1} & \text{by \MAROON{(c1)} and \DEF{6.7.2}}\\
               & = 1/x^r & \text{by \DEF{5.3.16}}
    \end{align*}
\item
    Suppose \(q > 0\), we have to show \(x > y\) if and only if \(x^q > y^q\).
    Again, we can let \(q = \lim_{n \toINF} q_n\), for some sequence of rationals \((q_n)_{n = 1}^{\infty}\) and the sequence is \emph{bounded away from zero} (\(q_n \ge M > 0\) for all \(n \ge 1\)).
    So by \DEF{6.7.2}, \(x^q = \lim_{n \toINF} x^{q_n}\), \(y^q = \lim_{n \toINF} y^{q_n}\) \MAROON{(d1)}.
    
    Now first, suppose \(x > y\).
    Since \(x > y\), for all \(n \ge 1\), by \LEM{5.6.9}(d), we have \(x^{q_n} > y^{q_n}\).
    Then by \LEM{6.4.13}, we have \(\limsup_{n \toINF} x^{q_n} \ge \limsup_{n \toINF} y^{q_n}\).
    But since by \LEM{6.7.1} \((x^{q_n})_{n = 1}^{\infty}\) and \((y^{q_n})_{n = 1}^{\infty}\) are \emph{convergent} sequence, by multiple applications of \PROP{6.4.12}, that implies \(\lim_{n \toINF} x^{q_n} \ge \lim_{n \toINF} y^{q_n}\).
    And by \MAROON{(d1)}, we have \(x^q \ge y^q\).
    
    Now what we left is to show \(x^q \ne y^q\).
    For the sake of contradiction, suppose that \(x^q = y^q\).
    Then \((x^q)^{1/q} = (y^q)^{1/q}\).
    But by part(b), LHS \(= x^{q \X 1/q} = x\), RHS \(= y^{q \X 1/q} = y\), so we have \(x = y\), which contradicts that \(x > y\).
    
    Now suppose \(x^q > y^q\), we have to show that \(x > y\).
    Again for the sake of contradiction, suppose \(x \le y\), i.e. \(y \ge x\).
    Then by the previous case we have \(y^q \ge x^q\), which contradicts \(x^q > y^q\).
    So we must have \(x > y\).
\item
    Suppose \(x > 1\).
    We have to show that \(x^q > x^r\) if and only if \(q > r\).
    So suppose \(x^q > x^r\), we have to show \(q > r\).
    Then
    \begin{align*}
             & x^q > x^r \\
        \iff & x^q - x^r > 0 \\
        \iff & x^r(x^{q - r} - 1) > 0 & \text{using part(b)} \\
        \iff & x^{q - r} - 1 > 0 & \text{by part(a) and \(x^r > 0\)} \\
        \iff & x^{q - r} > 1 \MAROON{(e1)}
    \end{align*}
    And from \MAROON{(e1)}, clearly \(q \ne r\).
    So for the sake of contradiction, suppose \(q < r\).
    Then \(r - q > 0\).
    Since \(x > 1\), from part(d), we have \(x^{r - q} > 1^{r - q} = 1\).
    And with \MAROON{(e1)} we have
    \begin{align*}
                 & x^{r - q} x^{q - r} > 1 \X 1 = 1 \\
        \implies & x^{(r - q) + (q - r)} > 1 & \text{by part(b)} \\
        \implies & x^0 = 1 > 1,
    \end{align*}
    which is impossible.
    So we must have \(q > r\)

    Now suppose \(q > r\), we have to show \(x^q > x^r\).
    But
    \begin{align*}
                 & q > r \\
        \implies & q - r > 0 \\
        \implies & x^{q - r} > 1^{q - r} = 1 & \text{since \(x > 1\) and part(d)} \\
        \implies & x^{q - r} - 1 > 0 \\
        \implies & x^r(x^{q - r} - 1) > x^r \X 0 = 0 \\
        \implies & x^q - x^r > 0 & \text{by part(b)} \\
        \implies & x^q > x^r.
    \end{align*}
    
    Now suppose \(x < 1\).
    We have to show that \(x^q > x^r\) if and only if \(q < r\).
    But
    \begin{align*}
             & x^q > x^r \\
        \iff & (x^q)^{-1} < (x^r)^{-1} & \text{by \PROP{5.4.8}} \\
        \iff & x^{-q} < x^{-r} & \text{by part(b)} \\
        \iff & (x^{-1})^q < (x^{-1})^r & \text{by part(b)} \\
        \iff & q < r. & \text{since \(x < 1 \iff x^{-1} > 1\), and by previous case}
    \end{align*}
\item
    \begin{align*}
        (xy)^q & = \lim_{n \toINF} (xy)^{q_n} & \text{by \DEF{6.7.2}} \\
               & = \lim_{n \toINF} x^{q_n} y^{q_n} & \text{by \LEM{5.6.9}(f)} \\
               & = \lim_{n \toINF} x^{q_n} \lim_{n \toINF} y^{q_n} & \text{by \THM{6.1.19}(b)} \\
               & = x^q y^q & \text{by \DEF{6.7.2}}
    \end{align*}
\end{enumerate}
\end{proof}

\exercisesection

\begin{exercise} \label{exercise 6.7.1}
Prove the remaining components of \PROP{6.7.3}.
\end{exercise}

\begin{proof}
See \PROP{6.7.3}.
\end{proof}
\renewcommand\thechapter{\Alph{chapter}}
\setcounter{chapter}{0}
\chapter{Appendix: the basics of mathematical logic}\label{ch a}

\setcounter{section}{6}
\section{Equality}\label{section a.7}

\setcounter{axiom}{3}
\begin{axiom} [Substitution axiom] \label{axm a.7.4}
Given any two objects \(x\) and \(y\) of the \emph{same type}, if \(x = y\), then \(f(x)= f(y)\) for all \emph{functions or operations} \(f\).
Similarly, for any property \(P(x)\) depending on \(x\),if \(x = y\), then \(P(x)\)and \(P(y)\) are equivalent statements.
\end{axiom}

\end{CJK*}
\end{document}
