\chapter{The real numbers} \label{ch 5}

\begin{note}
We defined the natural numbers using the five Peano axioms, and postulated that such a number system existed;
this is plausible, since the natural numbers correspond to the very intuitive and fundamental notion of \emph{sequential counting}.
\end{note}

\begin{note}
The symbols \(\SET{N}\), \(\SET{Q}\), and \(\SET{R}\) stand for ``natural'', ``quotient'', and ``real'' respectively.
\(\SET{Z}\) stands for ``Zahlen'', the German word for numbers.
There is also the \emph{complex numbers} \(\SET{C}\), which obviously stands for ``complex''.
\end{note}

\begin{note}
\emph{Formal} means ``having the form of'';
at the beginning of our construction the expression \(a \M b\) did not actually \emph{mean} the difference \(a - b\), since the symbol \(\M\) was meaningless.
It only had the \emph{form} of a difference.
Later on we defined subtraction and verified that the formal difference was equal to the actual difference, so this eventually became a non-issue, and our symbol for formal differencing was discarded.
Somewhat confusingly, this use of the term ``formal'' is unrelated to the notions of a formal argument and an informal argument.
\end{note}

\begin{note}
There is a fundamental area of mathematics where the rational number system \emph{does not suffice} - that of \emph{geometry} (the study of lengths, areas, etc.).
For instance, a right-angled triangle with both sides equal to \(1\) gives a hypotenuse of \(\sqrt{2}\), which is an \emph{irrational} number, i.e., not a rational number; see \PROP{4.4.4}.
Things get even worse when one starts to deal with the sub-field of geometry known as \emph{trigonometry}, when one sees numbers such as \(\pi\) or \(\cos(1)\), which turn out to be in some sense ``even more'' irrational than \(\sqrt{2}\).
(These numbers are known as \emph{transcendental numbers}, but to discuss this further would be far beyond the scope of this text.)
Thus, in order to have a number system which can adequately describe geometry
- or even something as simple as measuring lengths on a line
- one needs to replace the rational number system with the real number system.
\end{note}

\begin{note}
In the constructions of integers and rationals, the task was to introduce one more \emph{algebraic} operation to the number system
- e.g., one can get integers from naturals by introducing subtraction, and get the rationals from the integers by introducing division.
But to get the reals from the rationals is to pass \emph{from a ``discrete'' system to a ``continuous'' one}, and requires the introduction of a somewhat different notion
- that of a \emph{limit}.
\end{note}

\begin{note}
The limit is a concept which on one level is quite intuitive, but to pin down rigorously turns out to be quite difficult.
(Even such great mathematicians as Euler and Newton had difficulty with this concept.
It was only in the nineteenth century that mathematicians such as Cauchy and Dedekind figured out how to deal with limits rigorously.)
\end{note}

\begin{note}
In \SEC{4.4} we explored the ``gaps'' in the rational numbers;
now we shall fill in these gaps using limits to create the real numbers.
The real number system will end up being a lot like the rational numbers, but will have some new operations
- notably that of \emph{supremum}, which can then be used to define limits and thence to everything else that calculus needs.
\end{note}

\begin{note}
The procedure we give here of obtaining the real numbers as the limit of sequences of rational numbers may seem rather complicated.
However, it is in fact an instance of a very general and useful procedure, that of \emph{\href{https://www.wikiwand.com/en/Complete_metric_space}{completing} one metric space to form another}.
\end{note}

\newcommand{\LIM}{\text{LIM}}
\newcommand{\varE}{\varepsilon}
\newcommand{\toINF}{\to \infty}

\section{Cauchy sequences}

\begin{definition} [Sequences]  \label{def 5.1.1}
Let \(m\) be an integer.
A sequence \((a_n)_{n = m}^{\infty}\) of \emph{rational} numbers is any function from the set \( \{n \in \SET{Z} : n \ge m\} \) to \(\SET{Q}\),
i.e., a mapping which assigns to each integer \(n\) greater than or equal to \(m\), a rational number \(a_n\).
More informally, a sequence \((a_n)_{n = m}^{\infty}\)
of rational numbers is a collection of rationals \(a_m, a_{m + 1}, a_{m + 2},...\)
\end{definition}

\begin{example}
Simple example.
\end{example}

We want to define the \emph{real} numbers as the \emph{``limits''} of sequences of \emph{rational} numbers.
To do so, we have to distinguish which sequences of rationals are \emph{convergent} and which ones are not.
To do this we use the definition of \DEF{4.3.4} \(\varepsilon\)-closeness.

\begin{definition} [\(\varepsilon\)-steadiness] \label{def 5.1.3}
Let \(\varepsilon > 0\).
A sequence \((a_n)_{n = 0}^{\infty}\) is said to be \emph{\(\varepsilon\)-steady} iff each pair \(a_j, a_k\) of sequence elements is \(\varepsilon\)-close \emph{for every} natural number \(j, k\).
In other words, the sequence \(a_0, a_1, a_2,...\) is \(\varepsilon\)-steady iff \(d(a_j, a_k) \le \varepsilon\) for all \(j, k\).
\end{definition}

\begin{remark} \label{remark 5.1.4}
This definition is not standard in the literature;
we will not need it outside of this section;
similarly for the concept of ``eventual \(\varepsilon\)-steadiness'' below.
We have defined \(\varepsilon\)-steadiness for sequences whose index starts at \(0\), but clearly we can make a similar notion for sequences whose indices start from any other number:
a sequence \(a_N, a_{N + 1}, ...\) is \(\varepsilon\)-steady if one has \(d(a_j, a_k) \le \varepsilon\) for all \(j, k \ge N\).
\end{remark}

\begin{example} \label{example 5.1.5}
The sequence \(1, 0, 1, 0, 1,...\) is \(1\)-steady, but is not \(1/2\)-steady.
The sequence \(0.1, 0.01, 0.001, 0.0001,...\) is \(0.1\)-steady, but is not \(0.01\)-steady (why? Because the distance of the first two elements \(d(0.1, 0.01) = 0.09 > 0.01\)).
The sequence 1, 2, 4, 8, 16,... is not \(\varepsilon\)-steady for any \(\varepsilon\) (why? \MAROON{(1)})
The sequence 2, 2, 2, 2,... is \(\varepsilon\)-steady for every \(\varepsilon > 0\).
\end{example}

\begin{proof}
\MAROON{(1)} Given arbitrary \(\varepsilon\), by \PROP{4.4.1} there exists a natural number \(N\) s.t. \(N > \varepsilon\).
And by definition of the sequence, \(a_{N + 1} = 2^{N + 1}\), which \(\ge 2(N + 1)\), since \(N + 1\) must be positive and that satisfies \EXEC{4.3.5}.
And \(2(N + 1) - a_0 = 2N + 2 - a_0 = 2N + 2 - 1 = 2N + 1 > N\), which implies \(2^{N + 1} - a_0 > N\), that is, \(a_{N + 1} - a_0 > N\).
So we can find the pair \(a_{N + 1}, a_1\) s.t. \(d(a_{N + 1}, a_1) > N > \varepsilon\), so by \DEF{5.1.3}, the sequence is not \(\varepsilon\)-steadiness.
Since \(\varepsilon\) is arbitrary, for all \(\varepsilon\) the sequence is not \(\varepsilon\)-steady.
\end{proof}

\begin{note}
The notion of \(\varepsilon\)-steadiness of a sequence is simple, but does not really capture the \emph{limiting} behavior of a sequence, because it is \emph{too sensitive to the initial members} of the sequence.
So we need a more robust notion of steadiness that does not care about the initial members of a sequence.
\end{note}

\begin{definition} [Eventual \(\varepsilon\)-steadiness] \label{def 5.1.6}
Let \(\varepsilon > 0\).
A sequence \((a_n)_{n = 0}^{\infty}\) is said to be \emph{eventually \(\varepsilon\)-steady} iff the sequence \(a_N, a_{N + 1}, a_{N + 2},...\) is \(\varepsilon\)-steady \emph{for some} natural number \(N \ge 0\).
In other words, the sequence \(a_0, a_1, a_2, ...\) is eventually \(\varepsilon\)-steady iff there exists an \(N \ge 0\) such that \(d(a_j, a_k) \le \varepsilon\) for all \(j, k \ge N\).
\end{definition}

\begin{example} \label{example 5.1.7}
\raggedright The sequence \(a_1, a_2, ...\) defined by \(a_n := 1/n\), (that is, the sequence \(1, 1/2, 1/3, 1/4,...\)) is not \(0.1\)-steady, but is eventually \(0.1\)-steady, because the sequence \(a_{10}, a_{11}, a_{12}, ...\) (that is, \(1/10, 1/11, 1/12,...\)) is 0.1-steady.
The sequence \(10, 0, 0, 0, 0, ...\) is not \(\varepsilon\)-steady for any \(\varepsilon\) less than \(10\), but it is eventually \(\varepsilon\)-steady for every \(\varepsilon > 0\) (why? \MAROON{(1)}).
\end{example}

\begin{proof}
\MAROON{(1)} The distance of the first two elements of the sequence is \(d(10, 0) = 10\), so every \(\varepsilon < 10\) is less than \(d(10, 0)\).
But given any \(\varepsilon > 0\), let \(N = 1\), then for every \(j, k \ge N\), \(d(a_i, a_j) = d(0, 0) = 0 < \varepsilon\).
By \DEF{5.1.6}, the sequence is eventually \(\varepsilon\)-steady.
Since \(\varepsilon\) is arbitrary, for all \(\varepsilon > 0\) the sequence is eventual \(\varepsilon\)-steady.
\end{proof}

Now we can finally define the correct notion of what it means for a sequence of rationals to ``want'' to converge.

\begin{definition} [Cauchy sequences] \label{def 5.1.8}
A sequence \((a_n)_{n = 0}^{\infty}\) of \emph{rational} numbers is said to be a \emph{Cauchy sequence} iff for every rational \(\varepsilon > 0\), the sequence \((a_n)_{n = 0}^{\infty}\) is eventually \(\varepsilon\)-steady.
In other words, the sequence \(a_0, a_1, a_2,...\) is a Cauchy sequence iff for every \(\varepsilon > 0\), there exists an \(N \ge 0\) such that \(d(a_j, a_k) \le \varepsilon\) for all \(j, k \ge N\).
\end{definition}

\begin{remark} \label{remark 5.1.9}
At present, the parameter \(\varepsilon\) is restricted to be a positive \emph{rational};
we cannot take \(\varepsilon\) to be an arbitrary positive \emph{real} number, because the real numbers have not yet been constructed.
However, once we do construct the real numbers, we shall see that \DEF{5.1.8} will not change if we require \(\varepsilon\) to be real instead of rational(\PROP{6.1.4}).
In other words, we will eventually prove that
\begin{center}
    (a sequence is eventually \(\varepsilon\)-steady for every \emph{rational} \(\varepsilon > 0\)) if and only if (it is eventually \(\varepsilon'\)-steady for every \emph{real} \(\varepsilon' > 0\)).
\end{center}
(See \PROP{6.1.4}.)
This rather subtle distinction between a rational \(\varepsilon\) and a real \(\varepsilon'\) turns out \emph{not} to be very important in the long run, and the reader is advised not to pay too much attention as to what type of number \(\varepsilon\) should be.
\end{remark}

\begin{example} [\emph{Informal}] \label{example 5.1.10}
Consider the sequence \(1.4, 1.41, 1.414, 1.4142,...\) mentioned earlier.
This sequence is already \(0.1\)-steady.
If one discards the first element \(1.4\), then the remaining sequence \(1.41, 1.414, 1.4142,...\) is now \(0.01\)-steady, which means that the original sequence was eventually \(0.01\)-steady. 
Discarding the next element gives the \(0.001\)-steady sequence \(1.414, 1.4142,...\);
thus the original sequence was eventually \(0.001\)-steady. 
Continuing in this way it seems plausible that this sequence is in fact \(\varepsilon\)-steady for every \(\varepsilon > 0\), which seems to suggest that this is a Cauchy sequence.
However, this discussion is not rigorous for several reasons, for instance we have not precisely defined what this sequence \(1.4, 1.41, 1.414, \textbf{...}\) really is.
An example of a rigorous treatment follows next.
\end{example}

\begin{note}
我不確定課本說我們還沒定義\ \(1.4, 1.41, 1.414, ...\) 的意思是指什麼。也許可能是我們目前有的工具只有數學歸納法? 目前只能用數學歸納法(或遞迴,或者公式,嚴格來說還是遞迴)來定義\ sequence,但\ \(1.4, 1.41, 1.414, ...\) 其實跟數學歸納法無關,所以嚴格來說我們根本沒說清楚他是什麼?
\end{note}

\begin{proposition} \label{prop 5.1.11}
\sloppy The sequence \(a_1, a_2, a_3,...\) defined by \(a_n := 1/n\) (i.e., the sequence \(1, 1/2, 1/3,...\)) is a Cauchy sequence.
\end{proposition}

\begin{note}
Minor: the index is start from \(1\), not from \(0\).
\end{note}

\begin{proof}
We have to show that for every \(\varepsilon > 0\), the sequence \(a_1, a_2,...\) is eventually \(\varepsilon\)-steady.
So let \(\varepsilon > 0\) be \emph{arbitrary}.
We now have to find a number \(N \ge 1\) such that the sequence \(a_N, a_{N + 1},...\) is \(\varepsilon\)-steady.
This means that \(d(a_j, a_k) \le \varepsilon\) for every \(j, k \ge N\), i.e.
\begin{center}
    \(\abs{1 / j - 1 / k} \le \varepsilon\) for every \(j, k \ge N\).
\end{center}
Now since \(j, k \ge N\), we know that
\begin{align*}
             & 0 < 1 / j \le 1 / N \land 0 < 1 / k \le 1 / N \\
    \implies & 0 \le 1 / j \le 1 / N \land 0 \le 1 / k \le 1 / N & \text{in particular by \DEF{4.2.8}} \\
    \implies & 0 \le 1 / j \le 1 / N \land -(1 / N) \le - (1 / k) \le 0 & \text{by \EXEC{4.2.6}} \\
    \implies & -(1 / N) \le 1 / j - 1 / k \le 1 / N & \text{by \PROP{4.2.9}(e)} \\
    \implies & \abs{1 / j - 1 / k} \le 1 / N & \text{by \PROP{4.3.3}(b)}.
\end{align*}
So in order to force \(\abs{1 / j - 1 / k}\) to be less than or equal to \(\varepsilon\), it would be sufficient for \(1/N\) to be less than (or equal to) \(\varepsilon\).
So all we need to do is choose an \(N\) such that \(1 / N\) is less than \(\varepsilon\), or in other words that \(N\) is
greater than \(1/\varepsilon\).
But this can be done thanks to \PROP{4.4.1}.
(\(\varepsilon\) is rational, so \(1/\varepsilon\) is also rational, by \PROP{4.4.1} we can find a natural number \(N > 1/\varepsilon\).)
\end{proof}

\begin{note}
As you can see, verifying from first principles (i.e., without using any of the machinery of limits, etc.) that a sequence is a Cauchy sequence requires some effort, even for a sequence as simple as \(1/n\).
\emph{The part about selecting an \(N\) can be particularly difficult for beginners} - one has to \emph{think in reverse}, working out what conditions on \(N\) would suffice to force the sequence \(a_N, a_{N + 1}, a_{N + 2}, ...\) to be \(\varepsilon\)-steady, and then finding an \(N\) which obeys those conditions.
Later we will develop some limit laws which allow us to determine when a sequence is Cauchy more easily.
\end{note}

\begin{note}
先去想當\ \(N\) 滿足什麼條件時會導致整個\ sequence 是\ \(\varepsilon\)-steady,然後再找出這個\ \(N\)。
\end{note}

We now relate the notion of a Cauchy sequence to another basic notion, that of a \emph{bounded} sequence.

\begin{definition} [Bounded sequences] \label{def 5.1.12}
Let \(M \ge 0\) be rational.
A \emph{finite} sequence \(a_1, a_2, ..., a_n\) is \emph{bounded by} \(M\) iff \(\abs{a_i} \le M\) for all \(1 \le i \le n\).
An infinite sequence \((a_n)_{n = 1}^{\infty}\) is \emph{bounded by} \(M\) iff \(\abs{a_i} \le M\) for all \(i \ge 1\).
A sequence is said to be \emph{bounded} iff it is bounded by \(M\) \emph{for some} rational \(M \ge 0\).
\end{definition}

\begin{example} \label{example 5.1.13}
The finite sequence \(1, -2, 3, -4\) is bounded (in this case, it is bounded by \(4\), or indeed by any \(M\) greater than or equal to 4).
But the infinite sequence \(1, -2, 3, -4, 5, -6,...\) is unbounded. (Can you prove this? Use \PROP{4.4.1}.)
The sequence \(1, -1, 1, -1,...\) is bounded (e.g., by \(1\)), but is \emph{not} a Cauchy sequence(because in particular it's not eventual \(2.1\)-steady).
\end{example}

\begin{proof}
For the sake of contradiction, suppose the infinite sequence \(1, -2, 3, -4, 5, -6,...\) is bounded, i.e. bounded by some rational \(M\).
By \PROP{4.4.1}, we can find a natural number \(N\) s.t. \(N > M\).
WLOG, if we find \(N = 0\) then we redefine \(N := 1\) to make it positive.
Then it's clear that \(\abs{a_N} = N > M\), so by \DEF{5.1.12}, the sequence is not bounded by \(M\), a contradiction.
\end{proof}

\begin{lemma} [Finite sequences are bounded] \label{lem 5.1.14}
Every finite sequence \(a_1, a_2, ..., a_n\) is bounded.
\end{lemma}

\begin{proof}
We prove this by induction on \(n\).
When \(n = 1\) the arbitrary sequence \(a_1\) is clearly bounded, for if we choose \(M := \abs{a_1}\) then clearly we have \(\abs{a_i} \le M\) for all \(1 \le i \le n\).
Now suppose that we have already proved the lemma for some \(n \ge 1\);
that is, any finite sequence with \(n\) elements is bounded;
we now prove it for \(n + 1\), i.e., we prove every sequence with \(n + 1\) elements \(a_1, a_2, ..., a_{n + 1}\) is bounded.
So given arbitrary sequence of \(n + 1\) elements \(a_1, a_2, ..., a_n, a_{n + 1}\).
By the induction hypothesis we know that the first \(n\) elements \(a_1, a_2, ..., a_n\) is bounded by some \(M \le 0\);
in particular, it must be also bounded by \(M + \abs{a_{n + 1}}\) since \(M + \abs{a_{n + 1}} \ge M\).
On the other hand, \(a_{n + 1}\) is also bounded by \(\abs{a_{n + 1}}\), so is also bounded by \(M + \abs{a_{n + 1}}\) because the latter \(\ge\) the former.
Thus \(a_1, a_2, ... , a_n, a_{n + 1}\) is bounded by \(M +  \abs{a_{n + 1}}\), and is hence bounded.
This closes the induction.
\end{proof}

\begin{note}
Note that while this argument shows that every \textbf{finite} sequence is bounded, no matter how long the finite sequence is, it \textbf{does not} say anything about whether an \textbf{infinite} sequence is bounded or not;
infinity is not a natural number.
\end{note}

However, we have
\begin{lemma} [Cauchy sequences are bounded] \label{lem 5.1.15}
Every Cauchy sequence \((a_n)_{n = 1}^{\infty}\) is bounded.
\end{lemma}

\begin{proof}
Let \((a_n)_{n = 1}^{\infty}\) be an arbitrary Cauchy sequence, we have to show it is bounded by some rational \(M \ge 0\).

Now by \DEF{5.1.8}, given any \(\varepsilon > 0\), the sequence is eventually \(\varepsilon\)-steady.
By \DEF{5.1.6}, that means we can find an index \(N \ge 0\) s.t. \(d(a_j, a_k) \le \varepsilon\) for all \(j, k \ge N\) \MAROON{(1)}.
Now we split the sequence as the first \(a_1, ..., a_{N}\) and \(a_{N + 1}, ...\).

By \LEM{5.1.14}, the first part is finite, so it's bounded.

For the second part and \MAROON{(1)}, and in particular for the index \(j\) fixed as \(N + 1\), \(d(a_{N + 1}, a_k) \le \varepsilon\) for all \(k \ge N\) \MAROON{(2)}.
Now let \(M := \abs{a_{N + 1}} + \varepsilon\), we will show that every element \(a_k\) for all \(k \ge N\) is bounded by \(M\), that is, \(\abs{a_k} \le M\).
Note that \(M = \abs{a_{N + 1}} + \abs{\varepsilon}\) \MAROON{(3)} since \(\varepsilon > 0\).
So let \(k\) be arbitrary integer s.t. \(k \ge N\).
Then
\begin{align*}
             & d(a_{N + 1}, a_k) \le \varepsilon & \text{by \MAROON{(2)}} \\
    \implies & \abs{a_{N + 1} - a_k} \le \varepsilon & \text{by \DEF{4.3.2}} \\
    \implies & \abs{a_{N + 1} - a_k} \le \abs{\varepsilon} & \text{since \(\varepsilon > 0\)} \\
    \implies & \abs{a_{N + 1}} + \abs{a_{N + 1} - a_k} \le \abs{a_{N + 1}} + \abs{\varepsilon} & \text{by \PROP{4.2.9}(d)} \\
    \implies & \abs{a_{N + 1}} + \abs{a_{N + 1} - a_k} \le M & \text{by \MAROON{(3)}} \\
    \implies & \abs{-a_{N + 1}} + \abs{a_{N + 1} - a_k} \le M & \text{by \PROP{4.3.3}(d)} \\
    \implies & \abs{(-a_{N + 1}) + (a_{N + 1} - a_k)} \le \abs{-a_{N + 1}} + \abs{a_{N + 1} - a_k} \le M & \text{by \PROP{4.3.3}(b)} \\
    \implies & \abs{(-a_{N + 1}) + (a_{N + 1} - a_k)} \le M & \text{by \PROP{4.2.9}(c), transitive} \\
    \implies & \abs{-a_k} & \text{by algebra}. \\
    \implies & \abs{a_k} \le M & \text{by \PROP{4.3.3}(d)}
\end{align*}
Since \(k\) is arbitrary, \(\abs{a_k} \le M\) for all \(k \ge N\).
So the second part of the sequence is bounded by \(M\) and hence is bounded.

So the whole sequence is bounded.
\end{proof}

\begin{exercise} \label{exercise 5.1.1}
Prove \LEM{5.1.15}.
\end{exercise}

\begin{proof}
See \LEM{5.1.15}.
\end{proof}
\section{Equivalent Cauchy sequence} \label{sec 5.2}

Consider the two Cauchy sequences of rational numbers:
\[
    1.4, 1.41, 1.414, 1.4142, 1.41421,...
\]
and
\[
    1.5, 1.42, 1.415, 1.4143, 1.41422,...
\]
\emph{Informally}, both of these sequences seem to be converging to the same number, the square root
\(\sqrt{2} = 1.41421...\)
(though this statement is \emph{not yet rigorous} because we have not defined real numbers yet).
If we are to define the real numbers from the rationals as \emph{limits} of Cauchy sequences, \emph{we have to know when two Cauchy sequences of rationals give the same limit}, \textbf{without} first defining a real number
(since that would be circular).

\begin{note}
若你希望用柯西序列的極限來定義實數,那柯西序列的相等的定義就不能跟實數有任何關係,這樣會\ circular。
\end{note}

\begin{definition} [\(\varepsilon\)-close sequences] \label{def 5.2.1}
Let \((a_n)_{n = 0}^{\infty}\) and \((b_n)_{n = 0}^{\infty}\) be two sequences,
and let \(\varepsilon > 0\).
We say that the sequence \((a_n)_{n = 0}^{\infty}\) is \emph{\(\varepsilon\)-close} to \((b_n)_{n = 0}^{\infty}\) iff \(a_n\) is \(\varepsilon\)-close to \(b_n\) for each \(n \in \SET{N}\).
In other words, the sequence \(a_0, a_1, a_2,...\) is \(\varepsilon\)-close to the sequence \(b_0, b_1, b_2,...\) 
iff \(\abs{a_n - b_n} \le \varepsilon\) for all \(n = 0, 1, 2,...\).
\end{definition}

\begin{example} \label{example 5.2.2}
The two sequences
\[
    1, -1, 1, -1, 1,...
\]
and
\[
    1.1, -1.1, 1.1, -1.1, 1.1,...
\]
are \(0.1\)-close to each other.
(Note however that neither of them are \(0.1\)-steady).
\end{example}

\begin{definition} [Eventually \(\varepsilon\)-close sequences] \label{def 5.2.3}
Let \((a_n)_{n = 0}^{\infty}\) and \((b_n)_{n = 0}^{\infty}\) be two sequences, and let \(\varepsilon > 0\).
We say that the sequence \((a_n)_{n = 0}^{\infty}\) is \emph{eventually \(\varepsilon\)-close to} \((b_n)_{n = 0}^{\infty}\)
iff there exists an \(N \ge 0\) such that the sequences \((a_n)_{n = N}^{\infty}\) and \((b_n)_{n = N}^{\infty}\) are \(\varepsilon\)-close.
In other words, \(a_0, a_1, a_2,...\) is eventually \(\varepsilon\)-close to \(b_0, b_1, b_2,...\)
iff there exists an \(N \ge 0\) such that \(\abs{a_n - b_n} \le \varepsilon\) for all \(n \ge N\).
\end{definition}

\begin{remark} \label{remark 5.2.4}
Again, the notations for \(\varepsilon\)-close sequences and eventually \(\varepsilon\)-close sequences are not standard in the literature, and we will not use them outside of this section.
\end{remark}

\begin{example} \label{example 5.2.5}
Simple example that informally shows
\[
    1.1, 1.01, 1.001, 1.0001,...
\]
and
\[
    0.9, 0.99, 0.999, 0.9999,...
\]
are equivalent(see definition below).
\end{example}

\begin{definition} [Equivalent sequences] \label{def 5.2.6}
Two sequences \((a_n)_{n = 0}^{\infty}\) and \((b_n)_{n = 0}^{\infty}\) are \emph{equivalent}
iff for each \emph{rational} \(\varepsilon > 0\), the sequences \((a_n)_{n = 0}^{\infty}\) and \((b_n)_{n = 0}^{\infty}\) are eventually \(\varepsilon\)-close.
In other words, \(a_0, a_1, a_2,...\) and \(b_0, b_1, b_2,...\) are equivalent
iff for every rational \(\varepsilon > 0\), there exists an \(N \ge 0\) such that \(\abs{a_n - b_n} \le \varepsilon\) for all \(n \ge N\).
\end{definition}

\begin{remark} \label{remark 5.2.7}
As with \DEF{5.1.8}, the quantity \(\varepsilon > 0\) is currently restricted to be a positive \emph{rational}, rather than a positive \emph{real}.
However, we shall eventually see that it makes no difference whether \(\varepsilon\) ranges over the positive rationals or positive reals;
see \EXEC{6.1.10}.
\end{remark}

We now prove the sequences in \EXAMPLE{5.2.5} are equivalent rigorously.

\begin{proposition} \label{prop 5.2.8}.
Let \((a_n)_{n = 1}^{\infty}\) and \((b_n)_{n = 1}^{\infty}\) be the sequences
\(a_n := 1 + 10^{-n}\) and \(b_n = 1 - 10^{-n}\).
Then the sequences \(a_n, b_n\) are equivalent.
\end{proposition}

\begin{remark} \label{remark 5.2.9}
This Proposition, in \emph{decimal notation}, asserts that \(1.0000... = 0.9999...\); see \PROP{B.2.3}.
\end{remark}

\begin{proof}
We need to prove that for every \(\varepsilon > 0\), the two sequences \((a_n)_{n = 1}^{\infty}\) and \((b_n)_{n = 1}^{\infty}\) are eventually \(\varepsilon\)-close to each other.
So we let \(\varepsilon\) be an arbitrary rational such that \(\varepsilon > 0\).
We need to find an integer \(N \ge 1\) (where \(1\) is the starting index of the two sequences) such that \((a_n)_{n = N}^{\infty}\) and \((b_n)_{n = N}^{\infty}\) are \(\varepsilon\)-close;
in other words, we need to find an \(N \ge 1\) such that
\[
    \abs{a_n - b_n} \le \varepsilon \text{ for all } n \ge N.
\]
However, we have
\[
    \abs{a_n - b_n} = \abs{(1 + 10^{-n}) - (1 - 10^{-n})} = 2 \X 10^{-n}.
\]
Since \(10^{-n}\) is a \emph{decreasing} function of \(n\) (i.e., \(10^{-m} < 10^{-n}\) whenever \(m > n\);
this is easily proven by induction),
and since \(n \ge N\), we have \(2 \X 10^{-n} \le 2 \X 10^{-N}\).
Thus we have
\[
    \abs{a_n - b_n} \le 2 \X 10^{-N} \text{ for all } n \ge N.
\]
Thus in order to obtain \(\abs{a_n - b_n} \le \varepsilon\) for all \(n \ge N\), it will be sufficient to choose \(N\) so that \(2 \X 10^{-N} \le \varepsilon\).
This is easy to do using logarithms, but we have not yet developed logarithms yet, so we will use a cruder
method.
First, we observe \(10^N\) is always greater than \(N\) for any \(N \ge 1\) (see \EXEC{4.3.5}).
Thus \(10^{-N} \le 1/N\), and \(2 \X 10^{-N} \le 2/N\).

Thus to get \(2 \X 10^{-N} \le \varepsilon\), it will suffice to choose \(N\) so that \(2/N \le \varepsilon\), or equivalently that \(N \ge 2/\varepsilon\).
But by \PROP{4.4.1} since \(2/\varepsilon\) is rational, we can find \(N\) s.t. \(N > 2/\varepsilon\) and in particular \(N \ge 2/\varepsilon\).
So the two sequences are eventually \(\varepsilon\)-close.
Since \(\varepsilon\) are arbitrary, for all \(\varepsilon > 0\) the two sequences are eventually \(\varepsilon\)-close.
By \DEF{5.2.6}, the two sequences are equivalent.
\end{proof}


\exercisesection

\begin{exercise} \label{exercise 5.2.1}
Show that if \((a_n)_{n = 1}^{\infty}\) and \((b_n)_{n = 1}^{\infty}\) are equivalent sequences of rationals,
then \((a_n)_{n = 1}^{\infty}\) is a Cauchy sequence if and only if \((b_n)_{n = 1}^{\infty}\) is a Cauchy sequence.
\end{exercise}
\begin{proof}
Suppose \((a_n)_{n = 1}^{\infty}\) and \((b_n)_{n = 1}^{\infty}\) are equivalent sequences of rationals.
\begin{itemize}
    \item[\(\Longrightarrow\)]:
        Suppose \((a_n)_{n = 1}^{\infty}\) is Cauchy, we have to show \((b_n)_{n = 1}^{\infty}\) is Cauchy.
        So, let \(\varepsilon\) be arbitrary rational s.t. \(\varepsilon > 0\), we have to show \((b_n)_{n = 1}^{\infty}\) is eventual \(\varepsilon\)-steady.
        That is, we have to find a integer \(N \ge 1\), which is the starting index of the two sequences, s.t. \(d(b_j, b_k) \le \varepsilon\) for all \(j, k \ge N\).
        
        But by supposition \((a_n)_{n = 1}^{\infty}\) is Cauchy, so we can find a integer \(N_1 \ge 1\) s.t. \(d(a_j, a_k) \le \varepsilon\) for all \(j, k \ge N_1\) \GREEN{(1)}.
        And also since the two sequences are equivalent, we can find a integer \(N_2 \ge 1\) s.t. \(d(a_i, b_i) \le \varepsilon\) for all \(i \ge N_2\) \GREEN{(2)}.
        
        Now let \(N_3 := max(N_1, N_2)\), for all \(i, j \ge N_3\), by \GREEN{(1)} we have \(d(a_i, a_j) \le \varepsilon\) \MAROON{(1)};
        and by \GREEN{(2)} we have \(d(a_i, b_i) \le \varepsilon\) \MAROON{(2)} and \(d(a_j, b_j) \le \varepsilon)\) \MAROON{(3)}.
        And by \PROP{4.3.7}(f) and \MAROON{(2)}, we have \(d(a_i, b_i) = d(b_i, a_i)\) \MAROON{(4)}.
        So by adding both sides of the three inequalities \MAROON{(1) (3) (4)}, we have \(d(b_i, a_i) + d(a_i, a_j) + d(a_j, b_j) \le 3\varepsilon\).
        But by applying \PROP{4.3.7}(g) twice, we have \(d(b_i, b_j) \le d(b_i, a_i) + d(a_i, a_j) + d(a_j, b_j)\).
        So by transitivity, we have \(d(b_i, b_j) \le 3\varepsilon\) for all \(i, j \le N_3\).
        
        (I do this with purpose and first conclude that the sequence \(b_n\) is \(3\varepsilon\)-steady, because finding \(N\) for \(\varepsilon/3\) in the \emph{beginning} of the proof is somewhat anti-human.)
        
        Now let \(\varepsilon' = \varepsilon/3\).
        Then with the same argument above, we can conclude that we can find \(N_4 \ge 1\) s.t. \(d(b_i, b_j) \le 3\varepsilon'\) for all \(i, j \le N_4\).
        But \(3\varepsilon' = 3(\varepsilon/3) = \varepsilon\), so we have \(d(b_i, b_j) \le \varepsilon\) for all \(i, j \le N_4\).
        So by \DEF{5.1.6}, \((b_n)_{n = 1}^{\infty}\) is \(\varepsilon\)-steady.
        
        Since \(\varepsilon > 0\) is arbitrary, \((b_n)_{n = 1}^{\infty}\) is eventual \(\varepsilon\)-steady for all \(\varepsilon > 0\).
        By \DEF{5.1.8}, \((b_n)_{n = 1}^{\infty}\) is Cauchy!
    \item[\(\Longleftarrow\)]:
        The proof is similar with previous case, with \(a_{blabla}\) and \(b_{blabla}\) swapped.
\end{itemize}
\end{proof}

\begin{exercise} \label{exercise 5.2.2}
Let \(\varepsilon > 0\).
Show that if \((a_n)_{n = 1}^{\infty}\) and \((b_n)_{n = 1}^{\infty}\) are eventually \(\varepsilon\)-close,
then \((a_n)_{n = 1}^{\infty}\) is bounded if and only if \((b_n)_{n = 1}^{\infty}\) is bounded.
\end{exercise}
\begin{proof}
Let \(\varepsilon > 0\) and suppose \((a_n)_{n = 1}^{\infty}\) and \((b_n)_{n = 1}^{\infty}\) are eventually \(\varepsilon\)-close.
\begin{itemize}
    \item[\(\Longrightarrow\)]
        Suppose \((a_n)_{n = 1}^{\infty}\) is bounded, we have to show \((b_n)_{n = 1}^{\infty}\) is also bounded.

        Since \((a_n)_{n = 1}^{\infty}\) is bounded, we can find a rational \(M_1 \ge 0\) s.t. \(\abs{a_i} \le M_1\) for all \(i \ge 1\). And since the two sequence are eventually \(\varepsilon\)-close, we can find an integer \(N_1 \ge 1\) s.t. \(d(a_i, b_i) \le \varepsilon\) for all \(i \ge N_1\).
        
        Now again like the proof in \LEM{5.1.15}, we split \((b_n)_{n = 1}^{\infty}\) into \((b_n)_{n = 1}^{N_1}\) and \((b_n)_{N_1 + 1}^{\infty}\).
        By \LEM{5.1.14} the first finite part is bounded.
        
        Now let \(M_2 := M_1 + \varepsilon\), we will show that the second part of the sequence, \((b_n)_{n = N_1 + 1}^{\infty}\), is bounded by \(M_2\) and hence is bounded.
        So given arbitrary index \(i \ge N_1 + 1\):
        \begin{align*}
                     & d(a_i, b_i) \le \varepsilon & \text{since the two seq. are even. \(\varepsilon\)-close} \\
            \implies & d(b_i, a_i) \le \varepsilon & \text{by \PROP{4.3.3}(f)} \\
            \implies & \abs{b_i - a_i} \le \varepsilon & \text{by \DEF{4.3.2}} \\
            \implies & M_1 + \abs{b_i - a_i} \le M_1 + \varepsilon & \text{by \PROP{4.2.9}(d)} \\
            \implies & M_1 + \abs{b_i - a_i} \le M_2 & \text{since \(M_2 = M_1 + \varepsilon\)}\\
            \implies & \abs{a_i} + \abs{b_i - a_i} \le M_1 + \abs{b_i - a_i} \le M_2 & \text{since \(\abs{a_i} \le M_1\)} \\
            \implies & \abs{a_i} + \abs{b_i - a_i} \le M_2 & \text{transitive} \\
            \implies & \abs{a_i + (b_i - a_i)} \le \abs{a_i} + \abs{b_i - a_i} \le M_2 & \text{by \PROP{4.3.3}(b)} \\
            \implies & \abs{a_i + (b_i - a_i)} \le M_2 & \text{transitive}  \\
            \implies & \abs{b_i} \le M_2 & \text{trivial} \\
        \end{align*}
        Since \(i \ge N_1 + 1\) is arbitrary, so for all \(i \ge N_1 + 1\), \(\abs{b_i} \le M_2\).
        By \DEF{5.1.12}, \((b_n)_{n = N_1 + 1}^{\infty}\) is bounded by \(M_2\) and hence is bounded.

        So the whole sequence is bounded.
    \item[\(\Longleftarrow\)]:
        The proof is similar with previous case, with \(a_{blabla}\) and \(b_{blabla}\) swapped.
\end{itemize}
\end{proof}
\section{The construction of the real numbers} \label{sec 5.3}

\begin{definition} [Real numbers] \label{def 5.3.1}
A \emph{real number} is defined to be an \emph{object} of the \emph{form} \(\LIM_{n \toINF} a_n\),
where \((a_n)_{n = 1}^{\infty}\) is a Cauchy sequence of \emph{rational} numbers.
Two real numbers \(\LIM_{n \toINF} a_n\) and \(\LIM_{n \toINF} b_n\) are said to be \emph{equal} iff \((a_n)_{n = 1}^{\infty}\) and \((b_n)_{n = 1}^{\infty}\) are equivalent Cauchy sequences.
The set of all real numbers is denoted \(\SET{R}\).
\end{definition}

\begin{example} [Informal] \label{example 5.3.2}
Let \(a_1, a_2, a_3, ...\) denote the sequence
\[
    1.4, 1.41, 1.414, 1.4142, 1.41421,...
\]
and let \(b_1, b_2, b_3,...\) denote the sequence
\[
    1.5, 1.42, 1.415, 1.4143, 1.41422,...
\]
then \(\LIM_{n \toINF} a_n\) is a real number, and is \emph{the same} real number as \(\LIM_{n \toINF} b_n\),
because \((a_n)_{n = 1}^{\infty}\) and \((b_n)_{n = 1}^{\infty}\) are equivalent Cauchy sequences:
\(\LIM_{n \toINF} a_n = \LIM_{n \toINF} b_n\).
\end{example}
\begin{note}
到目前為止,我們還沒有證明:
\begin{itemize}
    \item 這兩個\ sequence 都是\ Cauchy sequences。
    \item 進一步來說,也無法知道他們對應的\ \LIM\ object 是\ real number。
    \item 所以也無法知道討論他們是否相等。
\end{itemize}
而且開天眼,這整節看完之後我們好像還是不能知道他們是\ Cauchy...。
\end{note}

\begin{note}
We will refer to \(\LIM_{n \toINF} a_n\) as the \emph{formal limit} of the sequence \((a_n)_{n = 1}^{\infty}\).
Later on we will define a genuine notion of limit, and show that the formal limit of a Cauchy sequence is the same as the limit of that sequence;
after that, we will not need formal limits ever again.
(The situation is much like what we did with \emph{formal} subtraction \(\M\) and \emph{formal} division \(\D\).)
\end{note}

\begin{note}
we need to check that the notion of equality in the definition obeys the first three laws of
equality:
\end{note}

\begin{proposition} [Formal limits are well-defined] \label{prop 5.3.3}
Let \(x = \LIM_{n \toINF} a_n,\ y = \LIM_{n \toINF} b_n\), and \(z = \LIM_{n \toINF} c_n\) be real numbers.
(This implies the corresponding sequences are Cauchy.)
Then, with the above \DEF{5.3.1} of equality for real numbers, we have \(x = x\).
Also, if \(x = y\), then \(y = x\).
Finally, if \(x = y\) and \(y = z\), then \(x = z\).
\end{proposition}

\begin{proof}
Reflexive: Since given \(\varE > 0\), \(\varE > 0 = d(a_n, a_n)\) for all \(n \ge 1\), by \DEF{5.2.6}, \((a_n)_{n = 1}^{\infty}\) and \((a_n)_{n = 1}^{\infty}\) are equivalent.
And by \DEF{5.3.1}, \(\LIM_{n \toINF} a_n = \LIM_{n \toINF} a_n\), that is, \(x = x\).

Symmetric: Suppose \(x = y\).
By \DEF{5.3.1}, \((a_n)_{n = 1}^{\infty}\) and \((b_n)_{n = 1}^{\infty}\) are equivalent.
And by \DEF{5.2.6}, For all \(\varE > 0\), there exists \(N \ge 1\) s.t. for all \(n \ge N\), \(d(a_n, b_n) \le \varE\) \MAROON{(1)}.
But by \PROP{4.3.3}(f), \(d(a_n, b_n) = d(b_n, a_n)\).
So with \MAROON{(1)}, we have For all \(\varE > 0\), there exists \(N \ge 1\) s.t. for all \(n \ge N\), \(\MAROON{d(b_n, a_n)} \le \varE\).
By \DEF{5.2.6}, \((b_n)_{n = 1}^{\infty}\) and \((a_n)_{n = 1}^{\infty}\) are equivalent.
By \DEF{5.3.1}, \(\LIM_{n \toINF} b_n = \LIM_{n \toINF} b_n\), that is, \(y = x\).

Transitive: Suppose \(x = y\) and \(y = z\).
By \DEF{5.3.1}, \((a_n)_{n = 1}^{\infty}, (b_n)_{n = 1}^{\infty}\) are equivalent and \((b_n)_{n = 1}^{\infty}, (c_n)_{n = 1}^{\infty}\) are equivalent.
Now given arbitrary \(\varE > 0 \), so \emph{in particular \(\varE/2 > 0\)}, we have
\begin{itemize}
    \item There exists \(N_1 \ge 1\) s.t. for all \(n \ge N_1\), \(d(a_n, b_n) \le \varE/2\) \MAROON{(2)}.
    \item There exists \(N_2 \ge 1\) s.t. for all \(n \ge N_2\), \(d(b_n, c_n) \le \varE/2\) \MAROON{(3)}.
\end{itemize}
Let \(N_3 := max(N_1, N_2)\), then by \MAROON{(2) (3)}, we have for all \(n \ge N_3\), \(d(a_n, b_n) \le \varE/2\) and \(d(b_n, c_n) \le \varE/2\).
But by \DEF{4.3.4}, that implies \(a_n, b_n\) and \(b_n, c_n\) are \(\varE/2\)-close.
By \PROP{4.3.7}, we have \(a_n, c_n\) are \(\varE/2 + \varE/2 = \varE\)-close.
By \DEF{4.3.4}, \(d(a_n, c_n) \le \varE\).
So for all \(n \ge N_3\), \(d(a_n, c_n) \le \varE\).
By \DEF{5.2.6}, \((a_n)_{n = 1}^{\infty}, (c_n)_{n = 1}^{\infty}\) are equivalent.
By \DEF{5.3.1}, \(\LIM_{n \toINF} a_n = \LIM_{n \toINF} c_n\), that is, \(x = z\).
\end{proof}

\begin{note}
By \PROP{5.3.3}, we now have well-defined equality between two real numbers.
Of course, when we define other operations on the reals, \emph{we have to check that they obey the law of substitution} \AXM{a.7.4}:
two real number inputs which are \emph{equal} should give equal outputs when applied to any operation on the real numbers.

Now we want to define all the usual arithmetic operations on the real numbers, such as addition and multiplication. 
\end{note}

\begin{definition} [Addition of reals] \label{def 5.3.4} 
Let \(x = \LIM_{n \toINF} a_n\) and \(y = \LIM_{n \toINF} b_n\) be real numbers.
Then we \emph{define} the \emph{sum} \(x + y\) to be \(x + y := \LIM_{n \toINF} (a_n + b_n)\).
\end{definition}

\begin{example} \label{example 5.3.5}
The sum of \(\LIM_{n \toINF} 1 + 1/n\) and \(\LIM_{n \toINF} 2 + 3 / n\) is \(\LIM_{n \toINF} 3 + 4/n\).
\end{example}

We now check that \DEF{5.3.4} is valid.
The first thing we need to do is to confirm that the sum of two real numbers is in fact a real number:

\begin{lemma} [Sum of Cauchy sequences is Cauchy] \label{lem 5.3.6}
Let \(x = \LIM_{n \toINF} a_n\) and \(y = \LIM_{n \toINF} b_n\) be real numbers.
Then \(x + y\) is also a real number. (i.e., \((a_n + b_n)_{n = 1}^{\infty}\) is a Cauchy sequence of rationals.)
\end{lemma}


\begin{proof}
We need to show that for every \(\varE > 0\), the sequence \((a_n + b_n)_{n = 1}^{\infty}\) is eventually \(\varE\)-steady.
Now from hypothesis and \DEF{5.3.1} we know that \((a_n)_{n = 1}^{\infty}\) is Cauchy and hence by \DEF{5.1.8} is eventually \(\varE\)-steady, and similarly \((b_n)_{n = 1}^{\infty}\) is eventually \(\varE\)-steady;
but it turns out that this is not quite enough
(this can be used to imply that \((a_n + b_n)_{n = 1}^{\infty}\) is eventually \(2\varE\)-steady, but that's not what we want).
So we need to do a little trick, which is to play with the value of \(\varE\).

We know that \((a_n)_{n = 1}^{\infty}\) is eventually \(\delta\)-steady for every value of \(\delta > 0\).
This implies not only that \((a_n)_{n = 1}^{\infty}\) is eventually \(\varE\)-steady, but it is also eventually \(\varE / 2\)-steady(just in particular let \(\delta = \varE/2\)).
Similarly, the sequence \((b_n)_{n = 1}^{\infty}\) is also eventually \(\varE / 2\)-steady.
This will turn out to be enough to conclude that \((a_n + b_n)_{n = 1}^{\infty}\) is eventually \(\varE\)-steady.

Since \((a_n)_{n = 1}^{\infty}\) is eventually \(\varE / 2\)-steady, we know that there exists an \(N \geq 1\) such that \((a_n)_{n = N}^{\infty}\) is \(\varE / 2\)-steady, i.e., \(a_n\) and \(a_m\) are \(\varE / 2\)-close for every \(n, m \geq N\).
Similarly there exists an \(M \geq 1\) such that \((b_n)_{n = M}^{\infty}\) is \(\varE / 2\)-steady, i.e., \(b_n\) and \(b_m\) are \(\varE / 2\)-close for every \(n, m \geq M\).

Let \(\max(N, M)\) be the larger of \(N\) and \(M\)
(we know from \PROP{2.2.13} that one has to be greater than or equal to the other).
If \(n, m \geq \max(N, M)\), then we know that \(a_n\) and \(a_m\) are \(\varE / 2\)-close, and \(b_n\) and \(b_m\) are \(\varE / 2\)-close, and so by \PROP{4.3.7}(d) we see that \(a_n + b_n\) and \(a_m + b_m\) are \(\varE/2 + \varE/2 = \varE\)-close for every \(n, m \geq \max(N, M)\).
This implies that the sequence \((a_n + b_n)_{n = 1}^{\infty}\) is eventually \(\varE\)-steady, as desired.
\end{proof}

The other thing we need to check is the axiom of substitution \AXM{a.7.4}:
if we replace a real number \(x\) by another number \(x'\) \emph{equal} to \(x\),  then we must have \(x' + y = x + y\)
(and similarly if we substitute \(y\) by another number \(y'\) equal to \(y\)).

\begin{lemma} [Sums of equivalent Cauchy sequences are equivalent]\label{lem 5.3.7}.
Let \(x = \LIM_{n \toINF} a_n, y = \LIM_{n \toINF} b_n\), and \(x' = \LIM_{n \toINF} a'_n\) be real numbers.
Suppose that \(x = x'\).
Then we have \(x + y = x' + y\).
\end{lemma}

\begin{proof}
We need to show \(x + y = x' + y\), 
that is by \DEF{5.3.4} \(\LIM_{n \toINF} (a_n + b_n) = \LIM_{n \toINF} (a'_n + b_n)\), 
that is by \DEF{5.3.1}, the sequences \((a_n + b_n)_{n = 1}^{\infty}\) and \((a'_n + b_n)_{n = 1}^{\infty}\) are eventually \(\varE\)-close for each \(\varE > 0\) \MAROON{(1)}. 
But since \(x = x'\), we know that the Cauchy sequences \((a_n)_{n = 1}^{\infty}\) and \((a'_n)_{n = 1}^{\infty}\) are equivalent, by similar definition derivation we know that there is an \(N \ge 1\) such that \((a_n)_{n = N}^{\infty}\) and \((a'_n)_{n = N}^{\infty}\) are \(\varE\)-close,
i.e., that \(a_n\) and \(a'_n\) are \(\varE\)-close for each \(n \ge N\).
Since \(b_n\) is of course \(0\)-close to \(b_n\)
(where we extend the notion of \(\varE\)-closeness(The author seems to mean \DEF{4.3.4}, \(\varE\)-closeness for two numbers) to include \(\varE=0\) in the obvious fashion),
we thus see from \PROP{4.3.7}(d)
(extended in the obvious manner to the \(\delta =  0\) case, or extended to cover the \(0\)-close case)
that \(a_n + b_n\) and \(a'_n + b_n\) are \(\varE + 0 = \varE\)-close for each \(n \ge N\).
This implies \MAROON{(1)} is true, and we are done.
\end{proof}

\begin{remark} \label{remark 5.3.8}
The above lemma verifies the axiom of substitution \AXM{a.7.4} for the ``\(x\)'' variable in \(x + y\),
but one can similarly prove the \AXM{a.7.4} for the ``\(y\)'' variable.
(A quick way is to observe from the definition of \(x + y\) that we certainly have \(x + y = y + x\), since \(a_n + b_n\) = \(b_n + a_n\).)
\end{remark}

We can define multiplication of real numbers in a manner similar to that of addition:

\begin{definition} [Multiplication of reals] \label{def 5.3.9}
Let \(x = \LIM_{n \toINF} a_n\) and \(y = \LIM_{n \toINF} b_n\) be real numbers.
Then we \emph{define} the product \(xy\) to be \(xy :=  \LIM_{n \toINF} a_n b_n\).
\end{definition}

\begin{proposition} [Multiplication is well defined] \label{prop 5.3.10}
Let \(x = \LIM_{n \toINF} a_n, y = \LIM_{n \toINF} b_n\), and \(x' = \LIM_{n \toINF} a'_n\) be real numbers.
Then \(xy\) is also a real number.
Furthermore, if \(x = x'\), then \(xy = x'y\). 
\end{proposition}

\begin{proof}
\begin{itemize}
    \item
        First we show \(xy\) is real;
        that is, \((a_n b_n)_{n = 1}^{\infty}\) is Cauchy;
        that is, for all \(\varE > 0\), there exists \(N \ge 1\) s.t. \(d(a_i b_i, a_j b_j) \le \varE\) for all \(i, j \ge N\).
        
        So let \(3 > \varE > 0\). \BLUE{(trick 1)}
        (The purpose of the upper bound is to make some trick in the latter proof).
        It's trivial that if \((a_n b_n)_{n = 1}^{\infty}\) is eventually \(\varE\)-close, then given any \(\delta \ge 3\), \((a_n b_n)_{n = 1}^{\infty}\) is also \(\delta\)-close.
        (By \PROP{4.3.7}(e) and the definition of closeness of the sequences and rational numbers.)
        So together we can conclude \((a_n b_n)_{n = 1}^{\infty}\) is eventually \(\varE\)-close for all \(\varE > 0\).
        
        Since \((a_n)_{n = 1}^{\infty}\) and \((b_n)_{n = 1}^{\infty}\) are Cauchy, by \LEM{5.1.15} they are bounded by some rational \(M_1, M_2 \ge 0\),
        that is, \(\abs{a_n} \le M_1\) and \(\abs{b_n} \le M_2\) for all \(n \ge 1\).
        Now let \(M = max(M_1, M_2, 1)\), and that trivially implies \(\abs{a_n} \le M\) \MAROON{(1)} and \(\abs{b_n} \le M\) \MAROON{(2)} for all \(n \ge 1\).
        (Note that \(M \ge 1\), this is also used as a trick in the latter proof \BLUE{(trick 2)}).
        
        And since \((a_n)_{n = 1}^{\infty}\) and \((b_n)_{n = 1}^{\infty}\) are Cauchy, there exists \(N_1 \ge 1\) s.t. \(d(a_i , a_j) \le \varE\) for all \(i, j \ge N_1\),
        and there exists \(N_2 \ge 1\) s.t. \(d(b_i , b_j) \le \varE\) for all \(i, j \ge N_2\).
        Let \(N = max(N_1, N_2)\), then \(d(a_i , a_j) \le \varE\) and \(d(b_i, b_j)\le \varE\) for all \(i, j \ge N\).
        And by \PROP{4.3.7}(h), we have \(d(a_i b_i, a_j b_j) \le \varE\abs{b_i} + \varE\abs{a_i} + \varE^2\).
        And by \MAROON{(1) (2)}
        \begin{align*}
                & \varE\abs{b_i} + \varE\abs{a_i} + \varE^2 \\
            \le &  \varE M + \varE M + \varE^2 \\
              = & \varE(2M + \varE),
        \end{align*}
        so we have \(d(a_i b_i, a_j b_j) \le \varE(2M + \varE)\) for all \(i, j \ge N\) \MAROON{(3)}.
        
        \sloppy Now let \(\varE' = \frac{\varE}{3M}\), then by similar argument until \MAROON{(3)},
        there exists \(N_3 \ge 1\) s.t. for all \(i, j \ge N_3\).
        \begin{align*}
            d(a_i b_i, a_j b_j) & \le \varE'(2M + \varE') \\
                                & = \frac{\varE}{3M}(2M + \frac{\varE}{3M}) \\
                                & = \frac{2\varE}{3} + \frac{\varE^2}{9M^2} \\
                                & \le \frac{2\varE}{3} + \frac{\varE^2}{9\X1^2} & \text{since \(M \ge 1\), trick \BLUE{(2)}} \\
                                & = \frac{2\varE}{3} + \frac{\varE^2}{9} \\
                                & = \frac{2\varE}{3} + \varE \X \frac{\varE}{9} \\
                                & \le \frac{2\varE}{3} + 3 \X \frac{\varE}{9} & \text{since \(\varE < 3\), trick \BLUE{(1)}} \\
                                & = \frac{2\varE}{3} + \frac{\varE}{3} \\
                                & = \varE
        \end{align*}
        By \DEF{5.1.6}, \((a_n b_n)_{n = 1}^{\infty}\) is eventually \(\varE\)-close.
        Since \(\varE\) is arbitrary between \(0\) and \(3\), \((a_n b_n)_{n = 1}^{\infty}\) is eventually \(\varE\)-close for all \(0 < \varE < 3\),
        and by the previous discussion it's trivial that \((a_n b_n)_{n = 1}^{\infty}\) is eventually \(\varE\)-close for all \(\varE > 0\).
        So by \DEF{5.1.8}, \((a_n b_n)_{n = 1}^{\infty}\) is Cauchy.
    \item
        Now we prove if \(x = x'\), then \(xy = x'y\).
        Since \(x = x'\), \((a_n)_{n = 1}^\infty\) and \((a'_n)_{n = 1}^\infty\) are equivalent sequences.
        Since \((b_n)_{n = 1}^\infty\) is Cauchy, there is some number \(M' \ge 0\) which bounds it.
        Now let \(M = max(M', 1)\), then it's trivial that \(M\) also bounds \((b_n)_{n = 1}^\infty\) and \(M\) is positive.

        Now let \(\varE > 0\), so in particular \(\varE/M > 0\).
        Then since \((a_n)_{n=1}^\infty\) and \((a'_n)_{n=1}^\infty\) are equivalent sequences, they are eventually \(\varE/M\)-close.
        So there exists \(N \ge 1\) s.t. \(a_i, a'_i\) are \(\varE/M\)-close for all \(i \ge N\).
        And by \PROP{4.3.7}(g), \(a_i b_i, a'_i b_i\) are \((\varE/M)\abs{b_i}\)-close for all \(i \ge N\).
        But since \(M\) bounds \((b_n)_{n = 1}^\infty\), we have \(\abs{b_i} \le M\), so \((\varE/M)\abs{b_i} \le (\varE/M)M = \varE\).
        So \(a_i b_i, a'_i b_i\) are \(\varE\)-close for all \(i \ge N\).
        So \((a_n b_n)_{n = 1}^\infty\) and \((a'_n b_n)_{n = 1}^\infty\) is eventually \(\varE\)-close.
        Since \(\varE > 0\) is arbitrary, by \DEF{5.2.6} \((a_n b_n)_{n = 1}^\infty\) and \((a'_n b_n)_{n = 1}^\infty\) are equivalent, that is, \(xy = x'y\).
\end{itemize}

\end{proof}

Of course we can prove a similar substitution rule when \(y\) is replaced by a real number \(y'\) which is equal to \(y\).

\begin{note}
At this point we \emph{embed the rationals back into the reals}, by equating every \emph{rational} number \(q\) with the \emph{real} number \(\LIM_{n \toINF} q\).
(That is, \(q \equiv \LIM_{n \toINF} q\).)
For instance, if \(a_1, a_2, a_3, ...\) is the sequence
\[
	0.5, 0.5, 0.5, 0.5, 0.5,...
\]
Then we let \(\LIM_{n \toINF} a_n\) equal to \(0.5\).

This embedding is consistent with our definitions of addition and multiplication, since for any rational numbers \(a, b\) we have
\begin{align*}
	a + b & = \LIM_{n \toINF} a + \LIM_{n \toINF} b & \text{by equating real and rational} \\
          & = \LIM_{n \toINF}(a + b) & \text{by \DEF{5.3.4}} \\
          & = a + b & \text{by equating again}
\end{align*}
And
\begin{align*}
	ab & = \LIM_{n \toINF} a \X \LIM_{n \toINF} b & \text{by equating} \\
          & = \LIM_{n \toINF}(ab) & \text{by \DEF{5.3.9}} \\
          & = ab & \text{by equating}
\end{align*}
(Note that the element \(a\) or \(b\) of the formula \(\LIM_{n \toINF} a\) or \(\LIM_{n \toINF} b\) is independent of the index \(n\).)

This means that when one wants to add or multiply two rational numbers \(a, b\) it does \emph{not} matter whether one thinks of these numbers as \emph{rationals} or as the \emph{real} numbers \(\LIM_{n \toINF} a, \LIM_{n \toINF} b\).
Also, this identification of rational numbers and real numbers is consistent with our definitions of equality(\EXEC{5.3.3}).
\end{note}

We can now easily define negation \(-x\) for real numbers \(x\) by the formula
\[
    -x := (-1) \X x,
\]
since \(-1\) is a rational number and is hence real.
(This definition is dependent on \DEF{5.3.9}, and we give it a label: \DEF{5.3.18}.)
Note that this is clearly consistent with our \emph{negation for rational} numbers since we have \(-q = (-1) \X q\) (by \AC{4.2.3}) for all rational numbers \(q\).
Also, from our definitions it is clear (why? see below) that
\begin{align*}
      & -(\LIM_{n \toINF} a_n) \\
    = & (-1) \X (\LIM_{n \toINF} a_n) & \text{by \DEF{5.3.18}, negation} \\
    = & \LIM_{n \toINF} (-1) \X \LIM_{n \toINF} a_n & \text{by equating rational and real} \\
    = & \LIM_{n \toINF} (-1) a_n & \text{by \DEF{5.3.9}} \\
    = & \LIM_{n \toINF} (-a_n)  & \text{by \AC{4.2.3}}
\end{align*}

Once we have addition and negation, we can define subtraction as usual by
\[
    x - y = x + (-y),
\]
(We label it as \DEF{5.3.19}.) Note that
\begin{align*}
      & \LIM_{n \toINF} a_n - \LIM_{n \toINF} b_n \\
    = & \LIM_{n \toINF} a_n + (-\LIM_{n \toINF} b_n) & \text{by \DEF{5.3.19}} \\
    = & \LIM_{n \toINF} a_n + \LIM_{n \toINF} (-b_n) & \text{shown above} \\
    = & \LIM_{n \toINF} (a_n + (-b_n)) & \text{by \DEF{5.3.4}} \\
    = & \LIM_{n \toINF} (a_n - b_n) & \text{by \DEF{4.2.13}, subtraction of rationals}
\end{align*}

We can now easily show that the \emph{real} numbers obey all the usual rules of algebra (\emph{except} perhaps for the laws involving \emph{division}
(since we currently have not defined the \emph{reciprocal} of the \emph{reals}), which we shall address shortly):

\begin{proposition} \label{prop 5.3.11}
All the laws of algebra from \PROP{4.1.6} (not \PROP{4.2.4}, not defining division yet) hold not only for the integers, but for the reals as well.
\end{proposition}

\begin{proof}
(The proof use the algebra laws of \emph{rationals}.)
We illustrate this with one such rule: \(x(y + z) = xy + xz\).
Let \(x = \LIM_{n \toINF} a_n\), \(y = \LIM_{n \toINF} b_n\),
and \(z = \LIM_{n \toINF} c_n\) be real numbers.
Then by \DEF{5.3.9}, \(xy = \LIM_{n \toINF} a_n b_n\) and \(xz = \LIM_{n \toINF} a_n c_n\),
and so by \DEF{5.3.4} \(xy + xz = \LIM_{n \toINF} (a_n b_n + a_n c_n)\).
A similar line of reasoning shows that \(x(y + z) = \LIM_{n \toINF} a_n (b_n + c_n)\).
But we already know that \(a_n (b_n + c_n)\) is equal to \(a_n b_n + a_n c_n\) for the \emph{rational} numbers \(a_n, b_n, c_n\), and the claim follows.
The other laws of algebra are proven similarly.
\end{proof}

The last basic arithmetic operation we need to define is reciprocation: \(x \to x^{-1}\).
This one is a little \emph{more subtle}.
On obvious first guess for how to proceed would be define
\[
    (\LIM_{n \toINF} a_n)^{-1} := \LIM_{n \toINF} a^{-1},
\]
but there are \emph{a few problems} with this.
For instance, let \(a_1, a_2, a_3,...\) be the \emph Cauchy sequence
\[
    0.1, 0.01, 0.001, 0.0001,...,
\]
and let \(x := \LIM_{n \toINF} a_n\).
Then by this definition, \(x^{-1}\) would be \(\LIM_{n \toINF} b_n\), where \(b_1, b_2, b_3,...\) is the sequence
\[
    10, 100, 1000, 10000,...
\]
but this is \emph{not} a Cauchy sequence (it isn't even bounded).
Of course, the problem here is that our original Cauchy sequence \((a_n)_{n = 1}^{\infty}\) was equivalent to the \emph{zero sequence} \((0)_{n = 1}^{\infty}\) (why? \MAROON{(1)}),
and hence that our real number \(x\) was in fact equal to \(0\).
So we should only allow the operation of reciprocal \emph{when \(x\) is non-zero}.

\begin{proof}
\MAROON{(1)} Suppose for the sake of contradiction, \((a_n)_{n = 1}^{\infty}\) was \emph{not} equivalent to \((0)_{n = 1}^{\infty}\).
Then by \DEF{5.2.6}, there exists \(\varE > 0\) s.t. \textbf{for all \(N \ge 1\)} there exists \(i \ge N\) s.t. \(\abs{a_i - 0} = \abs{a_i} = \abs{10^{-i}} > \varE\), or \(10^{-i} > \varE\), or \(10^i < 1/\varE\).
But since \(10^x\) an increasing function, and \(i \ge N\), we have \(10^N \le 10^i\), so we have \(10^N \le 1/\varE\).
And by \EXEC{4.3.5}, we have \(N \le 10^N\), wo we have \(N \le 1/\varE\).
So we conclude for all \(N \ge 1\), \(N \le 1/\varE\), which contradicts \PROP{4.4.1} that we can always find a natural number \(N\) s.t. \(N \ge 1/\varE\).
So the two sequences must be equivalent.
\end{proof}

However, even when we restrict ourselves to non-zero real numbers, we have a slight problem, because a non-zero real number might be the formal limit of a Cauchy sequence which \emph{contains zero elements}.
For instance, the number \(1\), which is rational and hence real, is the formal limit \(1 = \LIM_{n \toINF} a_n\) of the Cauchy sequence
\[
	0, 0.9, 0.99, 0.999, 0.9999,...
\]
(It can be proved that \(0, 0.9, 0.99,...\) and \(1, 1, 1, ...\) is equivalent.) but using our naive definition of reciprocal, we \emph{cannot} invert the real number \(1\), because we can’t invert the first element \(0\) of this Cauchy sequence!

To get around these problems we need to \emph{keep our Cauchy sequence away from zero}.
To do this we first need a definition.

\begin{definition} [Sequences bounded away from zero]  \label{def 5.3.12}
A sequence \((a_n)_{n = 1}^{\infty}\) of rational numbers is said to be \emph{bounded away from zero} iff there exists a \emph{rational} number \(c > 0\) such that \(\abs{a_n} \ge c\) for all \(n \ge 1\).
\end{definition}

\begin{example} \label{example 5.3.13}
The sequence \(1, -1, 1, -1, 1, -1, 1,...\) is bounded away from zero (all the coefficients have absolute value at least 1). 
But the sequence \(0.1, 0.01, 0.001,...\) is not bounded away from zero(proved by contradiction),
and neither is \(\textbf{0}, 0.9, 0.99, 0.999, 0.9999,....\)
The sequence \(10, 100, 1000,...\) is bounded away from zero, but is not bounded.
\end{example}

We now show that every \emph{non-zero} real number is the formal limit of a Cauchy sequence \emph{bounded away from zero}:
\begin{lemma} \label{lem 5.3.14}
Let \(x\) be a \emph{non-zero} real number.
Then \(x = \LIM_{n \toINF} a_n\) for some Cauchy sequence \((a_n)_{n = 1}^{\infty}\) which is \emph{bounded away from zero}.
\end{lemma}

\begin{proof}
Since \(x\) is real, we know that \(x = \LIM_{n \toINF} b_n\) for some Cauchy sequence \((b_n)_{n = 1}^{\infty}\).
But we are not yet done, because we do not know that \((b_n)_{n = 1}^{\infty}\) is bounded away from zero.

On the other hand, we are given that \(x \ne 0 = \LIM_{n \toINF} 0\), which means that the sequence \((b_n)_{n = 1}^{\infty}\) is \textbf{not} equivalent to \((0)_{n = 1}^{\infty}\).
Thus by \DEF{5.2.6} the sequence \((b_n)_{n = 1}^{\infty}\) \emph{cannot} be eventually \(\varE\)-close to \((0)_{n = 1}^{\infty}\) for every \(\varE > 0\).
Therefore we \emph{can find} an \(\varE >  0\) such that \((b_n)_{n = 1}^{\infty}\) is not eventually \(\varE\)-close to \((0)_{n = 1}^{\infty}\).

So let arbitrary \(\varE > 0\).
Since \((b_n)_{n = 1}^{\infty}\) is Cauchy, it is eventually \(\varE\)-steady.
Moreover, since \(\varE/2\) also \(> 0\) so \((b_n)_{n = 1}^{\infty}\) is eventually \(\varE/2\)-steady.
Thus there is an \(N \ge 1\) such that \(\abs{b_n - b_m} \le \varE/2\) for all \(n, m \ge N\) \MAROON{(1)}.
On the other hand, we cannot have \(\abs{b_n - 0} = \abs{b_n} \le \varE\) for all \(n \ge N\), 
since this would imply that \((b_n)_{n = 1}^{\infty}\) is eventually \(\varE\)-close to \((0)_{n = 1}^{\infty}\).
Thus there must be some \(n_0 \ge N\) for which \(\abs{b_{n_0} - 0} = \abs{b_{n_0}} > \varE\) \MAROON{(2)}.
Now from \MAROON{(1)}, in particular we have \(\abs{b_{n_0} - b_n} \le \varE/2\) for all \(n \ge N\),
and from the triangle inequality:
\begin{align*}
	         & \abs{b_{n_0} - b_n} \le \varE/2 \\
    \implies & -\abs{b_{n_0} - b_n} \ge -\varE/2 \\
    \implies & \abs{b_{n_0}} - \abs{b_{n_0} - b_n} \ge \abs{b_{n_0}} - \varE/2 \\
    \implies & \abs{b_{n_0}} - \abs{b_{n_0} - b_n} \ge \abs{b_{n_0}} - \varE/2 \ge \varE - \varE/2 = \varE/2 & \text{by \MAROON{(2)}}\\
    \implies & \abs{b_{n_0}} - \abs{b_{n_0} - b_n} \ge \varE/2 \\
    \implies & \abs{b_{n_0}} - \abs{b_n - b_{n_0}} \ge \varE/2 \\
    \implies & \abs{b_{n_0} + (b_n - b_{n_0})} \ge \abs{b_{n_0}} - \abs{b_n - b_{n_0}} \ge \varE/2 & \text{by \AC{4.3.1}} \\
    \implies & \abs{b_{n_0} + (b_n - b_{n_0})} \ge \varE/2 \\
    \implies & \abs{b_n} \ge \varE/2
\end{align*}
So \(\abs{b_n} \ge \varE/2\) for all \(n \ge N\).

This almost proves that \((b_n)_{n = 1}^{\infty}\) is bounded away from zero.
Actually, what it does is show that \((b_n)_{n = 1}^{\infty}\) is \emph{eventually} bounded away from zero.
But this is easily fixed, by \emph{defining a new sequence} \(a_n\), by setting \(a_n := \varE/2\) if \(n < N\) and \(a_n := b_n\) if \(n \ge N\).
Since \(b_n\) is a Cauchy sequence, it is not hard to verify that \(a_n\) is also a Cauchy sequence which is \emph{equivalent} to \(b_n\) (because the two sequences are eventually the same),
and so \(x = \LIM_{n \toINF} a_n\).
And since \(\abs{b_n} \ge \varE/2\) for all \(n \ge \textbf{N}\), we know that \(\abs{a_n} \ge \varE/2\) for all \(n \ge \textbf{1}\)
(splitting into the two cases \(n \ge N\) and \(n < N\) separately).
Thus we have a Cauchy sequence \(a_n\) which is bounded away from zero (by \(\varE/2\) instead of \(\varE\), but that’s still OK since \(\varE/2 > 0\)) and which has \(x\) as a formal limit, and so we are done.
\end{proof}

Once a sequence is bounded away from zero, we can take its \emph{reciprocal} without any difficulty:

\begin{lemma} \label{lem 5.3.15}
Suppose that \((a_n)_{n = 1}^{\infty}\) is a Cauchy sequence which is bounded away from zero.
Then the sequence \((a_n^{-1})_{n = 1}^{\infty}\) is also a Cauchy sequence.
\end{lemma}

\begin{proof}
Since \((a_n)_{n = 1}^{\infty}\) is bounded away from zero, by \DEF{5.3.12} we know that there is a \(c > 0\) such that \(\abs{a_n} \ge c\) for all \(n \ge 1\).

Now we need to show that \((a_n^{-1})_{n = 1}^{\infty}\) is eventually \(\varE\)-steady for each \(\varE > 0\).
Thus let arbitrary \(\varE > 0\);
our task is now to find an \(N \ge 1\) such that \(\abs{a_n^{-1} - a_m^{-1}} \le \varE\) for all \(n, m \ge N\).
But
\begin{align*}
    \abs{a_n^{-1} - a_m^{-1}} & = \abs{\frac1{a_n} - \frac1{a_m}} \\
                              & = \abs{\frac{a_m - a_n}{a_n a_m}} \\
                              & = \frac{\abs{a_m - a_n}}{\abs{a_n a_m}} \\
                              & = \frac{\abs{a_m - a_n}}{\abs{a_n}\abs{a_m}} \\
                              & \le \frac{\abs{a_m - a_n}}{c^2} & \text{since \(\abs{a_m}, \abs{a_n} \ge c\)}
\end{align*}
and so to make \(\abs{a_n^{-1} - a_m^{-1}}\) less than or equal to
\(\varE\), it will suffice to make \(\abs{a_m - a_n}\) less than or equal to \(c^2\varE\).

But since \((a_n)_{n = 1}^{\infty}\) is a Cauchy sequence, and \(c^2\varE > 0\), we can certainly find an \(N\) such that the sequence \((a_n)_{n = N}^{\infty}\) is \(c^2\varE\)-steady, 
i.e., \(\abs{a_m - a_n} \le c^2\varE\) for all \(m, n \ge N\).
By what we have said above, this shows that \(\abs{a_n^{-1} - a_m^{-1}} \le \varE\) for all \(m, n \ge N\), and hence the sequence \((a_n^{-1})_{n = 1}^{\infty}\) is eventually \(\varE\)-steady.
Since we have proven this for arbitrary \(\varE > 0\), we have that \((a_n^{-1})_{n = 1}^{\infty}\) is a Cauchy sequence, as desired.
\end{proof}

We are now ready to define reciprocation:
\begin{definition} [Reciprocals of \emph{real} numbers] \label{def 5.3.16}
Let \(x\) be a \emph{non-zero} real number.
Let \((a_n)_{n = 1}^{\infty}\) be a Cauchy sequence bounded away from zero such that \(x = \LIM_{n \toINF} a_n\)
(such a sequence exists by \LEM{5.3.14}).
Then we define the reciprocal \(x^{-1}\) by the formula \(x^{-1} := \LIM_{n \toINF} a_n^{-1}\).
(From \LEM{5.3.15} we know that \(x^{-1}\) is still a real number.)
\end{definition}

And we need to check that the reciprocals of the two different but \emph{equivalent}(in the sense of \DEF{5.2.6}) Cauchy sequences are still equivalent.

\begin{lemma} [Reciprocation is well defined] \label{lem 5.3.17}
Let \((a_n)_{n = 1}^{\infty}\) and \((b_n)_{n = 1}^{\infty}\) be two Cauchy sequences bounded away from zero such that \(\LIM_{n \toINF} a_n = \LIM_{n \toINF} b_n\)
(i.e., the two sequences are \emph{equivalent}).
Then \(\LIM_{n \toINF} a_n^{-1} = \LIM_{n \toINF} b_n^{-1}\).
\end{lemma}

\begin{proof}
Consider the following product \(P\) of three real numbers:
\[
    P := (\LIM_{n \toINF} a_n^{-1}) \X \BLUE{(\LIM_{n \toINF} a_n)} \X (\LIM_{n \toINF} b_n^{-1}).
\]
If we multiply (by \DEF{5.3.9}) this out, we obtain
\[
    P = \LIM_{n \toINF} a_n^{-1} a_n b_n^{-1} = \LIM_{n \toINF} b_n^{-1}.
\]
On the other hand, since \BLUE{\(\LIM_{n \toINF} a_n = \LIM_{n \toINF} b_n\)}, by \PROP{5.3.10}(multiplication is well-defined), we can write \(P\) in another way as
\[
     P = (\LIM_{n \toINF} a_n^{-1}) \X \BLUE{(\LIM_{n \toINF} b_n)} \X (\LIM_{n \toINF} b_n^{-1}).
\]
Multiplying things out again, we get
\[
    P = \LIM_{n \toINF} a_n^{-1} b_n b_n^{-1} = \LIM_{n \toINF} a_n^{-1}.
\]
Comparing our different formulae for \(P\) we see that \(\LIM_{n \toINF} a_n^{-1} = \LIM_{n \toINF} b_n^{-1}\), as desired.
\end{proof}

Thus reciprocal is well-defined (for each non-zero real number \(x\), we have \emph{exactly one definition of} the reciprocal \(x^{-1}\)).

\begin{note}
Note it is clear from the definition that \(xx^{-1} = x^{-1}x = 1\) (from the commutative law in \PROP{5.3.11} we only have to show \(xx^{-1} = 1\)):
\begin{align*}
    xx^{-1} & = \LIM_{n \toINF} a_n \LIM{n \toINF} a_n^{-1} \\
            & = \LIM_{n \toINF} a_n a_n^{-1} & \text{by \DEF{5.3.9}} \\
            & = \LIM_{n \toINF} 1 & \text{by \PROP{4.2.4}(10)} \\
            & = 1 & \text{by equating real to rational}
\end{align*}
Thus with this property, all the \emph{field} axioms (\PROP{4.2.4}) apply to the \emph{reals} as well as to the rationals.
\end{note}

\begin{note}
We of course cannot give \(0\) a reciprocal, since \(0\) multiplied by anything gives \(0\), not \(1\). 
\end{note}

\begin{note}
Also note that if \(q\) is a non-zero \emph{rational}, and hence equal to the real number \(\LIM_{n \toINF} q\),
then the reciprocal of \(\LIM_{n \toINF} q\) is \(\LIM_{n \toINF} q^{-1} = q^{-1}\);
That is,
\begin{align*}
    q^{-1} & = (\LIM_{n \toINF} q)^{-1} & \text{by equating rational and real} \\
           & = \LIM_{n \toINF} q^{-1} & \text{by \DEF{5.3.16}} \\
           & = q^{-1} & \text{by equating rational and real}
\end{align*}
thus the operation of \emph{reciprocal on real} numbers is consistent with the operation of \emph{reciprocal on rational} numbers.
\end{note}

Once one has reciprocal, one can define \emph{division} \(x/y\) of two real numbers \(x, y\), provided \(y\) is non-zero, by the formula
\[
    x/y := x \X y^{-1},
\]
just as we did with the rationals. (We label it as \DEF{5.3.20}.)
In particular, we have the cancellation law (label as \LEM{5.3.21}): if \(x, y, z\) are real numbers such that \(xz = yz\), and \(z\) is non-zero, then by dividing by \(z\) we conclude that \(x = y\).
Note that this cancellation law does not work when \(z\) is zero.

We now have all four of the basic arithmetic operations on the reals: addition, subtraction, multiplication, and division, with all the usual
rules of algebra.

\begin{definition} \label{def 5.3.18}
We define negation \(-x\) for real numbers \(x\) by the formula
\[
    -x := (-1) \X x,
\]
\end{definition}

\begin{definition} \label{def 5.3.19}
We define subtraction for real numbers \(x - y\) by the formula
\[
    x - y := x + (-y).
\]
\end{definition}

\begin{definition} \label{def 5.3.20}
We define division for real numbers \(x/y\), provided \(y \ne 0\), by the formula
\[
    x/y := x \X y^{-1}.
\]
\end{definition}

\begin{lemma} [Cancellation law for real number] \label{lem 5.3.21}
If \(x, y, z\) are real numbers such that \(xz = yz\), and \(z\) is non-zero, then by dividing by \(z\) we conclude that \(x = y\).
\end{lemma}

\exercisesection

\begin{exercise} \label{exercise 5.3.1}
Prove \PROP{5.3.3}.
\end{exercise}

\begin{proof}
See \PROP{5.3.3}.
\end{proof}

\begin{exercise} \label{exercise 5.3.2}
Prove \PROP{5.3.10}.
\end{exercise}

\begin{proof}
See \PROP{5.3.10}.
\end{proof}

\begin{exercise} \label{exercise 5.3.3}
Let \(a, b\) be \emph{rational} numbers.
Show that \(a = b\) if and only if \(\LIM_{n \toINF} a = \LIM_{n \toINF} b\)
(i.e., the Cauchy sequences \(a, a, a, a, ...\) and \(b, b, b, b, ...\) are equivalent if and only if \(a = b\)).
This allows us to \emph{embed the rational numbers inside the real numbers in a well-defined manner}.
\end{exercise}

\begin{proof}
\begin{align*}
         & a = b \\
    \iff & a - b = 0 \\
    \iff & \abs{a - b} = 0 \\
    \iff & \abs{a - b} < \varE\ \forall \varE > 0 \\
    \iff & \abs{a - b} < \varE\ \forall \varE > 0, \forall i \ge 1 & \text{just give a dummy index} \\
    \iff & (a)_{n = 1}^{\infty} = (b)_{n = 1}^{\infty} & \text{by \DEF{5.2.6}} \\
    \iff & \LIM_{n \toINF} a_n = \LIM_{n \toINF} b_n & \text{by \DEF{5.3.1}}
\end{align*}
\end{proof}

\begin{exercise} \label{exercise 5.3.4}
Let \((a_n)_{n = 0}^{\infty}\) be a sequence of rational numbers which is bounded.
Let \((b_n)_{n = 0}^{\infty}\) be another sequence of rational numbers which is \emph{equivalent} to \((a_n)_{n = 0}^{\infty}\).
Show that \((b_n)_{n = 0}^{\infty}\) is also bounded.
(Hint: use \EXEC{5.2.2}.)
\end{exercise}

\begin{proof}
Since \((a_n)_{n = 0}^{\infty}\) and \((b_n)_{n = 0}^{\infty}\) are equivalent, they are eventually \(\varE\)-closes for all \(\varE > 0\).
And since \((a_n)_{n = 0}^{\infty}\) is bounded, the conditions in the \EXEC{5.2.2} are satisfied, and by \EXEC{5.2.2}, \((b_n)_{n = 0}^{\infty}\) is bounded.
\end{proof}

\begin{note}
若\ seq A 是\ bounded,且\ seq B 等價於\ seq A,則\ seq B 也是\ bounded。
\end{note}

\begin{exercise} \label{exercise 5.3.5}
Show that \(\LIM_{n \toINF} 1 / n = 0\).
\end{exercise}

\begin{proof}
We have to show \(\LIM_{n \toINF} 1 / n = 0 = \LIM_{n \toINF} 0\).
Suppose for the sake of contradiction that \(\LIM_{n \toINF} 1 / n \ne \LIM_{n \toINF} 0\).
Then by \DEF{5.2.6}, There exists \(\varE > 0\), such that \textbf{for all \(N \ge 1\)}, there exists \(i \ge N\) s.t. \(\abs{1/i - 0} > \varE\).
And \(\abs{1/i - 0} = \abs{1/i}\) is positive, so we have \(1/i > \varE\), or \(i < 1/\varE\).
But since \(i \ge N\), we have \(N < 1/\varE\).
So we have for all \(N \ge 1\), \(N < 1/\varE\), which contradicts \PROP{4.4.1} that we can always find a natural number \(N\) s.t. \(N > 1/\varE\).
So \(\LIM_{n \toINF} 1 / n = 0\).
\end{proof}