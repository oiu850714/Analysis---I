\chapter{The real numbers} \label{ch 5}

\begin{note}
We defined the natural numbers using the five Peano axioms, and postulated that such a number system existed;
this is plausible, since the natural numbers correspond to the very intuitive and fundamental notion of \emph{sequential counting}.
\end{note}

\begin{note}
The symbols \(\SET{N}\), \(\SET{Q}\), and \(\SET{R}\) stand for ``natural'', ``quotient'', and ``real'' respectively.
\(\SET{Z}\) stands for ``Zahlen'', the German word for numbers.
There is also the \emph{complex numbers} \(\SET{C}\), which obviously stands for ``complex''.
\end{note}

\begin{note}
\emph{Formal} means ``having the form of'';
at the beginning of our construction the expression \(a \M b\) did not actually \emph{mean} the difference \(a - b\), since the symbol \(\M\) was meaningless.
It only had the \emph{form} of a difference.
Later on we defined subtraction and verified that the formal difference was equal to the actual difference, so this eventually became a non-issue, and our symbol for formal differencing was discarded.
Somewhat confusingly, this use of the term ``formal'' is unrelated to the notions of a formal argument and an informal argument.
\end{note}

\begin{note}
There is a fundamental area of mathematics where the rational number system \emph{does not suffice} - that of \emph{geometry} (the study of lengths, areas, etc.).
For instance, a right-angled triangle with both sides equal to \(1\) gives a hypotenuse of \(\sqrt{2}\), which is an \emph{irrational} number, i.e., not a rational number; see \PROP{4.4.4}.
Things get even worse when one starts to deal with the sub-field of geometry known as \emph{trigonometry}, when one sees numbers such as \(\pi\) or \(\cos(1)\), which turn out to be in some sense ``even more'' irrational than \(\sqrt{2}\).
(These numbers are known as \emph{transcendental numbers}, but to discuss this further would be far beyond the scope of this text.)
Thus, in order to have a number system which can adequately describe geometry
- or even something as simple as measuring lengths on a line
- one needs to replace the rational number system with the real number system.
\end{note}

\begin{note}
In the constructions of integers and rationals, the task was to introduce one more \emph{algebraic} operation to the number system
- e.g., one can get integers from naturals by introducing subtraction, and get the rationals from the integers by introducing division.
But to get the reals from the rationals is to pass \emph{from a ``discrete'' system to a ``continuous'' one}, and requires the introduction of a somewhat different notion
- that of a \emph{limit}.
\end{note}

\begin{note}
The limit is a concept which on one level is quite intuitive, but to pin down rigorously turns out to be quite difficult.
(Even such great mathematicians as Euler and Newton had difficulty with this concept.
It was only in the nineteenth century that mathematicians such as Cauchy and Dedekind figured out how to deal with limits rigorously.)
\end{note}

\begin{note}
In \SEC{4.4} we explored the ``gaps'' in the rational numbers;
now we shall fill in these gaps using limits to create the real numbers.
The real number system will end up being a lot like the rational numbers, but will have some new operations
- notably that of \emph{supremum}, which can then be used to define limits and thence to everything else that calculus needs.
\end{note}

\begin{note}
The procedure we give here of obtaining the real numbers as the limit of sequences of rational numbers may seem rather complicated.
However, it is in fact an instance of a very general and useful procedure, that of \emph{\href{https://www.wikiwand.com/en/Complete_metric_space}{completing} one metric space to form another}.
\end{note}

\newcommand{\LIM}{\text{LIM}}
\newcommand{\varE}{\varepsilon}
\newcommand{\toINF}{\to \infty}

\section{Cauchy sequences}

\begin{definition} [Sequences]  \label{def 5.1.1}
Let \(m\) be an integer.
A sequence \((a_n)_{n = m}^{\infty}\) of \emph{rational} numbers is any function from the set \( \{n \in \SET{Z} : n \ge m\} \) to \(\SET{Q}\),
i.e., a mapping which assigns to each integer \(n\) greater than or equal to \(m\), a rational number \(a_n\).
More informally, a sequence \((a_n)_{n = m}^{\infty}\)
of rational numbers is a collection of rationals \(a_m, a_{m + 1}, a_{m + 2},...\)
\end{definition}

\begin{example}
Simple example.
\end{example}

We want to define the \emph{real} numbers as the \emph{``limits''} of sequences of \emph{rational} numbers.
To do so, we have to distinguish which sequences of rationals are \emph{convergent} and which ones are not.
To do this we use the definition of \DEF{4.3.4} \(\varepsilon\)-closeness.

\begin{definition} [\(\varepsilon\)-steadiness] \label{def 5.1.3}
Let \(\varepsilon > 0\).
A sequence \((a_n)_{n = 0}^{\infty}\) is said to be \emph{\(\varepsilon\)-steady} iff each pair \(a_j, a_k\) of sequence elements is \(\varepsilon\)-close \emph{for every} natural number \(j, k\).
In other words, the sequence \(a_0, a_1, a_2,...\) is \(\varepsilon\)-steady iff \(d(a_j, a_k) \le \varepsilon\) for all \(j, k\).
\end{definition}

\begin{remark} \label{remark 5.1.4}
This definition is not standard in the literature;
we will not need it outside of this section;
similarly for the concept of ``eventual \(\varepsilon\)-steadiness'' below.
We have defined \(\varepsilon\)-steadiness for sequences whose index starts at \(0\), but clearly we can make a similar notion for sequences whose indices start from any other number:
a sequence \(a_N, a_{N + 1}, ...\) is \(\varepsilon\)-steady if one has \(d(a_j, a_k) \le \varepsilon\) for all \(j, k \ge N\).
\end{remark}

\begin{example} \label{example 5.1.5}
The sequence \(1, 0, 1, 0, 1,...\) is \(1\)-steady, but is not \(1/2\)-steady.
The sequence \(0.1, 0.01, 0.001, 0.0001,...\) is \(0.1\)-steady, but is not \(0.01\)-steady (why? Because the distance of the first two elements \(d(0.1, 0.01) = 0.09 > 0.01\)).
The sequence 1, 2, 4, 8, 16,... is not \(\varepsilon\)-steady for any \(\varepsilon\) (why? \MAROON{(1)})
The sequence 2, 2, 2, 2,... is \(\varepsilon\)-steady for every \(\varepsilon > 0\).
\end{example}

\begin{proof}
\MAROON{(1)} Given arbitrary \(\varepsilon\), by \PROP{4.4.1} there exists a natural number \(N\) s.t. \(N > \varepsilon\).
And by definition of the sequence, \(a_{N + 1} = 2^{N + 1}\), which \(\ge 2(N + 1)\), since \(N + 1\) must be positive and that satisfies \EXEC{4.3.5}.
And \(2(N + 1) - a_0 = 2N + 2 - a_0 = 2N + 2 - 1 = 2N + 1 > N\), which implies \(2^{N + 1} - a_0 > N\), that is, \(a_{N + 1} - a_0 > N\).
So we can find the pair \(a_{N + 1}, a_1\) s.t. \(d(a_{N + 1}, a_1) > N > \varepsilon\), so by \DEF{5.1.3}, the sequence is not \(\varepsilon\)-steadiness.
Since \(\varepsilon\) is arbitrary, for all \(\varepsilon\) the sequence is not \(\varepsilon\)-steady.
\end{proof}

\begin{note}
The notion of \(\varepsilon\)-steadiness of a sequence is simple, but does not really capture the \emph{limiting} behavior of a sequence, because it is \emph{too sensitive to the initial members} of the sequence.
So we need a more robust notion of steadiness that does not care about the initial members of a sequence.
\end{note}

\begin{definition} [Eventual \(\varepsilon\)-steadiness] \label{def 5.1.6}
Let \(\varepsilon > 0\).
A sequence \((a_n)_{n = 0}^{\infty}\) is said to be \emph{eventually \(\varepsilon\)-steady} iff the sequence \(a_N, a_{N + 1}, a_{N + 2},...\) is \(\varepsilon\)-steady \emph{for some} natural number \(N \ge 0\).
In other words, the sequence \(a_0, a_1, a_2, ...\) is eventually \(\varepsilon\)-steady iff there exists an \(N \ge 0\) such that \(d(a_j, a_k) \le \varepsilon\) for all \(j, k \ge N\).
\end{definition}

\begin{example} \label{example 5.1.7}
\raggedright The sequence \(a_1, a_2, ...\) defined by \(a_n := 1/n\), (that is, the sequence \(1, 1/2, 1/3, 1/4,...\)) is not \(0.1\)-steady, but is eventually \(0.1\)-steady, because the sequence \(a_{10}, a_{11}, a_{12}, ...\) (that is, \(1/10, 1/11, 1/12,...\)) is 0.1-steady.
The sequence \(10, 0, 0, 0, 0, ...\) is not \(\varepsilon\)-steady for any \(\varepsilon\) less than \(10\), but it is eventually \(\varepsilon\)-steady for every \(\varepsilon > 0\) (why? \MAROON{(1)}).
\end{example}

\begin{proof}
\MAROON{(1)} The distance of the first two elements of the sequence is \(d(10, 0) = 10\), so every \(\varepsilon < 10\) is less than \(d(10, 0)\).
But given any \(\varepsilon > 0\), let \(N = 1\), then for every \(j, k \ge N\), \(d(a_i, a_j) = d(0, 0) = 0 < \varepsilon\).
By \DEF{5.1.6}, the sequence is eventually \(\varepsilon\)-steady.
Since \(\varepsilon\) is arbitrary, for all \(\varepsilon > 0\) the sequence is eventual \(\varepsilon\)-steady.
\end{proof}

Now we can finally define the correct notion of what it means for a sequence of rationals to ``want'' to converge.

\begin{definition} [Cauchy sequences] \label{def 5.1.8}
A sequence \((a_n)_{n = 0}^{\infty}\) of \emph{rational} numbers is said to be a \emph{Cauchy sequence} iff for every rational \(\varepsilon > 0\), the sequence \((a_n)_{n = 0}^{\infty}\) is eventually \(\varepsilon\)-steady.
In other words, the sequence \(a_0, a_1, a_2,...\) is a Cauchy sequence iff for every \(\varepsilon > 0\), there exists an \(N \ge 0\) such that \(d(a_j, a_k) \le \varepsilon\) for all \(j, k \ge N\).
\end{definition}

\begin{remark} \label{remark 5.1.9}
At present, the parameter \(\varepsilon\) is restricted to be a positive \emph{rational};
we cannot take \(\varepsilon\) to be an arbitrary positive \emph{real} number, because the real numbers have not yet been constructed.
However, once we do construct the real numbers, we shall see that \DEF{5.1.8} will not change if we require \(\varepsilon\) to be real instead of rational(\PROP{6.1.4}).
In other words, we will eventually prove that
\begin{center}
    (a sequence is eventually \(\varepsilon\)-steady for every \emph{rational} \(\varepsilon > 0\)) if and only if (it is eventually \(\varepsilon'\)-steady for every \emph{real} \(\varepsilon' > 0\)).
\end{center}
(See \PROP{6.1.4}.)
This rather subtle distinction between a rational \(\varepsilon\) and a real \(\varepsilon'\) turns out \emph{not} to be very important in the long run, and the reader is advised not to pay too much attention as to what type of number \(\varepsilon\) should be.
\end{remark}

\begin{example} [\emph{Informal}] \label{example 5.1.10}
Consider the sequence \(1.4, 1.41, 1.414, 1.4142,...\) mentioned earlier.
This sequence is already \(0.1\)-steady.
If one discards the first element \(1.4\), then the remaining sequence \(1.41, 1.414, 1.4142,...\) is now \(0.01\)-steady, which means that the original sequence was eventually \(0.01\)-steady. 
Discarding the next element gives the \(0.001\)-steady sequence \(1.414, 1.4142,...\);
thus the original sequence was eventually \(0.001\)-steady. 
Continuing in this way it seems plausible that this sequence is in fact \(\varepsilon\)-steady for every \(\varepsilon > 0\), which seems to suggest that this is a Cauchy sequence.
However, this discussion is not rigorous for several reasons, for instance we have not precisely defined what this sequence \(1.4, 1.41, 1.414, \textbf{...}\) really is.
An example of a rigorous treatment follows next.
\end{example}

\begin{note}
我不確定課本說我們還沒定義\ \(1.4, 1.41, 1.414, ...\) 的意思是指什麼。也許可能是我們目前有的工具只有數學歸納法? 目前只能用數學歸納法(或遞迴,或者公式,嚴格來說還是遞迴)來定義\ sequence,但\ \(1.4, 1.41, 1.414, ...\) 其實跟數學歸納法無關,所以嚴格來說我們根本沒說清楚他是什麼?
\end{note}

\begin{proposition} \label{prop 5.1.11}
\sloppy The sequence \(a_1, a_2, a_3,...\) defined by \(a_n := 1/n\) (i.e., the sequence \(1, 1/2, 1/3,...\)) is a Cauchy sequence.
\end{proposition}

\begin{note}
Minor: the index is start from \(1\), not from \(0\).
\end{note}

\begin{proof}
We have to show that for every \(\varepsilon > 0\), the sequence \(a_1, a_2,...\) is eventually \(\varepsilon\)-steady.
So let \(\varepsilon > 0\) be \emph{arbitrary}.
We now have to find a number \(N \ge 1\) such that the sequence \(a_N, a_{N + 1},...\) is \(\varepsilon\)-steady.
This means that \(d(a_j, a_k) \le \varepsilon\) for every \(j, k \ge N\), i.e.
\begin{center}
    \(\abs{1 / j - 1 / k} \le \varepsilon\) for every \(j, k \ge N\).
\end{center}
Now since \(j, k \ge N\), we know that
\begin{align*}
             & 0 < 1 / j \le 1 / N \land 0 < 1 / k \le 1 / N \\
    \implies & 0 \le 1 / j \le 1 / N \land 0 \le 1 / k \le 1 / N & \text{in particular by \DEF{4.2.8}} \\
    \implies & 0 \le 1 / j \le 1 / N \land -(1 / N) \le - (1 / k) \le 0 & \text{by \EXEC{4.2.6}} \\
    \implies & -(1 / N) \le 1 / j - 1 / k \le 1 / N & \text{by \PROP{4.2.9}(e)} \\
    \implies & \abs{1 / j - 1 / k} \le 1 / N & \text{by \PROP{4.3.3}(b)}.
\end{align*}
So in order to force \(\abs{1 / j - 1 / k}\) to be less than or equal to \(\varepsilon\), it would be sufficient for \(1/N\) to be less than (or equal to) \(\varepsilon\).
So all we need to do is choose an \(N\) such that \(1 / N\) is less than \(\varepsilon\), or in other words that \(N\) is
greater than \(1/\varepsilon\).
But this can be done thanks to \PROP{4.4.1}.
(\(\varepsilon\) is rational, so \(1/\varepsilon\) is also rational, by \PROP{4.4.1} we can find a natural number \(N > 1/\varepsilon\).)
\end{proof}

\begin{note}
As you can see, verifying from first principles (i.e., without using any of the machinery of limits, etc.) that a sequence is a Cauchy sequence requires some effort, even for a sequence as simple as \(1/n\).
\emph{The part about selecting an \(N\) can be particularly difficult for beginners} - one has to \emph{think in reverse}, working out what conditions on \(N\) would suffice to force the sequence \(a_N, a_{N + 1}, a_{N + 2}, ...\) to be \(\varepsilon\)-steady, and then finding an \(N\) which obeys those conditions.
Later we will develop some limit laws which allow us to determine when a sequence is Cauchy more easily.
\end{note}

\begin{note}
先去想當\ \(N\) 滿足什麼條件時會導致整個\ sequence 是\ \(\varepsilon\)-steady,然後再找出這個\ \(N\)。
\end{note}

We now relate the notion of a Cauchy sequence to another basic notion, that of a \emph{bounded} sequence.

\begin{definition} [Bounded sequences] \label{def 5.1.12}
Let \(M \ge 0\) be rational.
A \emph{finite} sequence \(a_1, a_2, ..., a_n\) is \emph{bounded by} \(M\) iff \(\abs{a_i} \le M\) for all \(1 \le i \le n\).
An infinite sequence \((a_n)_{n = 1}^{\infty}\) is \emph{bounded by} \(M\) iff \(\abs{a_i} \le M\) for all \(i \ge 1\).
A sequence is said to be \emph{bounded} iff it is bounded by \(M\) \emph{for some} rational \(M \ge 0\).
\end{definition}

\begin{example} \label{example 5.1.13}
The finite sequence \(1, -2, 3, -4\) is bounded (in this case, it is bounded by \(4\), or indeed by any \(M\) greater than or equal to 4).
But the infinite sequence \(1, -2, 3, -4, 5, -6,...\) is unbounded. (Can you prove this? Use \PROP{4.4.1}.)
The sequence \(1, -1, 1, -1,...\) is bounded (e.g., by \(1\)), but is \emph{not} a Cauchy sequence(because in particular it's not eventual \(2.1\)-steady).
\end{example}

\begin{proof}
For the sake of contradiction, suppose the infinite sequence \(1, -2, 3, -4, 5, -6,...\) is bounded, i.e. bounded by some rational \(M\).
By \PROP{4.4.1}, we can find a natural number \(N\) s.t. \(N > M\).
WLOG, if we find \(N = 0\) then we redefine \(N := 1\) to make it positive.
Then it's clear that \(\abs{a_N} = N > M\), so by \DEF{5.1.12}, the sequence is not bounded by \(M\), a contradiction.
\end{proof}

\begin{lemma} [Finite sequences are bounded] \label{lem 5.1.14}
Every finite sequence \(a_1, a_2, ..., a_n\) is bounded.
\end{lemma}

\begin{proof}
We prove this by induction on \(n\).
When \(n = 1\) the arbitrary sequence \(a_1\) is clearly bounded, for if we choose \(M := \abs{a_1}\) then clearly we have \(\abs{a_i} \le M\) for all \(1 \le i \le n\).
Now suppose that we have already proved the lemma for some \(n \ge 1\);
that is, any finite sequence with \(n\) elements is bounded;
we now prove it for \(n + 1\), i.e., we prove every sequence with \(n + 1\) elements \(a_1, a_2, ..., a_{n + 1}\) is bounded.
So given arbitrary sequence of \(n + 1\) elements \(a_1, a_2, ..., a_n, a_{n + 1}\).
By the induction hypothesis we know that the first \(n\) elements \(a_1, a_2, ..., a_n\) is bounded by some \(M \le 0\);
in particular, it must be also bounded by \(M + \abs{a_{n + 1}}\) since \(M + \abs{a_{n + 1}} \ge M\).
On the other hand, \(a_{n + 1}\) is also bounded by \(\abs{a_{n + 1}}\), so is also bounded by \(M + \abs{a_{n + 1}}\) because the latter \(\ge\) the former.
Thus \(a_1, a_2, ... , a_n, a_{n + 1}\) is bounded by \(M +  \abs{a_{n + 1}}\), and is hence bounded.
This closes the induction.
\end{proof}

\begin{note}
Note that while this argument shows that every \textbf{finite} sequence is bounded, no matter how long the finite sequence is, it \textbf{does not} say anything about whether an \textbf{infinite} sequence is bounded or not;
infinity is not a natural number.
\end{note}

However, we have
\begin{lemma} [Cauchy sequences are bounded] \label{lem 5.1.15}
Every Cauchy sequence \((a_n)_{n = 1}^{\infty}\) is bounded.
\end{lemma}

\begin{proof}
Let \((a_n)_{n = 1}^{\infty}\) be an arbitrary Cauchy sequence, we have to show it is bounded by some rational \(M \ge 0\).

Now by \DEF{5.1.8}, given any \(\varepsilon > 0\), the sequence is eventually \(\varepsilon\)-steady.
By \DEF{5.1.6}, that means we can find an index \(N \ge 0\) s.t. \(d(a_j, a_k) \le \varepsilon\) for all \(j, k \ge N\) \MAROON{(1)}.
Now we split the sequence as the first \(a_1, ..., a_{N}\) and \(a_{N + 1}, ...\).

By \LEM{5.1.14}, the first part is finite, so it's bounded.

For the second part and \MAROON{(1)}, and in particular for the index \(j\) fixed as \(N + 1\), \(d(a_{N + 1}, a_k) \le \varepsilon\) for all \(k \ge N\) \MAROON{(2)}.
Now let \(M := \abs{a_{N + 1}} + \varepsilon\), we will show that every element \(a_k\) for all \(k \ge N\) is bounded by \(M\), that is, \(\abs{a_k} \le M\).
Note that \(M = \abs{a_{N + 1}} + \abs{\varepsilon}\) \MAROON{(3)} since \(\varepsilon > 0\).
So let \(k\) be arbitrary integer s.t. \(k \ge N\).
Then
\begin{align*}
             & d(a_{N + 1}, a_k) \le \varepsilon & \text{by \MAROON{(2)}} \\
    \implies & \abs{a_{N + 1} - a_k} \le \varepsilon & \text{by \DEF{4.3.2}} \\
    \implies & \abs{a_{N + 1} - a_k} \le \abs{\varepsilon} & \text{since \(\varepsilon > 0\)} \\
    \implies & \abs{a_{N + 1}} + \abs{a_{N + 1} - a_k} \le \abs{a_{N + 1}} + \abs{\varepsilon} & \text{by \PROP{4.2.9}(d)} \\
    \implies & \abs{a_{N + 1}} + \abs{a_{N + 1} - a_k} \le M & \text{by \MAROON{(3)}} \\
    \implies & \abs{-a_{N + 1}} + \abs{a_{N + 1} - a_k} \le M & \text{by \PROP{4.3.3}(d)} \\
    \implies & \abs{(-a_{N + 1}) + (a_{N + 1} - a_k)} \le \abs{-a_{N + 1}} + \abs{a_{N + 1} - a_k} \le M & \text{by \PROP{4.3.3}(b)} \\
    \implies & \abs{(-a_{N + 1}) + (a_{N + 1} - a_k)} \le M & \text{by \PROP{4.2.9}(c), transitive} \\
    \implies & \abs{-a_k} & \text{by algebra}. \\
    \implies & \abs{a_k} \le M & \text{by \PROP{4.3.3}(d)}
\end{align*}
Since \(k\) is arbitrary, \(\abs{a_k} \le M\) for all \(k \ge N\).
So the second part of the sequence is bounded by \(M\) and hence is bounded.

So the whole sequence is bounded.
\end{proof}

\begin{exercise} \label{exercise 5.1.1}
Prove \LEM{5.1.15}.
\end{exercise}

\begin{proof}
See \LEM{5.1.15}.
\end{proof}
\section{Equivalent Cauchy sequence} \label{sec 5.2}

Consider the two Cauchy sequences of rational numbers:
\[
    1.4, 1.41, 1.414, 1.4142, 1.41421,...
\]
and
\[
    1.5, 1.42, 1.415, 1.4143, 1.41422,...
\]
\emph{Informally}, both of these sequences seem to be converging to the same number, the square root
\(\sqrt{2} = 1.41421...\)
(though this statement is \emph{not yet rigorous} because we have not defined real numbers yet).
If we are to define the real numbers from the rationals as \emph{limits} of Cauchy sequences, \emph{we have to know when two Cauchy sequences of rationals give the same limit}, \textbf{without} first defining a real number
(since that would be circular).

\begin{note}
若你希望用柯西序列的極限來定義實數,那柯西序列的相等的定義就不能跟實數有任何關係,這樣會\ circular。
\end{note}

\begin{definition} [\(\varepsilon\)-close sequences] \label{def 5.2.1}
Let \((a_n)_{n = 0}^{\infty}\) and \((b_n)_{n = 0}^{\infty}\) be two sequences,
and let \(\varepsilon > 0\).
We say that the sequence \((a_n)_{n = 0}^{\infty}\) is \emph{\(\varepsilon\)-close} to \((b_n)_{n = 0}^{\infty}\) iff \(a_n\) is \(\varepsilon\)-close to \(b_n\) for each \(n \in \SET{N}\).
In other words, the sequence \(a_0, a_1, a_2,...\) is \(\varepsilon\)-close to the sequence \(b_0, b_1, b_2,...\) 
iff \(\abs{a_n - b_n} \le \varepsilon\) for all \(n = 0, 1, 2,...\).
\end{definition}

\begin{example} \label{example 5.2.2}
The two sequences
\[
    1, -1, 1, -1, 1,...
\]
and
\[
    1.1, -1.1, 1.1, -1.1, 1.1,...
\]
are \(0.1\)-close to each other.
(Note however that neither of them are \(0.1\)-steady).
\end{example}

\begin{definition} [Eventually \(\varepsilon\)-close sequences] \label{def 5.2.3}
Let \((a_n)_{n = 0}^{\infty}\) and \((b_n)_{n = 0}^{\infty}\) be two sequences, and let \(\varepsilon > 0\).
We say that the sequence \((a_n)_{n = 0}^{\infty}\) is \emph{eventually \(\varepsilon\)-close to} \((b_n)_{n = 0}^{\infty}\)
iff there exists an \(N \ge 0\) such that the sequences \((a_n)_{n = N}^{\infty}\) and \((b_n)_{n = N}^{\infty}\) are \(\varepsilon\)-close.
In other words, \(a_0, a_1, a_2,...\) is eventually \(\varepsilon\)-close to \(b_0, b_1, b_2,...\)
iff there exists an \(N \ge 0\) such that \(\abs{a_n - b_n} \le \varepsilon\) for all \(n \ge N\).
\end{definition}

\begin{remark} \label{remark 5.2.4}
Again, the notations for \(\varepsilon\)-close sequences and eventually \(\varepsilon\)-close sequences are not standard in the literature, and we will not use them outside of this section.
\end{remark}

\begin{example} \label{example 5.2.5}
Simple example that informally shows
\[
    1.1, 1.01, 1.001, 1.0001,...
\]
and
\[
    0.9, 0.99, 0.999, 0.9999,...
\]
are equivalent(see definition below).
\end{example}

\begin{definition} [Equivalent sequences] \label{def 5.2.6}
Two sequences \((a_n)_{n = 0}^{\infty}\) and \((b_n)_{n = 0}^{\infty}\) are \emph{equivalent}
iff for each \emph{rational} \(\varepsilon > 0\), the sequences \((a_n)_{n = 0}^{\infty}\) and \((b_n)_{n = 0}^{\infty}\) are eventually \(\varepsilon\)-close.
In other words, \(a_0, a_1, a_2,...\) and \(b_0, b_1, b_2,...\) are equivalent
iff for every rational \(\varepsilon > 0\), there exists an \(N \ge 0\) such that \(\abs{a_n - b_n} \le \varepsilon\) for all \(n \ge N\).
\end{definition}

\begin{remark} \label{remark 5.2.7}
As with \DEF{5.1.8}, the quantity \(\varepsilon > 0\) is currently restricted to be a positive \emph{rational}, rather than a positive \emph{real}.
However, we shall eventually see that it makes no difference whether \(\varepsilon\) ranges over the positive rationals or positive reals;
see \EXEC{6.1.10}.
\end{remark}

We now prove the sequences in \EXAMPLE{5.2.5} are equivalent rigorously.

\begin{proposition} \label{prop 5.2.8}.
Let \((a_n)_{n = 1}^{\infty}\) and \((b_n)_{n = 1}^{\infty}\) be the sequences
\(a_n := 1 + 10^{-n}\) and \(b_n = 1 - 10^{-n}\).
Then the sequences \(a_n, b_n\) are equivalent.
\end{proposition}

\begin{remark} \label{remark 5.2.9}
This Proposition, in \emph{decimal notation}, asserts that \(1.0000... = 0.9999...\); see \PROP{B.2.3}.
\end{remark}

\begin{proof}
We need to prove that for every \(\varepsilon > 0\), the two sequences \((a_n)_{n = 1}^{\infty}\) and \((b_n)_{n = 1}^{\infty}\) are eventually \(\varepsilon\)-close to each other.
So we let \(\varepsilon\) be an arbitrary rational such that \(\varepsilon > 0\).
We need to find an integer \(N \ge 1\) (where \(1\) is the starting index of the two sequences) such that \((a_n)_{n = N}^{\infty}\) and \((b_n)_{n = N}^{\infty}\) are \(\varepsilon\)-close;
in other words, we need to find an \(N \ge 1\) such that
\[
    \abs{a_n - b_n} \le \varepsilon \text{ for all } n \ge N.
\]
However, we have
\[
    \abs{a_n - b_n} = \abs{(1 + 10^{-n}) - (1 - 10^{-n})} = 2 \X 10^{-n}.
\]
Since \(10^{-n}\) is a \emph{decreasing} function of \(n\) (i.e., \(10^{-m} < 10^{-n}\) whenever \(m > n\);
this is easily proven by induction),
and since \(n \ge N\), we have \(2 \X 10^{-n} \le 2 \X 10^{-N}\).
Thus we have
\[
    \abs{a_n - b_n} \le 2 \X 10^{-N} \text{ for all } n \ge N.
\]
Thus in order to obtain \(\abs{a_n - b_n} \le \varepsilon\) for all \(n \ge N\), it will be sufficient to choose \(N\) so that \(2 \X 10^{-N} \le \varepsilon\).
This is easy to do using logarithms, but we have not yet developed logarithms yet, so we will use a cruder
method.
First, we observe \(10^N\) is always greater than \(N\) for any \(N \ge 1\) (see \EXEC{4.3.5}).
Thus \(10^{-N} \le 1/N\), and \(2 \X 10^{-N} \le 2/N\).

Thus to get \(2 \X 10^{-N} \le \varepsilon\), it will suffice to choose \(N\) so that \(2/N \le \varepsilon\), or equivalently that \(N \ge 2/\varepsilon\).
But by \PROP{4.4.1} since \(2/\varepsilon\) is rational, we can find \(N\) s.t. \(N > 2/\varepsilon\) and in particular \(N \ge 2/\varepsilon\).
So the two sequences are eventually \(\varepsilon\)-close.
Since \(\varepsilon\) are arbitrary, for all \(\varepsilon > 0\) the two sequences are eventually \(\varepsilon\)-close.
By \DEF{5.2.6}, the two sequences are equivalent.
\end{proof}


\exercisesection

\begin{exercise} \label{exercise 5.2.1}
Show that if \((a_n)_{n = 1}^{\infty}\) and \((b_n)_{n = 1}^{\infty}\) are equivalent sequences of rationals,
then \((a_n)_{n = 1}^{\infty}\) is a Cauchy sequence if and only if \((b_n)_{n = 1}^{\infty}\) is a Cauchy sequence.
\end{exercise}
\begin{proof}
Suppose \((a_n)_{n = 1}^{\infty}\) and \((b_n)_{n = 1}^{\infty}\) are equivalent sequences of rationals.
\begin{itemize}
    \item[\(\Longrightarrow\)]:
        Suppose \((a_n)_{n = 1}^{\infty}\) is Cauchy, we have to show \((b_n)_{n = 1}^{\infty}\) is Cauchy.
        So, let \(\varepsilon\) be arbitrary rational s.t. \(\varepsilon > 0\), we have to show \((b_n)_{n = 1}^{\infty}\) is eventual \(\varepsilon\)-steady.
        That is, we have to find a integer \(N \ge 1\), which is the starting index of the two sequences, s.t. \(d(b_j, b_k) \le \varepsilon\) for all \(j, k \ge N\).
        
        But by supposition \((a_n)_{n = 1}^{\infty}\) is Cauchy, so we can find a integer \(N_1 \ge 1\) s.t. \(d(a_j, a_k) \le \varepsilon\) for all \(j, k \ge N_1\) \GREEN{(1)}.
        And also since the two sequences are equivalent, we can find a integer \(N_2 \ge 1\) s.t. \(d(a_i, b_i) \le \varepsilon\) for all \(i \ge N_2\) \GREEN{(2)}.
        
        Now let \(N_3 := max(N_1, N_2)\), for all \(i, j \ge N_3\), by \GREEN{(1)} we have \(d(a_i, a_j) \le \varepsilon\) \MAROON{(1)};
        and by \GREEN{(2)} we have \(d(a_i, b_i) \le \varepsilon\) \MAROON{(2)} and \(d(a_j, b_j) \le \varepsilon)\) \MAROON{(3)}.
        And by \PROP{4.3.7}(f) and \MAROON{(2)}, we have \(d(a_i, b_i) = d(b_i, a_i)\) \MAROON{(4)}.
        So by adding both sides of the three inequalities \MAROON{(1) (3) (4)}, we have \(d(b_i, a_i) + d(a_i, a_j) + d(a_j, b_j) \le 3\varepsilon\).
        But by applying \PROP{4.3.7}(g) twice, we have \(d(b_i, b_j) \le d(b_i, a_i) + d(a_i, a_j) + d(a_j, b_j)\).
        So by transitivity, we have \(d(b_i, b_j) \le 3\varepsilon\) for all \(i, j \le N_3\).
        
        (I do this with purpose and first conclude that the sequence \(b_n\) is \(3\varepsilon\)-steady, because finding \(N\) for \(\varepsilon/3\) in the \emph{beginning} of the proof is somewhat anti-human.)
        
        Now let \(\varepsilon' = \varepsilon/3\).
        Then with the same argument above, we can conclude that we can find \(N_4 \ge 1\) s.t. \(d(b_i, b_j) \le 3\varepsilon'\) for all \(i, j \le N_4\).
        But \(3\varepsilon' = 3(\varepsilon/3) = \varepsilon\), so we have \(d(b_i, b_j) \le \varepsilon\) for all \(i, j \le N_4\).
        So by \DEF{5.1.6}, \((b_n)_{n = 1}^{\infty}\) is \(\varepsilon\)-steady.
        
        Since \(\varepsilon > 0\) is arbitrary, \((b_n)_{n = 1}^{\infty}\) is eventual \(\varepsilon\)-steady for all \(\varepsilon > 0\).
        By \DEF{5.1.8}, \((b_n)_{n = 1}^{\infty}\) is Cauchy!
    \item[\(\Longleftarrow\)]:
        The proof is similar with previous case, with \(a_{blabla}\) and \(b_{blabla}\) swapped.
\end{itemize}
\end{proof}

\begin{exercise} \label{exercise 5.2.2}
Let \(\varepsilon > 0\).
Show that if \((a_n)_{n = 1}^{\infty}\) and \((b_n)_{n = 1}^{\infty}\) are eventually \(\varepsilon\)-close,
then \((a_n)_{n = 1}^{\infty}\) is bounded if and only if \((b_n)_{n = 1}^{\infty}\) is bounded.
\end{exercise}
\begin{proof}
Let \(\varepsilon > 0\) and suppose \((a_n)_{n = 1}^{\infty}\) and \((b_n)_{n = 1}^{\infty}\) are eventually \(\varepsilon\)-close.
\begin{itemize}
    \item[\(\Longrightarrow\)]
        Suppose \((a_n)_{n = 1}^{\infty}\) is bounded, we have to show \((b_n)_{n = 1}^{\infty}\) is also bounded.

        Since \((a_n)_{n = 1}^{\infty}\) is bounded, we can find a rational \(M_1 \ge 0\) s.t. \(\abs{a_i} \le M_1\) for all \(i \ge 1\). And since the two sequence are eventually \(\varepsilon\)-close, we can find an integer \(N_1 \ge 1\) s.t. \(d(a_i, b_i) \le \varepsilon\) for all \(i \ge N_1\).
        
        Now again like the proof in \LEM{5.1.15}, we split \((b_n)_{n = 1}^{\infty}\) into \((b_n)_{n = 1}^{N_1}\) and \((b_n)_{N_1 + 1}^{\infty}\).
        By \LEM{5.1.14} the first finite part is bounded.
        
        Now let \(M_2 := M_1 + \varepsilon\), we will show that the second part of the sequence, \((b_n)_{n = N_1 + 1}^{\infty}\), is bounded by \(M_2\) and hence is bounded.
        So given arbitrary index \(i \ge N_1 + 1\):
        \begin{align*}
                     & d(a_i, b_i) \le \varepsilon & \text{since the two seq. are even. \(\varepsilon\)-close} \\
            \implies & d(b_i, a_i) \le \varepsilon & \text{by \PROP{4.3.3}(f)} \\
            \implies & \abs{b_i - a_i} \le \varepsilon & \text{by \DEF{4.3.2}} \\
            \implies & M_1 + \abs{b_i - a_i} \le M_1 + \varepsilon & \text{by \PROP{4.2.9}(d)} \\
            \implies & M_1 + \abs{b_i - a_i} \le M_2 & \text{since \(M_2 = M_1 + \varepsilon\)}\\
            \implies & \abs{a_i} + \abs{b_i - a_i} \le M_1 + \abs{b_i - a_i} \le M_2 & \text{since \(\abs{a_i} \le M_1\)} \\
            \implies & \abs{a_i} + \abs{b_i - a_i} \le M_2 & \text{transitive} \\
            \implies & \abs{a_i + (b_i - a_i)} \le \abs{a_i} + \abs{b_i - a_i} \le M_2 & \text{by \PROP{4.3.3}(b)} \\
            \implies & \abs{a_i + (b_i - a_i)} \le M_2 & \text{transitive}  \\
            \implies & \abs{b_i} \le M_2 & \text{trivial} \\
        \end{align*}
        Since \(i \ge N_1 + 1\) is arbitrary, so for all \(i \ge N_1 + 1\), \(\abs{b_i} \le M_2\).
        By \DEF{5.1.12}, \((b_n)_{n = N_1 + 1}^{\infty}\) is bounded by \(M_2\) and hence is bounded.

        So the whole sequence is bounded.
    \item[\(\Longleftarrow\)]:
        The proof is similar with previous case, with \(a_{blabla}\) and \(b_{blabla}\) swapped.
\end{itemize}
\end{proof}
\section{The construction of the real numbers} \label{sec 5.3}

\begin{definition} [Real numbers] \label{def 5.3.1}
A \emph{real number} is defined to be an \emph{object} of the \emph{form} \(\LIM_{n \toINF} a_n\),
where \((a_n)_{n = 1}^{\infty}\) is a Cauchy sequence of \emph{rational} numbers.
Two real numbers \(\LIM_{n \toINF} a_n\) and \(\LIM_{n \toINF} b_n\) are said to be \emph{equal} iff \((a_n)_{n = 1}^{\infty}\) and \((b_n)_{n = 1}^{\infty}\) are equivalent Cauchy sequences.
The set of all real numbers is denoted \(\SET{R}\).
\end{definition}

\begin{example} [Informal] \label{example 5.3.2}
Let \(a_1, a_2, a_3, ...\) denote the sequence
\[
    1.4, 1.41, 1.414, 1.4142, 1.41421,...
\]
and let \(b_1, b_2, b_3,...\) denote the sequence
\[
    1.5, 1.42, 1.415, 1.4143, 1.41422,...
\]
then \(\LIM_{n \toINF} a_n\) is a real number, and is \emph{the same} real number as \(\LIM_{n \toINF} b_n\),
because \((a_n)_{n = 1}^{\infty}\) and \((b_n)_{n = 1}^{\infty}\) are equivalent Cauchy sequences:
\(\LIM_{n \toINF} a_n = \LIM_{n \toINF} b_n\).
\end{example}
\begin{note}
到目前為止,我們還沒有證明:
\begin{itemize}
    \item 這兩個\ sequence 都是\ Cauchy sequences。
    \item 進一步來說,也無法知道他們對應的\ \LIM\ object 是\ real number。
    \item 所以也無法知道討論他們是否相等。
\end{itemize}
而且開天眼,這整節看完之後我們好像還是不能知道他們是\ Cauchy...。
\end{note}

\begin{note}
We will refer to \(\LIM_{n \toINF} a_n\) as the \emph{formal limit} of the sequence \((a_n)_{n = 1}^{\infty}\).
Later on we will define a genuine notion of limit, and show that the formal limit of a Cauchy sequence is the same as the limit of that sequence;
after that, we will not need formal limits ever again.
(The situation is much like what we did with \emph{formal} subtraction \(\M\) and \emph{formal} division \(\D\).)
\end{note}

\begin{note}
we need to check that the notion of equality in the definition obeys the first three laws of
equality:
\end{note}

\begin{proposition} [Formal limits are well-defined] \label{prop 5.3.3}
Let \(x = \LIM_{n \toINF} a_n,\ y = \LIM_{n \toINF} b_n\), and \(z = \LIM_{n \toINF} c_n\) be real numbers.
(This implies the corresponding sequences are Cauchy.)
Then, with the above \DEF{5.3.1} of equality for real numbers, we have \(x = x\).
Also, if \(x = y\), then \(y = x\).
Finally, if \(x = y\) and \(y = z\), then \(x = z\).
\end{proposition}

\begin{proof}
Reflexive: Since given \(\varE > 0\), \(\varE > 0 = d(a_n, a_n)\) for all \(n \ge 1\), by \DEF{5.2.6}, \((a_n)_{n = 1}^{\infty}\) and \((a_n)_{n = 1}^{\infty}\) are equivalent.
And by \DEF{5.3.1}, \(\LIM_{n \toINF} a_n = \LIM_{n \toINF} a_n\), that is, \(x = x\).

Symmetric: Suppose \(x = y\).
By \DEF{5.3.1}, \((a_n)_{n = 1}^{\infty}\) and \((b_n)_{n = 1}^{\infty}\) are equivalent.
And by \DEF{5.2.6}, For all \(\varE > 0\), there exists \(N \ge 1\) s.t. for all \(n \ge N\), \(d(a_n, b_n) \le \varE\) \MAROON{(1)}.
But by \PROP{4.3.3}(f), \(d(a_n, b_n) = d(b_n, a_n)\).
So with \MAROON{(1)}, we have For all \(\varE > 0\), there exists \(N \ge 1\) s.t. for all \(n \ge N\), \(\MAROON{d(b_n, a_n)} \le \varE\).
By \DEF{5.2.6}, \((b_n)_{n = 1}^{\infty}\) and \((a_n)_{n = 1}^{\infty}\) are equivalent.
By \DEF{5.3.1}, \(\LIM_{n \toINF} b_n = \LIM_{n \toINF} b_n\), that is, \(y = x\).

Transitive: Suppose \(x = y\) and \(y = z\).
By \DEF{5.3.1}, \((a_n)_{n = 1}^{\infty}, (b_n)_{n = 1}^{\infty}\) are equivalent and \((b_n)_{n = 1}^{\infty}, (c_n)_{n = 1}^{\infty}\) are equivalent.
Now given arbitrary \(\varE > 0 \), so \emph{in particular \(\varE/2 > 0\)}, we have
\begin{itemize}
    \item There exists \(N_1 \ge 1\) s.t. for all \(n \ge N_1\), \(d(a_n, b_n) \le \varE/2\) \MAROON{(2)}.
    \item There exists \(N_2 \ge 1\) s.t. for all \(n \ge N_2\), \(d(b_n, c_n) \le \varE/2\) \MAROON{(3)}.
\end{itemize}
Let \(N_3 := max(N_1, N_2)\), then by \MAROON{(2) (3)}, we have for all \(n \ge N_3\), \(d(a_n, b_n) \le \varE/2\) and \(d(b_n, c_n) \le \varE/2\).
But by \DEF{4.3.4}, that implies \(a_n, b_n\) and \(b_n, c_n\) are \(\varE/2\)-close.
By \PROP{4.3.7}, we have \(a_n, c_n\) are \(\varE/2 + \varE/2 = \varE\)-close.
By \DEF{4.3.4}, \(d(a_n, c_n) \le \varE\).
So for all \(n \ge N_3\), \(d(a_n, c_n) \le \varE\).
By \DEF{5.2.6}, \((a_n)_{n = 1}^{\infty}, (c_n)_{n = 1}^{\infty}\) are equivalent.
By \DEF{5.3.1}, \(\LIM_{n \toINF} a_n = \LIM_{n \toINF} c_n\), that is, \(x = z\).
\end{proof}

\begin{note}
By \PROP{5.3.3}, we now have well-defined equality between two real numbers.
Of course, when we define other operations on the reals, \emph{we have to check that they obey the law of substitution} \AXM{a.7.4}:
two real number inputs which are \emph{equal} should give equal outputs when applied to any operation on the real numbers.

Now we want to define all the usual arithmetic operations on the real numbers, such as addition and multiplication. 
\end{note}

\begin{definition} [Addition of reals] \label{def 5.3.4} 
Let \(x = \LIM_{n \toINF} a_n\) and \(y = \LIM_{n \toINF} b_n\) be real numbers.
Then we \emph{define} the \emph{sum} \(x + y\) to be \(x + y := \LIM_{n \toINF} (a_n + b_n)\).
\end{definition}

\begin{example} \label{example 5.3.5}
The sum of \(\LIM_{n \toINF} 1 + 1/n\) and \(\LIM_{n \toINF} 2 + 3 / n\) is \(\LIM_{n \toINF} 3 + 4/n\).
\end{example}

We now check that \DEF{5.3.4} is valid.
The first thing we need to do is to confirm that the sum of two real numbers is in fact a real number:

\begin{lemma} [Sum of Cauchy sequences is Cauchy] \label{lem 5.3.6}
Let \(x = \LIM_{n \toINF} a_n\) and \(y = \LIM_{n \toINF} b_n\) be real numbers.
Then \(x + y\) is also a real number. (i.e., \((a_n + b_n)_{n = 1}^{\infty}\) is a Cauchy sequence of rationals.)
\end{lemma}


\begin{proof}
We need to show that for every \(\varE > 0\), the sequence \((a_n + b_n)_{n = 1}^{\infty}\) is eventually \(\varE\)-steady.
Now from hypothesis and \DEF{5.3.1} we know that \((a_n)_{n = 1}^{\infty}\) is Cauchy and hence by \DEF{5.1.8} is eventually \(\varE\)-steady, and similarly \((b_n)_{n = 1}^{\infty}\) is eventually \(\varE\)-steady;
but it turns out that this is not quite enough
(this can be used to imply that \((a_n + b_n)_{n = 1}^{\infty}\) is eventually \(2\varE\)-steady, but that's not what we want).
So we need to do a little trick, which is to play with the value of \(\varE\).

We know that \((a_n)_{n = 1}^{\infty}\) is eventually \(\delta\)-steady for every value of \(\delta > 0\).
This implies not only that \((a_n)_{n = 1}^{\infty}\) is eventually \(\varE\)-steady, but it is also eventually \(\varE / 2\)-steady(just in particular let \(\delta = \varE/2\)).
Similarly, the sequence \((b_n)_{n = 1}^{\infty}\) is also eventually \(\varE / 2\)-steady.
This will turn out to be enough to conclude that \((a_n + b_n)_{n = 1}^{\infty}\) is eventually \(\varE\)-steady.

Since \((a_n)_{n = 1}^{\infty}\) is eventually \(\varE / 2\)-steady, we know that there exists an \(N \geq 1\) such that \((a_n)_{n = N}^{\infty}\) is \(\varE / 2\)-steady, i.e., \(a_n\) and \(a_m\) are \(\varE / 2\)-close for every \(n, m \geq N\).
Similarly there exists an \(M \geq 1\) such that \((b_n)_{n = M}^{\infty}\) is \(\varE / 2\)-steady, i.e., \(b_n\) and \(b_m\) are \(\varE / 2\)-close for every \(n, m \geq M\).

Let \(\max(N, M)\) be the larger of \(N\) and \(M\)
(we know from \PROP{2.2.13} that one has to be greater than or equal to the other).
If \(n, m \geq \max(N, M)\), then we know that \(a_n\) and \(a_m\) are \(\varE / 2\)-close, and \(b_n\) and \(b_m\) are \(\varE / 2\)-close, and so by \PROP{4.3.7}(d) we see that \(a_n + b_n\) and \(a_m + b_m\) are \(\varE/2 + \varE/2 = \varE\)-close for every \(n, m \geq \max(N, M)\).
This implies that the sequence \((a_n + b_n)_{n = 1}^{\infty}\) is eventually \(\varE\)-steady, as desired.
\end{proof}

The other thing we need to check is the axiom of substitution \AXM{a.7.4}:
if we replace a real number \(x\) by another number \(x'\) \emph{equal} to \(x\),  then we must have \(x' + y = x + y\)
(and similarly if we substitute \(y\) by another number \(y'\) equal to \(y\)).

\begin{lemma} [Sums of equivalent Cauchy sequences are equivalent]\label{lem 5.3.7}.
Let \(x = \LIM_{n \toINF} a_n, y = \LIM_{n \toINF} b_n\), and \(x' = \LIM_{n \toINF} a'_n\) be real numbers.
Suppose that \(x = x'\).
Then we have \(x + y = x' + y\).
\end{lemma}

\begin{proof}
We need to show \(x + y = x' + y\), 
that is by \DEF{5.3.4} \(\LIM_{n \toINF} (a_n + b_n) = \LIM_{n \toINF} (a'_n + b_n)\), 
that is by \DEF{5.3.1}, the sequences \((a_n + b_n)_{n = 1}^{\infty}\) and \((a'_n + b_n)_{n = 1}^{\infty}\) are eventually \(\varE\)-close for each \(\varE > 0\) \MAROON{(1)}. 
But since \(x = x'\), we know that the Cauchy sequences \((a_n)_{n = 1}^{\infty}\) and \((a'_n)_{n = 1}^{\infty}\) are equivalent, by similar definition derivation we know that there is an \(N \ge 1\) such that \((a_n)_{n = N}^{\infty}\) and \((a'_n)_{n = N}^{\infty}\) are \(\varE\)-close,
i.e., that \(a_n\) and \(a'_n\) are \(\varE\)-close for each \(n \ge N\).
Since \(b_n\) is of course \(0\)-close to \(b_n\)
(where we extend the notion of \(\varE\)-closeness(The author seems to mean \DEF{4.3.4}, \(\varE\)-closeness for two numbers) to include \(\varE=0\) in the obvious fashion),
we thus see from \PROP{4.3.7}(d)
(extended in the obvious manner to the \(\delta =  0\) case, or extended to cover the \(0\)-close case)
that \(a_n + b_n\) and \(a'_n + b_n\) are \(\varE + 0 = \varE\)-close for each \(n \ge N\).
This implies \MAROON{(1)} is true, and we are done.
\end{proof}

\begin{remark} \label{remark 5.3.8}
The above lemma verifies the axiom of substitution \AXM{a.7.4} for the ``\(x\)'' variable in \(x + y\),
but one can similarly prove the \AXM{a.7.4} for the ``\(y\)'' variable.
(A quick way is to observe from the definition of \(x + y\) that we certainly have \(x + y = y + x\), since \(a_n + b_n\) = \(b_n + a_n\).)
\end{remark}

We can define multiplication of real numbers in a manner similar to that of addition:

\begin{definition} [Multiplication of reals] \label{def 5.3.9}
Let \(x = \LIM_{n \toINF} a_n\) and \(y = \LIM_{n \toINF} b_n\) be real numbers.
Then we \emph{define} the product \(xy\) to be \(xy :=  \LIM_{n \toINF} a_n b_n\).
\end{definition}

\begin{proposition} [Multiplication is well defined] \label{prop 5.3.10}
Let \(x = \LIM_{n \toINF} a_n, y = \LIM_{n \toINF} b_n\), and \(x' = \LIM_{n \toINF} a'_n\) be real numbers.
Then \(xy\) is also a real number.
Furthermore, if \(x = x'\), then \(xy = x'y\). 
\end{proposition}

\begin{proof}
\begin{itemize}
    \item
        First we show \(xy\) is real;
        that is, \((a_n b_n)_{n = 1}^{\infty}\) is Cauchy;
        that is, for all \(\varE > 0\), there exists \(N \ge 1\) s.t. \(d(a_i b_i, a_j b_j) \le \varE\) for all \(i, j \ge N\).
        
        So let \(3 > \varE > 0\). \BLUE{(trick 1)}
        (The purpose of the upper bound is to make some trick in the latter proof).
        It's trivial that if \((a_n b_n)_{n = 1}^{\infty}\) is eventually \(\varE\)-close, then given any \(\delta \ge 3\), \((a_n b_n)_{n = 1}^{\infty}\) is also \(\delta\)-close.
        (By \PROP{4.3.7}(e) and the definition of closeness of the sequences and rational numbers.)
        So together we can conclude \((a_n b_n)_{n = 1}^{\infty}\) is eventually \(\varE\)-close for all \(\varE > 0\).
        
        Since \((a_n)_{n = 1}^{\infty}\) and \((b_n)_{n = 1}^{\infty}\) are Cauchy, by \LEM{5.1.15} they are bounded by some rational \(M_1, M_2 \ge 0\),
        that is, \(\abs{a_n} \le M_1\) and \(\abs{b_n} \le M_2\) for all \(n \ge 1\).
        Now let \(M = max(M_1, M_2, 1)\), and that trivially implies \(\abs{a_n} \le M\) \MAROON{(1)} and \(\abs{b_n} \le M\) \MAROON{(2)} for all \(n \ge 1\).
        (Note that \(M \ge 1\), this is also used as a trick in the latter proof \BLUE{(trick 2)}).
        
        And since \((a_n)_{n = 1}^{\infty}\) and \((b_n)_{n = 1}^{\infty}\) are Cauchy, there exists \(N_1 \ge 1\) s.t. \(d(a_i , a_j) \le \varE\) for all \(i, j \ge N_1\),
        and there exists \(N_2 \ge 1\) s.t. \(d(b_i , b_j) \le \varE\) for all \(i, j \ge N_2\).
        Let \(N = max(N_1, N_2)\), then \(d(a_i , a_j) \le \varE\) and \(d(b_i, b_j)\le \varE\) for all \(i, j \ge N\).
        And by \PROP{4.3.7}(h), we have \(d(a_i b_i, a_j b_j) \le \varE\abs{b_i} + \varE\abs{a_i} + \varE^2\).
        And by \MAROON{(1) (2)}
        \begin{align*}
                & \varE\abs{b_i} + \varE\abs{a_i} + \varE^2 \\
            \le &  \varE M + \varE M + \varE^2 \\
              = & \varE(2M + \varE),
        \end{align*}
        so we have \(d(a_i b_i, a_j b_j) \le \varE(2M + \varE)\) for all \(i, j \ge N\) \MAROON{(3)}.
        
        \sloppy Now let \(\varE' = \frac{\varE}{3M}\), then by similar argument until \MAROON{(3)},
        there exists \(N_3 \ge 1\) s.t. for all \(i, j \ge N_3\).
        \begin{align*}
            d(a_i b_i, a_j b_j) & \le \varE'(2M + \varE') \\
                                & = \frac{\varE}{3M}(2M + \frac{\varE}{3M}) \\
                                & = \frac{2\varE}{3} + \frac{\varE^2}{9M^2} \\
                                & \le \frac{2\varE}{3} + \frac{\varE^2}{9\X1^2} & \text{since \(M \ge 1\), trick \BLUE{(2)}} \\
                                & = \frac{2\varE}{3} + \frac{\varE^2}{9} \\
                                & = \frac{2\varE}{3} + \varE \X \frac{\varE}{9} \\
                                & \le \frac{2\varE}{3} + 3 \X \frac{\varE}{9} & \text{since \(\varE < 3\), trick \BLUE{(1)}} \\
                                & = \frac{2\varE}{3} + \frac{\varE}{3} \\
                                & = \varE
        \end{align*}
        By \DEF{5.1.6}, \((a_n b_n)_{n = 1}^{\infty}\) is eventually \(\varE\)-close.
        Since \(\varE\) is arbitrary between \(0\) and \(3\), \((a_n b_n)_{n = 1}^{\infty}\) is eventually \(\varE\)-close for all \(0 < \varE < 3\),
        and by the previous discussion it's trivial that \((a_n b_n)_{n = 1}^{\infty}\) is eventually \(\varE\)-close for all \(\varE > 0\).
        So by \DEF{5.1.8}, \((a_n b_n)_{n = 1}^{\infty}\) is Cauchy.
    \item
        Now we prove if \(x = x'\), then \(xy = x'y\).
        Since \(x = x'\), \((a_n)_{n = 1}^\infty\) and \((a'_n)_{n = 1}^\infty\) are equivalent sequences.
        Since \((b_n)_{n = 1}^\infty\) is Cauchy, there is some number \(M' \ge 0\) which bounds it.
        Now let \(M = max(M', 1)\), then it's trivial that \(M\) also bounds \((b_n)_{n = 1}^\infty\) and \(M\) is positive.

        Now let \(\varE > 0\), so in particular \(\varE/M > 0\).
        Then since \((a_n)_{n=1}^\infty\) and \((a'_n)_{n=1}^\infty\) are equivalent sequences, they are eventually \(\varE/M\)-close.
        So there exists \(N \ge 1\) s.t. \(a_i, a'_i\) are \(\varE/M\)-close for all \(i \ge N\).
        And by \PROP{4.3.7}(g), \(a_i b_i, a'_i b_i\) are \((\varE/M)\abs{b_i}\)-close for all \(i \ge N\).
        But since \(M\) bounds \((b_n)_{n = 1}^\infty\), we have \(\abs{b_i} \le M\), so \((\varE/M)\abs{b_i} \le (\varE/M)M = \varE\).
        So \(a_i b_i, a'_i b_i\) are \(\varE\)-close for all \(i \ge N\).
        So \((a_n b_n)_{n = 1}^\infty\) and \((a'_n b_n)_{n = 1}^\infty\) is eventually \(\varE\)-close.
        Since \(\varE > 0\) is arbitrary, by \DEF{5.2.6} \((a_n b_n)_{n = 1}^\infty\) and \((a'_n b_n)_{n = 1}^\infty\) are equivalent, that is, \(xy = x'y\).
\end{itemize}

\end{proof}

Of course we can prove a similar substitution rule when \(y\) is replaced by a real number \(y'\) which is equal to \(y\).

\begin{note}
At this point we \emph{embed the rationals back into the reals}, by equating every \emph{rational} number \(q\) with the \emph{real} number \(\LIM_{n \toINF} q\).
(That is, \(q \equiv \LIM_{n \toINF} q\).)
For instance, if \(a_1, a_2, a_3, ...\) is the sequence
\[
	0.5, 0.5, 0.5, 0.5, 0.5,...
\]
Then we let \(\LIM_{n \toINF} a_n\) equal to \(0.5\).

This embedding is consistent with our definitions of addition and multiplication, since for any rational numbers \(a, b\) we have
\begin{align*}
	a + b & = \LIM_{n \toINF} a + \LIM_{n \toINF} b & \text{by equating real and rational} \\
          & = \LIM_{n \toINF}(a + b) & \text{by \DEF{5.3.4}} \\
          & = a + b & \text{by equating again}
\end{align*}
And
\begin{align*}
	ab & = \LIM_{n \toINF} a \X \LIM_{n \toINF} b & \text{by equating} \\
          & = \LIM_{n \toINF}(ab) & \text{by \DEF{5.3.9}} \\
          & = ab & \text{by equating}
\end{align*}
(Note that the element \(a\) or \(b\) of the formula \(\LIM_{n \toINF} a\) or \(\LIM_{n \toINF} b\) is independent of the index \(n\).)

This means that when one wants to add or multiply two rational numbers \(a, b\) it does \emph{not} matter whether one thinks of these numbers as \emph{rationals} or as the \emph{real} numbers \(\LIM_{n \toINF} a, \LIM_{n \toINF} b\).
Also, this identification of rational numbers and real numbers is consistent with our definitions of equality(\EXEC{5.3.3}).
\end{note}

We can now easily define negation \(-x\) for real numbers \(x\) by the formula
\[
    -x := (-1) \X x,
\]
since \(-1\) is a rational number and is hence real.
(This definition is dependent on \DEF{5.3.9}, and we give it a label: \DEF{5.3.18}.)
Note that this is clearly consistent with our \emph{negation for rational} numbers since we have \(-q = (-1) \X q\) (by \AC{4.2.3}) for all rational numbers \(q\).
Also, from our definitions it is clear (why? see below) that
\begin{align*}
      & -(\LIM_{n \toINF} a_n) \\
    = & (-1) \X (\LIM_{n \toINF} a_n) & \text{by \DEF{5.3.18}, negation} \\
    = & \LIM_{n \toINF} (-1) \X \LIM_{n \toINF} a_n & \text{by equating rational and real} \\
    = & \LIM_{n \toINF} (-1) a_n & \text{by \DEF{5.3.9}} \\
    = & \LIM_{n \toINF} (-a_n)  & \text{by \AC{4.2.3}}
\end{align*}

Once we have addition and negation, we can define subtraction as usual by
\[
    x - y = x + (-y),
\]
(We label it as \DEF{5.3.19}.) Note that
\begin{align*}
      & \LIM_{n \toINF} a_n - \LIM_{n \toINF} b_n \\
    = & \LIM_{n \toINF} a_n + (-\LIM_{n \toINF} b_n) & \text{by \DEF{5.3.19}} \\
    = & \LIM_{n \toINF} a_n + \LIM_{n \toINF} (-b_n) & \text{shown above} \\
    = & \LIM_{n \toINF} (a_n + (-b_n)) & \text{by \DEF{5.3.4}} \\
    = & \LIM_{n \toINF} (a_n - b_n) & \text{by \DEF{4.2.13}, subtraction of rationals}
\end{align*}

We can now easily show that the \emph{real} numbers obey all the usual rules of algebra (\emph{except} perhaps for the laws involving \emph{division}
(since we currently have not defined the \emph{reciprocal} of the \emph{reals}), which we shall address shortly):

\begin{proposition} \label{prop 5.3.11}
All the laws of algebra from \PROP{4.1.6} (not \PROP{4.2.4}, not defining division yet) hold not only for the integers, but for the reals as well.
\end{proposition}

\begin{proof}
(The proof use the algebra laws of \emph{rationals}.)
We illustrate this with one such rule: \(x(y + z) = xy + xz\).
Let \(x = \LIM_{n \toINF} a_n\), \(y = \LIM_{n \toINF} b_n\),
and \(z = \LIM_{n \toINF} c_n\) be real numbers.
Then by \DEF{5.3.9}, \(xy = \LIM_{n \toINF} a_n b_n\) and \(xz = \LIM_{n \toINF} a_n c_n\),
and so by \DEF{5.3.4} \(xy + xz = \LIM_{n \toINF} (a_n b_n + a_n c_n)\).
A similar line of reasoning shows that \(x(y + z) = \LIM_{n \toINF} a_n (b_n + c_n)\).
But we already know that \(a_n (b_n + c_n)\) is equal to \(a_n b_n + a_n c_n\) for the \emph{rational} numbers \(a_n, b_n, c_n\), and the claim follows.
The other laws of algebra are proven similarly.
\end{proof}

The last basic arithmetic operation we need to define is reciprocation: \(x \to x^{-1}\).
This one is a little \emph{more subtle}.
On obvious first guess for how to proceed would be define
\[
    (\LIM_{n \toINF} a_n)^{-1} := \LIM_{n \toINF} a^{-1},
\]
but there are \emph{a few problems} with this.
For instance, let \(a_1, a_2, a_3,...\) be the \emph Cauchy sequence
\[
    0.1, 0.01, 0.001, 0.0001,...,
\]
and let \(x := \LIM_{n \toINF} a_n\).
Then by this definition, \(x^{-1}\) would be \(\LIM_{n \toINF} b_n\), where \(b_1, b_2, b_3,...\) is the sequence
\[
    10, 100, 1000, 10000,...
\]
but this is \emph{not} a Cauchy sequence (it isn't even bounded).
Of course, the problem here is that our original Cauchy sequence \((a_n)_{n = 1}^{\infty}\) was equivalent to the \emph{zero sequence} \((0)_{n = 1}^{\infty}\) (why? \MAROON{(1)}),
and hence that our real number \(x\) was in fact equal to \(0\).
So we should only allow the operation of reciprocal \emph{when \(x\) is non-zero}.

\begin{proof}
\MAROON{(1)} Suppose for the sake of contradiction, \((a_n)_{n = 1}^{\infty}\) was \emph{not} equivalent to \((0)_{n = 1}^{\infty}\).
Then by \DEF{5.2.6}, there exists \(\varE > 0\) s.t. \textbf{for all \(N \ge 1\)} there exists \(i \ge N\) s.t. \(\abs{a_i - 0} = \abs{a_i} = \abs{10^{-i}} > \varE\), or \(10^{-i} > \varE\), or \(10^i < 1/\varE\).
But since \(10^x\) an increasing function, and \(i \ge N\), we have \(10^N \le 10^i\), so we have \(10^N \le 1/\varE\).
And by \EXEC{4.3.5}, we have \(N \le 10^N\), wo we have \(N \le 1/\varE\).
So we conclude for all \(N \ge 1\), \(N \le 1/\varE\), which contradicts \PROP{4.4.1} that we can always find a natural number \(N\) s.t. \(N \ge 1/\varE\).
So the two sequences must be equivalent.
\end{proof}

However, even when we restrict ourselves to non-zero real numbers, we have a slight problem, because a non-zero real number might be the formal limit of a Cauchy sequence which \emph{contains zero elements}.
For instance, the number \(1\), which is rational and hence real, is the formal limit \(1 = \LIM_{n \toINF} a_n\) of the Cauchy sequence
\[
	0, 0.9, 0.99, 0.999, 0.9999,...
\]
(It can be proved that \(0, 0.9, 0.99,...\) and \(1, 1, 1, ...\) is equivalent.) but using our naive definition of reciprocal, we \emph{cannot} invert the real number \(1\), because we can’t invert the first element \(0\) of this Cauchy sequence!

To get around these problems we need to \emph{keep our Cauchy sequence away from zero}.
To do this we first need a definition.

\begin{definition} [Sequences bounded away from zero]  \label{def 5.3.12}
A sequence \((a_n)_{n = 1}^{\infty}\) of rational numbers is said to be \emph{bounded away from zero} iff there exists a \emph{rational} number \(c > 0\) such that \(\abs{a_n} \ge c\) for all \(n \ge 1\).
\end{definition}

\begin{example} \label{example 5.3.13}
The sequence \(1, -1, 1, -1, 1, -1, 1,...\) is bounded away from zero (all the coefficients have absolute value at least 1). 
But the sequence \(0.1, 0.01, 0.001,...\) is not bounded away from zero(proved by contradiction),
and neither is \(\textbf{0}, 0.9, 0.99, 0.999, 0.9999,....\)
The sequence \(10, 100, 1000,...\) is bounded away from zero, but is not bounded.
\end{example}

We now show that every \emph{non-zero} real number is the formal limit of a Cauchy sequence \emph{bounded away from zero}:
\begin{lemma} \label{lem 5.3.14}
Let \(x\) be a \emph{non-zero} real number.
Then \(x = \LIM_{n \toINF} a_n\) for some Cauchy sequence \((a_n)_{n = 1}^{\infty}\) which is \emph{bounded away from zero}.
\end{lemma}

\begin{proof}
Since \(x\) is real, we know that \(x = \LIM_{n \toINF} b_n\) for some Cauchy sequence \((b_n)_{n = 1}^{\infty}\).
But we are not yet done, because we do not know that \((b_n)_{n = 1}^{\infty}\) is bounded away from zero.

On the other hand, we are given that \(x \ne 0 = \LIM_{n \toINF} 0\), which means that the sequence \((b_n)_{n = 1}^{\infty}\) is \textbf{not} equivalent to \((0)_{n = 1}^{\infty}\).
Thus by \DEF{5.2.6} the sequence \((b_n)_{n = 1}^{\infty}\) \emph{cannot} be eventually \(\varE\)-close to \((0)_{n = 1}^{\infty}\) for every \(\varE > 0\).
Therefore we \emph{can find} an \(\varE >  0\) such that \((b_n)_{n = 1}^{\infty}\) is not eventually \(\varE\)-close to \((0)_{n = 1}^{\infty}\).

So let arbitrary \(\varE > 0\).
Since \((b_n)_{n = 1}^{\infty}\) is Cauchy, it is eventually \(\varE\)-steady.
Moreover, since \(\varE/2\) also \(> 0\) so \((b_n)_{n = 1}^{\infty}\) is eventually \(\varE/2\)-steady.
Thus there is an \(N \ge 1\) such that \(\abs{b_n - b_m} \le \varE/2\) for all \(n, m \ge N\) \MAROON{(1)}.
On the other hand, we cannot have \(\abs{b_n - 0} = \abs{b_n} \le \varE\) for all \(n \ge N\), 
since this would imply that \((b_n)_{n = 1}^{\infty}\) is eventually \(\varE\)-close to \((0)_{n = 1}^{\infty}\).
Thus there must be some \(n_0 \ge N\) for which \(\abs{b_{n_0} - 0} = \abs{b_{n_0}} > \varE\) \MAROON{(2)}.
Now from \MAROON{(1)}, in particular we have \(\abs{b_{n_0} - b_n} \le \varE/2\) for all \(n \ge N\),
and from the triangle inequality:
\begin{align*}
	         & \abs{b_{n_0} - b_n} \le \varE/2 \\
    \implies & -\abs{b_{n_0} - b_n} \ge -\varE/2 \\
    \implies & \abs{b_{n_0}} - \abs{b_{n_0} - b_n} \ge \abs{b_{n_0}} - \varE/2 \\
    \implies & \abs{b_{n_0}} - \abs{b_{n_0} - b_n} \ge \abs{b_{n_0}} - \varE/2 \ge \varE - \varE/2 = \varE/2 & \text{by \MAROON{(2)}}\\
    \implies & \abs{b_{n_0}} - \abs{b_{n_0} - b_n} \ge \varE/2 \\
    \implies & \abs{b_{n_0}} - \abs{b_n - b_{n_0}} \ge \varE/2 \\
    \implies & \abs{b_{n_0} + (b_n - b_{n_0})} \ge \abs{b_{n_0}} - \abs{b_n - b_{n_0}} \ge \varE/2 & \text{by \AC{4.3.1}} \\
    \implies & \abs{b_{n_0} + (b_n - b_{n_0})} \ge \varE/2 \\
    \implies & \abs{b_n} \ge \varE/2
\end{align*}
So \(\abs{b_n} \ge \varE/2\) for all \(n \ge N\).

This almost proves that \((b_n)_{n = 1}^{\infty}\) is bounded away from zero.
Actually, what it does is show that \((b_n)_{n = 1}^{\infty}\) is \emph{eventually} bounded away from zero.
But this is easily fixed, by \emph{defining a new sequence} \(a_n\), by setting \(a_n := \varE/2\) if \(n < N\) and \(a_n := b_n\) if \(n \ge N\).
Since \(b_n\) is a Cauchy sequence, it is not hard to verify that \(a_n\) is also a Cauchy sequence which is \emph{equivalent} to \(b_n\) (because the two sequences are eventually the same),
and so \(x = \LIM_{n \toINF} a_n\).
And since \(\abs{b_n} \ge \varE/2\) for all \(n \ge \textbf{N}\), we know that \(\abs{a_n} \ge \varE/2\) for all \(n \ge \textbf{1}\)
(splitting into the two cases \(n \ge N\) and \(n < N\) separately).
Thus we have a Cauchy sequence \(a_n\) which is bounded away from zero (by \(\varE/2\) instead of \(\varE\), but that’s still OK since \(\varE/2 > 0\)) and which has \(x\) as a formal limit, and so we are done.
\end{proof}

Once a sequence is bounded away from zero, we can take its \emph{reciprocal} without any difficulty:

\begin{lemma} \label{lem 5.3.15}
Suppose that \((a_n)_{n = 1}^{\infty}\) is a Cauchy sequence which is bounded away from zero.
Then the sequence \((a_n^{-1})_{n = 1}^{\infty}\) is also a Cauchy sequence.
\end{lemma}

\begin{proof}
Since \((a_n)_{n = 1}^{\infty}\) is bounded away from zero, by \DEF{5.3.12} we know that there is a \(c > 0\) such that \(\abs{a_n} \ge c\) for all \(n \ge 1\).

Now we need to show that \((a_n^{-1})_{n = 1}^{\infty}\) is eventually \(\varE\)-steady for each \(\varE > 0\).
Thus let arbitrary \(\varE > 0\);
our task is now to find an \(N \ge 1\) such that \(\abs{a_n^{-1} - a_m^{-1}} \le \varE\) for all \(n, m \ge N\).
But
\begin{align*}
    \abs{a_n^{-1} - a_m^{-1}} & = \abs{\frac1{a_n} - \frac1{a_m}} \\
                              & = \abs{\frac{a_m - a_n}{a_n a_m}} \\
                              & = \frac{\abs{a_m - a_n}}{\abs{a_n a_m}} \\
                              & = \frac{\abs{a_m - a_n}}{\abs{a_n}\abs{a_m}} \\
                              & \le \frac{\abs{a_m - a_n}}{c^2} & \text{since \(\abs{a_m}, \abs{a_n} \ge c\)}
\end{align*}
and so to make \(\abs{a_n^{-1} - a_m^{-1}}\) less than or equal to
\(\varE\), it will suffice to make \(\abs{a_m - a_n}\) less than or equal to \(c^2\varE\).

But since \((a_n)_{n = 1}^{\infty}\) is a Cauchy sequence, and \(c^2\varE > 0\), we can certainly find an \(N\) such that the sequence \((a_n)_{n = N}^{\infty}\) is \(c^2\varE\)-steady, 
i.e., \(\abs{a_m - a_n} \le c^2\varE\) for all \(m, n \ge N\).
By what we have said above, this shows that \(\abs{a_n^{-1} - a_m^{-1}} \le \varE\) for all \(m, n \ge N\), and hence the sequence \((a_n^{-1})_{n = 1}^{\infty}\) is eventually \(\varE\)-steady.
Since we have proven this for arbitrary \(\varE > 0\), we have that \((a_n^{-1})_{n = 1}^{\infty}\) is a Cauchy sequence, as desired.
\end{proof}

We are now ready to define reciprocation:
\begin{definition} [Reciprocals of \emph{real} numbers] \label{def 5.3.16}
Let \(x\) be a \emph{non-zero} real number.
Let \((a_n)_{n = 1}^{\infty}\) be a Cauchy sequence bounded away from zero such that \(x = \LIM_{n \toINF} a_n\)
(such a sequence exists by \LEM{5.3.14}).
Then we define the reciprocal \(x^{-1}\) by the formula \(x^{-1} := \LIM_{n \toINF} a_n^{-1}\).
(From \LEM{5.3.15} we know that \(x^{-1}\) is still a real number.)
\end{definition}

And we need to check that the reciprocals of the two different but \emph{equivalent}(in the sense of \DEF{5.2.6}) Cauchy sequences are still equivalent.

\begin{lemma} [Reciprocation is well defined] \label{lem 5.3.17}
Let \((a_n)_{n = 1}^{\infty}\) and \((b_n)_{n = 1}^{\infty}\) be two Cauchy sequences bounded away from zero such that \(\LIM_{n \toINF} a_n = \LIM_{n \toINF} b_n\)
(i.e., the two sequences are \emph{equivalent}).
Then \(\LIM_{n \toINF} a_n^{-1} = \LIM_{n \toINF} b_n^{-1}\).
\end{lemma}

\begin{proof}
Consider the following product \(P\) of three real numbers:
\[
    P := (\LIM_{n \toINF} a_n^{-1}) \X \BLUE{(\LIM_{n \toINF} a_n)} \X (\LIM_{n \toINF} b_n^{-1}).
\]
If we multiply (by \DEF{5.3.9}) this out, we obtain
\[
    P = \LIM_{n \toINF} a_n^{-1} a_n b_n^{-1} = \LIM_{n \toINF} b_n^{-1}.
\]
On the other hand, since \BLUE{\(\LIM_{n \toINF} a_n = \LIM_{n \toINF} b_n\)}, by \PROP{5.3.10}(multiplication is well-defined), we can write \(P\) in another way as
\[
     P = (\LIM_{n \toINF} a_n^{-1}) \X \BLUE{(\LIM_{n \toINF} b_n)} \X (\LIM_{n \toINF} b_n^{-1}).
\]
Multiplying things out again, we get
\[
    P = \LIM_{n \toINF} a_n^{-1} b_n b_n^{-1} = \LIM_{n \toINF} a_n^{-1}.
\]
Comparing our different formulae for \(P\) we see that \(\LIM_{n \toINF} a_n^{-1} = \LIM_{n \toINF} b_n^{-1}\), as desired.
\end{proof}

Thus reciprocal is well-defined (for each non-zero real number \(x\), we have \emph{exactly one definition of} the reciprocal \(x^{-1}\)).

\begin{note}
Note it is clear from the definition that \(xx^{-1} = x^{-1}x = 1\) (from the commutative law in \PROP{5.3.11} we only have to show \(xx^{-1} = 1\)):
\begin{align*}
    xx^{-1} & = \LIM_{n \toINF} a_n \LIM{n \toINF} a_n^{-1} \\
            & = \LIM_{n \toINF} a_n a_n^{-1} & \text{by \DEF{5.3.9}} \\
            & = \LIM_{n \toINF} 1 & \text{by \PROP{4.2.4}(10)} \\
            & = 1 & \text{by equating real to rational}
\end{align*}
Thus with this property, all the \emph{field} axioms (\PROP{4.2.4}) apply to the \emph{reals} as well as to the rationals.
\end{note}

\begin{note}
We of course cannot give \(0\) a reciprocal, since \(0\) multiplied by anything gives \(0\), not \(1\). 
\end{note}

\begin{note}
Also note that if \(q\) is a non-zero \emph{rational}, and hence equal to the real number \(\LIM_{n \toINF} q\),
then the reciprocal of \(\LIM_{n \toINF} q\) is \(\LIM_{n \toINF} q^{-1} = q^{-1}\);
That is,
\begin{align*}
    q^{-1} & = (\LIM_{n \toINF} q)^{-1} & \text{by equating rational and real} \\
           & = \LIM_{n \toINF} q^{-1} & \text{by \DEF{5.3.16}} \\
           & = q^{-1} & \text{by equating rational and real}
\end{align*}
thus the operation of \emph{reciprocal on real} numbers is consistent with the operation of \emph{reciprocal on rational} numbers.
\end{note}

Once one has reciprocal, one can define \emph{division} \(x/y\) of two real numbers \(x, y\), provided \(y\) is non-zero, by the formula
\[
    x/y := x \X y^{-1},
\]
just as we did with the rationals. (We label it as \DEF{5.3.20}.)
In particular, we have the cancellation law (label as \LEM{5.3.21}): if \(x, y, z\) are real numbers such that \(xz = yz\), and \(z\) is non-zero, then by dividing by \(z\) we conclude that \(x = y\).
Note that this cancellation law does not work when \(z\) is zero.

We now have all four of the basic arithmetic operations on the reals: addition, subtraction, multiplication, and division, with all the usual
rules of algebra.

\begin{definition} \label{def 5.3.18}
We define negation \(-x\) for real numbers \(x\) by the formula
\[
    -x := (-1) \X x,
\]
\end{definition}

\begin{definition} \label{def 5.3.19}
We define subtraction for real numbers \(x - y\) by the formula
\[
    x - y := x + (-y).
\]
\end{definition}

\begin{definition} \label{def 5.3.20}
We define division for real numbers \(x/y\), provided \(y \ne 0\), by the formula
\[
    x/y := x \X y^{-1}.
\]
\end{definition}

\begin{lemma} [Cancellation law for real number] \label{lem 5.3.21}
If \(x, y, z\) are real numbers such that \(xz = yz\), and \(z\) is non-zero, then by dividing by \(z\) we conclude that \(x = y\).
\end{lemma}

\exercisesection

\begin{exercise} \label{exercise 5.3.1}
Prove \PROP{5.3.3}.
\end{exercise}

\begin{proof}
See \PROP{5.3.3}.
\end{proof}

\begin{exercise} \label{exercise 5.3.2}
Prove \PROP{5.3.10}.
\end{exercise}

\begin{proof}
See \PROP{5.3.10}.
\end{proof}

\begin{exercise} \label{exercise 5.3.3}
Let \(a, b\) be \emph{rational} numbers.
Show that \(a = b\) if and only if \(\LIM_{n \toINF} a = \LIM_{n \toINF} b\)
(i.e., the Cauchy sequences \(a, a, a, a, ...\) and \(b, b, b, b, ...\) are equivalent if and only if \(a = b\)).
This allows us to \emph{embed the rational numbers inside the real numbers in a well-defined manner}.
\end{exercise}

\begin{proof}
\begin{align*}
         & a = b \\
    \iff & a - b = 0 \\
    \iff & \abs{a - b} = 0 \\
    \iff & \abs{a - b} < \varE\ \forall \varE > 0 \\
    \iff & \abs{a - b} < \varE\ \forall \varE > 0, \forall i \ge 1 & \text{just give a dummy index} \\
    \iff & (a)_{n = 1}^{\infty} = (b)_{n = 1}^{\infty} & \text{by \DEF{5.2.6}} \\
    \iff & \LIM_{n \toINF} a_n = \LIM_{n \toINF} b_n & \text{by \DEF{5.3.1}}
\end{align*}
\end{proof}

\begin{exercise} \label{exercise 5.3.4}
Let \((a_n)_{n = 0}^{\infty}\) be a sequence of rational numbers which is bounded.
Let \((b_n)_{n = 0}^{\infty}\) be another sequence of rational numbers which is \emph{equivalent} to \((a_n)_{n = 0}^{\infty}\).
Show that \((b_n)_{n = 0}^{\infty}\) is also bounded.
(Hint: use \EXEC{5.2.2}.)
\end{exercise}

\begin{proof}
Since \((a_n)_{n = 0}^{\infty}\) and \((b_n)_{n = 0}^{\infty}\) are equivalent, they are eventually \(\varE\)-closes for all \(\varE > 0\).
And since \((a_n)_{n = 0}^{\infty}\) is bounded, the conditions in the \EXEC{5.2.2} are satisfied, and by \EXEC{5.2.2}, \((b_n)_{n = 0}^{\infty}\) is bounded.
\end{proof}

\begin{note}
若\ seq A 是\ bounded,且\ seq B 等價於\ seq A,則\ seq B 也是\ bounded。
\end{note}

\begin{exercise} \label{exercise 5.3.5}
Show that \(\LIM_{n \toINF} 1 / n = 0\).
\end{exercise}

\begin{proof}
We have to show \(\LIM_{n \toINF} 1 / n = 0 = \LIM_{n \toINF} 0\).
Suppose for the sake of contradiction that \(\LIM_{n \toINF} 1 / n \ne \LIM_{n \toINF} 0\).
Then by \DEF{5.2.6}, There exists \(\varE > 0\), such that \textbf{for all \(N \ge 1\)}, there exists \(i \ge N\) s.t. \(\abs{1/i - 0} > \varE\).
And \(\abs{1/i - 0} = \abs{1/i}\) is positive, so we have \(1/i > \varE\), or \(i < 1/\varE\).
But since \(i \ge N\), we have \(N < 1/\varE\).
So we have for all \(N \ge 1\), \(N < 1/\varE\), which contradicts \PROP{4.4.1} that we can always find a natural number \(N\) s.t. \(N > 1/\varE\).
So \(\LIM_{n \toINF} 1 / n = 0\).
\end{proof}
\section{Ordering the reals} \label{sec 5.4}

Since a real number \(x\) is just a formal limit of rationals \(a_n\), it is tempting to make the following definition:
a real number \(x = \LIM{n \toINF} a_n\) is positive if \emph{all of} the \(a_n\) are positive,
and negative if all of the \(a_n\) are negative (and zero if all of the \(a_n\) are zero).
However, one soon realizes some problems with this definition. For instance, the sequence \((a_n)_{n = 1}^{\infty}\) defined by \(a_n := 10^{-n}\), thus
\[
    0.1, 0.01, 0.001, 0.0001,...
\]
consists \emph{entirely of positive} numbers, but this sequence is equivalent to the zero sequence \(0, 0, 0, 0,...\) and thus \(\LIM_{n \toINF} a_n = 0\).
Thus even though all the rationals were positive, the real formal limit of these rationals was zero rather than positive.

Another example is
\[
    0.1, -0.01, 0.001, -0.0001,...;
\]
this sequence is a \emph{hybrid} of positive and negative numbers, but again the formal limit is zero.

The trick, as with the reciprocals in the previous section, is to \emph{limit one’s attention to sequences which are bounded away from zero.}.

\begin{definition} \label{def 5.4.1}
Let \((a_n)_{n = 1}^{\infty}\) be a sequence of rationals.
We say that this sequence is \emph{positively bounded away} from zero iff we have a positive rational \(c > 0\) such that \(a_n \ge c\) for all \(n \ge 1\)
(in particular, the sequence is entirely positive).
The sequence is \emph{negatively bounded away} from zero iff we have a negative rational \(-c < 0\) such that \(a_n \le -c\) for all \(n \ge 1\).
(in particular, the sequence is entirely negative).
\end{definition}

\begin{note}
You can compare \DEF{5.3.12} with \DEF{5.4.1}.
\end{note}

\begin{example} \label{example 5.4.2}
The sequence \(1.1, 1.01, 1.001, 1.0001,...\) is positively bounded away from zero (all terms are greater than or equal to \(1\)).
The sequence \(-1.1, -1.01, -1.001, -1.0001,...\) is negatively bounded away from zero.
The sequence \(1, -1, 1, -1, 1, -1,...\) is bounded away from zero, but is neither positively bounded away from zero nor negatively bounded away from zero.
\end{example}

\begin{note}
It is clear that any sequence which is positively or negatively bounded away from zero, is bounded away from zero.
Also, a sequence cannot be both positively bounded away from zero and negatively bounded away from zero at the same time
(or it would contradict with the trichotomy law of \emph{rational} numbers).
\end{note}

\begin{definition} \label{def 5.4.3}
A real number \(x\) is \emph{said} to be \emph{positive} iff it can be written as \(x = \LIM_{n \toINF} a_n\) for some Cauchy sequence \((a_n)_{n = 1}^{\infty}\) which is positively bounded away from zero.
\(x\) is said to be \emph{negative} iff it can be written as \(x = \LIM_{n \toINF} a_n\) for some sequence \((a_n)_{n = 1}^{\infty}\) which is negatively bounded away from zero.
\end{definition}

\begin{proposition} [Basic properties of positive reals] \label{prop 5.4.4}
For every real number \(x\), exactly one of the following three statements is true:
\begin{enumerate}
    \item \(x\) is zero
    \item \(x\) is positive
    \item \(x\) is negative.
\end{enumerate}
A real number \(x\) is negative if and only if \(-x\) is positive. 
If \(x\) and \(y\) are positive, then so are \(x + y\) and \(xy\).
\end{proposition}

\begin{proof}
First we show the trichotomy property.

We show given any real \(x\), at least one of \(x = 0\), \(x\) is positive, \(x\) is negative is true.
If \(x = 0\), then \(x = 0\) of course is true, so we suppose \(x \neq 0\).
Since \(x \neq 0\), by \LEM{5.3.14} \(x\) is the formal limit of some Cauchy sequence \((a_n)_{n = 1}^{\infty}\) which is bounded away from zero.
So by \DEF{5.3.12} there exist \(c > 0\) s.t. \(\abs{a_n} \ge c\) for all \(n \ge 1\) \MAROON{(1)}.
And since \((a_n)_{n = 1}^{\infty}\) is Cauchy, given \(c/2\), there exists \(N \ge 1\) s.t. \(\abs{a_i - a_j} \le c/2\) for all \(i, j \ge N\).
And in particular for \(j = N\) we have \(\abs{a_i - a_N} \le c/2\) for all \(i \ge N\).
That is, by \PROP{4.3.3}(c),
\[
    -c/2 \le a_i - a_N \le c/2 \text{\ for all\ } i \ge N \MAROON{\ (2)}
\]
Now we consider the value of \(a_N\).
By \MAROON{(1)} we have \(a_N \neq 0\), and by trichotomy law of rationals \(a_N\) is either positive or negative.
\begin{itemize}
    \item
        If \(a_N\) is positive, then \(\abs{a_N} = a_N\).
        And since \(\abs{a_N} \ge c\), we have \(a_N \ge c\).
        And with \MAROON{(2)}, we have
        \begin{align*}
                     & -c/2 \le a_i - a_N \\
            \implies & -c/2 \le a_i - a_N \le a_i - c & \text{since \(a_N \ge c\)} \\
            \implies & -c/2 \le a_i - c \\
            \implies & -c/2 + c \le a_i - c + c \\
            \implies & c/2 \le a_i
        \end{align*}
        So \(a_i \ge c/2 > 0\) for all \(i \ge N\).
        This almost shows that \((a_n)_{n = 1}^{\infty}\) is \emph{positively} bounded away from zero.
        With the similar argument from \LEM{5.3.14} we can define another sequence \((b_n)_{n = 1}^{\infty}\), where if \(n < N\) then we let \(b_n := 2/c\), else we let \(b_n := a_n\).
        Then it's clear that \((a_n)_{n = 1}^{\infty}\) and \((b_n)_{n = 1}^{\infty}\) are equivalent since they are eventually the same, and the latter is positively bounded away from zero.
        So \(x\) is also equal to the formal limit of a Cauchy sequence \((b_n)_{n = 1}^{\infty}\) which is positively bounded away from zero, and so by \DEF{5.4.3} is positive.
    \item
        If \(a_N\) is negative, then \(\abs{a_N} = -a_N\).
        And since \(\abs{a_N} \ge c\), we have \(-a_N \ge c\), so \(a_N \le -c\).
        And with \MAROON{(2)}, we have
        \begin{align*}
                     & (a_i - a_N) \le 2/c \\
            \implies & a_i \le 2/c + a_N \\
            \implies & a_i \le 2/c + a_N \le 2/c + (-c) & \text{since \(a_N \le -c\)} \\
            \implies & a_i \le -(2/c)
        \end{align*}
        So \(a_i \le -(c/2) < 0\) for all \(i \ge N\).
        Again, this almost shows that \((a_n)_{n = 1}^{\infty}\) is \emph{negatively} bounded away from zero.
        With the similar argument from \LEM{5.3.14} we can define another sequence \((b_n)_{n = 1}^{\infty}\), where if \(n < N\) then we let \(b_n := -2/c\), else we let \(b_n := a_n\).
        Then it's clear that \((a_n)_{n = 1}^{\infty}\) and \((b_n)_{n = 1}^{\infty}\) are equivalent sine they are eventually the same, and the latter is negatively bounded away from zero.
        So \(x\) is also equal to the formal limit of a Cauchy sequence \((b_n)_{n = 1}^{\infty}\) which is negatively bounded away from zero, and so by \DEF{5.4.3} is negative.
\end{itemize}
So in all cases at least one of \(x = 0\), \(x\) is positive, \(x\) is negative is true.

Then we prove at most one of the statement is true.
That is, (1) \(x = 0\), \(x\) is positive cannot both be true (2) \(x = 0\), \(x\) is negative cannot both be true, and (3) \(x\) is positive and \(x\) is negative cannot both be true.
Let \(x\) be the formal limit of some Cauchy sequence \((a_n)_{n = 1}^{\infty}\). Then
\begin{itemize}
    \item [(1) is true:] Then by \DEF{5.4.3}, \((a_n)_{n = 1}^{\infty}\) is positively bounded away from zero, so trivially is bounded away from zero;
    and \((a_n)_{n = 1}^{\infty}\) also equals to 0, which is impossible by \DEF{5.3.12}.
    \item [(2) is true:] Then by \DEF{5.4.3}, \((a_n)_{n = 1}^{\infty}\) is negatively bounded away from zero, so trivially is bounded away from zero, which is impossible by the same argument in the previous case.
    \item [(3) is true:] Then we have that for some \(c_1 > 0\), \(a_n \ge c_1\) for all \(n \ge 1\) and that for some \(-c_2 < 0\), \(a_n \ge -c_2\) for all \(n \ge 1\).
    In particular we have \(a_1 > c_1 > 0\) and \(a_1 < c_2 < 0\), which contradicts the trichotomy property for rationals.
\end{itemize}
So in all cases, exactly one of the three statements is true.

Now we show \(x\) is negative iff \(-x\) is positive.
Then \(x\) is negative, iff (by \DEF{5.4.3}) \(x\) is the formal limit of the sequence \((a_n)_{n = 1}^{\infty}\), which is negatively bounded away from zero, iff (by \DEF{5.4.1}) there exists \(-c_1 < 0\) s.t. \(a_i < -c_1\) for all \(i > 1\), iff (by algebra) there exists \(c_1 > 0\) s.t. \(-a_i > c_1\) for all \(i > 1\), iff (by \DEF{5.4.1}) \((-a_n)_{n = 1}^{\infty}\) is positively bounded away from zero, iff (by \DEF{5.3.18}, negation, and \DEF{5.4.3}) \(-x\) is positive.

Now we show if \(x, y\) are positive then \(x + y\) and \(xy\) are positive.
Suppose \(x, y\) are positive, the by \DEF{5.4.3} \(x, y\) are the formal limit of some Cauchy sequence \((a_n)_{n = 1}^{\infty}\), \((b_n)_{n = 1}^{\infty}\) which are positively bounded away from zero.
For addition, by \DEF{5.3.4}, \(x + y\) is equal to the formal limit of the Cauchy sequence \((a_n + b_n)_{n = 1}^{\infty}\)
And by \DEF{5.4.1}, there are \(c_1, c_2 > 0\) s.t. \(a_i \ge c_1\) and \(b_i \ge c_2\) for all \(i \ge 1\).
Then it's clear that \(a_i + b_i \ge c_1 + c_2\) for all \(i \ge 1\), where \(c_1 + c_2 > 0\).
So again by \DEF{5.4.1}, \((a_n + b_n)_{n = 1}^{\infty}\) is positively bounded away from zero.
By \DEF{5.4.3}, \(x + y\) is positive.

Now for multiplication, by \DEF{5.3.9}, \(xy\) is equal to the formal limit of the Cauchy sequence \((a_n b_n)_{n = 1}^{\infty}\)
And it's clear that \(a_i b_i \ge c_1 c_2\) for all \(i \ge 1\), where \(c_1 c_2 > 0\).
So again by \DEF{5.4.1}, \((a_n b_n)_{n = 1}^{\infty}\) is positively bounded away from zero.
By \DEF{5.4.3}, \(xy\) is positive.
\end{proof}

\begin{note}
If \(q\) is a positive rational number, then the Cauchy sequence \(q, q, q, ...\) is positively bounded away from zero, and hence \(\LIM_{n \toINF} q = q\) is a positive real number.
Thus the notion of positivity for \emph{rationals} is \emph{consistent with} that for \emph{reals}.
Similarly, the notion of negativity for rationals is consistent with that for reals.
\end{note}

\begin{definition} [Absolute value] \label{def 5.4.5}
Let \(x\) be a real number.
We define the absolute value \(\abs{x}\) of \(x\) to equal \(x\) if \(x\) is positive, \(-x\) when \(x\) is negative, and \(0\) when \(x\) is zero.
\end{definition}

\begin{definition} [Ordering of the real numbers] \label{def 5.4.6}
Let \(x\) and \(y\) be real numbers.
We say that \(x\) is greater than \(y\), and write \(x > y\), iff \(x - y\) is a positive real number, and \(x < y\) iff \(x - y\) is a negative real number.
We define \(x \ge y\) iff \(x > y\) or \(x = y\), and similarly define \(x \le y\).
\end{definition}

\begin{note}
With the \emph{consistent behavior} of positivity and negativity between rational and real numbers,
comparing this with the definition of order on the rationals from \DEF{4.2.8}
we see that \emph{order on the reals} is consistent with \emph{order on the rationals},
i.e., if two rational numbers \(q, q'\) are such that \(q\) is less than \(q'\) in the \emph{rational} number system,
then \(q\) is still less than \(q'\) in the \emph{real} number system,
and similarly for ``greater than''.
In the same way we see that the definition of absolute value given here is consistent with that of rationals in \DEF{4.3.1}.
\end{note}

\begin{proposition} \label{prop 5.4.7}
All the claims in \PROP{4.2.9} which held for \emph{rationals}, continue to hold for \emph{real} numbers.
\end{proposition}

\begin{proof}
\begin{enumerate}
    \item (Order trichotomy) Exactly one of the three statements \(x = y\), \(x < y\), or \(x > y\) is true.

        Let \(a = x - y\). Then by \PROP{5.4.4}, exactly one of \(a = 0\), \(a\) is positive, \(a < 0\) is true.
        \begin{itemize}
            \item [>>] \(a = 0\):
                Then we have \(x - y = 0\), which implies \(x = y\) since \(\SET{R}\) is a field.
                It's trivial that \(x > y\) or \(x < y\) would contradict \(a = 0\).
            \item [>>] \(a > 0\):
                Then we have \(x - y > 0\), which implies \(x > y\) since \(\SET{R}\) is a field.
                It's trivial that \(x = y\) or \(x < y\) would contradict \(a > 0\).
            \item [>>] \(a < 0\):
                Then we have \(x - y < 0\), which implies \(x < y\) since \(\SET{R}\) is a field.
                It's trivial that \(x > y\) or \(x = y\) would contradict \(a < 0\).
        \end{itemize}
        So in all cases, exactly one of the statements is true.
    \item (Order is \emph{anti-symmetric}) One has \(x < y\) if and only if \(y > x\).

        \begin{align*}
                 & x < y \\
            \iff & x - y \text{ is negative} & \text{by \DEF{5.4.6}} \\
            \iff & -(x - y) \text{ is positive} & \text{by \PROP{5.4.4}} \\
            \iff & y - x \text{ is positive} & \text{by field algebra} \\
            \iff & y > x & \text{by \DEF{5.4.6}}
        \end{align*}
    \item (Order is transitive) If \(x < y\) and \(y < z\), then \(x < z\).

        \begin{align*}
                     & x < y \land y < z \\
            \implies & x - y \text{ is negative} \land y - z \text{ is negative} & \text{by \DEF{5.4.6}} \\
            \implies & (x - y) + (y - z) \text{ is negative} & \text{by \PROP{5.4.4}} \\
            \implies & x - z \text{ is negative} & \text{by field algebra} \\
            \implies & x < z & \text{by \DEF{5.4.6}}
        \end{align*}
    \item (Addition preserves order) If \(x < y\), then \(x + z < y + z\).

        \begin{align*}
                     & x < y \\
            \implies & x - y \text{ is negative} & \text{by \DEF{5.4.6}} \\
            \implies & (x - y) + (z - z) \text{ is negative} & \text{by field algebra} \\
            \implies & (x + z) - (y + z) \text{ is negative} & \text{by field algebra} \\
            \implies & x + z < y + z & \text{by \DEF{5.4.6}}
        \end{align*}
    \item (Positive multiplication preserves order) If \(x < y\) and \(z\) is positive, then \(xz < yz\).

        Suppose we have \(x < y\) and \(z\) a positive real, and want to conclude that \(xz < yz\).
        Since \(x < y\), \(y - x\) is positive by \DEF{5.4.6}, hence by \PROP{5.4.4} we have \((y - x)z = yz - xz\) is positive, hence by \DEF{5.4.6} \(xz < yz\).
\end{enumerate}
\end{proof}

\begin{proposition} \label{prop 5.4.8}
Let \(x\) be a positive real number.
Then \(x^{-1}\) is also positive.
Also, if \(y\) is another positive number and \(x > y\), then \(x^{-1} < y^{-1}\).
\end{proposition}

\begin{proof}
Suppose \(x\) is positive, then \(x^{-1}\) cannot be \(0\) since \(x \X 0 = 0\), but \(xx^{-1} = 1\).
Also, if \(x^{-1}\) is negative, then by \PROP{5.4.4} \(-x^{-1}\) is positive, and
\begin{align*}
             & xx^{-1} = 1 \\
    \implies & -(xx^{-1}) = -1 \\
    \implies & x(-x^{-1}) = -1 & \text{by field algebra}
\end{align*}
which implies the multiplication of positive \(x\) and ``positive'' \(-x^{-1}\) is \(-1\), which is negative, contradicting \PROP{5.4.4}.
So by trichotomy of reals in \PROP{5.4.4}, \(x^{-1}\) must be positive.

Now suppose \(x > y\) \MAROON{(1)} and let \(y\) be positive \MAROON{(2)} as well.
Then by previous discussion we know \(x^{-1}\) \MAROON{(3)} and \(y^{-1}\) are also positive.
Now for the sake of contradiction suppose \(x^{-1} \ge y^{-1}\) \MAROON{(4)}.
Then by \PROP{5.4.7}(e) using \MAROON{(1)(3)} we have \(xx^{-1} > yx^{-1}\) and using \MAROON{(4)(2)} we have \(x^{-1}y \ge y^{-1}y\), or \(yx^{-1} \ge yy^{-1}\).
So together we have \(xx^{-1} > yx^{-1} \ge yy^{-1}\), or \(1 > yx^{-1} \ge 1\), or \(1 > 1\), which is impossible.
So \(x^{-1} < y^{-1}\) must be true.
\end{proof}

Another application is that the laws of \emph{exponentiation} (with \emph{integer} exponent) (\PROP{4.3.12}) that were previously proven for rationals, are also true for reals;
see \SEC{5.6}.

We have already seen that the formal limit of positive rationals need not be positive; it could be zero, as the example \(0.1, 0.01, 0.001,...\) showed.
\textbf{However}, the formal limit of \emph{non-negative} rationals (i.e., rationals that are either positive or zero) is non-negative.

\begin{proposition} [The non-negative reals are \emph{closed}] \label{prop 5.4.9}
Let \(a_1, a_2, a_3,...\) be a Cauchy sequence of \emph{non-negative} rational numbers.
Then \(\LIM_{n \toINF} a_n\) is a \emph{non-negative} real number.
\end{proposition}

\begin{note}
Eventually, we will see a \emph{better explanation} of this fact: 
the set of non-negative reals is \emph{closed}, whereas the set of positive reals is \emph{open}. (Currently I'm totally obscured with this.)
See \SEC{11.4}.
\end{note}

\begin{proof}
We argue by contradiction, and suppose that the real number \(x := \LIM_{n \toINF} a_n\) is a \emph{negative} number.
Then by \DEF{5.4.3}, we have \(x = \LIM_{n \toINF} b_n\) for some Cauchy sequence \((b_n)_{n = 1}^{\infty}\) which is \emph{negatively bounded away} from zero,
i.e., there is a negative rational \(-c < 0\) such that \(b_n \le -c\) for all \(n \ge 1\).
On the other hand, we have \(a_n \ge 0\) for all \(n \ge 1\), by hypothesis.
Thus the numbers \(a_n\) and \(b_n\) are never \(c/2\)-close, since \(c/2 < c \le \abs{a_n - b_n}\).
Thus the sequences \((a_n)_{n = 1}^{\infty}\) and \((b_n)_{n = 1}^{\infty}\) are not eventually \(c/2\)-close.
Since \(c/2 > 0\), this implies that \((a_n)_{n = 1}^{\infty}\) and \((b_n)_{n = 1}^{\infty}\) are \emph{not equivalent}.
But this contradicts the fact that both these sequences have \(x\) as their formal limit.
\end{proof}

\begin{corollary} \label{corollary 5.4.10}
Let \((a_n)_{n = 1}^{\infty}\) and \((b_n)_{n = 1}^{\infty}\) be Cauchy sequences of rationals such that \(a_n \ge b_n\) for all \(n \ge 1\).
Then \(\LIM_{n \toINF} a_n \ge \LIM_{n \toINF} b_n\).
\end{corollary}

\begin{proof}
It's trivial that \(a_n - b_n \ge 0\) for all \(n \ge 1\).
So by \PROP{5.4.9}, \(\LIM_{n \toINF} (a_n - b_n)\) is non-negative.
But by \DEF{5.3.19} (subtraction of reals), that implies \(\LIM_{n \toINF} (a_n - b_n) = \LIM_{n \toINF} a_n - \LIM_{n \toINF} b_n\) is non-negative, that is, \(\LIM_{n \toINF} a_n - \LIM_{n \toINF} b_n\) is positive \(\lor \LIM_{n \toINF} a_n = \LIM_{n \toINF} b_n\).
So by \DEF{5.4.6}, \(\LIM_{n \toINF} a_n \ge \LIM_{n \toINF} b_n\).
\end{proof}

\begin{remark} \label{remark 5.4.11}
Note that \CORO{5.4.10} does \emph{not} work if the \(\ge\) signs in the description of \CORO{5.4.10} are replaced by \(>\):
for instance if \(a_n := 1 + 1 / n\) and \(b_n := 1 - 1/n\), then \(a_n\) is always \emph{strictly} greater than \(b_n\), but the formal limit of \(a_n\) is not greater than the formal limit of \(b_n\), instead they are equal.
\end{remark}

\begin{note}
We now define distance \(d(x, y) := \abs{x - y}\) (It it actually defined as \DEF{6.1.1}) just as we did for the rationals.
In fact, \PROP{4.3.3} and \PROP{4.3.7} hold not only for the rationals, but for the reals; (we label it as \PROP{5.4.16}.)
the proof is \emph{identical}, since the real numbers obey all the laws of algebra and order that the rationals do.
\end{note}

We now observe that while positive \emph{real} numbers can be arbitrarily large or small,
they cannot be larger than all of the positive \emph{integers}, 
or smaller in magnitude than all of the positive \emph{rationals}:
\begin{proposition} [Bounding of reals by rationals] \label{prop 5.4.12}
Let \(x\) be a \emph{positive real} number.
Then there exists a \emph{positive rational} number \(q\) such that \(q \le x\),
and there exists a \emph{positive integer} \(N\) such that \(x \le N\).
\end{proposition}

\begin{proof}
Since \(x\) is a positive real, by \DEF{5.4.3} it is the formal limit of some Cauchy sequence \((a_n)_{n = 1}^{\infty}\) which is \emph{positively} bounded away from zero.
Also, since \(a_n\) is Cauchy, by \LEM{5.1.15}, this sequence is bounded.
Thus we have rationals \(q > 0\) (by ``positively bounded away'') and \(r\) (by ``bounded'') such that \(q \le a_n \le r\) for all \(n \ge 1\).
But by \PROP{4.4.1} we know that there is some \emph{integer} \(N\) such that \(r \le N\);
since \(q\) is positive and \(q \le r \le N\), we see that \(N\) is positive.
Thus \(q \le a_n \le N\) for all \(n \ge 1\).
Applying \CORO{5.4.10} twice, we obtain that \(\LIM_{n \toINF} q \le \LIM_{n \toINF} a_n \le \LIM_{n \toINF} N\), that is, \(q \le x \le N\), as desired.
\end{proof}

\begin{note}
一個正的實數,不管多靠近\ 0,你都能找到一個比它還靠近\ 0 的正有理數。不管它多大,你都能找到一個比它還大的正整數。
\end{note}

\begin{corollary} [Archimedean property] \label{corollary 5.4.13} 
Let \(x\) and \(\varE\) be any \emph{positive} \emph{real} numbers.
Then there exists a positive integer \(M\) such that \(M\varE > x\).
\end{corollary}

\begin{proof}
It's trivial that \(x/\varE\) is positive, and hence by \PROP{5.4.12} there exists a positive integer \(N\) such that \(x/\varE \le N\).
If we set \(M := N + 1\), then \(x/\varE < M\).
Now multiply by \(\varE\).
\end{proof}

\begin{note}
This property is \emph{quite important};
it says that no matter how large \(x\) is and how small \(\varE\) is, if one keeps adding \(\varE\) to itself, one will eventually
overtake \(x\).
\end{note}

\begin{proposition} \label{prop 5.4.14}
Given any two real numbers \(x < y\), we can find a rational number \(q\) such that \(x < q < y\).
This is called \href{https://www.youtube.com/watch?v=A9L6YVtoAsQ}{\(\SET{Q}\) is dense in \(\SET{R}\)}.
\end{proposition}

\begin{proof}
See \EXEC{5.4.5}.
\end{proof}

We have now \emph{completed our construction} of the real numbers.
This number system contains the rationals, and has almost everything that the rational number system has: the arithmetic operations, the laws of algebra, the laws of order.
\emph{However}, we have not yet demonstrated any advantages that the real numbers have over the rationals;
so far, even after much effort, \emph{all we have done is shown that they are at least as good as the rational number system}.
But in the next few sections we show that the real numbers can do more things than rationals:
for example, we can take square roots in a real number system.

\begin{remark} \label{remark 5.4.15}
Up until now, we have \emph{not} addressed the fact that real numbers \emph{can} be expressed using the decimal system.
For instance, the formal limit of
\[
    1.4, 1.41, 1.414, 1.4142, 1.41421,...
\]
is more conventionally represented as the \emph{decimal} \(1.41421 ...\).
We will address this in an Appendix B, but for now let us just remark that \emph{there are some subtleties} in the decimal system,
for instance \(0.9999...\) and \(1.000...\) are in fact the same real number.
\end{remark}

\begin{proposition} \label{prop 5.4.16}
In fact, \PROP{4.3.3} and \PROP{4.3.7} hold not only for the rationals, but for the reals;
the proof is \emph{identical}, since the real numbers obey all the laws of algebra and order that the rationals do.
\end{proposition}

\exercisesection

\begin{exercise} \label{exercise 5.4.1}
Prove \PROP{5.4.4}.
(Hint: if \(x\) is not zero, and \(x\) is the formal limit of some sequence \((a_n)_{n = 1}^{\infty}\), then this sequence cannot be eventually \(\varE\)-close to the zero sequence \((0)_{n = 1}^{\infty}\) for every single \(\varE > 0\).
Use this to show that the sequence \((a_n)_{n = 1}^{\infty}\) is eventually either \emph{positively bounded away} from zero or \emph{negatively bounded away} from zero.)
\end{exercise}

\begin{proof}
See \PROP{5.4.4}.
\end{proof}

\begin{exercise} \label{exercise 5.4.2}
Prove the remaining claims in \PROP{5.4.7}.
\end{exercise}

\begin{proof}
See \PROP{5.4.7}.
\end{proof}

\begin{exercise} \label{exercise 5.4.3}
Show that for every \emph{real} number \(x\) there is \emph{exactly one integer} \(N\) such that \(N \le x < N + 1\).
(This integer \(N\) is called the \emph{integer part} of \(x\), and is sometimes denoted \(N = \FLOOR{x})\).
\end{exercise}

\begin{proof}
Let \(x\) equal to the formal limit of a Cauchy sequence \((a_n)_{n = 1}^{\infty}\).
Then given \(\varE = 1/2\), there exists an integer \(N\) s.t. \(\abs{a_i - a_j} \le 1/2\) for all \(i, j \ge N\).
And in particular fixing \(j = N\), we get \(\abs{a_i - a_N} \le 1/2\) for all \(i \ge N\).
Then by \PROP{5.4.16} (or \PROP{4.3.3}(c)) we have
\[
    -1/2 \le a_i - a_N \le 1/2 \text{ for all } i \ge N.
\]
So we have \(a_N - 1/2 \le a_i\) \MAROON{ (1)} and \(a_i \le a_N + 1/2\) \MAROON{(2)} for all \(i \ge N\).

Now we defined another sequence \((b_n)_{n = 1}^{\infty}\) where \(b_n := a_N\) if \(n < N\) and \(b_n := a_n\) if \(n \ge N\).
Then \((a_n)_{n = 1}^{\infty}\) and \((b_n)_{n = 1}^{\infty}\) are equivalent since they are eventually the same
(similar argument in \LEM{5.3.14}),
so \(x\) is also the formal limit of \((b_n)_{n = 1}^{\infty}\) \MAROON{(3)}.
And by definition of \((b_n)_{n = 1}^{\infty}\), from \MAROON{(1)(2)} we have \(a_N - 1/2 \le b_i \le a_N + 1/2\) \emph{for all \(i \ge \BLUE{1}\)} \MAROON{(4)}.

Also, by \PROP{4.4.1} we can find an \emph{unique} integer \(M\) s.t. \(M \le a_N < M + 1\).
So with \MAROON{(4)} we have \(M - 1/2 \le b_i \le (M + 1) + 1/2\) for all \(i \ge 1\) \MAROON{(5)}.

Since \(M\) is constant, \(M - 1\) and \(M + 2\) are also constant and automatically define Cauchy sequences(that is, \((M)_{n = 1}^{\infty}\) and \((M + 1)_{n = 1}^{\infty}\) are Cauchy).
So from \MAROON{(3)(5)} and \CORO{5.4.10} we have \(M - 1/2 \le x \le M + 3/2\).
And it's trivial that we have \(M - 1 < x < M + 2\) \MAROON{(6)}.

Now by trichotomy \PROP{5.4.7}(a), we have the following cases:
\begin{itemize}
    \item[>>] \(x < M\):
        Then from \MAROON{(6)} we have \(M - 1 < x < M\), and in particular \(M - 1 \le x < M\).
        So we have found an integer \(M' := M - 1\) s.t. \(M' \le x < M' + 1\).
    \item[>>] \(x = M\):
        Then we clear have \(M \le x < M + 1\).
    \item[>>] \(x > M\):
        Then from \MAROON{(6)} we have \(M < x < M + 2\) \MAROON{(7)}.
        Again by \PROP{5.4.7}(a) we have the following cases:
        \begin{itemize}
            \item[>>] \(x < M + 1\):
                Then by \MAROON{(7)} we have \(M < x < M + 1\), in particular \(M \le x < M + 1\).
            \item[>>] \(x \ge M + 1\):
                Then by \MAROON{(7)} we have \(M + 1 \le x < M + 2\).
                So we have found an integer \(M' := M + 1\) s.t. \(M' \le x < M' + 1\).
        \end{itemize}
\end{itemize}
So in all cases we have found the integer as desired.

For the uniqueness part, suppose there is another integer \(N'\) s.t. \(N' \le x < N' + 1\) but \(N' \ne N\).
Then again by \LEM{4.1.11}(f) we must have \(N < N'\) or \(N' < N\):
\begin{itemize}
    \item \(N < N'\):
        \begin{align*}
                     & N < N' \\
            \implies & N + 1 \le N' & \text{since \(N, N'\) are integers} \\
            \implies & x < N + 1 \leq N' \le x \\
            \implies & x < x
        \end{align*}
    \item \(N' < N\):
        \begin{align*}
                     & N' < N \\
            \implies & N' + 1 \le N & \text{since \(N, N'\) are integers} \\
            \implies & x < N' + 1 \le N \le x \\
            \implies & x < x
        \end{align*}
So we get a contradiction in both cases.
\end{itemize}
So the integer as desired is unique.
\end{proof}

\begin{exercise} \label{exercise 5.4.4}
Show that for any positive real number \(x > 0\) there exists a positive integer \(N\) such that \(x > 1/N > 0\).
\end{exercise}

\begin{proof}
Since \(x > 0\), by \PROP{5.4.8}, \(x^{-1} > 0\), and by \CORO{5.4.13}, for \(\varE = 1\) we can find integer \(N\) s.t. \(N\varE > x^{-1} > 0\);
But \(N\varE = N \X 1 = N\), so we have \(N > x^{-1} > 0\) \MAROON{(1)}, and trivially \(N^{-1} > 0\) \MAROON{(2)} also.
Now from \MAROON{(1)}, by \PROP{5.4.8}, we have \(N^{-1} < (x^{-1})^{-1}\), that is, \(N^{-1} < x\).
Together with \MAROON{(2)} we have \(0 < N^{-1} < x\).
\end{proof}

\begin{exercise} \label{exercise 5.4.5}
Prove \PROP{5.4.14}.
(Hint: use \EXEC{5.4.4}.
You may also need to argue by contradiction.)
\end{exercise}

\begin{proof}
Since \(x < y\) Then by \DEF{5.4.6}, \(y - x\) is a positive \emph{real} number.
So by \EXEC{5.4.4}, we can find an \emph{integer} \(N\) s.t. \(0 < 1/N < y - x\), so \(y > x + 1/N\) \MAROON{(1)}.
And given integer(thus real) \(N\) and real \(x\), by \DEF{5.3.9}, \(Nx\) is real.
So by \EXEC{5.4.3}, there exists integer \(M\) s.t. \(M \le Nx < M + 1\).
And
\begin{align*}
             & M \le Nx < M + 1 \\
    \implies & \frac{M}{N} \le x < \frac{M + 1}{N} \\
    \implies & \frac{M}{N} \le x \land x < \frac{M + 1}{N} \\
    \implies & \BLUE{\frac{M + 1}{N}} \le x + \frac{1}{N} \land x < \BLUE{\frac{M + 1}{N}} \\
    \implies & x < \BLUE{\frac{M + 1}{N}} \le x + \frac{1}{N} \\
    \implies & x < \frac{M + 1}{N} \le x + \frac{1}{N} < y & \text{by \MAROON{(1)}} \\
    \implies & x < \frac{M + 1}{N} < y
\end{align*}
Where each step is derived from \PROP{5.4.7}.
So we have found a rationals \(r := \frac{M + 1}{N}\) s.t. \(x < r < y\), as desired (It's trivial that \(\frac{M + 1}{N}\) \emph{is} rational).
\end{proof}

\begin{exercise} \label{exercise 5.4.6}.
Let \(x, y\) be real numbers and let \(\varE > 0\) be a positive real.
Show that \(\abs{x - y} < \varE\) if and only if \(y - \varE < x < y + \varE\),
and that \(\abs{x - y} \le \varE\) if and only if \(y - \varE \le x \le y + \varE\).
\end{exercise}

\begin{proof}
First from \PROP{5.4.16} (or \PROP{4.3.3}(c), plus another argument that mimics \PROP{4.3.3}(c) but removes the case of \(x = y\) for strict inequality),
we have \(\abs{x - y} < \varE\) if and only if \(-\varE < x - y < \varE\) \MAROON{(1)},
and \(\abs{x - y} \le \varE\) if and only if \(-\varE \le x - y \le \varE\) \MAROON{(2)}.
But from \MAROON{(1)} we have \(-\varE + y < x < \varE + y\) or \(y - \varE < x < y + \varE\);
and from \MAROON{(2)} we have \(-\varE + y \le x \le \varE + y\) or \(y - \varE \le x \le y + \varE\), as desired.
\end{proof}

\begin{note}
\(x, y\) 的距離小於(等於) \(\varE\),若且唯若\ \(y\) 只比\ \(x\) 多\ \(\varE\) 或少\ \(\varE\)。
\end{note}

\begin{exercise} \label{exercise 5.4.7}
Let \(x\) and \(y\) be real numbers.
Show that (1) \(x \le y + \varE\) for all real numbers \(\varE > 0\) if and only if \(x \le y\).
Show that (2) \(\abs{x - y} \le \varE\) for all real numbers \(\varE >  0\) if and only if \(x = y\).
\end{exercise}

\begin{proof}
First we show (1).
\begin{itemize}
    \item[\(\Longrightarrow\)]
        For the sake of contradiction, suppose \(x \le y + \varE\) for all real numbers \(\varE > 0\) but \(x > y\).
        Then by \DEF{5.4.6}, \(x - y\) is positive.
        Also by \PROP{5.4.8}, \(\frac{x - y}{2}\) is also positive.
        Then in particular let \(\varE = \frac{x - y}{2}\), by supposition we have \(x \le y + \varE\), that is, \(x \le y + \frac{x - y}{2}\).
        By algebra, we then have \(x - x/2 \le y - y/2\) and \(x/2 \le y/2\), or \(x \le y\), which contradicts the supposition that \(x > y\).
    \item[\(\Longleftarrow\)]
        For the sake of contradiction, suppose \(x \le y\) but there exists \(\varE > 0\) s.t. \(x > y + \varE\).
        Then since \(x \le y\), by \DEF{5.4.6} we have \(x - y\) is zero or negative.
        But since also \(x - y > \varE > 0\), by \DEF{5.4.6} we also have \(x - y\) is positive, which contradicts \PROP{5.4.7}(a) since \(x - y\) cannot be both positive and non-negative.
\end{itemize}

Now we show (2).
\begin{align*}
         & \abs{x - y} \le \varE\ \forall \varE > 0 \\
    \iff & y -\varE \le x \le y + \varE\ \forall \varE > 0 &  \text{by \EXEC{5.4.6}} \\
    \iff & y -\varE \le x \land x \le y + \varE\ \forall \varE > 0 \\
    \iff & y \le x + \varE \land x \le y + \varE\ \forall \varE > 0 \\
    \iff & y \le x \land x \le y & \text{by (1)} \\
    \iff & y = x & \text{trivial}
\end{align*}
\end{proof}

\begin{exercise} \label{exercise 5.4.8}
Let \((a_n)_{n = 1}^{\infty}\) be a Cauchy sequence of rationals, and let \(x\) be a \emph{real} number.
Show that if \(a_n \le x\) for all \(n \le 1\), then \(\LIM_{n \toINF} a_n \le x\).
Similarly, show that if \(a_n \ge x\) for all \(n \ge 1\), then \(\LIM_{n \toINF} a_n \ge x\).
(Hint: prove by contradiction.
Use \PROP{5.4.14} to find a \emph{rational} between \(\LIM_{n \toINF} a_n\) and \(x\), and then use \PROP{5.4.9} or \CORO{5.4.10}.)
\end{exercise}

\begin{proof}
We prove the first statement by contradiction.
Suppose \(a_n \le x\) for all \(n \ge 1\) \MAROON{(1)}, but \(\LIM_{n \toINF} a_n > x\).
Then by \PROP{5.4.14}, we can find a \emph{rational} \(r = \LIM_{n \toINF} r\) s.t. \(\LIM_{n \toINF} a_n > \LIM_{n \toINF} r > x\) \MAROON{(2)}.
Since \(r > x\), with \MAROON{(1)} we have \(a_n \le r\) for all \(n \ge 1\).
Since \(a_n\) for all \(n \ge 1\) and \(r\) are rational, by \CORO{5.4.10} we have \(\LIM_{n \toINF} a_n \le \LIM_{n \toINF} r\).
But that contradicts \PROP{5.4.7}(a) since by \MAROON{(2)} we also have \(\LIM_{n \toINF} a_n > \LIM_{n \toINF} r\).

Now We prove the second statement using first statement.
Suppose \(a_n \ge x\) for all \(n \geq 1\).
Then We have
\begin{align*}
             & a_n \ge x\ \forall n \ge 1 \\
    \implies & -a_n \le -x\ \forall n \ge 1 & \text{by algebra} \\
    \implies & \LIM_{n \toINF} -a_n \le -x & \text{from the first statement} \\
    \implies & -(\LIM_{n \toINF} a_n) \le -x & \text{by \DEF{5.3.18}, negation of reals} \\
    \implies & \LIM_{n \toINF} a_n \ge x & \text{by algebra}
\end{align*}
\end{proof}
\section{The least upper bound property} \label{sec 5.5}

We now give one of the most basic advantages of the real numbers over the rationals;
one can take the \emph{least upper bound} \(\sup(E)\) of any subset \(E\) of the real numbers \(\SET{R}\).

\begin{note}
開天眼,我覺得上一段話有點嘴,又不是每個實數子集合都有最小上界(\THM{5.5.9})。
\end{note}

\begin{definition} [Upper bound] \label{def 5.5.1}  Let \(E\) be a subset of \(\SET{R}\), and let \(M\) be a real number.
We say that \(M\) is ``an'' \emph{upper bound} for E, iff we have \(x \le M\) for every element \(x\) in \(E\).
\end{definition}

\begin{example}
Simple example.
\end{example}

\begin{example} \label{example 5.5.3}
Let \(\SET{R}^+\) be the set of positive reals:
\(\SET{R}^+ := \{x \in \SET{R} : x > 0\} \).
Then \(\SET{R}^+\) does \emph{not} have any upper bounds at all.
\end{example}

\begin{proof}
Suppose we have an upper bound \(M\) for the set.
Then by definition for all \(x \in \SET{R}^+, x \le M\), and that implies, \(M > 0\) and \(M + 1 > 0\).
So by definition of \(\SET{R}^+\), \(M + 1 \in \SET{R}^+\), so we have found a real number \(M + 1 \in \SET{R}^+\) and \(M < M + 1\), which contradicts that \(M\) is an upper bound for \(\SET{R}^+\).
\end{proof}

\begin{example} \label{example 5.5.4}
Let \(\emptyset\) be the empty set.
Then \emph{every} number \(M\) is an upper bound for \(\emptyset\), because \(M\) is greater than ``every element'' of the ``empty'' set
(this is a vacuously true statement, but still true).
\end{example}

\begin{note}
It is clear that if \(M\) is an upper bound of \(E\), then any larger number \(M' \geq M\) is also an upper bound of \(E\).
On the other hand, it is \emph{not so clear} whether it is also possible for any number \emph{smaller} than \(M\) to also be an upper bound of \(E\).
\end{note}

\begin{definition} [Least upper bound] \label{definition 5.5.5}
Let \(E\) be a subset of \(\SET{R}\), and \(M\) be a real number.
We say that \(M\) is ``a'' least upper bound for \(E\) iff
\begin{enumerate}
    \item \(M\) is an upper bound for \(E\), and also
    \item any other upper bound \(M'\) for \(E\) must be larger than or equal to \(M\).
\end{enumerate}
\end{definition}

\begin{note}
We will prove the least upper bound is \emph{unique}(if exist) (\PROP{5.5.8}).
\end{note}

\begin{example} \label{example 5.5.6}
Let \(E\) be the interval \(E := \{x \in \SET{R} : 0 \le x \le 1 \} \).
Then, as noted before, \(E\) has many upper bounds, indeed every number greater than or equal to \(1\) is an upper bound.
But only \(1\) is \emph{the} least upper bound;
all other upper bounds are larger than \(1\).
\end{example}

\begin{example} \label{example 5.5.7}
The empty set does not have a least upper bound.
\end{example}

\begin{proof}
Suppose not.
Then we have a least upper bound \(M\) for the empty set.
And since \(M\) is real, \(M - 1\) is real, by \EXAMPLE{5.5.4}, \(M - 1\) is also an upper bound of the empty set.
But \(M - 1 < M\), which contradicts that \(M\) is a \emph{least} upper bound for the empty set.
\end{proof}

\begin{proposition} [Uniqueness of least upper bound] \label{prop 5.5.8}
Let \(E\) be a subset of \(R\).
Then \(E\) can have \emph{at most one} least upper bound.
\end{proposition}

\begin{proof}
Let \(M_1\) and \(M_2\) be two \emph{least} upper bounds.
Since \(M_1\) is a \emph{least} upper bound and \(M_2\) is (also just) an upper bound, then by definition of least upper bound we have \(M_2 \ge M_1\).
Since \(M_2\) is a \emph{least} upper bound and \(M_1\) is (also just) an upper bound, we similarly have \(M_1 \ge M_2\).
Thus \(M_1 = M_2\).
Thus there is at most one least upper bound.
\end{proof}

Now we come to an important(worth calling ``theorem'') property of the real numbers:

\begin{theorem} [Existence of least upper bound] \label{thm 5.5.9}
Let \(E\) be a nonempty subset of \(\SET{R}\).
\emph{If} \(E\) has an upper bound, (i.e., \(E\) has some upper bound \(M\)), \emph{then it must have exactly one \emph{least} upper bound}.
\end{theorem}

\begin{proof}
This theorem will take quite a bit of effort to prove, and many of the steps will be left as exercises(\EXEC{5.5.2}, \EXEC{5.5.3}, \EXEC{5.5.4}).

Let \(E\) be a non-empty subset of \(R\) with an upper bound \(M\).
By \PROP{5.5.8}, we know that \(E\) has at most one least upper bound;
we have to show that \(E\) \emph{has at least one} least upper bound.

Since \(E\) is non-empty, we can choose some element \(x_0\) in \(E\).
Let \(n\) be arbitrary positive integer(i.e. \(n \ge 1\)).
We know that \(E\) has an upper bound \(M\).
Since \(n\) is an integer, so trivially \(1/n\) is rational.
By the Archimedean property (\CORO{5.4.13}), we can find an integer \(K\) such that \(K \X (1/n) = K/n > M\), and hence \(K/n\) is also an upper bound for \(E\) \MAROON{(1)}.
Now we consider the value of \(x_0\):
\begin{itemize}
    \item [>>] \(x_0 \ge 0\):
        Then let \(L = -1\), so \(L/n\) is negative, so \(L/n < 0\).
    \item [>>] \(x_0 < 0\):
        Then \(-x_0 > 0\), so again by the Archimedean property (\CORO{5.4.13}), we can find an integer \(L'\) s.t. \(L' \X 1/n > -x_0\), or \(-L' \X 1/n < x_0\).
        So let \(L = -L'\), then we have \(L/n < x_0\).
\end{itemize}
So in all cases we can find \(L\) s.t. \(L/n < x_0\).
That is, we can find an element \(x_0 \in E\) s.t. \(x_0 < L/n\), so \(L/n\) is \emph{not} an upper bound for \(E\).
But by \MAROON{(1)}, \(K/n\) is an upper bound for \(E\), so \(K/n \ge L/n\), so \(K \ge L\).

Since \(K/n\) is an upper bound for \(E\) and \(L/n\) is not, we can find an integer \(L < m_n \le K\) with the property that \(m_n/n\) \emph{is} an upper bound for \(E\), but \((m_n - 1)/n\) \emph{is not} (see \EXEC{5.5.2}) \MAROON{(2)}.
In fact, this integer \(m_n\) is unique (\EXEC{5.5.3}).
We \emph{subscript} \(m_n\) by \(n\) to emphasize the fact that \emph{this integer \(m\) depends on the choice of \(n\).}
So for any integer \(n \ge 1\), the \emph{mapping} \(m_n\) is unique.
So by definition of sequence (\DEF{5.1.1}), this gives a well-defined (and unique) sequence \(m_1, m_2, m_3,...\) of integers, and by \MAROON{(2)},
with each of the \(m_n/n\) being upper bounds(of \(E\)) and each of the \((m_n-1)/n\) not being upper bounds(of \(E\)).

(BTW, the choice of \(K\) and \(L\) are also dependent on \(n\), but we don't use them later, so we do not add subscript on them.)

Now let \(N \ge 1\) be a positive integer, and let \(n, n' \ge N\) be integers.
So of course \(n, n' \ge 1\), so in particular for \(n\), \(m_n/n\) is an upper bound for \(E\), and for \(n'\), \((m_{n'} - 1)/n'\) is not, so we must have \(m_n/n > (m_{n'} - 1)/n'\)
\begin{itemize}
    \item
        why?
        Because \((m_{n'} - 1)/n'\) is not an upper bound, we can find another real number \(a \in E\) s.t. \(a > (m_{n'} - 1)/n'\).
        But \(m_n/n\) is an upper bound for \(E\), so by definition \(m_n/n \ge a\), so we have \(m_n/n \ge a > (m_{n'} - 1)/n'\), so \(m_n/n > (m_{n'} - 1)/n'\).
\end{itemize}
And from that we have:
\begin{align*}
             & \frac{m_n}{n} > \frac{m_{n'} - 1}{n'} \\
    \implies & \frac{m_n}{n} - \frac{m_{n'}}{n'} > \frac{-1}{n'} \\
    \implies & \frac{m_n}{n} - \frac{m_{n'}}{n'} > \frac{-1}{n'} \ge \frac{-1}{N} \MAROON{\ (3)} & \text{since \(n' \ge N\), we have \(1/n' \le 1/N\), and \(-1/n' \ge -1/N\)}
\end{align*}
Similarly, since \(m_{n'}/n'\) is an upper bound for \(E\) and \((m_n - 1)/n\)) is not, we have \(m_{n'}/n' > (m_n - 1)/n\), and,
\begin{align*}
             & \frac{m_{n'}}{n'} > \frac{m_n - 1}{n} \\
    \implies & \frac{m_{n'}}{n'} - \frac{m_n}{n} > \frac{-1}{n} \\
    \implies & -(\frac{m_{n'}}{n'} - \frac{m_n}{n}) < -(\frac{-1}{n}) \\
    \implies & \frac{m_n}{n} - \frac{m_{n'}}{n'} < \frac{1}{n} \\
    \implies & \frac{m_n}{n} - \frac{m_{n'}}{n'} < \frac{1}{n} \le \frac{1}{N} \MAROON{\ (4)} & \text{since \(n \ge N\), we have \(1/n \le 1/N\)}
\end{align*}
So with \MAROON{(3) (4)} and \PROP{4.3.3}(c)(note that all operand are rationals, so we just use property for rationals)we have
\[
    \abs{\frac{m_n}{n} - \frac{m_{n'}}{n'}} \le \frac{1}{N} \text{ for all } n, n' \ge N \ge 1
\]
This implies that \((m_n/n)_{n = 1}^{\infty}\) is a Cauchy sequence (\EXEC{5.5.4}).
So since \((m_n/n)_{n = 1}^{\infty}\) is Cauchy, we can take the formal limit of it and let
\[
    S := \LIM_{n \toINF}\frac{m_n}{n}.
\]
From \EXEC{5.3.5} we also have \(\LIM_{n \toINF}1/n = 0\), So we can derive that
\begin{align*}
    S & = \LIM_{n \toINF}\frac{m_n}{n} \\
      & = \LIM_{n \toINF}\frac{m_n}{n} - 0 \\
      & = \LIM_{n \toINF}\frac{m_n}{n} - \LIM_{n \toINF}\frac{1}{n} \\
      & = \LIM_{n \toINF}\frac{m_n - 1}{n}
\end{align*}

To finish the proof of the theorem, we need to show that \(S\) is the least upper bound for \(E\).
First we show that it is an upper bound.
Let \(x\) be arbitrary element of \(E\).
Then, \emph{for all} \(n \ge 1\), since \(m_n/n\) is an upper bound for \(E\), we have \(x \le m_n/n\) \emph{for all} \(n \ge 1\).
Applying \EXEC{5.4.8}, we conclude that \(x \le \LIM_{n \toINF} m_n/n = S\).
Thus \(S\) is indeed an upper bound for \(E\).

Now we show it is a \emph{least} upper bound.
Suppose \(y\) is an arbitrary upper bound for \(E\).
Since \emph{for all} \(n \ge 1\), \((m_n - 1)/n\) is \emph{not} an upper bound, we conclude that \(y \ge (m_n - 1)/n\) for all \(n \ge 1\).
Applying \EXEC{5.4.8}, we conclude that \(y \ge \LIM_{n \toINF}(m_n - 1)/n = S\).
Thus the upper bound \(S\) is less than or equal to \emph{every} upper bound of \(E\), and \(S\) is thus a least upper bound of \(E\).
\end{proof}

\begin{definition}[Supremum]\label{5.5.10}
Let \(E\) be a subset of the real numbers.
If \(E\) is non-empty and has some upper bound, we define \(\sup(E)\) to be \emph{the} least upper bound of \(E\)
(this is well-defined by \THM{5.5.9}).
We introduce two additional symbols, \(+\infty\) and \(-\infty\).
If \(E\) is non-empty and \emph{has no} upper bound, we set \(\sup(E) := +\infty\);
if \(E\) is empty, we set \(\sup(E) := -\infty\).
We refer to \(\sup(E)\) as the \emph{supremum} of \(E\), and also denote it by \(\sup E\).
\end{definition}

\begin{remark} \label{remark 5.5.11}
At present, \(+\infty\) and \(-\infty\) are meaningless symbols; 
we have no operations on them at present, and none of our results involving \emph{real numbers} apply to \(+\infty\) and \(-\infty\), because these are \emph{not} real numbers.
In \SEC{6.2} we add \(+\infty\) and \(-\infty\) to the reals to form the \emph{extended real number system}, but this system is not as convenient to work with as the real number system, because many of the laws of algebra break down.
For instance, it is not a good idea to try to define \(+\infty + -\infty\); setting this equal to 0 causes some problems.
\end{remark}

Now we give an example of how the least upper bound property is useful.

\begin{proposition} \label{prop 5.5.12}
There exists a positive \emph{real} number \(x\) such that \(x^2 = 2\).
\end{proposition}

\begin{remark} \label{remark 5.5.13}
Comparing \PROP{5.5.12} with \PROP{4.4.4}, we see that certain numbers are real but not rational.
The proof of this proposition also shows that the rationals \(\SET{Q}\) do not obey the least upper bound property, \emph{otherwise} one could use that property to construct a (\emph{rational} that is) square root of \(2\), which by \PROP{4.4.4} is not possible.
\end{remark}

\begin{proof}
Let \(E\) be the set \( \{ y \in \SET{R} : y \ge 0 \text{ and } y^2 < 2 \} \);
thus \(E\) is the set of all non-negative real numbers whose square is less than \(2\).
Observe that \(E\) has an upper bound of \(2\) (because if \(y > 2\), then \(y^2 > 4 > 2\) and hence \(y \notin E\)).
Also, \(E\) is non-empty (for instance, \(1\) is an element of \(E\)).
Thus by the least upper bound property \THM{5.5.9}, we have a real number \(x := \sup(E)\) which is the least upper bound of \(E\) \MAROON{(1)}.
Then \(x\) is greater than or equal to \(1\)(since \(1 \in E\)) and less than or equal to \(2\) (since \(2\) is an upper bound for E).
So \(x\) is positive.
Now we show that \(x^2 = 2\).

We argue this by contradiction.
We show that both \(x^2 < 2\) and \(x^2 > 2\) lead to contradictions.
First suppose that \(x^2 < 2\). Let \(0 < \varE < 1\) be a small number;
then we have
\begin{align*}
    (x + \varE)^2 & = x^2 + 2\varE x + \varE^2 & \text{by algebra} \\
                  & \le x^2 + 4\varE + \varE^2 & \text{since \(x \le 2\), an upper bound of \(E\)} \\
                  & \le x^2 + 4\varE + \varE & \text{since \(0 < \varE < 1\), so \(\varE^2 < \varE\) }\\
                  & = x^2 + 5\varE \MAROON{(2)}
\end{align*}
Now from \(x^2 < 2\), we have
\begin{align*}
             & x^2 < 2 \\
    \implies & 0 < 2 - x^2 \\
    \implies & 0/5 < (2 - x^2)/5 \\
    \implies & 0 < (2 - x^2)/5
\end{align*}
And by \PROP{5.4.14} we can find a rational \(\varE\) s.t. \(0 < \varE < (2 - x^2)/5\), and
\begin{align*}
             & 0 < \varE < (2 - x^2)/5 \\
    \implies & 0 < 5\varE < (2 - x^2) \\
    \implies & x^2 + 0 < x^2 + 5\varE < 2 \\
    \implies & x^2 < x^2 + 5\varE < 2 \MAROON{(3)}
\end{align*}
So we can choose \(\varE\) s.t. \MAROON{(3)} is true.
And with the derivation in \MAROON{(2)} we have \((x + \varE)^2 < 2\).
By construction(definition) of \(E\), this means \(x + \varE \in E\); but this contradicts \MAROON{(1)} that \(x\) is an upper bound of \(E\) since \(x < x + \varE\).

Now suppose that \(x^2 > 2\).
Let \(0 < \varE < 1\) be a small number;
then we have
\begin{align*}
    (x - \varE)^2 & = x^2 - 2\varE x + \varE^2 & \text{by algebra} \\
                  & \ge x^2 - 2\varE x & \text{of course} \\
                  & \ge x^2 - 4\varE \MAROON{(4)} & \text{since \(x \le 2\), an upper bound of \(E\)}
\end{align*}
Again, since \(x^2 > 2\), we can choose \(0 < \varE < 1\) such that by \PROP{5.4.14} \(x^2 > 2 + 4\varE > 2\), which implies \(x^2 - 4\varE > 2\), and thus by the derivation in \MAROON{(4)} we have \((x - \varE)^2 > 2\) \MAROON{(5)}.
But then this implies that \(x - \varE \ge y\) for all \(y \in E\). (Why? If \(x - \varE < y\) then \((x - \varE)^2 < y^2 \le 2\) by construction of \(E\), so \((x - \varE)^2 < 2\), contradicting \MAROON{(5)}.)
Thus \(x - \varE\) is an upper bound for \(E\), which contradicts the fact that \(x\) is the \emph{least} upper bound of \(E\).

From these two contradictions we see that \(x^2 = 2\), as desired.
\end{proof}

\begin{note}
這裡有一個很\ mind=blown 的論述: 若要證明一個實數\ \(r\) 是某集合的\ upper bound,你可以證明所有比\ \(r\) 大的實數都不在這個集合裡。
\end{note}

\begin{remark} \label{remark 5.5.14}
In \CH{6} we will use the least upper bound property to develop the theory of limits, which allows us to do many more
things than just take square roots.
\end{remark}

\begin{remark} \label{remark 5.5.15}
We can of course talk about lower bounds, and \emph{greatest lower bounds}, of sets \(E\);
the greatest lower bound of a set \(E\) is also known as the \emph{infimum} of \(E\) and is denoted \(\inf(E)\) or \(\inf E\).
Everything we say about suprema has a counterpart for infima;
we will usually leave such statements to the reader.
A precise relationship between the two notions is given by \EXEC{5.5.1}.
See also \SEC{6.2}.
\end{remark}

\begin{note}
Supremum means ``highest'' and infimum means ``lowest'', and the plurals are suprema and infima.
Supremum is to superior, and infimum to inferior, as maximum is to major, and minimum to minor.
The root words are ``super'', which means ``above'', and ``infer'', which means ``below''
(this usage only survives in a few rare English words such as ``infernal'', with the Latin prefix ``sub'' having mostly replaced ``infer'' in English).
\end{note}

\exercisesection

\begin{exercise} \label{exercise 5.5.1}
Let \(E\) be a subset of the real numbers \(\SET{R}\), and suppose that \(E\) has a least upper bound \(M\) which is a real number, i.e., \(M = \sup(E)\).
Let \(-E\) be the set \(-E := \{-x : x \in E\} \).
Show that \(-M\) is the greatest lower bound of \(-E\), i.e., \(-M = \inf(-E)\).
\end{exercise}

\begin{proof}
We first show \(-M\) is a lower bound of \(-E\).
Let arbitrary \(x' \in -E\).
By construction of \(-E\), \(x' = -x\) for some \(x \in E\).
So we have \(x \le M\) \MAROON{(1)} since \(M\) is a least upper bound of \(E\).
And \MAROON{(1)} implies \(-M \le -x\), that is, \(-M \le x'\).
So for all \(x' \in -E\), \(-M \le x'\), so \(-M\) is a lower bound of \(-E\).

Now we show \(-M\) is a greatest lower bound of \(-E\).
Suppose not, then there exists another lower bound \(-M'\) of \(-E\) s.t. \(-M' > -M\), which implies \(M' < M\).
But now given arbitrary \(x \in E\), since \(-x \in -E\), and we have \(-M'\) as a lower bound, we can conclude that \(-x \ge -M'\), which implies \(x \le M'\).
So for all \(x \in E\), \(x \le M'\), so \(M'\) is also an upper bound of \(E\);
but \(M' < M\), which contradicts that \(M\) is a least upper bound of \(E\).

It is similar as \PROP{5.5.8} to show the uniqueness of ``the'' greatest lower bound.
\end{proof}

\begin{exercise} \label{exercise 5.5.2}
Let \(E\) be a non-empty subset of \(\SET{R}\), let \(n \ge 1\) be an integer, and let \(L < K\) be integers.
Suppose that \(K/n\) is an upper bound for E, but that \(L/n\) is \emph{not} an upper bound for \(E\).
Without using \THM{5.5.9}(\emph{otherwise circular}), show that there exists an integer \(L < m \le K\) such that \(m/n\) is an upper bound for \(E\), but that \((m - 1)/n\) is not an upper bound for \(E\).
(Hint: prove by contradiction, and use induction.
It may also help to draw a picture of the situation.)
\end{exercise}

\begin{proof}
\href{https://taoanalysis.wordpress.com/2020/05/14/exercise-5-5-2/}{Reference}

The statement of the exercise:
\begin{align*}
    \forall\ n \in \SET{N} \text{ s.t. } \ge 1 & \\
             & (\exists L, K \in \SET{Z} \text{ s.t. } L < K \\
             &  \land K/n \text{ is an upper bound of } E \\
             &  \land L/n \text{ is not an upper bound of } E) \\
    \implies & \\
             & (\exists m \in \SET{Z} \text{ s.t. } L < m \le K \\
             &  \land\ m/n \text{ is an upper bound of } E \\
             &  \land\ (m - 1)/n \text{ is not an upper bound of } E).
\end{align*}

So for the sake of contradiction \BLUE{(1)}: Suppose there exists an integer \(n \ge 1\) s.t. there exist integers \(L < K\) s.t. \(K/n\) is an upper bound of \(E\) but \(L/n\) is not, \emph{but, for the sake of contradiction}, there is no integer \(m\) such that \(L < m \le K\), and \(m/n\) is an upper bound of \(E\) and \((m - 1)/n\) is not an upper bound of \(E\)

We will show by induction on \(j\) that \BLUE{(2)} \((K - j)/n\) is an upper bound for \(E\) for all natural numbers \(j\).

For base case \(j = 0\), \((K - j)/n = (K - 0)/n = K/n\), which is an upper bound of \(E\) by supposition, so the base case is true.

Now suppose inductively for some \(j \ge 0\) we have \((K - j)/n\) is an upper bound for \(E\).
We have to show \((K - (j + 1))/n\) is also an upper bound for \(E\) \MAROON{(1)}.
Now by supposition, \(L/n\) is \emph{not} an upper bound of \(E\), so we can find \(x \in E\) s.t. \(L/n < x\).
But since by inductive hypothesis \((K - j)/n\) is an upper bound for \(E\), \(x \le (K - j)/n\).
So together we have \(L/n < x \le (K - j)/n\), so \(L/n < (K - j)/n\), so \(L < K - j\) \MAROON{(2)}.
And since \(j\) is a natural number, i.e. \(j \ge 0\), so \(K - j \le K - 0 = K\), so can further derive that \(L < K - j \le K\).
Suppose for the sake of contradiction that \MAROON{(1)} is false, i.e. \((K - (j + 1))/n\) is \emph{not} an upper bound of \(E\).
Then currently we have
\begin{enumerate}
    \item[(1)] (by inductive hypothesis) \((K - j)/n\) is an upper bound for \(E\)
    \item[(2)] (suppose for contradiction) \((K - (j + 1))/n\) is not an upper bound for \(E\) \MAROON{(3)}.
\end{enumerate}
Now let \(m := K - j\), then with \MAROON{(2)} we have \(L < m\),
and replacing \(m\) into above, we have \(m/n\) is an upper bound for \(E\) and \((m - 1)/n\) is not an upper bound for \(E\), which contradicts the very first supposition \BLUE{(1)}.
So \((K - (j + 1))/n\) must also be an upper bound for \(E\).
This closes the induction, so \BLUE{(2)} is true.
In particular, since integers \(L < K\), \(K - L > 0\) i.e. \(K - L\) is a natural number;
so from \BLUE{(2)} we have \((K - (K - L))/n\) is an upper bound for \(E\), that is, \(L/n\) is an upper bound for \(E\), which is a contradiction!
So the very first supposition \BLUE{(1)} is false, so the statement of the exercise is true.
\end{proof}

\begin{exercise} \label{exercise 5.5.3}
Let \(E\) be a non-empty subset of \(R\), let \(n \ge 1\) be an integer, and let \(m, m'\) be integers with the properties that \(m/n\) and \(m'/n\) are upper bounds for \(E\), but \((m - 1)/n\) and \((m' - 1)/n\) are \emph{not} upper bounds for \(E\).
Show that \(m = m'\).
This shows that the integer \(m\) constructed in \EXEC{5.5.2} is \emph{unique}.
(Hint: again, drawing a picture will be helpful.)
\end{exercise}

\begin{proof}
Since \(\frac{m' - 1}{n}\) is \emph{not} an upper bound for \(E\), there exists \(x \in E\) s.t. \(\frac{m' - 1}{n} < x\);
and since \(\frac{m}{n}\) is an upper bound for \(E\), \(x \le \frac{m}{n}\), so together we have \(\frac{m' - 1}{n} < x \le \frac{m}{n}\), so \(\frac{m' - 1}{n} < \frac{m}{n}\) \MAROON{(1)}.
Similarly since \(\frac{m - 1}{n}\) is \emph{not} an upper bound for \(E\) and \(\frac{m'}{n}\) is an upper bound for \(E\), we have \(\frac{m - 1}{n} < \frac{m'}{n}\) \MAROON{(2)}.
And since \(n \ge 1\), by multiplying \(n\) on both sides of \(\MAROON{(1) (2)}\) we have \(m' - 1 < m\) and \(m - 1 < m'\) \MAROON{(3)}.
But since \(m, m'\) are integer, by similar property of \PROP{2.2.12}(e)(this is for natural numbers, but the property can be proved for integers), from \MAROON{(3)} we have \(m' \le m\) and \(m \le m'\), which implies \(m' = m\).
\end{proof}

\begin{exercise} \label{exercise 5.5.4}
Let \(q_1, q_2, q_3,...\) be a sequence of rational numbers with the property that \(\abs{q_n - q_{n'}} \le \frac{1}{M}\) whenever \(M \ge 1\) is an integer and \(n, n' \ge M\) \BLUE{(property 1)}.
Show that \(q_1, q_2, q_3,...\) is a Cauchy sequence.
Furthermore, if \(S := \LIM_{n \toINF} q_n\), show that \(\abs{q_M - S} \le \frac{1}{M}\) for every \(M \ge 1\).
(Hint: use \EXEC{5.4.8}.)
\end{exercise}

\begin{proof}
We first show \(q_1, q_2, q_3,...\) is a Cauchy sequence.
So let arbitrary \(\varE > 0\), we have to show there exists \(N \ge 1\) s.t. \(\abs{q_n - q_{n'}} \ge \varE\) for all \(n, n' \ge N\).
But from \EXEC{5.4.4}, there exists a positive integer \(N\) s.t. \(\varE > 1/N > 0\) \MAROON{(1)}.
And from \BLUE{(property 1)} of \(q_1, q_2, q_3,...\), \(\abs{q_n - q_{n'}} \le \frac{1}{N}\) \MAROON{(2)} for all \(n, n' \ge N\).
So with \MAROON{(1) (2)}, we can conclude that \(\abs{q_n - q_{n'}} \le 1/N < \varE\) for all \(n, n' \ge N\), as desired.
So the sequence \(q_1, q_2, q_3,...\) is Cauchy.

Now we let \(S\) be the formal limit of the sequence \(q_1, q_2, q_3,...\):
\[
    S := \LIM_{n \toINF} q_n.
\]
Let arbitrary integer \(M \ge 1\), we have to show \(\abs{q_M - S} \le \frac{1}{M}\);
Now from \BLUE{(property 1)}, given \(n \ge M\), we have \(\abs{q_M - q_n} \le \frac{1}{M}\).
By \EXEC{5.4.6} we have \(-\frac{1}{M} \le q_M - q_n \le \frac{1}{M}\).
Now we define another sequence \((q'_n)_{n = 1}^{\infty}\) that \(q'_n := q_M\) if \(n < M\) else \(q'_n := q_n\).
Then the sequence \((q_n)_{n = 1}^{\infty}\) and \((q'_n)_{n = 1}^{\infty}\) are equivalent since they are eventually the same(similar argument as \LEM{5.3.14}).
So \(S\) is also equal to the formal limit of \((q'_n)_{n = 1}^{\infty}\) \MAROON{(3)}, and it's trivial that \(-\frac{1}{M} \le q_M - q'_n \le \frac{1}{M}\) for all \(n \ge 1\)(not from \(M\), but from \(1\)).
With \EXEC{5.4.8}, that implies \(-\frac{1}{M} \le q_M - \LIM_{n \toINF} (q'_n) \le \frac{1}{M}\).
That is, by \MAROON{(3)}, \(-\frac{1}{M} \le q_M - S \le \frac{1}{M}\).
So again by \EXEC{5.4.6}, \(\abs{q_M - S} \le \frac{1}{M}\).

Since \(M \ge 1\) is arbitrary, we have \(\abs{q_M - S} \le \frac{1}{M}\) for all \(M \ge 1\).
\end{proof}

\begin{note}
第二個敘述是說,這個序列的第\ \(M\) 項跟這個序列的極限的距離小於\ \(1/M\)。
\end{note}

\begin{exercise} \label{exercise 5.5.5}
Establish an analogue of \PROP{5.4.14}, in which ``rational'' is replaced by ``irrational''.
That is, given any two real numbers \(x < y\), we can find a irrational number \(r\) such that \(x < r < y\).
\end{exercise}

\begin{proof}
Let \(x, y\) be arbitrary real numbers s.t. \(x < y\).
Let \(r = \sqrt{2}\), which is well-defined by \PROP{5.5.12}.
And since \(x < y\), we have \(x - r < y - r\) (by \PROP{5.4.7}).
Now by \PROP{5.4.14}, we can find a \emph{rational} \(r'\) s.t. \(x - r < r' < y - r\), so we have \(x < r' + r < y\).
Then \(r' + r\) must be \emph{irrational}, otherwise if \(r' + r\) is rational, then \((r' + r) - r'\) is also a rational by definition of the \emph{rational} subtraction, so \((r' + r) - r' = r\) is a rational, which is impossible since \(r = \sqrt{2}\). 
\end{proof}

\section{Real exponentiation, part I} \label{sec 5.6}

In \SEC{4.3} we defined exponentiation \(x^n\) when \(x\) is \emph{rational} and \(n\) is a \emph{natural} number, or when \(x\) is a \emph{non-zero} rational and \(n\) is an \emph{integer}(i.e. can be negative).
Now that we have all the arithmetic operations on the reals
(and \PROP{5.4.7} assures us that the arithmetic properties of the rationals that we are used to, continue to hold for the reals),
we can similarly define \emph{exponentiation} of the \emph{reals}.

\begin{definition} [Exponentiating a real by a \emph{natural} number] \label{def 5.6.1}
Let \(x\) be a real number.
To raise \(x\) to the power \(0\), we define \(x^0 := 1\).
Now suppose recursively(inductively) that \(x^n\) has been defined for some natural number \(n\), then we define \(x^{n + 1} := x^n \X x\).
\end{definition}

\begin{definition} [Exponentiating a real by an (negative) \emph{integer}] \label{def 5.6.2}
Let \(x\) be a \emph{non-zero} real number.
Then for any \emph{negative} integer \(-n\), we define \(x^{-n} := 1/x^n\).
\end{definition}

Clearly these definitions are consistent with the definition of rational exponentiation given earlier (\DEF{4.3.9} and \DEF{4.3.11}).
We can then assert:

\begin{proposition} \label{prop 5.6.3}
All the properties in \PROP{4.3.10} and \PROP{4.3.12} remain valid if \(x\) and \(y\) are assumed to be \emph{real numbers} instead of \emph{rational numbers}.
\end{proposition}

\begin{note}
Instead of giving an actual proof of this proposition, we shall give a \emph{meta-proof} (an argument appealing to \emph{the nature of proofs}, rather than the nature of real and rational numbers).
\end{note}

\begin{meta-proof}
If one inspects the proof of \PROP{4.3.10} and \PROP{4.3.12} we see that they rely on the laws of algebra and the laws of order for the rationals (\PROP{4.2.4} and \PROP{4.2.9}).
But by \PROP{5.3.11}, \PROP{5.4.7}, and the identity \(xx^{-1} = x^{-1}x = 1\)(in the middle of page 110)
we know that all these laws of algebra and order continue to hold for real numbers as well as rationals.
Thus we can modify the proof of \PROP{4.3.10} and \PROP{4.3.12} to hold in the case when \(x\) and \(y\) are \emph{real}.
\end{meta-proof}

Now we consider exponentiation to exponents which are \emph{not} integers.
We begin with the notion of an \(n^{\text{th}}\) root, which we can \emph{define using our notion of supremum}.

\begin{definition} \label{def 5.6.4}.
Let \(x \ge 0\) be a \emph{non-negative} \emph{real}, and let \(n \ge 1\) be a \emph{positive integer}.
We \emph{define} \(x^{1/n}\), also known as the \(n^{\text{th}}\) root of \(x\), by the
formula
\[
    x^{1 / n} := \sup \{y \in \SET{R} : y \ge 0 \text{ and } y^n \le x\}.
\]
We often write \(\sqrt{x}\) for \(x^{1/2}\).
\end{definition}

\begin{note}
Note we do not define the \(n^{\text{th}}\) roots of a \emph{negative} number.
In fact, we will \emph{leave} the \(n^{\text{th}}\) roots of negative numbers \emph{undefined} for the rest of the text
(one can define these \(n^{\text{th}}\) roots once one defines the \emph{complex numbers}, but we shall refrain from doing so).
\end{note}

\begin{lemma} [Existence of nth roots] \label{lem 5.6.5}
Let \(x \ge 0\) be a non-negative real, and let \(n \ge 1\) be a positive integer.
Then the set \(E := \{ y \in \SET{R} : y \ge 0 \text{ and } y^n \le x \} \) is non-empty and is also bounded above. (So by \THM{5.5.9} the \(\sup E\) exits.)
In particular, \(x^{1/n} = \sup E\) is a real number.
\end{lemma}

\begin{proof}
The set \(E\) contains \BLUE{0}, so it is certainly not empty.
(why? Because (1) \(\BLUE{0} \ge 0\) (2) given \emph{positive} integer \(n = m + 1\)(for some natural number \(m\), by \LEM{2.2.10}), by \DEF{5.6.2}, \(\BLUE{0}^n = 0^m \X 0 = \GREEN{0}\), but \(x \ge \GREEN{0}\), so \(\BLUE{0}^n \le x\).
So together we have \(\BLUE{0} \ge 0\) and \(\BLUE{0}^n \le x\), so by construction of \(E\), \(\BLUE{0} \in E\).)

Now we show it has an upper bound.
We divide into two cases: \(x \le 1\) and \(x > 1\).

First suppose that we are in the case where \(x \le 1\).
Then we claim that the set \(E\) is bounded above by \(1\).
To see this, suppose for sake of contradiction that there was an element \(y \in E\) for which \(y > 1\). But then \(y^n > 1\)
(why? It can be proved by induction that given \(y > 1\), \(y^m > 1\) for all natural number \(m\).
So in particular for positive integer(i.e. natural number) \(n\), \(y^n > 1\)),
and hence \(y^n > x\), so by construction of \(E\), \(y \notin E\), a contradiction.
Thus \(E\) has an upper bound \(1\).

Now suppose that we are in the case where  \(x > 1\). Then we claim that the set \(E\) is bounded above by \(x\).
To see this, suppose for contradiction that there was an element \(y \in E\) for which \(y > x\).
Since \(x > 1\), we thus have \(y > 1\). Since \(y > x\) and \(y > 1\), we have \(y^n > x\)
(why? Again it can be proved by induction that given \(y > x\), \(y^m > x\) for all natural number \(m\).
So in particular for positive integer(i.e. natural number) \(n\), \(y^n > x\)),
so by construction of \(E\), \(y \notin E\), a contradiction.
Thus \(E\) has an upper bound \(x\).

Thus in both cases \(E\) has an upper bound, hence \(x^{1/n} = \sup E\) exists.
\end{proof}

We list some basic properties of \(n^{\text{th}}\) roots below.

\begin{lemma} \label{lem 5.6.6}
Let \(x, y \ge 0\) be non-negative reals, and let \(n, m \ge 1\) be \emph{positive} integers.
\begin{enumerate}
    \item If \(y = x^{1 / n}\), then \(y^n = x\), so \(x = y^n = (x^{1/n})^n\).
    \item Conversely, if \(y^n = x\), then \(y = x^{1 / n}\), so \(y = x^{1/n} = (y^n)^{1/n}\).
    \item \(x^{1 / n}\) is a non-negative real number, and is positive if and only if \(x\) is positive.
    \item We have \(x > y\) if and only if \(x^{1 / n} > y^{1 / n}\).
    \item Let \(k\) be a positive integer.
        If \(x > 1\), then \(x^{1 / k}\) is a (strictly) \emph{decreasing} function of \(k\).
        If \(0 < x < 1\), then \(x^{1 / k}\) is an (strictly) \emph{increasing} function of \(k\).
        If \(x = 1\), then \(x^{1 / k} = 1\) for all \(k\).
    \item We have \((xy)^{1 / n} = x^{1 / n}y^{1 / n}\).
    \item We have \((x^{1 / n})^{1 / m} = x^{1 / nm}\).
\end{enumerate}
\end{lemma}

\begin{note}
The part (c) of the lemma is different from the textbook, which seems wrong.
For part(a), the point is to make sure the \DEF{5.6.4} is really what we expect.
That is, not only the \(n^{\text{th}}\) root of \(x\) is a real number, it also satisfies that the \(n^{\text{th}}\) power of the \(n^{\text{th}}\) root of \(x\) is actually equal to \(x\).
\end{note}

\begin{note}
(a) 先開\ \(n\) 次根再取\ \(n\) 次方會等於自己。
(b) 先取\ \(n\) 次方再開\ \(n\) 次根後會等於自己。
(c) 非負實數開幾次根都還是非負;而正實數根幾次根都是正數。
(d) 若\ \(x > y\),則\ \(x\) 開\ \(n\) 次根大於\ \(y\) 開\ \(n\) 次根。
(f) 先相乘再開\ \(n\) 次根等於先各自開\ \(n\) 次根再相乘。
(g) 先開\ \(n\) 次根再開\ \(m\) 次根等於直接開\ \(n \X m\) 次根。
\end{note}

\begin{proof}
Let \(n\) be arbitrary positive integer, and \(x, y\) be non-negative real.
\begin{enumerate}
\item
    Suppose \(y = x^{1 / n} = \sup E\) where \(E := \{ z \in \SET{R}: z \ge 0 \land z^n \le x \}) \).
    We have to show \(y^n = x\).
    Like \PROP{5.5.12}, we will show that both \(y^n < x\) and \(y^n > x\) lead to contradictions.
    \begin{itemize}
    \item [\(y^n < x\)]:
        Then we first show that given arbitrary positive integer \(n\), and given arbitrary non-negative real number \(y\), there exists a positive real number \(M\) s.t.
        \emph{for all real} \(0 < \delta < 1\), \((y + \delta)^n \le y^n + \delta M\) \MAROON{(*)}.
        
        We prove this by induction with the base case \(n = 1\).
        
        For the base case \(n = 1\), given arbitrary \(y \ge 0\); then for all \(0 < \delta < 1\),
        \begin{align*}
        (y + \delta)^1 & = y + \delta & \text{from \DEF{5.6.1}} \\
                       & = y^1 + \delta & \text{from \DEF{5.6.1}} \\
                       & = y^1 + 1 \X \delta, & \text{by \PROP{4.2.4}(7), with \(\delta\) as real number} \\
                       & \le y^1 + 1 \X \delta, & \text{in particular}
        \end{align*}
        so we can find \(M = 1\) as required (BTW this derivation in fact does not depend on the value of \(\delta\)).
        
        Suppose inductively that for some integer \(n \ge 1\), given arbitrary \(y \ge 0\), we can find some positive real \(M\) s.t. for all \(0 < \delta < 1\), \((y + \delta)^n \le y^n + M\delta\);
        we have to find another positive real \(M'\) s.t. for all \(0 < \delta < 1\), \((y + \delta)^{n + 1} \le y^{n + 1} + M' \delta \).
        But
        \begin{align*}
            (y + \delta)^{n + 1} & = (y + \delta)(y + \delta)^n & \text{of course} \\
                                 & \le (y + \delta)(y^n + M\delta) & \text{by inductive hypothesis} \\
                                 & = y^{n + 1} + \delta y^n + yM\delta + M\delta^2 & \text{expand by algebra} \\
                                 & \le y^{n + 1} + \delta y^n + yM\delta + M\delta & \text{since \(0 < \delta < 1\), \(\delta^2 < \delta\)} \\
                                 & = y^{n + 1} + (y^n + yM + M)\delta & \text{of course}
        \end{align*}
        Now let \(M' = y^n + yM + M\), then of course \(M'\) is positive, and for all \(0 < \delta < 1\) we have the inequalities as desired.
        So this closes the induction.
        
        So, from the induction \MAROON{(*)}, given positive integer \(n\) and non-negative real \(y\), we can get that \(M\);
        and once we find \(M\), we want to find a particular \(\varE\) s.t. \(0 < \varE < 1\) and:
        \begin{align*}
                     & y^n < x \\
            \implies & 0 < x - y^n \\
            \implies & 0 < (x - y^n)/M \\
            \implies & 0 < \varE < (x - y^n)/M & \text{for some \(\varE\) by \PROP{5.4.14} or \EXEC{5.5.5}} \\
            \implies & 0 < \varE M < x - y^n \\
            \implies & y^n < y^n + \varE M < x.
        \end{align*}
        So we can found that \(1 > \varE > 0\) s.t. \(y^n + \varE M < x\) \MAROON{(1)}.
        So in particular, \(0 < \varE < 1\), and from the induction \MAROON{(*)} we have \((y + \varE)^n \le y^n + \varE M\), and with \MAROON{(1)} we conclude that \((y + \varE)^n < x\).
        So by construction of \(E\), \(y + \varE \in E\), but that contradicts \(y = x^{1/n} = \sup E\) is the least upper bound of \(E\).
    \item [\(y^n > x\)]:
        Then similarly as previous case, we first show that given arbitrary positive integer \(n\), and given arbitrary non-negative real number \(y\), there exists a positive real number \(M\) s.t.
        \emph{for all real} \(0 < \delta < 1\), \((y - \delta)^n \ge y^n - \delta M\) \MAROON{(**)}.
        
        We prove this by induction with the base case \(n = 1\).
        
        For the base case \(n = 1\), given arbitrary \(y \ge 0\); then for all \(0 < \delta < 1\),
        \begin{align*}
        (y - \delta)^1 & = y - \delta & \text{of course} \\
                       & = y^1 - \delta & \text{of course} \\
                       & = y^1 - 1 \X \delta, & \text{of course!!} \\
                       & \ge y^1 - 1 \X \delta, & \text{in particular!}
        \end{align*}
        so we can find \(M = 1\) as required (BTW this derivation in fact does not depend on the value of \(\delta\)).
        
        Suppose inductively that for some integer \(n \ge 1\), given arbitrary \(y \ge 0\), we can find some positive real \(M\) s.t. for all \(0 < \delta < 1\), \((y - \delta)^n \ge y^n - M\delta\);
        we have to find another positive real \(M'\) s.t. for all \(0 < \delta < 1\), \((y - \delta)^{n + 1} \ge y^{n + 1} - M' \delta \).
        But
        \begin{align*}
            (y - \delta)^{n + 1} & = (y - \delta)(y - \delta)^n & \text{of course} \\
                                 & \ge (y - \delta)(y^n - M\delta) & \text{by inductive hypothesis} \\
                                 & = y^{n + 1} - \delta y^n - yM\delta + M\delta^2 & \text{expand by algebra} \\
                                 & \ge y^{n + 1} - \delta y^n - yM\delta & \text{since \(M\delta^2 > 0\)} \\
                                 & = y^{n + 1} - (y^n + yM)\delta & \text{of course}
        \end{align*}
        Now let \(M' = y^n + yM\), then of course \(M'\) is positive, and for all \(0 < \delta < 1\) we have the inequalities as desired.
        So this closes the induction.
        
        So, from the induction \MAROON{(**)}, given positive integer \(n\) and non-negative real \(y\), we can get that \(M\);
        and once we find \(M\), we want to find a particular \(\varE = -\varE'\) s.t. \(0 > \varE' > -1\) (hence \(0 < \varE < 1\)) and:
        \begin{align*}
                     & y^n > x \\
            \implies & 0 > x - y^n \\
            \implies & 0 > (x - y^n)/M \\
            \implies & 0 > \varE' > (x - y^n)/M & \text{for some \(\varE'\) by \PROP{5.4.14} or \EXEC{5.5.5}} \\
            \implies & 0 > -\varE > (x - y^n)/M \\
            \implies & 0 > -\varE M > x - y^n \\
            \implies & y^n > y^n - \varE M > x.
        \end{align*}
        So we can found that \(1 > \varE > 0\) s.t. \(y^n - \varE M > x\) \MAROON{(2)}.
        So in particular, \(0 < \varE < 1\), and from the induction \MAROON{(**)} we have \((y - \varE)^n \ge y^n - \varE M\), and with \MAROON{(2)} we conclude that \((y - \varE)^n > x\).
        So by construction of \(E\), that implies \((y - \varE)^n\) is an upper bound of \(E\)(since \(x\) is an upper bound of \(E\));
        but that contradicts \(y = x^{1/n} = \sup E\) is the \emph{least} upper bound of \(E\).
    \end{itemize}
    So both \(y^n < x\) and \(y^n > x\) lead to contradictions, so we have \(y^n = x\).
\item
    Suppose \(x, y, n\) satisfy the given condition.
    Now suppose \(y^n = x\).
    For the sake of contradiction, suppose \(y \neq x^{1/n}\). We will show both \(y < x^{1/n}\) and \(y > x^{1/n}\) lead to contradictions.
    
    \begin{itemize}
        \item[\(y < x^{1/n}\)]:
            Then
            \begin{align*}
                         & y < x^{1/n} \\
                \implies & y^n < (x^{1/n})^n & \text{\(n\)'s positive int, with \PROP{4.3.10}(c)} \\
                \implies & y^n < x & \text{by part (a)}
            \end{align*}
            which is a contradiction because \(y^n = x\).
        \item[\(y > x^{1/n}\)]:
            Then
            \begin{align*}
                         & y > x^{1/n} \\
                \implies & y^n > (x^{1/n})^n & \text{\(n\)'s positive int, with \PROP{4.3.10}(c)} \\
                \implies & y^n > x & \text{by part (a)}
            \end{align*}
            which is a contradiction because \(y^n = x\).
    \end{itemize}
\item
    By definition, \(x^{1/n} = \sup E\) where \(E = \{ y \in \SET{R}: y \ge 0 \land y^n \le x \}\).
    But in the proof of \LEM{5.6.5}, we have concluded that \(E\) contains \(0\), so the least upper bound of \(E\) must be greater than or equal to \(0\).
    That is, \(x^{1/n} \ge 0\).

    Now suppose \(x^{1/n}\) is positive.
    Then by \PROP{4.3.10}(c), we can conclude \((x^{1/n})^{n} > 0\).
    But by part(a), we know \((x^{1/n})^n = x\), so \(x > 0\).

    Now suppose \(x > 0\).
    Suppose for the sake of contradiction that \(x^{1/n}\) is not greater than \(0\).
    Since we have shown that \(x^{1/n} \ge 0\), so \(x^{1/n} = 0\).
    But by \PROP{4.3.10}(b), since \(x^{1/n} = 0\), we have \((x^{1/n})^n = 0\).
    By part(a), that implies \(x = (x^{1/n})^n = 0\), a contradiction.
\item
    Suppose \(x > y\) \MAROON{(3)}.
    Suppose for the sake of contradiction that \(x^{1/n} \le y^{1/n}\).
    If \(x^{1/n} \le y^{1/n}\), then from part(c) we have shown that both \(x^{1/n}, y^{1/n}\) are non-negative, so we have \(0 \le x^{1/n} \le y^{1/n}\).
    From \PROP{4.3.10}(c), we have \(0 \le (x^{1/n})^n \le (y^{1/n})^n\).
    By part(a), we have \(0 \le x \le y\), contradicting \MAROON{(3)}.
    So \(x^{1/n} > y^{1/n}\).

    Suppose \(x^{1/n} > y^{1/n}\).
    Then again by part(c), we have \(x^{1/n} > y^{1/n} \ge 0\).
    Also, with \PROP{4.3.10}(c), we have \((x^{1/n})^n > (y^{1/n})^n \ge 0\), in particular  \((x^{1/n})^n > (y^{1/n})^n\).
    From part(a), we have \(x > y\).
\item
    \begin{itemize}
    \item[\(x > 1\)]:
        We have to show \(x^{1/k}\) is a strictly decreasing function of positive integer \(k\).
        That is, we have to show given arbitrary positive integer \(k_1 > k_2\), we have \(x^{1/k_1} < x^{1/k_2}\).
        Suppose for the sake of contradiction that there exist positive integers \(k_1 > k_2\) but \(x^{1/k_1} \ge x^{1/k_2}\).
        By part(c), \(x^{1/k_1} \ge x^{1/k_2} \ge 0\). And
        \begin{align*}
                     & x^{1/k_1} \ge x^{1/k_2} \ge 0 \\
            \implies & (x^{1/k_1})^{k_1} \ge (x^{1/k_2})^{k_1} \ge 0 & \text{by \PROP{4.3.10}(c)} \\
            \implies & x \ge (x^{1/k_2})^{k_1} \ge 0 & \text{by part(a)} \\
            \implies & x^{k_2} \ge ((x^{1/k_2})^{k_1})^{k_2} \ge 0 & \text{by \PROP{4.3.10}(c)} \\
            \implies & x^{k_2} \ge (x^{1/k_2})^{k_1 k_2} \ge 0 & \text{by \PROP{4.3.10}(a)} \\
            \implies & x^{k_2} \ge (x^{1/k_2})^{k_2 k_1} \ge 0 & \text{of course} \\
            \implies & x^{k_2} \ge ((x^{1/k_2})^{k_2})^{k_1} \ge 0 & \text{by \PROP{4.3.10}(a)} \\
            \implies & x^{k_2} \ge x^{k_1} \ge 0 \MAROON{(4)} & \text{by part(a)}
        \end{align*}
        But \(k_1 > k_2\), so \(k_1 = k_2 + r\) for some positive integer \(r\).
        And
        \begin{align*}
            x^{k_1} & = x^{k_2 + r} \\
                    & = x^{k_2} x^{r} & \text{by \PROP{4.3.10}(a)} \\
                    & > x^{k_2} 1 & \text{\(x > 1 \land \text{ (integer) } r > 1 \implies x^r > 1\)} \\
                    & = x^{k_2}
        \end{align*}
        which contradicts \MAROON{(4)} that \(x^{k_2} \ge x^{k_1}\).
        So for \(x > 1\), \(x^{1/k}\) is an strictly decreasing function of positive integer \(k\).
    \item[\(0 < x < 1\)]:
        We have to show \(x^{1/k}\) is a strictly increasing function of positive integer \(k\).
        That is, we have to show given arbitrary positive integer \(k_1 > k_2\), we have \(x^{1/k_1} > x^{1/k_2}\).
        Suppose for the contradiction that there exist positive integers \(k_1 > k_2\) but \(x^{1/k_1} \le x^{1/k_2}\).
        By part(c), \(0 \le x^{1/k_1} \le x^{1/k_2}\). And
        \begin{align*}
                     & 0 \le x^{1/k_1} \le x^{1/k_2} \\
            \implies & 0 \le (x^{1/k_1})^{k_1} \le (x^{1/k_2})^{k_1} & \text{by \PROP{4.3.10}(c)} \\
            \implies & 0 \le x \le (x^{1/k_2})^{k_1} & \text{by part(a)} \\
            \implies & 0 \le x^{k_2} \le ((x^{1/k_2})^{k_1})^{k_2} & \text{by \PROP{4.3.10}(c)} \\
            \implies & 0 \le x^{k_2} \le (x^{1/k_2})^{k_1 k_2} & \text{by \PROP{4.3.10}(a)} \\
            \implies & 0 \le x^{k_2} \le (x^{1/k_2})^{k_2 k_1} & \text{of course} \\
            \implies & 0 \le x^{k_2} \le ((x^{1/k_2})^{k_2})^{k_1} & \text{by \PROP{4.3.10}(a)} \\
            \implies & 0 \le x^{k_2} \le x^{k_1} \MAROON{(5)} & \text{by part(a)}
        \end{align*}
        But \(k_1 > k_2\), so \(k_1 = k_2 + r\) for some positive integer \(r\).
        And
        \begin{align*}
            x^{k_1} & = x^{k_2 + r} \\
                    & = x^{k_2} x^{r} & \text{by \PROP{4.3.10}(a)} \\
                    & < x^{k_2} 1 & \text{\(0 < x < 1 \land \text{ (integer) } r > 1 \implies x^r < 1\)} \\
                    & = x^{k_2}
        \end{align*}
        which contradicts \MAROON{(5)} that \(x^{k_2} \le x^{k_1}\).
        So for \(0 < x < 1\), \(x^{1/k}\) is an strictly increasing function of positive integer \(k\).
    \item[\(x = 1\)]:
        We have to show \(x^{1/k} = 1\) for all positive integer \(k\).
        Suppose for the sake of contradiction there exists a positive integer \(k\) s.t. \(x^{1/k} \neq 1\). Then either \(x^{1/k} > 1\) or \(x^{1/k} < 1\).
        \begin{itemize}
            \item[\(x^{1/k} > 1\)]:
                In particular \(x^{1/k} > 1 \ge 0\).
                Then
                \begin{align*}
                             & x^{1/k} > 1 \ge 0 \\
                    \implies & (x^{1/k})^k > 1^k > 0 & \text{by \PROP{4.3.10}(c)} \\
                    \implies & (x^{1/k})^k > 1 > 0 & \text{of course} \\
                    \implies & x > 1 > 0, & \text{by part(a)}
                \end{align*}
                which contradicts that \(x = 1\).
            \item[\(x^{1/k} < 1\)]:
                Again by part(c), \(0 \le x^{1/k} < 1\). And
                \begin{align*}
                             & 0 \le x^{1/k} < 1 \\
                    \implies & 0 \le (x^{1/k})^k < 1^k & \text{by \PROP{4.3.10}(c)} \\
                    \implies & 0 \le (x^{1/k})^k < 1 & \text{of course} \\
                    \implies & 0 \le x < 1, & \text{by part(a)}
                \end{align*}
                which contradicts that \(x = 1\).
        \end{itemize}
        So \(x^{1/k} = 1\) for all positive integer \(k\).
    \end{itemize}
\item
    By part(a), we have \(((xy)^{1/n})^n = xy\).
    And
    \begin{align*}
        (x^{1/n}y^{1/n})^n & = (x^{1/n})^n (y^{1/n})^n & \text{by \PROP{4.3.10}(a)} \\
                           & = x (y^{1/n})^n & \text{by part(a)} \\
                           & = x y & \text{by part(a)}
    \end{align*}
    So we have \(((xy)^{1/n})^n = (x^{1/n}y^{1/n})^n\),
    and with part(b), we have \((xy)^{1/n} = ((x^{1/n}y^{1/n})^n)^{1/n}\).
    But again by part(b), RHS is equal to \(x^{1/n}y^{1/n}\).
    So we have \((xy)^{1/n} = x^{1/n}y^{1/n}\), as desired.
\item
    \begin{align*}
        ((x^{1/n})^{1/m})^{nm} & = ((x^{1/n})^{1/m})^{mn} & \text{of course} \\
                               & = (((x^{1/n})^{1/m})^m)^n & \text{by \PROP{4.3.10}(a)} \\
                               & = (x^{1/n})^n & \text{by part(a), \(((x^{1/n})^{1/m})^m = x^{1/n}\)} \\
                               & = x & \text{by part(a)}
    \end{align*}
    And by part(a), \((x^{1/mn})^{mn} = x\).
    So together we have \((x^{1/n})^{1/m})^{nm} = (x^{1/mn})^{mn}\),
    and with part(b), we have \((x^{1/n})^{1/m} = ((x^{1/mn})^{mn})^{1/mn}\).
    But again by part(b), RHS is equal to \(x^{1/mn}\).
    So we have \((x^{1/n})^{1/m} = x^{1/mn}\), as desired.
\end{enumerate}
\end{proof}

\begin{note}
The observant reader may note that this definition of \(x^{1/n}\) might possibly be inconsistent with our previous notion of \(x^n\) when \(n = 1\).
The point is that by \DEF{5.6.1}, we have defined what \(x^1\) means.
But since \(1 = 1/1\), we can also define \(x^{1/1}\) using \DEF{5.6.4}.
So we have to check these two definitions give the same real number.
But it is easy to check since they both give the real number \(x\)
(why? \(x^1\) by \DEF{5.6.1} is trivially equal to \(x\).
For \(x^{1/1}\), for \(n = 1\), by \LEM{5.6.6}(a), \((x^{1/n})^n = x\), that is, \((x^{1/1})^1 = x\), but again by \DEF{5.6.1} \((x^{1/1})^1\) is equal to \(x^{1/1}\), so we have \(x^{1/1} = x\).
So both \(x^1\) and \(x^{1/1}\) equal to the real number \(x\) therefore are equal to each other.),
so there is no inconsistency.
\end{note}

\begin{note}
One consequence of \LEM{5.6.6}(b) is the following cancellation law:
if \(y\) and \(z\) are positive and \(y^n = z^n\), then \(y = z\). (Why? From \LEM{5.6.6}(b), we have \(y = (z^n)^{1/n}\). But also from \LEM{5.6.6}(b), we also have \(z = (z^n)^{1/n}\), so \(y = z\).)
Note that this only works when \(y\) and \(z\) are positive;
for instance, \((-3)^2 = 3^2\), but we cannot conclude from this that \(-3 = 3\).
\end{note}

Now we define how to raise a positive number \(x\) to a \emph{rational} exponent \(q\).

\begin{definition} \label{def 5.6.7}
Let \(x > 0\) be a positive real number, and let \(q\) be a rational number.
To define \(x^q\), we write \(q = a/b\) for some integer \(a\) and \emph{positive} integer \(b\), and define
\[
    x^q := (x^{1/b})^a.
\]
\end{definition}

\begin{note}
Note that every rational \(q\), whether positive, negative, or zero, can be written in the form \(a/b\) where \(a\) is an integer and \(b\) is \emph{positive} (why? This is a direct consequence of \DEF{4.2.6}).
However, the rational number \(q\) can be expressed in the form \(a/b\) in more than one way, for instance \(1/2\) can also be expressed as \(2/4\) or \(3/6\).
So to ensure that this definition is well-defined, \emph{we need to check that different expressions \(a/b\) give the same formula for \(x^q\)}:
\end{note}

\begin{note}
意思就是若\ \(a/b = a'/b'\),則根據 \DEF{5.6.7},我們要確定先開\ \(b\) 次根再取\ \(a\) 次方,跟先開\ \(b'\) 次根再取\ \(a'\) 次方,得到的結果要是一樣的。
\end{note}

\begin{lemma} \label{lem 5.6.8}
Let \(a, a'\) be integers and \(b, b'\) be positive integers such that \(a/b = a'/b'\),
and let \(x\) be a positive real number.
Then we have \((x^{1/b'})^{a'} = (x^{1/b})^a\) (so that by \DEF{5.6.7} \(x^{a/b} = x^{a'/b'}\)).
\end{lemma}

\begin{proof}
There are three cases: \(a = 0, a > 0, a < 0\).
If \(a = 0\), then we must have \(a' = 0\)
(why? By \DEF{4.2.1}, \(a/b = a'/b'\) iff \(ab' = a'b\); but LHS = \(0 \X b' = 0\), so RHS \(a'b = 0\); but \(b\) is positive, by \PROP{4.1.8}, we have \(a' = 0\).)
and so both \((x^{1/b'})^{a'} = (x^{1/b'})^0 = 1\) and \((x^{1/b})^a = (x^{1/b})^0 = 1\), so we are done.

Now suppose that \(a > 0\).
Then \(a' > 0\)
(why? Again by \DEF{4.2.1} we have \(ab' = a'b\). Since both \(a, b' > 0\), \(ab' > 0\), so LHS is positive, so RHS \(a'b\) is positive;
and \(b\) also positive, so \(a'\) must be positive.
We can use \AC{4.2.5} and \AC{4.2.6} in each step, although they are for rationals, but integers are also rationals.),
and by \DEF{4.2.1}, we have \(ab' = a'b\), which is equal to \(ba'\).
So write \(y := x^{1/(ab')} = x^{1/(ba')}\).
By \LEM{5.6.6}(g) we have \(y = (x^{1/b'})^{1/a}\) and \(y = (x^{1/b})^{1/a'}\);
by \LEM{5.6.6}(a) we thus have \(y^a = x^{1/b'}\) \MAROON{(1)} and \(y^{a'} = x^{1/b}\) \MAROON{(2)}.
Thus we have
\begin{align*}
    (x^{1/b'})^{a'} & = (y^a)^{a'} & \text{by \MAROON{(1)}} \\
                    & = y^{aa'} & \text{by \PROP{4.3.10}(a)} \\
                    & = (y^{a'})^a & \text{by \PROP{4.3.10}(a)} \\
                    & = (x^{1/b})^a & \text{by \MAROON{(2)}}
\end{align*}
as desired.

Finally, suppose that \(a < 0\). Then we have \((-a)/b = (-a')/b\).
But \(-a\) is positive, so the previous case applies and we have \((x^{1/b'})^{-a'} = (x^{1/b})^{-a}\).
And thus \(((x^{1/b'})^{-a'})^{-1} = ((x^{1/b})^{-a})^{-1}\).
By \PROP{4.3.12}(a), we have \((x^{1/b'})^{-a' \X -1} = (x^{1/b})^{-a \X -1}\), that is, \((x^{1/b'})^{a'} = (x^{1/b})^{a}\), as desired.
\end{proof}

Thus \(x^q\) is well-defined for every rational \(q\).
Note that this new definition is consistent with our old definition(\DEF{5.6.4}) for \(x^{1/n}\)
(why? since
\begin{align*}
    x^{\MAROON{1}/\BLUE{n}} & = (x^{1/\BLUE{n}})^{\MAROON{1}} & \text{by \DEF{5.6.7}} \\
                            & = x^{1/n} & \text{of course}
\end{align*}
)
and is also consistent with our old definition(\DEF{5.6.1}) for \(x^n\)
(why? since
\begin{align*}
    x^n & = x^{\MAROON{n}/\BLUE{1}} & \text{of course} \\
        & = (x^{1/\BLUE{1}})^{\MAROON{n}} & \text{by \DEF{5.6.7}} \\
        & = x^{\MAROON{n}} & \text{we have checked the consistency below the page \(123\) that \(x^{1/1} = x^1 = x\)}
\end{align*}
).

Some basic facts about rational exponentiation:

\begin{lemma} \label{lem 5.6.9}
Let \(x, y > 0\) be positive reals, and let \(q, r\) be rationals.
\begin{enumerate}
    \item \(x^q\) is a \emph{positive} real.
    \item \(x^{q+r}\ = x^q x^r\) and \((x^q)^r = x^{qr}\).
    \item \(x^{-q} = 1/x^q\).
    \item If \(q > 0\), then \(x > y\) if and only if \(x^q > y^q\).
    \item If \(x > 1\), then \(x^q > x^r\) if and only if \(q > r\). If \(x < 1\), then \(x^q > x^r\) if and only if \(q < r\).
    \item \((xy)^q = x^q y^q\).
\end{enumerate}
\end{lemma}

\begin{proof}
\begin{enumerate}
\item
    Let arbitrary \(q = a/b\) where \(a\) is integer and \(b\) is positive integer.
    Then
    \begin{align*}
        x^q & = x^{a/b} \\
            & = (x^{1/b})^a & \text{by \DEF{5.6.7}}
    \end{align*}
    Since \(x\) is positive and \(b\) is positive integer, by \LEM{5.6.6}(c), \(x^{1/b}\) is positive.
    And since \(x^{1/b}\) is positive and \(a\) is just an integer, by \PROP{5.6.3}(or precisely, \PROP{4.3.12}(b)), \((x^{1/b})^a\) is still positive.
    So \(x^q\) is positive for all rationals \(q\).
\item
    Let \(q = a / b\) and \(r = c / d\) where \(a, c\) are integers and \(b, d\) are positive integers.
    Then
    \begin{align*}
        x^{q + r} & = x^{a/b + c/d} \\
                  & = x^{(ad + bc)/bd} & \text{by \DEF{4.2.2}} \\
                  & = (x^{1/bd})^{ad + bc} & \text{by \DEF{5.6.7}} \\
                  & = (x^{1/bd})^{ad} (x^{1/bd})^{bc} & \text{by \PROP{4.3.12}(a)} \\
                  & = x^{ad/bd} x^{bc/bd} & \text{by \DEF{5.6.7}} \\
                  & = x^{a/b} x^{c/d} & \text{\(ad/bd = a/b, bc/bd = c/d\), and \LEM{5.6.8}} \\
                  & = x^q x^r
    \end{align*}

    For the equation \(x^{qr} = (x^q)^r\), we first show that \(((x^q)^r)^{bd} = (x^{qr})^{bd} = x^{ac}\):
    \begin{align*}
        ((x^q)^r)^{bd} & = ((x^{a/b})^{c/d})^{bd} \\
                       & = ((((x^{1/b})^a)^{1/d})^c)^{bd} & \text{by \DEF{5.6.7}} \\
                       & = (((x^{1/b})^a)^{1/d})^{c \X bd} & \text{\(c, bd\) are int, with \PROP{4.3.12}(a)} \\
                       & = ((\BLUE{((x^{1/b})^a)}^{1/d})^d)^{bc} & \text{again by \PROP{4.3.12}(a)} \\
                       & = ((x^{1/b})^a)^{bc} & \text{\(d\)'s positive int, with \LEM{5.6.6}(a)} \\
                       & = ((\BLUE{x}^{1/b})^b)^{ac} & \text{\(a, b, c\) int, apply \PROP{4.3.12}(a) several times} \\
                       & = x^{ac} & \text{\(b\)'s positive int, with \LEM{5.6.6}(a)}
    \end{align*}
    and
    \begin{align*}
        (x^{qr})^{bd} & = (x^{ac/bd})^{bd} \\
                      & = ((x^{1/(bd)})^{ac})^{bd} & \text{by \DEF{5.6.7}} \\
                      & = ((x^{1/(bd)})^{bd})^{ac} & \text{\(ac, bd\) are int, apply \PROP{4.3.12}(a) two times} \\
                      & = x^{ac} & \text{by \LEM{5.6.6}(a)}
    \end{align*}
    Thus \(((x^q)^r)^{bd} = (x^{qr})^{bd}\).
    Since \((x^q)^r\) and \(x^{qr}\) are positive by part(a), and \(bd\) is integer not equal to zero, by \PROP{4.3.12}(c) we have \((x^q)^r = x^{qr}\), as desired.
\item
    We first show that \(x^{-n} = 1/x^n\) \BLUE{(*)} for an integer \(n\), \textbf{regardless of the sign} of \(n\).
    If \(n\) is positive, then this follows by \DEF{5.6.2}.
    If \(n = 0\), then \(-n = 0\) so both sides of \BLUE{(*)} are equal to \(1\).
    If \(n\) is negative, then \(n = -m\) for some positive integer \(m\).
    We have \(x^{-n} = x^{m}\) and
    \begin{align*}
        1/x^n & = 1/x^{-m} \\
              & = 1/(1/x^m) & \text{by previous case that \(m\) is positive} \\
              & = x^m & \text{need to prove but trivial} \\
              & = x^{-n}.
    \end{align*}
    So again \BLUE{(*)} is satisfied.

    Now let \(q = a/b\) for some integer \(a\) and positive integer \(b\).
    Then
    \begin{align*}
        x^{-q} & = x^{-(a/b)} \\
               & = x^{(-a)/b} & \text{by \AC{4.2.3}} \\
               & = (x^{1/b})^{-a} & \text{by \DEF{5.6.7}} \\
               & = 1/(x^{1/b})^{a} & \text{by \BLUE{(*)}} \\
               & = 1/x^q & \text{by \DEF{5.6.7}}
    \end{align*}
\item
    Suppose rational \(q = a/b > 0\) for some \emph{positive} integer \(a, b\).
    \begin{itemize}
    \item[\(\Longrightarrow\)]
        \begin{align*}
                     & x > y \\
            \implies & x^{1/b} > y^{1/b} & \text{\(b\)'s positive int,} \\
                     &                   & \text{by \LEM{5.6.6}(d)'s \(\Longrightarrow\) direction} \\
            \implies & (x^{1/b})^a > (y^{1/b})^a & \text{\(a\)'s positive int, by \PROP{4.3.10}(c)} \\
            \implies & x^q > y^q & \text{by \DEF{5.6.7}}
        \end{align*}
    \item[\(\Longleftarrow\)]
        \begin{align*}
                     & x^q > y^q \\
            \implies & x^{a/b} > y^{a/b} \\
            \implies & (x^{1/b})^a > (y^{1/b})^a & \text{by \DEF{5.6.7}} \\
            \implies & ((x^{1/b})^a)^{1/a} > ((y^{1/b})^a)^{1/a} & \text{\(a\)'s positive int,} \\
                     &                                           & \text{by \LEM{5.6.6}(d)'s \(\Longrightarrow\) direction} \\
            \implies & x^{1/b} = ((x^{1/b})^a)^{1/a} > ((y^{1/b})^a)^{1/a} = y^{1/b} & \text{by \LEM{5.6.6}(b)} \\
            \implies & x^{1/b} > y^{1/b} & \text{simplify} \\
            \implies & x > y & \text{\(b\)'s positive int,} \\
                     &       & \text{by \LEM{5.6.6}(d)'s \(\Longleftarrow\) direction}
        \end{align*}
    \end{itemize}
\item
    Let \(q = a/b, r = c/d\) for some integers \(a, b\) and for some positive integers \(b, d\).
    \begin{itemize}
    \item
        Suppose \(x > 1\).

        Suppose \(x^q > x^r\), we have to show \(q > r\).
        Then
        \begin{align*}
                     & x^q > x^r \\
            \implies & x^q x^{-r} > x^r x^{-r} & \text{\(x^{-r}\) is positive by part(a)}\\
            \implies & x^{q - r} > x^{r - r} & \text{by part(b)} \\
            \implies & x^{q - r} > x^0 = 1 & \text{of course} \\
            \implies & x^{a/b - c/d} > 1 & \\
            \implies & x^{(ad - bc)/bd} > 1 & \text{by \DEF{4.2.2}} \\
            \implies & x^{(ad - bc) \X (1/bd)} > 1 \\
            \implies & (x^{ad - bc})^{1/bd} > 1 & \text{by part(b)} \\
            \implies & ((x^{ad - bc})^{1/bd})^{bd} > 1^{bd} = 1 & \text{\(bd > 0\), by part(d)} \\
            \implies & x^{ad - bc} = ((x^{ad - bc})^{1/bd})^{bd} > 1 & \text{\(bd\)'s positive int, with \LEM{5.6.6}(a)} \\
            \implies & x^{ad - bc} > 1 \MAROON{(*)}
        \end{align*}
        Now we claim that \(ad - bc > 0\).
        It's trivial that \(ad - bc \neq 0\).
        If \(ad - bc < 0\), since \(x > 1\), from \PROP{4.3.12}(b), we have \(x^{ad - bc} < 1^{ad - bc} = 1\), which contradicts \MAROON{(*)}.
        And hence
        \begin{align*}
                     & ad - bc > 0 \\
            \implies & (ad - bc)/bd > 0 \X 1/bd = 0 & \text{(\(1/bd > 0\), with \PROP{4.2.9}(e))} \\
            \implies & (a/b - c/d) > 0 & \text{by \DEF{4.2.2}} \\
            \implies & a/b > c/d \\
            \implies & q > r
        \end{align*}
    
        Now suppose \(q > r\), we have to show \(x^q > x^r\).
        Then
        \begin{align*}
                     & q > r \\
            \implies & q - r > 0 \\
            \implies & x^{q - r} > 1^{q - r} & \text{\(q - r > 0 \land x > 1\), with part(d)} \\
            \implies & x^{q - r} > 1^{(ad - bc)/bd} & \text{by \DEF{4.2.2}} \\
            \implies & x^{q - r} > (1^{1/bd})^{ad - bc} & \text{by \DEF{5.6.7}} \\
            \implies & x^{q - r} > 1^{ad - bc} = 1 & \text{\(bd\)'s positive integer, with \LEM{5.6.6}(e)} \\
            \implies & (x^{q - r}) \X x^r > 1 \X x^r = x^r & \text{\(x^r > 0\), by part(a)} \\
            \implies & x^{q - r + r} > x^r & \text{by part(b)} \\
            \implies & x^q > x^r
        \end{align*}
    \item
        Suppose \(x < 1\).

        Suppose \(x^q > x^r\), we have to show \(q < r\).
        Then
        \begin{align*}
                     & x^q < x^r \\
            \implies & x^q x^{-r} < x^r x^{-r} & \text{\(x^{-r}\) is positive by part(a)}\\
            \implies & x^{q - r} < x^{r - r} & \text{by part(b)} \\
            \implies & x^{q - r} < x^0 = 1 & \text{of course} \\
            \implies & x^{a/b - c/d} < 1 & \\
            \implies & x^{(ad - bc)/bd} < 1 & \text{by \DEF{4.2.2}} \\
            \implies & x^{(ad - bc) \X (1/bd)} < 1 \\
            \implies & (x^{ad - bc})^{1/bd} < 1 & \text{by part(b)} \\
            \implies & ((x^{ad - bc})^{1/bd})^{bd} < 1^{bd} = 1 & \text{\(bd > 0\), by part(d)} \\
            \implies & x^{ad - bc} = ((x^{ad - bc})^{1/bd})^{bd} < 1 & \text{\(bd\)'s positive int, with \LEM{5.6.6}(a)} \\
            \implies & x^{ad - bc} < 1 \MAROON{(**)}
        \end{align*}
        Now we claim that \(ad - bc < 0\).
        It's trivial that \(ad - bc \neq 0\).
        If \(ad - bc > 0\), since \(x > 1\), from \PROP{4.3.12}(b), we have \(x^{ad - bc} > 1^{ad - bc} = 1\), which contradicts \MAROON{(**)}.
        And hence
        \begin{align*}
                     & ad - bc < 0 \\
            \implies & (ad - bc)/bd < 0 \X 1/bd = 0 & \text{(\(1/bd > 0\), with \PROP{4.2.9}(e))} \\
            \implies & (a/b - c/d) < 0 & \text{by \DEF{4.2.2}} \\
            \implies & a/b < c/d \\
            \implies & q < r
        \end{align*}
        
        Suppose \(q < r\), we have to show \(x^q > x^r\).
        Then
        \begin{align*}
                     & q < r \\
            \implies & r - q > 0 \\
            \implies & 1^{r - q} > x^{r - q} & \text{\(r - q > 0 \land 1 > x\), with part(d)} \\
            \implies & x^{r - q} < 1^{r - q} \\
            \implies & x^{r - q} < 1^{c/d - a/b} \\
            \implies & x^{r - q} > 1^{(cb - da)/db} & \text{by \DEF{4.2.2}} \\
            \implies & x^{r - q} > (1^{1/db})^{cb - da} & \text{by \DEF{5.6.7}} \\
            \implies & x^{r - q} > 1^{cb - da} = 1 & \text{\(db\)'s positive integer, with \LEM{5.6.6}(e)} \\
            \implies & (x^{r - q}) \X x^q > 1 \X x^q = x^q & \text{\(x^q > 0\), by part(a)} \\
            \implies & x^{r - q + q} > x^q & \text{by part(b)} \\
            \implies & x^r > x^q
        \end{align*}
    \end{itemize}
\item
    Let \(q = a/b\) for some integer \(a\) and positive integer \(b\).
    Then
    \begin{align*}
        (xy)^{q} & = ((xy)^{1/b})^a & \text{by \DEF{5.6.7}} \\
                 & = (x^{1/b} y^{1/b})^a & \text{\(b\)'s positive integer, with \LEM{5.6.6}(f)} \\
                 & = (x^{1/b})^a (y^{1/b})^a & \text{\(a\)'s integer, with \PROP{4.3.12}(a)} \\
                 & = x^q y^q & \text{by \DEF{5.6.7}}
    \end{align*}
\end{enumerate}
\end{proof}

We still have to do \emph{real} exponentiation;
in other words, we still have to define \(x^y\) where \(x > 0\) and \(y\) is a \emph{real} number,
but we will defer that until \SEC{6.7}, once we have formalized the concept of limit.

\exercisesection

\begin{exercise} \label{exercise 5.6.1}
Prove \LEM{5.6.6}.
(Hints: review the proof of \PROP{5.5.12}.
Also, you will find proof by contradiction a useful tool, especially when combined with the trichotomy of order in \PROP{5.4.7} and \PROP{5.4.12}.
The earlier parts of the lemma can be used to prove later parts of the lemma.
With part(e), first show that if \(x > 1\) then \(x^{1/n} > 1\), and if \(x < 1\) then \(x^{1/n} < 1\).)
\end{exercise}

\begin{proof}
See \LEM{5.6.6}.
\end{proof}

\begin{exercise} \label{exercise 5.6.2}
Prove \LEM{5.6.9}.
(Hint: you should rely mainly on \LEM{5.6.6} and on algebra.)
\end{exercise}

\begin{proof}
See \LEM{5.6.9}.
\end{proof}

\begin{exercise} \label{exercise 5.6.3}
If \(x\) is a real number, show that \(\abs{x} = (x^2)^{1/2}\).
\end{exercise}

\begin{proof}
We split the case into \(x \ge 0\), \(x < 0\).
\begin{itemize}
    \item [\(x \ge 0\)]:
        Then by \LEM{5.6.6}(b) we have \(x = (x^2)^{1/2}\).
        But \(\abs{x} = x\), so \(\abs{x} = (x^2)^{1/2}\).
    \item [\(x < 0\)]:
        Then \(-x > 0\) and by \LEM{5.6.6}(b) we have \(-x = ((-x)^2)^{1/2}\).
        But \((-x)^2 = x^2\), so \(-x = (x^2)^{1/2}\).
        But \(\abs{x} = -x\), so \(\abs{x} = (x^2)^{1/2}\).
\end{itemize}
\end{proof}