\section{Gaps in the rational numbers} \label {sec 4.4}

\begin{note}
Like \SEC{4.3}, I will also skip some proofs of algebraic equations or some equivalent inequalities.
\end{note}

\begin{proposition} [Interspersing of integers by rationals] \label{prop 4.4.1}
Let \(x\) be a \emph{rational} number.
Then there exists an \emph{integer} \(n\) such that \(n \le x < n + 1\).
In fact, this integer is \emph{unique} (i.e., for each \(x\) there is only one \(n\) for which \(n \le x < n + 1\)).
In particular, there exists a \emph{natural} number \(N\) such that \(N > x\) (i.e., there is no such thing as a \emph{rational} number which is larger than all the \emph{natural} numbers).
\end{proposition}

\begin{note}
最後一段的意思就是,給定任一個有理數,我都找得到比它大的自然數。
\end{note}

\begin{remark} \label{remark 4.4.2}
The integer \(n\) for which \(n \le x < n + 1\) is sometimes referred to as the \emph{integer} part of \(x\) and is sometimes denoted \(n = \FLOOR{x}\).
\end{remark}

\begin{proof}
Let \(x\) be an arbitrary rational number.
Then by \LEM{4.2.7}, exactly one of the following cases is true:
\begin{itemize}
    \item \(x\) is positive:
        Then by \DEF{4.2.6} \(x = a/b\) for some \emph{positive} integers \(a, b\).
        Also since \(a, b\) are positive integers(i.e. naturals), by \PROP{2.3.9}(Euclidean division), \(a = qb + r\) for some natural number \(q, r\) where \(0 \le r < b\).
        So
        \begin{align*}
            x & = a/b = (qb + r)/b = qb/b + r/b \\
              & = q + r/b
        \end{align*}
        Since \(r < b\), \(r/b < 1\).
        Since both \(r, b\) are natural(i.e. non-negative), and \(b \neq 0\), again by \DEF{4.2.6} \(r/b \ge 0\).
        Together we have \(q \le q + r/b < q + 1\), that is, \(q \le x < q + 1\), where \(q\) is a natural number, i.e. integer.
    \item \(x = 0\):
        Then \(0 \le x < 0 + 1\).
    \item \(x\) is negative:
        Then \(-x\) is positive.
        By what we have shown, there exist a natural number \(q\) s.t. \(q \le -x < q + 1\), which implies \(-(q + 1) < x \le -q\).
        Now there are two cases:
        \begin{itemize}
            \item[>>] \(x \neq -q\):
                Then we have \(-(q + 1) < x \MAROON{\ <} -q\), and in particular by \DEF{4.2.8} \(-(q + 1) \MAROON{\ \le\ } x < -q\).
                Let \(q' := -(q + 1)\), then we have \(q' \le x < q' + 1\).
            \item[>>] \(x = -q\):
                Then we have \(-q \le x < -q + 1\).
                Let \(q' := -q\), then we have \(q' \le x < q' + 1\).
        \end{itemize}
\end{itemize}

Now we show the uniqueness.
For the sake of contradiction, suppose there exists an rational number \(x\) such that there exist integers \(n, m\) such that \(n \neq m\) and \(n \le x < n + 1\) \MAROON{(1)} and \(m \le x < m + 1\) \MAROON{(2)}.
WLOG, suppose \(n < m\)(or if \(m < n\), we can rename or relabel the variable).
Then with \MAROON{(1)(2)} and \(n < m\), we have \(n < m \le x < n + 1\), that implies \(n < m < n + 1\).
Since \(n, m\) are integers, by \DEF{4.1.10} and \LEM{4.1.11}(a), \(m = n + d_1\) and \(n + 1 = m + d_2\) for some \emph{positive} integers \(d_1, d_2\).
Replacing \(m\) with \(n + d_1\) in the second equation, we have \(n + 1 = (n + d_1) + d_2\), which implies \(1 = d_1 + d_2\).
But since \(d_1, d_2\) are positive, that implies \(d_1 \ge 1\) and \(d_2 \ge 1\) and \(d_1 + d_2 \ge 2 > 1\), so \(1 = d_2 + d_2 > 1\), which is impossible!
So we must have \(n = m\).

Finally we have to find a \textbf{natural} number \(N\) such that \(N > x\).
If \(x \ge 0\), then by what we have shown, \(n \le x < n + 1\) for some integer \(n\).
And since \(x \ge 0\), \(n + 1 > x\) is a \emph{positive} integer, i.e. a natural number, so we have found a natural number \(N := n + 1 > x\).
If \(x < 0\) then \(N := 0\) is what we need.
\end{proof}

Also, between every two rational numbers there is at least one additional rational:

\begin{proposition} [Interspersing of rationals by rationals] \label{prop 4.4.3}
If \(x\) and \(y\) are two rationals such that \(x < y\), then there exists a third rational \(z\) such that \(x < z < y\).
\end{proposition}

\begin{proof}
We set \(z := (x + y)/2\).
Since \(x < y\), and \(1 / 2\) is positive, by \PROP{4.2.9}(e) that \(x / 2 < y / 2\).
If we add \(y / 2\) to both sides using \PROP{4.2.9}(d) we obtain \(x / 2 + y / 2 < y / 2 + y / 2\), i.e., \(z < y\).
If we instead add \(x / 2\) to both sides we obtain \(x / 2 + x / 2 < y / 2 + x / 2\), i.e., \(x < z\).
Thus \(x < z < y\) as desired.
\end{proof}

\begin{note}
Despite the rationals having this \href{https://en.wikipedia.org/wiki/Dense_set}{denseness} property, they are still incomplete;
there are still an infinite number of ``gaps'' or ``holes'' between the rationals, although this denseness property does ensure that these holes are in some sense infinitely small.
For instance, we will now show that the rational numbers do not contain any square root of two.
\end{note}

\begin{note}
``有理數線'' 上面有\ ``無限個'' ``無限小'' 的洞。
\end{note}

\begin{additional corollary} \label{ac 4.4.1}
Let \(n\) be a natural number.
We define \(n\) to be \emph{even} if \(n = 2k\) for some natural number \(k\)., and \emph{odd} if \(n = 2r + 1\) for some natural number \(r\).
Then every natural number is either even or odd, but not both.
\end{additional corollary}

\begin{proof}
We first show that every natural number is either even or odd.
We use induction on \(n\).

For \(n = 0\), since \(\GREEN{0} = 2 \X \BLUE{0}\) for some natural number \BLUE{\(0\)}, by definition \GREEN{\(0\)} is even, so the base case is true.

Suppose inductively that for some natural number \(n \ge 0\), \(n\) is either even or odd, We have to show \(n + 1\) is even or odd.
If \(n\) is odd, then \(n = 2k + 1\) for some natural number \(k\), and \(n + 1 = (2k + 1) + 1 = 2(k +1)\).
By definition, \(n + 1\) is even.
If \(n\) is even, then \(n = 2r\) for some natural number \(r\), and \(n + 1 = 2r + 1\).
By definition, \(n + 1\) is odd.

Now we prove any natural number cannot be both odd and even.
For the sake of contradiction, suppose there exists a natural number \(n\) which is both odd and even.
By definition, \(n = 2k + 1\) for some natural number \(k\) and \(n = 2r\) for some natural number \(r\).
So \(2k + 1 = 2r\).
But viewing them as rationals, we can get \(k + 1/2 = r\), so \(1/2 = r - k\), which is impossible since \(r, k\) are both integers.
So \(n\) cannot be both odd and even.
\end{proof}

\begin{additional corollary} \label{ac 4.4.2}
Let \(n\) be a natural number.
If \(n\) is even, then \(n^2\) is also even.
If \(n\) is odd, then \(n^2\) is also odd.
\end{additional corollary}

\begin{proof}
\begin{align*}
             & n \text{ is even} \\
    \implies & n = 2k \text{ for some natural number k} & \text{by definition of even} \\
    \implies & n^2 = (2k)^2 = 4k^2 \\
    \implies & n^2 = 2(2k^2) \\
    \implies & n^2 \text{ is even} & \text{by definition of even}
\end{align*}

\begin{align*}
             & n \text{ is dd} \\
    \implies & n = 2k + 1 \text{ for some natural number k} & \text{by definition of even} \\
    \implies & n^2 = (2k + 1)^2 = 4k^2 + 4k + 1 \\
    \implies & n^2 = 2(2k^2 + 2k) + 1 \\
    \implies & n^2 \text{ is odd} & \text{by definition of odd}
\end{align*}
\end{proof}

\begin{additional corollary} \label{ac 4.4.3}
Let \(p, q\) be positive integers.
If \(p^2 = 2q^2\) \MAROON{(1)}, then \(q < p\). 
\end{additional corollary}

\begin{proof}
Suppose \(p = q\), then by substitution \AXM{a.7.4} \(p \X p = q \X p\) and \(p \X q = q \X q\).
So we have \(p \X p = q \X p = p \X q = q \X q\), i.e. \(p^2 = q^2\).
Since \(q\) is positive, \(q^2\) is positive, or \(0 < q^2\).
By \LEM{4.1.11}(b), \(0 + p^2 < q^2 + q^2\), or \(p^2 < 2q^2\), contradicting \MAROON{(1)}.

Suppose \(p < q\), then by \LEM{4.1.11}(c) \(p \X p < q \X p\) and \(p \X q < q \X q\).
So we have \(p \X p < q \X p = p \X q < q \X q\), so \(p^2 < q^2\).
And similarly it's trivial that \(p^2 < 2q^2\), contradicting \MAROON{(1)}.

So by order trichotomy of integers(\LEM{4.1.11}(f)), \(q < p\) must be true.
\end{proof}

\begin{proposition} \label{prop 4.4.4}
There does not exist any \emph{rational} number \(x\) for which \(x^2 = 2\).
\end{proposition}

\begin{proof}
Suppose for sake of contradiction that we had a rational number \(x\) for which \(x^2 = 2\).
By \PROP{4.3.10}(b), \(x\) is not zero.
We may assume that \(x\) is positive, for if \(x\) were negative then we could just replace \(x\) by \(-x\) (since \(x^2 = (-x)^2\)).
Thus by \DEF{4.2.6} \(x = p/q\) for some \emph{positive} integers \(p, q\), so \((p/q)^2 = 2\), which we can rearrange as \(p^2 = 2q^2\) \MAROON{(1)}.

Now suppose \(p\) is odd so by \AC{4.4.2} \(p^2\) is odd.
But by \MAROON{(1)} \(p^2 = 2(q^2\) for some integer \(q^2\), so by definition of even, \(p^2\) is also even, contradicting \AC{4.4.1} that \(p^2\) cannot be both odd and even; So similarly, with \AC{4.4.1}, \(p\) must be even.
Then by definition of even, \(p = 2k\) for some natural number \(k\).
Since \(p\) is positive, \(k\) must also be positive.
Replace \(2k\) with \(p\) into \MAROON{(1)} we obtain \(4k^2 =2q^2\), so that \(q^2 = 2k^2\).

So far, we started with a pair \((p, q)\) of positive integers such that \(p^2 = 2q^2\), and ended up with a pair \((q, k)\) of positive integers such that \(q^2 =2k^2\).
From \MAROON{(1)}, by \AC{4.3.3} we have \(q < p\).
If we rewrite \(p' := q\) and \(q' := k\), we thus can pass from one solution \((p, q)\) to the equation \(p^2 = 2q^2\) to a new solution \((p', q')\) to the same equation which has a smaller value of \(p\).
But then we can repeat this procedure again and again, obtaining a sequence \((p'', q''), (p''', q''')\), etc. of solutions to \(p^2 = 2q^2\), each one with a smaller value of \(p\) than the previous, and each one consisting of \emph{positive integers}.
But this contradicts \href{https://www.wikiwand.com/en/Proof_by_infinite_descent}{\emph{the principle of infinite descent}} (\EXEC{4.4.2}).
This contradiction shows that we could not have had a rational \(x\) for which \(x^2 = 2\).
\end{proof}

\begin{note}
我記得在離散課本看到的反證法是說\ WLOG,我們可以假設\ \(p, q\) 互質;若不是的話就先把公因數都除掉;然後就會推出\ \(p, q\) 都是偶數,所以有公因數\ 2,但\ \(p, q\) 互質,於是矛盾。

但這本分析沒有在討論數論,所以也沒定義公因數等等。
\end{note}

On the other hand, we can get rational numbers which are \textbf{arbitrarily close} to a square root of \(2\):

\begin{proposition} \label{prop 4.4.5}
For every rational number \(\varepsilon > 0\), there exists a non-negative rational number \(x\) such that \(x^2 < 2 < (x + \varepsilon)^2\).
\end{proposition}

\begin{proof}
Let \(\varepsilon > 0\) be rational.
Suppose for sake of contradiction that there is no non-negative rational number \(x\) for which \(x^2 < 2 < (x + \varepsilon)^2\).
This means that whenever \(x\) is non-negative and \(x^2 < 2\), we must also have \((x + \varepsilon)^2 < 2\) \BLUE{(ARG1)}
(note that \((x + \varepsilon)^2\) cannot equal \(2\), by \PROP{4.4.4}).
So in particular for \(x := 0\) we have \(0^2 < 2\) and \((0 + \varepsilon)^2 < 2\), i.e. \(\varepsilon^2 < 2\) \MAROON{(1)}.
On the other hand if we let \(x' := \varepsilon\), then by \MAROON{(1)} we have \(x'^2 < \varepsilon\) and \((x' + \varepsilon)^2 < 2\), that is \((\varepsilon + \varepsilon)^2 < 2\), that is \((2\varepsilon)^2 < 2\).
And indeed a simple induction shows that \((n\varepsilon)^2 < 2\) for every natural number \(n\). Also, note that \(n\varepsilon\) is non-negative for every natural number \(n\), by \AC{4.2.5}.

Anyway, this induction is:
\begin{itemize}
    \item
        For \(n = 0\), \((0\varepsilon)^2 = 0^2 = 0 < 2\).
    \item
        Suppose inductively for some \(n \ge 0\), \((n\varepsilon)^2 < 2\) \MAROON{(1)}, we have to show \(((n + 1)\varepsilon)^2 < 2\).
        Then \(((n + 1)\varepsilon)^2 = (n\varepsilon + \varepsilon)^2\).
        But by \MAROON{(1)} and \(n\varepsilon >= 0\), with similar argument in the beginning of this proof(I mean \BLUE{(ARG1)}), we have \((n\varepsilon + \varepsilon)^2 < 2\), as desired.
\end{itemize}
So this closes the induction.

But, by \PROP{4.4.1} we can find an integer \(n\) such that \(n > 2 / \varepsilon\), which implies that \(n\varepsilon > 2\), which implies that \((n\varepsilon)^2 > 4 > 2\), contradicting the claim that \((n\varepsilon)^2 < 2\) for all natural numbers \(n\).
This contradiction gives the proof.
\end{proof}

\begin{note}
The remaining text uses the notation of decimal numbers. Refer to Appendix B.
\end{note}

\begin{example} \label{example 4.4.6}
If \(\varepsilon = 0.001\), we can take \(x = 1.414\), since \(x^2 = 1.999396\) and \((x + \varepsilon)^2 = 2.002225\).
\end{example}

\PROP{4.4.5} indicates that, while the set \(\SET{Q}\) of rationals does not actually have \(\sqrt{2}\) as a member, we can get \textbf{as close as we wish} to \(\sqrt{2}\).
For instance, the \emph{sequence} of \emph{rationals}
\[
    1.4, 1.41, 1.414, 1.4142, 1.41421,...
\]
seem to get closer and closer to \(\sqrt{2}\), as their squares indicate:
\[
    1.96, 1.9881, 1.99396, 1.99996164, 1.9999899241,...
\]
Thus it seems that we can create a square root of \(2\) by taking a \emph{``limit'' of a sequence} of \emph{rationals}.
This is how we shall construct the real numbers in the next chapter.
Other ways are \href{https://www.wikiwand.com/en/Dedekind_cut}{Dedekind cuts} or \emph{infinite} decimal expansions.

\exercisesection

\begin{exercise} \label{exercise 4.4.1}
Prove \PROP{4.4.1}.
\end{exercise}

\begin{proof}
See \PROP{4.4.1}.
\end{proof}

\begin{exercise}\label{exercise 4.4.2}
A definition: a sequence \(a_0, a_1, a_2, ...\) of numbers (natural numbers, integers, rationals, or reals) is said to be in \emph{infinite descent} if we have \(a_n > a_{n + 1}\) for all natural numbers \(n\)
(i.e., \(a_0 > a_1 > a_2 > \dots\)).
\begin{enumerate}
    \item
        Prove the \emph{principle of infinite descent}:
        that it is not possible to have a sequence of \emph{natural numbers} which is in infinite descent.
        (Hint: assume for sake of contradiction that you can find a sequence of \emph{natural numbers} which is in infinite descent.
        Since all the \(a_n\) are natural numbers, you know that \(a_n \ge 0\) for all \(n\).
        Now use induction to show in fact that \(a_n \ge k\) for all \(k \in \SET{N}\) and all \(n \in \SET{N}\), and obtain a contradiction.)
    \item
        Does the principle of infinite descent work if the sequence \(a_1, a_2, a_3, \dots\) is allowed to take integer values instead of natural number values?
        What about if it is allowed to take positive rational values instead of natural numbers? Explain.
\end{enumerate}
\end{exercise}

\begin{proof}
\begin{enumerate}
    \item
        For the sake of contradiction, suppose there exists a sequence of \emph{natural numbers} which is in infinite descent.
        We will prove the statement in the hint.
        We use induction on \(k\).
        For \(k = 0\), since all the \(a_n\) are natural numbers, \(a_n \ge 0 = k\), so the base case is true.
        
        Suppose inductively all the \(a_n \ge k\), we have to show all the \(a_n \ge k + 1\).
        Suppose not.
        Then for some \(n_0 \in \SET{N}\), \(a_{n_0} < k + 1\), and by \PROP{2.2.12}(e), \((a_{n_0})\INC \le k + 1\), that is, \(a_{n_0} + 1 \le k + 1\), that is \(a_{n_0} \le k\).
        But since the sequence is infinite descent, \(a_{n_0 + 1} < a_{n_0}\), so \(a_{n_0 + 1} < k\), a contradiction.
        So all the \(a_n \ge k + 1\).

        So for all \(k \in \SET{N}\), \(a_n \ge k\) for all \(n \in \SET{N}\).
        
        But in particular, \(a_0 \ge k\) for all \(k \in \SET{N}\), which contradicts that there is no largest natural number(on page 27).
    \item
        The principle of infinite descent does not work for both sequence of integers and sequence of rationals.
        That is, there \emph{exist} sequence of integers and sequence of rationals, both of which are infinite descent.

        Example: Let the sequence of integers be \(a_n := -n\) for each \(n \in \SET{N}\)
        Then it's trivial(or by induction) that \(a_{n + 1} < a_n\) since \(-(n + 1) < -n\), so the sequence is infinite descent.

        Now let the sequence of rationals be \(a_n := 1/n\) for each \(n \in \SET{N}\).
        Then \(a_{n + 1} < a_n\) since \(1/(n + 1) < 1/n\), so the sequence is infinite descent.
\end{enumerate}
\end{proof}

\begin{exercise} \label{exercise 4.4.3}
Fill in the gaps marked (why?) in the proof of \PROP{4.4.4}.
\end{exercise}

\begin{proof}
They correspond to \AC{4.4.1} to \AC{4.4.3}.
\end{proof}

\begin{note}
\EXEC{4.4.2}(a) 的\ hint 是要證明那個假設存在的\ sequence 會符合以下陳述: 給定任意一個自然數\ \(k\),這個\ sequence 的每一項都大於等於\ \(k\)。
所以可以\ induction on \(k\),base case \(k = 0\) 是整個\ sequence 都\ \(\ge 0\);inductive hypothesis 是假設整個\ sequence 都\ \(\ge k\),然後得到整個\ sequence 都\ \(\ge k + 1\)。
\end{note}