\section{The rationals} \label{sec 4.2}

Of course, just as two differences \(a - b\) and \(c - d\) can be equal if \(a+d = c+b\), we know (from more advanced knowledge) that two quotients \(a / b\) and \(c / d\) can be equal if \(ad = bc\). (or \(ad = cb\), to be consistent with the definition below.)
Thus, in analogy with the integers, we create a new meaningless symbol \(//\) (which will eventually be superseded by division), and define:

\begin{definition} \label{def 4.2.1}
A \emph{rational number} is an expression of the form \(a \D b\), where \(a\) and \(b\) are \emph{integers} and \(b\) is \emph{non-zero}; \(a \D 0\) is not considered to be a rational number.
Two rational numbers are considered to be equal, \(a \D b = c \D d\), if and only if \(ad = cb\).
The set of all rational numbers is denoted \(\SET{Q}\).
\end{definition}

\begin{note}
Like integers, you can refer to \href{https://www.wikiwand.com/en/Rational_number#/Formal_construction}{wiki}
\end{note}

This is a valid definition of equality (\EXEC{4.2.1}).

\begin{additional corollary} \label{ac 4.2.1}
The definition of equality for the rational numbers is reflexive, symmetric and transitive.
\end{additional corollary}

\begin{proof}
Reflexive: Suppose \(a, b\) are integers where \(b \neq 0\), we have to show \(a \D b = a \D b\).
But \(\GREEN{a}\BLUE{b} = \BLUE{a}\GREEN{b}\) by reflexivity of integers. So by \DEF{4.2.1}, \(\GREEN{a \D b} = \BLUE{a \D b}\).

Symmetric: Suppose \(a, b, c, d\) are integers such that \(b, d \neq 0\) and \(a \D b = c \D d\), we have to show \(c \D d = a \D b\). Then
\begin{align*}
             & a \D b = c \D d \\
    \implies & ad = cb & \text{by \DEF{4.2.1}} \\
    \implies & cb = ad & \text{since integers are symmetric} \\
    \implies & c \D d = a \D b & \text{by \DEF{4.2.1}}
\end{align*}

Transitive: Suppose \(a, b, c, d, e, f\) are integers such that \(b, d, f \neq 0\) and \(a \D b = c \D d\) and \(c \D d = e \D f\), we have to show \(a \D b = e \D f\). Then
\begin{align*}
             & a \D b = c \D d \land c \D d = e \D f \\
    \implies & ad = cb \land cf = ed & \text{by \DEF{4.2.1}} \\
    \implies & adf = cbf \land cfb = edb & \text{multiply first equation by \(f\), second by \(b\)}\\
    \implies & adf = cbf \land cbf = edb & \text{by \PROP{4.1.6}(5)} \\
    \implies & adf = edb & \text{since integers are transitive} \\
    \implies & afd = ebd & \text{by \PROP{4.1.6}(5)} \\
    \implies & af = eb & \text{by \PROP{4.1.8} and \(d \neq 0\)} \\
    \implies & a \D b = e \D f & \text{by \DEF{4.2.1}}
\end{align*}
\end{proof}

\begin{definition} \label{def 4.2.2}
If \(a \D b\) and \(c \D d\) are rational numbers, we \emph{define} their sum
\[
    (a \D b) + (c \D d) := (ad + bc) \D (bd)
\]
their product
\[
    (a \D b) \X (c \D d) := (ac) \D (bd)
\]
and the negation
\[
    -(a \D b) := (-a) \D b.
\]
\end{definition}

\begin{note}
If \(b\) and \(d\) are non-zero, then \(bd\) is also non-zero, by \PROP{4.1.8}, so the sum or product of two rational numbers remains a
rational number.
\end{note}

\begin{note}
我覺得\ negation 是用定義的,挺神奇的。另外這個定義是把有理數的\ negation reduce 給整數的\ negation。
\end{note}

\begin{lemma} \label{lem 4.2.3}
The sum, product, and negation operations on rational numbers are well-defined,
in the sense that if one replaces \(a \D b\) with another rational number \(a' \D b'\) which \emph{is equal to} \(a \D b\), then the output
of the above operations remains unchanged, and similarly for \(c \D d\) replaced by \(c' \D d'\) which \emph{is equal to} \(c \D d\).
\end{lemma}

\begin{proof} 
Let \(a, b, a', b', c, d, c' d'\) be integers s.t. \(b, b', d, d' \neq 0\) and \(a \D b = a' \D b'\) and \(c \D d = c' \D d'\).
Note that by \DEF{4.2.1} we have \(ab' = a'b\) \MAROON{(1)} and \(cd' = c'd\) \MAROON{(2)}.

\begin{align*}
         & (a \D b) + (c \D d) = (a' \D b') + (c \D d) \\
    \iff & (ad + bc) \D (bd) = (a'd + b'c) \D (b'd) & \text{by \DEF{4.2.2}} \\
    \iff & (ad + bc)(b'd) = (a'd + b'c)(bd) & \text{by \DEF{4.2.1}} \\
    \iff & adb'd + bcb'd = a'dbd + b'cbd & \text{by \PROP{4.1.6}(9)} \\
    \iff & ab'dd + bb'cd = a'bdd + bb'cd & \text{by \PROP{4.1.6}(5)(6)} \\
    \iff & ab'dd + bb'cd - bb'cd = a'bdd + bb'cd - bb'cd \\
    \iff & ab'dd = a'bdd & \text{by \PROP{4.1.6}(3)(4)} \\
    \iff & ab' = a'b & \text{by \CORO{4.1.9}, cancel \(d\) twice} \\
\end{align*}
which is true by \MAROON{(1)}.

\begin{align*}
         & (a \D b) + (c \D d) = (a \D b) + (c' \D d') \\
    \iff & (ad + bc) \D (bd) = (ad' + bc') \D (bd') & \text{by \DEF{4.2.2}} \\
    \iff & (ad + bc)(bd') = (ad' + bc')(bd) & \text{by \DEF{4.2.1}} \\
    \iff & adbd' + bcbd' = ad'bd + bc'bd & \text{by \PROP{4.1.6}(9)} \\
    \iff & abdd' + bbcd' = abdd' + bbc'd & \text{by \PROP{4.1.6}(5)(6)} \\
    \iff & (-(abdd')) + abdd' + bbcd' = (-(abdd')) + abdd' + bbc'd \\
    \iff & bbcd' = bbc'd & \text{by \PROP{4.1.6}(3)(4)} \\
    \iff & cd'bb = c'dbb & \text{by \PROP{4.1.6}(5)} \\
    \iff & cd' = c'd & \text{by \CORO{4.1.9}, cancel \(b\) twice} \\
\end{align*}
which is true by \MAROON{(2)}.

\begin{align*}
         & (a \D b) \X (c \D d) = (a' \D b') \X (c \D d) \\
    \iff & (ac) \D (bd) = (a'c) \D (b'd) & \text{by \DEF{4.2.2}} \\
    \iff & acb'd = a'cbd & \text{by \DEF{4.2.1}} \\
    \iff & ab'cd = a'bcd & \text{by \PROP{4.1.6}(5)(6)} \\
    \iff & ab' = a'b & \text{by \CORO{4.1.9}, cancel \(c\) and \(d\)}
\end{align*}
which is true by \MAROON{(1)}.

\begin{align*}
         & (a \D b) \X (c \D d) = (a \D b) \X (c' \D d') \\
    \iff & (ac) \D (bd) = (ac') \D (bd') & \text{by \DEF{4.2.2}} \\
    \iff & acbd' = ac'bd & \text{by \DEF{4.2.1}} \\
    \iff & cd'ab = c'dab & \text{by \PROP{4.1.6}(5)(6)} \\
    \iff & cd' = c'd & \text{by \CORO{4.1.9}, cancel \(a\) and \(b\)}
\end{align*}
which is true by \MAROON{(2)}.

\begin{align*}
         & -(a \D b) = -(a' \D b') \\
    \iff & (-a) \D b = (-a') \D b' & \text{by \DEF{4.2.2}} \\
    \iff & (-a)b' = (-a')b & \text{by \DEF{4.2.1}} \\
    \iff & (-1)ab' = (-1)a'b & \text{by \AC{4.1.3}} \\
    \iff & ab' = a'b & \text{by \CORO{4.1.9}, cancel \(-1\)}
\end{align*}
which is true by \MAROON{(1)}
\end{proof}

We note that the \emph{rational} numbers \(a \D 1\) \textbf{behave} in a manner identical to the \emph{integers} \(a\):
\[
    (a \D 1) + (b \D 1) = (a \X 1 + 1 \X b) \D (1 \X 1) = (a + b) \D 1
\]
\[
    (a \D 1) \X (b \D 1) = (a \X b) \D (1 \X 1) = (ab \D 1)
\]
\[
    -(a \D 1) = (-a) \D 1.
\]
Also, \(a \D 1\) and \(b \D 1\) are only equal when \(a\) and \(b\) are equal.
Because of this, we will identify a with \(a \D 1\) for each integer \(a\):
\[
    a \equiv a \D 1;
\]
the above identities then guarantee that \emph{the arithmetic of the integers is consistent with the arithmetic of the rationals}.
Thus just as we embedded the \emph{natural} numbers inside the \emph{integers}, we embed the \emph{integers} inside the \emph{rational} numbers.
In particular, all \emph{natural} numbers are \emph{rational} numbers, for instance \(0\) is equal to \(0 \D 1\) and \(1\) is equal to \(1 \D 1\).

\begin{additional corollary} \label{ac 4.2.2}
Observe that
\begin{align*}
         & a \D b = 0 \\
    \iff & a \D b = 0 \D 1 & \text{since \(0 \equiv 0 \D 1\)} \\
    \iff & a \X 1 = 0 \X b & \text{by \DEF{4.2.1}} \\
    \iff & a = 0 & \text{simplify}
\end{align*}
Thus if \(a\) and \(b\) are non-zero (note that \(b\) must be non-zero) then so is \(a \D b\). This is used in \DEF{4.1.11}.
\end{additional corollary}

We now define a new operation on the rationals: reciprocal. See \DEF{4.1.11} (just give the definition a label to make it convenient to refer to).

\begin{proposition} [Laws of algebra for rationals] \label{prop 4.2.4}
Let \(x, y, z\) be rationals.
Then the following laws of algebra hold:
\begin{align*}
          x + y & = y + x \MAROON{\ (1)} \\
    (x + y) + z & = x + (y + z) \MAROON{\ (2)} \\
          x + 0 & = 0 + x = x \MAROON{\ (3)} \\
       x + (-x) & = (-x) + x = 0 \MAROON{\ (4)} \\
             xy & = yx \MAROON{\ (5)} \\
          (xy)z & = x(yz) \MAROON{\ (6)} \\
        x1 = 1x & = x \MAROON{\ (7)} \\
       x(y + z) & = xy + xz \MAROON{\ (8)} \\
       (y + z)x & = yx + zx \MAROON{\ (9)}
\end{align*}
If \(x\) is non-zero, we also have
\begin{align*}
    xx^{-1} = x^{-1}x = 1. \MAROON{\ (10)}
\end{align*}
\end{proposition}

\begin{proof}
Let \(x = (a \D b), y = (c \D d)\), and \(z = (e \D f)\) for some integers \(a, b, c, d, e, f\) where \(b, d, f \neq 0\).

\MAROON{(1)} \(x + y = y + x\):
\begin{align*}
    x + y & = (a \D b) + (c \D d) \\
          & = (ad + bc) \D (bd) & \text{by \DEF{4.2.2}} \\
          & = (cb + da) \D (db) & \text{by properties of integers \PROP{4.1.6}} \\
          & = (c \D d) + (a \D b) & \text{by \DEF{4.2.2}} \\
          & = y + x
\end{align*}

\MAROON{(2)} \((x + y) + z = x + (y + z)\):
\begin{align*}
    (x + y) + z & = ((a \D b) + (c \D d)) + (e \D f) \\
                & = ((ad + bc) \D (bd)) + (e \D f) & \text{by \DEF{4.2.2}} \\
                & = ((ad + bc)f + (bd)e) \D ((bd)f) & \text{by \DEF{4.2.2}} \\
                & = (adf + bcf + bde) \D (bdf) \GREEN{\ (1)} & \text{by properties of integers \PROP{4.1.6}}
\end{align*}
\begin{align*}
    x + (y + z) & = (a \D b) + ((c \D d) + (e \D f)) \\
                & = (a \D b) + ((cf + de) \D (df) & \text{by \DEF{4.2.2}} \\
                & = (a(df) + b(cf + de)) \D (b(df)) & \text{by \DEF{4.2.2}} \\
                & = (adf + bcf + bde) \D (bdf) & \text{by properties of integers \PROP{4.1.6}} \\
                & = \GREEN{(1)}
\end{align*}
So \((x + y) + z = x + (y + z)\).

\MAROON{(3)} \(x + 0 = 0 + x = x\):
By \MAROON{(1)} we have \(x + 0 = 0 + x\), so we only have to show \(x + 0 = x\).
Then
\begin{align*}
    x + 0 & = (a \D b) + 0 \\
          & = (a \D b) + (0 \D 1) & \text{since (integer) \(0 \equiv (0 \D 1)\) (rational)} \\
          & = (a \X 1 + b \X 0) \D (b \X 1) & \text{by \DEF{4.2.2}} \\
          & = a \D b & \text{by properties of integers \PROP{4.1.6}} \\
          & = x
\end{align*}

\MAROON{(4)} \(x + (-x) = (-x) + x = 0\):
By \MAROON{(1)} we have \(x + (-x) = (-x) + x\), so we only have to show \(x + (-x) = 0\).
Then
\begin{align*}
    x + (-x) & = (a \D b) + -(a \D b) \\
             & = (a \D b) + ((-a) \D b) & \text{by \DEF{4.2.2}} \\
             & = (ab + b(-a)) \D (bb) & \text{by \DEF{4.2.2}} \\
             & = (0) \D (bb) & \text{by properties of integers \PROP{4.1.6}} \\
             & = 0 & \text{since (integer) \( 0 \equiv (0) \D (bb)\) (rational)}
\end{align*}

\MAROON{(5)} \(xy = yx\):
\begin{align*}
    xy & = (a \D b)(c \D d) \\
       & = (ac) \D (bd) \text{\ \GREEN{(2)}} & \text{by \DEF{4.2.2}}
\end{align*}
And 
\begin{align*}
    yx & = (c \D d)(a \D b) \\
       & = (ca) \D (db) & \text{by \DEF{4.2.2}} \\
       & = (ac) \D (bd) & \text{by properties of integers \PROP{4.1.6}} \\
       & = \text{\GREEN{(2)}}
\end{align*}
So \(xy = yx\).

\MAROON{(6)} \((xy)z = x(yz)\):
\begin{align*}
    (xy)z & = ((a \D b)(c \D d))(e \D f) \\
          & = ((ac) \D (bd))(e \D f) & \text{by \DEF{4.2.2}} \\
          & = ((ac)e) \D ((bd)f) & \text{by \DEF{4.2.2}} \\
          & = (a(ce)) \D (b(df)) & \text{by \PROP{4.1.6}(6)} \\
          & = (a \D b)((ce) \D (df)) & \text{by \DEF{4.2.2}} \\
          & = (a \D b)((c \D d)(e \D f)) & \text{by \DEF{4.2.2}} \\
          & = x(yz)
\end{align*}

\MAROON{(7)} \(x1 = 1x = x\):
Again by \MAROON{(1)}, we only have to show \(x1 = x\).
\begin{align*}
    x1 & = (a \D b) \X 1 \\
       & = (a \D b) \X (1 \D 1) & \text{since (integer) \(1 \equiv 1 \D 1\) (rational)} \\
       & = (a \X 1) \D (b \X 1) & \text{by \DEF{4.2.2}} \\
       & = a \D b & \text{by properties of integers \PROP{4.1.6}} \\
       & = x
\end{align*}

\MAROON{(8)} \(x(y + z) = xy + xz\):
\begin{align*}
    x(y + z) & = (a \D b)((c \D d) + (e \D f)) \\
             & = (a \D b)((cf + de) \D (df)) & \text{by \DEF{4.2.2}} \\
             & = (a(cf + de)) \D (b(df)) & \text{by \DEF{4.2.2}} \\
             & = (acf + ade) \D (bdf) \GREEN{\ (3)} & \text{by properties of integers \PROP{4.1.6}}
\end{align*}
And
\begin{align*}
    xy + xz & = (a \D b)(c \D d) + (a \D b)(e \D f) \\
            & = ((ac) \D (bd)) + ((ae) \D (bf)) & \text{by \DEF{4.2.2}} \\
            & = ((ac)(bf) + (bd)(ae)) \D ((bd)(bf)) & \text{by \DEF{4.2.2}} \\
            & = (acbf + abde) \D (bdbf) & \text{by properties of integers \PROP{4.1.6}} \\
            & = (b(acf + ade)) \D (b(bdf)) & \text{by \PROP{4.1.6}(8)} \\
            & = (b \D b)((acf + ade) \D (bdf)) & \text{by \DEF{4.2.2}} \\
            & = (1 \D 1)((acf + ade) \D (bdf)) & \text{by \DEF{4.2.1}, \(\GREEN{1 \D 1} = \BLUE{b \D b} \iff \GREEN{1} \X \BLUE{b} = \BLUE{b} \X \GREEN{1} \)} \\
            & = 1((acf + ade) \D (bdf)) & \text{since (integer) \(1 \equiv 1 \D 1\) (rational)} \\
            & = (acf + ade) \D (bdf) & \text{by \MAROON{(7)}} \\
            & = \GREEN{(3)}
\end{align*}
So \(x(y + z) = xy + xz\).

\MAROON{(9)} \((y + z)x = yx + zx\):
\begin{align*}
    (y + z)x & = x(y + z) & \text{by \MAROON{(1)}} \\
             & = xy + xz & \text{by \MAROON{(8)}} \\
             & = yx + zx & \text{by \MAROON{(1)}}
\end{align*}

\MAROON{(10)} \(xx^{-1} = x^{-1}x = 1\):
Suppose \(x^{-1}\) is well-defined, i.e. \(x = a \D b\) where both \(a, b \neq 0\).
Again by \MAROON{(1)} we only have to show \(xx^{-1} = 1\).
Then
\begin{align*}
    xx^-1 & = (a \D b)(b \D a) \\
          & = (ab) \D (ba) & \text{by \DEF{4.2.2}} \\
          & = (ab) \D (ab) & \text{by \PROP{4.1.6}} \\
          & = 1 \D 1 & \text{by \DEF{4.2.1}, \(\GREEN{1 \D 1} = \BLUE{ab \D ab} \iff \GREEN{1} \X \BLUE{ab} = \BLUE{ab} \X \GREEN{1} \)} \\
          & = 1 & \text{since (integer) \(1 \equiv 1 \D 1\) (rational)}
\end{align*}
\end{proof}

\begin{remark} \label{remark 4.2.5}
The above set of ten identities have a name; they are asserting that the rationals \(\SET{Q}\) form a \emph{field}.
This is \emph{better} than being a \emph{commutative ring} because of \emph{the tenth identity} \(xx^{-1} = x^{-1}x = 1\).
Note that this proposition \emph{supersedes} \PROP{4.1.6}.
\end{remark}

We can now define the \emph{quotient} \(x / y\) of two rational numbers: see \DEF{4.2.12}. And we also define the \emph{subtraction} of twp rational numbers; see \DEF{4.2.13}.

\begin{note}
\PROP{4.2.4} allows us to use all the normal rules of algebra; we will now proceed to do so without further comment.
\end{note}

\begin{definition} \label{def 4.2.6}
A rational number \(x\) is said to be \emph{positive} iff we have \(x = a/b\) for some positive integers \(a\) and \(b\).
It is said to be \emph{negative} iff we have \(x = -y\) for some positive \emph{rational} \(y\) (i.e., \(x =(-a)/b\) for some \emph{positive} integers \(a\) and \(b\)).
\end{definition}

\begin{note}
From \DEF{4.2.6}, note that every positive \emph{integer} is a positive \emph{rational} number,
and every negative \emph{integer} is a negative \emph{rational} number,
so our new definition is consistent with our old one.
\end{note}

\begin{additional corollary} \label{ac 4.2.3}
Let \(x = a / b\) be a rational number where \(a, b\) are integers and \(b \neq 0\).
Then
\begin{align*}
    -x & = (-a) / b & \text{by \DEF{4.2.2}} \\
       & = a / (-b) \MAROON{\ (1)} \\
       & = (-1)(a / b) = (-1)x \MAROON{\ (2)}.
\end{align*}
And \(-(-x) = x\) \MAROON{(3)}.
\end{additional corollary}

\begin{proof}
\begin{align*}
    -x & = -(a / b) \\
       & = (-a) / b & \text{by \DEF{4.2.2}} \\
       & = ((-1) \X a) / b & \text{by \AC{4.1.3}} \\
       & = ((-1) \X a) / (1 \X b) & \text{by \PROP{4.1.6}(7)} \\
       & = (-1 / 1) \X (a / b) & \text{by \DEF{4.2.2}} \\
       & = (-1) \X (a / b) & \text{since (integer) \(-1 \equiv -1 / 1\) (rational)} \\
       & = (-1)x & \text{proved \MAROON{(2)}}
\end{align*}
Now, since \(\GREEN{-1} \X \BLUE{-1} = \BLUE{1} \X \GREEN{1}\) by the properties of integers, by \DEF{4.2.1}, \(\GREEN{(-1)} / \GREEN{1} = \BLUE{1} / \BLUE{(-1)}\).
So from \MAROON{(2)}, we have
\begin{align*}
        (-1)x
    & = ((-1) / 1)x & \text{since (integer) \(-1 \equiv -1 / 1\) (rational)} \\
    & = (1 / (-1))x & \text{we have shown \((-1)/1 = 1/(-1)\)} \\
    & = (1 / (-1))(a/b) \\
    & = (1 \X a)/((-1) \X b) & \text{by \DEF{4.2.2}} \\
    & = a/(-b) & \text{by properties of integer}
\end{align*}
So we have proved \MAROON{(1)}

Finally,
\begin{align*}
    -(-x) = -(-(a / b))
        & = -((-a) / b) & \text{by \DEF{4.2.2}} \\
        & = (-(-a)) / b & \text{by \DEF{4.2.2}} \\
        & = a / b & \text{by \AC{4.1.4}} \\
        & = x
\end{align*}
So we proved \MAROON{(3)}.
\end{proof}

\begin{lemma} [Trichotomy of rationals] \label{lem 4.2.7}
Let \(x\) be a rational number.
Then \emph{exactly one} of the following three statements is true:
\begin{enumerate}
    \item \(x\) is equal to \(0\).
    \item \(x\) is a positive rational number.
    \item \(x\) is a negative rational number.
\end{enumerate}
number.
\end{lemma}

\begin{proof}
We first show at least one of these cases are true.

Let \(x\) be a rational number.
By \DEF{4.2.1}, \(x = a / b\) for some integers \(a, b\) and \(b \neq 0\).
By \LEM{4.1.11}(f), exactly one of \(a = 0, a < 0, a > 0\) is true, and similarly since \(b \neq 0\), exactly one of \(b > 0, b < 0\) is true.
\begin{itemize}
    \item \(a = 0\):
        Then by \AC{4.2.2}, we have \(a / b = 0\).
    \item \(a < 0\):
        \begin{itemize}
            \item
                If \(b < 0\), then by \LEM{4.1.5}, \(a = -c\), \(b = -d\) for some positive natural numbers(i.e. integers) \(c, d\).
                So
                \begin{align*}
                    x & = a / b \\
                      & = (-c)/(-d) \\
                      & = ((-1)(c))/((-1)(d)) & \text{by \AC{4.1.3}} \\
                      & = ((-1)/(-1))(c/d)) & \text{by \DEF{4.2.2}} \\
                      & = (1/1)(c/d) & \text{similar cases showed many times} \\
                      & = 1(c/d) & \text{\(1 \equiv 1 / 1\)} \\
                      & = c/d. & \text{by \PROP{4.2.4}(7)} \\
                \end{align*}
                So \(x = a/ b = c/d\) for some positive integers \(c, d\).
                By \DEF{4.2.6}, \(x\) is positive.
            \item 
                If \(b > 0\), then by \LEM{4.1.5}, only \(a = -c\) for some positive natural number(i.e. integer).
                So \(x = a / b = (-c) / b\) for some positive integers \(c, b\).
                By \DEF{4.2.6}, \(x\) is negative.
        \end{itemize}
    \item \(a > 0\):
        \begin{itemize}
            \item 
                If \(b < 0\), then by \LEM{4.1.5}, only \(b = -c\) for some positive natural number(i.e. integer).
                So
                \begin{align*}
                    x & = a / b \\
                      & = a / (-c) \\
                      & = (-a)/c & \text{by \AC{4.2.3}} \\
                \end{align*} for some positive integers \(a, c\).
                By \DEF{4.2.6}, \(x\) is negative.
            \item 
                If \(b > 0\), then \(x = a/b\) for some positive integers \(a, b\). By \DEF{4.2.6}, \(x\) is positive.
        \end{itemize}
\end{itemize}

Now we show at most one of these cases is true, that is, (a)(b) can not be both true and (a)(c) can not be both true and (b)(c) can not be both true.
For the sake of contradiction, suppose:
\begin{itemize}
    \item
        If (a)(b) are both true, i.e. \(x = 0\) and \(x\) is positive, then we have \(x = 0/1\) and by \DEF{4.2.6} we have \(x = a / b\) for some positive integers \(a, b\).
        So \(0/1 = a/b\), by \DEF{4.2.1}, \(0 \X b = a \X 1\), so \(0 = a\), contradicting that \(a\) is positive.
    \item
        If (a)(c) are both true, i.e. \(x = 0\) and \(x\) is negative, then we have \(x = 0/1\) and by \DEF{4.2.6} we have \(x = (-a) / b\) for some positive integers \(a, b\).
        So \(0/1 = (-a)/b\), by \DEF{4.2.1}, \(0 \X b = (-a) \X 1\), so \(0 = -a\), contradicting that \(a\) is positive.
    \item
        If (b)(c) are both true, i.e. \(x\) is positive and \(x\) is negative, and by \DEF{4.2.6} we have \(x = a/b\) for some positive integers \(a, b\) and \(x = (-c)/d\) for some positive integers \(c, d\).
        So \(a/d = (-c)/d\), by \DEF{4.2.1}, \(ad = (-c)d\), and by \CORO{4.1.9}, since \(d \neq 0\), we have \(a = -c\), contradicting that \(a\) is positive.
\end{itemize}
\end{proof}

\begin{definition} [Ordering of the rationals] \label{def 4.2.8}
Let \(x\) and \(y\) be rational numbers.
We say that \(x > y\) iff \(x - y\) is a positive rational number,
and \(x < y\) iff \(x - y\) is a negative rational number.
We write \(x \ge y\) iff either \(x > y\) or \(x = y\), and similarly define \(x \le y\).
\end{definition}

\begin{note}
From \DEF{4.2.8}, \(x < y\) and \(y > x\) is \emph{not defined} to be equivalent, We need to \emph{prove} it(see \PROP{4.2.9}).
\end{note}
\begin{note}
It is trivial that \(x > 0\) and \(-x < 0\) for some positive rational number \(x\).
\end{note}

\begin{proposition} [Basic properties of order on the rationals] \label{prop 4.2.9}
Let \(x, y, z\) be rational numbers.
Then the following properties hold.
\begin{enumerate}
    \item \MAROON{(1)}
        (Order trichotomy) Exactly one of the three statements \(x = y\), \(x < y\), or \(x > y\) is true.
    \item \MAROON{(2)}
        (Order is \emph{anti-symmetric}) One has \(x < y\) if and only if \(y > x\).
    \item \MAROON{(3)}
        (Order is transitive) If \(x < y\) and \(y < z\), then \(x < z\).
    \item \MAROON{(4)}
        (Addition preserves order) If \(x < y\), then \(x + z < y + z\).
    \item \MAROON{(5)}
        (Positive multiplication preserves order) If \(x<y\) and \(z\) is positive, then \(xz < yz\).
\end{enumerate}
\end{proposition}

\begin{proof}
\begin{enumerate}
    \item \MAROON{(1)}
        Given rational numbers \(x, y\), let \(a = x - y\).
        Then by \LEM{4.2.7}, exactly one of \(a = 0, a < 0, a > 0\) is true.
        \begin{itemize}
            \item If \(a = 0\), then
                \begin{align*}
                         & a = 0 \\
                    \iff & x - y = 0 \\
                    \iff & x - y + y = 0 + y = y \\
                    \iff & x = y
                \end{align*}
                Suppose \(x > y\), then by \DEF{4.2.8}, \(x - y\) is positive, i.e. \(a\) is positive, contradicting \(a = 0\) and \LEM{4.2.7}.
                Suppose \(x < y\), then by \DEF{4.2.8}, \(x - y\) is negative, i.e. \(a\) is negative, contradicting \(a = 0\) and \LEM{4.2.7}.
            \item If \(a < 0\), then
                \begin{align*}
                         & a < 0 \\
                    \iff & x - y < 0 \\
                    \iff & x - y + y < 0 + y = y \\
                    \iff & x < y
                \end{align*}
                Suppose \(x = y\) then trivially we can get \(a = 0\), contradicting \(a < 0\) and \LEM{4.2.7}.
                Suppose \(x > y\) then trivially we can get \(a > 0\), contradicting \(a < 0\) and \LEM{4.2.7}.
            \item If \(a > 0\), then
                \begin{align*}
                         & a > 0 \\
                    \iff & x - y > 0 \\
                    \iff & x - y + y > 0 + y = y \\
                    \iff & x > y
                \end{align*}
                Similar with the previous case, we will get contradiction if \(a = 0\) or \(a < 0\).
        \end{itemize}
        So in all cases, exactly one of \(x = y, x > y, x < y\) is true.
    \item \MAROON{(2)}
        \begin{align*}
                 & x < y \\
            \iff & x - y = -a \text{\ for some positive rational \(a\)} & \text{by \DEF{4.2.6}, \DEF{4.2.8}} \\
            \iff & (-1)(x - y) = (-1)(-a) \\
            \iff & (-1)(x + (-y)) = (-1)(-a) & \text{by \DEF{4.2.13}} \\
            \iff & (-1)x + (-1)(-y) = (-1)(-a) & \text{by \PROP{4.2.4}(8)} \\
            \iff & (-x) + (-(-y)) = (-(-a)) & \text{by \AC{4.2.3}} \\
            \iff & (-x) + y = a & \text{by \AC{4.2.3}} \\
            \iff & y + (-x) = a & \text{by \PROP{4.2.4}(1)} \\
            \iff & y - x = a & \text{by \DEF{4.2.13}} \\
            \iff & y > x & \text{by \DEF{4.2.8}}
        \end{align*}
    \item \MAROON{(3)}
        \begin{align*}
                     & x < y \land y < z \\
            \implies & y > x \land z > y & \text{by \MAROON{(2)}} \\
            \implies & y - x = a \land z - y = b \land a, b > 0 & \text{by \DEF{4.2.8}} \\
            \implies & y + (-x) = a \land z + (-y) = b & \text{by \DEF{4.2.13}} \\
            \implies & y + (-x) + z + (-y) = a + b & \text{add both side} \\
            \implies & z + (-x) + y + (-y) = a + b & \text{by \PROP{4.2.4}(1)(2)} \\
            \implies & z + (-x) = a + b & \text{by \PROP{4.2.4}(3)(4)} \\
            \implies & z - x = a + b & \text{by \DEF{4.2.13}} \\
            \implies & z - x \text{\ is positive} & \text{it's trivial that \(a + b\) is positive} \\
            \implies & z > x & \text{by \DEF{4.2.8}} \\
            \implies & x < z & \text{by \MAROON{(2)}}
        \end{align*}
    \item \MAROON{(4)}
        \begin{align*}
                     & x < y \\
            \implies & x - y = -a, a > 0 & \text{by \DEF{4.2.8}} \\
            \implies & x - y + 0 = -a & \text{by \PROP{4.2.4}(3)} \\
            \implies & x - y + z + (-z) = -a & \text{by \PROP{4.2.4}(4)} \\
            \implies & x + (-y) + z + (-z) = -a & \text{by \DEF{4.2.13}} \\
            \implies & x + z + (-y) + (-z) = -a & \text{by \PROP{4.2.4}(1)(2)} \\
            \implies & (x + z) + (-1)(y) + (-1)z = -a & \text{by \AC{4.2.3}} \\
            \implies & (x + z) + (-1)(y + z) = -a & \text{by \PROP{4.2.4}(8)} \\
            \implies & (x + z) + (-(y + z)) = -a & \text{by \AC{4.2.3}} \\
            \implies & (x + z) - (y + z) = -a & \text{by \DEF{4.2.13}} \\
            \implies & x + z < y + z & \text{by \DEF{4.2.8}}
        \end{align*}
    \item \MAROON{(5)}
        \begin{align*}
                     & x < y \land z > 0 \\
            \implies & x - y = -a, a > 0 \\
            \implies & (x - y)z = (-a)z \\
            \implies & (x + (-y))z = (-a)z & \text{by \DEF{4.2.13}} \\
            \implies & xz + (-y)z = (-a)z & \text{by \PROP{4.2.4}(9)} \\
            \implies & xz + ((-1)y)z) = ((-1)a)z & \text{by \AC{4.2.3}} \\
            \implies & xz + (-1)(yz) = (-1)az & \text{by \PROP{4.2.4}(6)} \\
            \implies & xz + (-(yz)) = -(az) & \text{by \AC{4.2.3}} \\
            \implies & xz - yz = -(az) & \text{by \DEF{4.2.13}} \\
            \implies & xz < yz & \text{by \DEF{4.2.8}, where \(az > 0\) is trivial}
        \end{align*}
\end{enumerate}
\end{proof}

\begin{remark} \label{remark 4.2.10}
The above five properties in \PROP{4.2.9}, combined with the field axioms in \PROP{4.2.4}, have a name:
they assert that the rationals \(\SET{Q}\) form an \emph{ordered field}.
It is important to keep in mind that \PROP{4.2.9}(e) only works when \(z\) is \emph{positive}, see \EXEC{4.2.6}.
\end{remark}

\begin{definition} [Reciprocal] \label{def 4.1.11}
If \(x = a \D b\) is a \emph{non-zero} rational (so that \(a, b \neq 0\)) then we define the reciprocal \(x^{-1}\) of \(x\) to be the rational number
\[
    x^{-1} := b \D a.
\]
\end{definition}

It is easy to check that this operation is consistent with our notion of equality: if two rational numbers \(a \D b\), \(a' \D b'\) are equal, then their reciprocals are also equal.
\begin{additional corollary} \label{ac 4.2.4}
The reciprocal operation on rational numbers is well-defined:
if two rational numbers \(a // b\), \(a' // b'\) are equal, then their reciprocals are also equal.
We however leave the reciprocal of \(0\) undefined, i.e. when \(a = 0\), we leave the reciprocal of \(a \D b\) undefined.
\end{additional corollary}

\begin{definition} [Quotient] \label{def 4.2.12}
The \emph{quotient} \(x / y\) of two \emph{rational} numbers \(x\) and \(y\), provided that \(y\) is \emph{non-zero}, is defined by the formula
\[
    x / y := x \X y^{-1}.
\]
\end{definition}
Thus, for instance,
\[
    (3 \D 4) / (5 \D 6) = (3 \D 4) \X (6 \D 5) = (18 \D 20) = (9 \D 10).
\]

Using this formula, it is easy to see that \(a / b = a \D b\) for every \textbf{integer} \(a\) and every non-zero \textbf{integer} \(b\).
Thus we can now discard the \(\D\) notation, and use the more customary \(a / b\) instead of \(a \D b\).

\begin{definition} \label{def 4.2.13}
In a similar spirit, we define \emph{subtraction} on the rationals by the formula
\[
    x - y := x + (-y),
\]
just as we did with the integers.
\end{definition}

\exercisesection

\begin{exercise} \label{exercise 4.2.1}
Show that the definition of equality for the rational numbers is reflexive, symmetric, and transitive.
\end{exercise}
\begin{proof}
See \AC{4.2.1}.
\end{proof}

\begin{exercise} \label{exercise 4.2.2}
Prove the remaining components of \LEM{4.2.3}.
\end{exercise}
\begin{proof}
See \LEM{4.2.3}.
\end{proof}

\begin{exercise} \label{exercise 4.2.3}
Prove the remaining components of \PROP{4.2.4}.
\end{exercise}
\begin{proof}
See \PROP{4.2.4}.
\end{proof}

\begin{exercise} \label{exercise 4.2.4}
Prove \LEM{4.2.7}.
\end{exercise}
\begin{proof}
See \LEM{4.2.7}.
\end{proof}

\begin{exercise}\label{exercise 4.2.5}
Prove \PROP{4.2.9}.
\end{exercise}
\begin{proof}
See \PROP{4.2.9}.
\end{proof}

\begin{exercise}\label{exercise 4.2.6}
Show that if \(x, y, z\) are rational numbers such that \(x < y\) and \(z\) is negative, then \(xz > yz\).
\end{exercise}
\begin{proof}
\begin{align*}
                 & x < y \land z < 0 \\
        \implies & x - y = -a \land z - 0 = z = -b \land a, b > 0 & \text{by \DEF{4.2.8}}\\
        \implies & (x - y)z = (-a)z \\
        \implies & (x - y)z = (-a)(-b) \\
        \implies & (x + (-y))z = (-a)(-b) & \text{by \DEF{4.2.13}} \\
        \implies & xz + (-y)z = (-a)(-b) & \text{by \PROP{4.2.4}(9)} \\
        \implies & xz + ((-1)y)z) = ((-1)a)((-1)b) & \text{by \AC{4.2.3}} \\
        \implies & xz + (-1)(yz) = (-1)(-1)ab & \text{by \PROP{4.2.4}(5)(6)} \\
        \implies & xz + (-(yz)) = ab & \text{by \AC{4.2.3}} \\
        \implies & xz - yz = ab & \text{by \DEF{4.2.13}} \\
        \implies & xz > yz & \text{by \DEF{4.2.8}, where \(ab > 0\) is trivial}
    \end{align*}
\end{proof}