\section{Images and inverse images}

\begin{note}
我覺得這節標題有點爛,因為這節不只講\ image 阿?
\end{note}

Functions \(f : X \rightarrow Y\) can also takes subsets of \(X\) to subsets of \(Y\).

\begin{definition} [Images of sets] \label{def 3.4.1}
If \(f : X \rightarrow Y\) is a function from \(X\) to \(Y\), and \(S\) is a subset of \(X\), we define \(f(S)\) to be the set
\begin{center}
    \(f(S) := \{ f(x): x \in S \} \)
\end{center}
this set is a subset of \(Y\), and is sometimes called the \emph{image} of \(S\) under the map \(f\).
\end{definition}

We sometimes call \(f(S)\) the \emph{forward image} of \(S\) to distinguish it from the concept of the \emph{inverse image} \(f^{-1}(S)\) of \(S\), which is defined below.

Note that the set \(f(S)\) is well-defined thanks to the axiom of replacement (\AXM{3.6}). One can also define \(f(S)\) using the axiom of specification (\AXM{3.5}) instead of replacement, but we leave this as a
exercise to the reader.

\begin{note}
這邊兩個 examples 都是幹話,跳過。
\end{note}

\begin{additional corollary}
\(y \in f(S) \iff y = f(x) \text{\ for some\ } x \in S\)
\end{additional corollary}

\begin{proof}

\(y \in f(S) \implies y = f(x) \text{\ for some\ } x \in S\): this is explicitly stated in \DEF{3.4.1}.

\(y = f(x) \text{\ for some\ } x \in S \implies y \in f(S)\): the hypothesis is the identity of the image set \(f(S)\), so \(y \in f(S)\).
\end{proof}

\setcounter{theorem}{3}
\begin{definition}[Inverse images] \label{def 3.4.4}
If \(U\) is a subset of \(Y\), we define the set \(f^{-1}(U)\) to be the set
\[
    f^{-1}(U) := \{x \in X : f(x) \in U\}.
\]
In other words, \(f^{-1}(U)\) \textbf{consists of all the elements of \(X\) which map into \(U\)}:
\[
    f(x) \in U \iff x \in f^{-1}(U).
\]
We call \(f^{-1}(U)\) the \emph{inverse image} of \(U\).
\end{definition}

\begin{note}
Note that \(f\) \textbf{does not have to be invertible} in order for \(f^{-1}(U)\) to make sense.
\end{note}

\begin{note}
\(U\) 的\ reverse image 就是所有符合以下條件的\ elements x: (1) \(x \in X\) (2) \(f(x) \in U\).

另外\ \(U\) 裡面有些\ elements 一樣可以不被任何的\ \(x\) map 到。可以觀察\ \EXAMPLE{3.4.5} 的\ \( f^{-1}( \{ 1, 2, 3 \} ) \)。
\end{note}

\begin{example} \label{example 3.4.5}
If \(f : N \rightarrow N\) is the map \(f(x) = 2x\), then \( f( \{ 1, 2, 3 \} ) = \{ 2, 4, 6 \} \), but \( f^{-1}( \{ 1, 2, 3 \} )= \{ 1 \} \).
Thus the forward image of \( \{1, 2, 3 \} \) and the inverse image of \( \{ 1, 2, 3 \} \) are quite different sets.
Also note that \(f(f^{-1}( \{ 1, 2, 3 \} )) \neq \{1, 2, 3\} \), because \( f^{-1}( \{ 1, 2, 3 \} ) = \{1\} \), and \( f( \{1\} ) = \{2\} \neq \{1, 2, 3\} \).
\end{example}

\begin{example}
幹話。
\end{example}

\begin{note}
In general, the two examples above imply that, given any subset \(U\) of the codomain, \(f(f^{-1}(U))\) may not be equal to \(U\).
This happens when \(f\) is not onto.
If \(f\) is not onto, then there exist \(y\) \(f does not map\), then if \(y \in U\), then \(y \notin f(f^{-1}(U))\), so \(U \neq f(f^{-1}(U))\).
\end{note}

\begin{remark} \label{remark 3.4.7}
If \(f\) is a \emph{bijective} function, then we have defined \(f^{−1}\) in two slightly different ways, but this is not an issue because both
definitions are equivalent (\EXEC{3.4.1}).
\end{remark}

\begin{note}
我覺得這個\ remark 寫的不夠精準,要搭配\ \EXEC{3.4.1} 一起看;主要是因為 \(f^-1\) 被\ ``overload'' 了,這樣無法知道\ \(f^{-1}\) 代表的是\ \(f\) 的\ inverse function 還是\ \(f\) 的某個\ inverse image。而從\ \EXEC{3.4.1} 就可以知道這邊想討論的是\ \(f\) 的\ inverse image 跟\ \(f^{-1}\) 的\ forward image,「符號上」都寫成\  \(f^{-1}(V)\),也有各自的定義,但是實際上等價。
\end{note}

As remarked earlier, functions are not sets.
However, we do consider functions to be a type of \emph{object}, and in particular we should be able to consider \emph{sets of functions}.
In particular, we should be able to consider the \textbf{set of all functions from a set \(X\) to a set \(Y\)}.
To do this we need to introduce another axiom to set theory:

\begin{axiom} [Power set axiom] \label{axm 3.10}
Let \(X\) and \(Y\) be sets.
Then there exists a set, denoted \(Y^X\), which consists of all the functions from \(X\) to \(Y\), thus
\[
    f \in Y^X \iff (f \text{\ is a function with domain \(X\) and codomain \(Y\)}).
\]
\end{axiom}

\begin{example}
The reason we use the notation \(Y^X\) to denote this set is that if \(Y\) (codomain) has \(n\) elements and \(X\) (domain) has \(m\) elements, then one can show that \(Y^X\) has \(n^m\) elements; see  \PROP{3.6.14}(f).
\end{example}

\begin{note}
注意上面的定義又是跟\ \SEC{3.6} 有關的;換句話說那個\ \(n^m\) elements 是\ ``cardinality''。
\end{note}

\begin{lemma} \label{lem 3.4.9}
Let \(X\) be a set. Then the set
\[
    \{ Y : Y \text{\ is a subset of\ } X \}
\]
is a set.
See \EXEC{3.4.6}
\end{lemma}

\begin{remark} \label{remark 3.4.10}
The set \( \{ Y : Y \text{\ is a subset of\ X} \} \) is known as the \emph{power set of \(X\)} and is denoted \(2^X\).
For instance, if \(a, b, c\) are distinct objects, we have
\[
    2^{ \{a,b,c\} } = \{
        \emptyset,
        \{a\},
        \{b\},
        \{c\},
        \{a, b\},
        \{a, c\},
        \{b, c\},
        \{a, b, c\}
    \}.
\]
Note that while \( \{a, b, c\} \) has 3 elements, \(2^{ \{a, b , c\} } \) has \(2^3 = 8\) elements.
This gives a hint as to why we refer to the power set of \(X\) as \(2^X\); we return to this issue in \CH{8}.
\end{remark}

\begin{axiom} [Union] \label{axm 3.11}
Let \(A\) be a set, \emph{all of whose elements are themselves sets}.
Then there exists a set \(\bigcup A\) whose elements are precisely those objects which are \textbf{elements of elements of} \(A\), thus for all object \(x\),
\[
    x \in \bigcup A \iff (x \in S \text{\ for some \(S \in A\)}).
\]
\end{axiom}

\begin{example}
幹話。
\end{example}

The axiom of union, combined with the axiom of pair set(\AXM{3.3}), implies the axiom of pairwise union(\AXM{3.4}) (\EXEC{3.4.8}).
Another important consequence of this axiom is that if one has some set \(I\), and for every element \(\alpha \in I\) we have some set \(A_{\alpha}\), then we can form the union set \(\bigcup_{\alpha \in I} A_{\alpha}\) by defining
\[
    \bigcup_{\alpha \in I} A_{\alpha} := \bigcup \{ A_{\alpha} : \alpha \in I \},
\]
which is a set thanks to the axiom of replacement \AXM{3.6} and the axiom of union \AXM{3.11}.
Thus for instance, if \(I = \{1, 2, 3\} \), \(A_1 := \{2, 3\} \), \(A_2 := \{3, 4\} \), and \(A_3 := \{4, 5\} \), then \(\bigcup_{\alpha \in \{1, 2, 3\}} A_{\alpha} = \{2, 3, 4, 5\} \).
\setcounter{equation}{1}
More generally, we see that for any object \(y\),
\begin{align} \label{formula 3.2}
    y \in \bigcup_{\alpha \in I} A_{\alpha} \iff (y \in A_{\alpha} \text{\ \textbf{for some} \(\alpha \in I\)}).
\end{align}

\begin{note}
In situations like this, we often refer to \(I\) as an \emph{index set}, and the elements \(\alpha\) of this index set as \emph{labels}; the sets \(A_{\alpha}\) are then called a \(family of sets\), and are \emph{indexed} by the labels \(\alpha \in I\).
\end{note}

Note that if \(I\) was empty, then \(\bigcup_{\alpha \in I} A_{\alpha}\) would automatically also be empty (why?) because if it is not empty, then there exists object \(y\) in it;
by the identity of \(\bigcup_{\alpha \in I} A_{\alpha}\) above, \(y \in A_{\alpha} \text{\ for some \(\alpha \in I\)}\), and that implies there exists \(\alpha \in I\), which implies \(I\) is not empty, a contradiction.

If the index set \(I\) is non-empty, then we can similarly form intersections of families of sets.
More specifically, given any non-empty set \(I\), and given an assignment of a set \(A_{\alpha}\) to each \(\alpha \in I\), we can define the \textbf{intersection} \(\bigcap_{\alpha \in I} A_{\alpha} \) by \textbf{first choosing} some element \(\beta \in I\) (which we can do since \(I\) is non-empty), and setting(defining)
\begin{align}
    \bigcap_{\alpha \in I} A_{\alpha} := \{x \in A_{\beta} : x \in A_{\alpha} for all \alpha \in I\},
\end{align}
which is a set by the axiom of specification \AXM{3.5}.
This definition may look like it depends on the choice of \(\beta\), but it does not (\EXEC{3.4.9}).

\begin{note}
這邊會需要先挑出一個\ \(\beta \in I\) 的原因是因為我們必須從一個\textbf{已知是集合}的\ object(i.e. \(A_{\beta}\)) 作\ specification。
\end{note}

Observe that for any object \(y\),
\begin{align} \label{formula 3.4}
    y \in \bigcap_{\alpha \in I} A_{\alpha} \iff (y \in A_{\alpha} \text{\ \textbf{for all} \(\alpha \in I\))}.
\end{align}

\begin{note}
The formula \ref{formula 3.4} cannot be derived directly and needs to be proved; again, see \EXEC{3.4.9}。
\end{note}

\begin{note}
上面的\ formula \ref{formula 3.4} 去跟\ union 的定義\ \ref{formula 3.2} 比較一下;一個是\ ``for some'' 一個是\ ``for all''。
\end{note}

\begin{remark}
The axioms of set theory that we have introduced (\AXM{3.1} - \AXM{3.11}, excluding the dangerous \AXM{3.8}) are known as the \textbf{Zermelo-Fraenkel} axioms of set theory, after Ernest Zermelo (1871– 1953) and Abraham Fraenkel (1891–1965).
There is one further axiom we will eventually need, the famous \textbf{axiom of choice} (see \SEC{8.4}, \AXM{8.1}), giving rise to the Zermelo-Fraenkel-\textbf{Choice} (ZFC) axioms of set theory,
but we will not need this axiom for some time.
\end{remark}

\exercisesection

\begin{exercise} \label{exercise 3.4.1}
Let \(f : X \to Y\) be a bijective function, and let \(f^{-1} : Y \to X\) be its inverse.
Let \(V\) be any subset of \(Y\).
Prove that the forward image of \(V\) under \(f^{-1}\) \textbf{is the same} set as the inverse image of \(V\) under \(f\);
thus the fact that \emph{both sets are denoted} by \(f^{-1}(V)\) will not lead to any inconsistency.
\end{exercise}

\begin{proof}
Let \(A\) be the forward image of \(V\) under \(f^{-1}\), \(B\) be the inverse image of \(V\) under \(f\), we have to show \(A = B\).
Suppose \(x\) is an arbitrary object, then
\begin{align*}
         & x \in A \\
    \iff & x \in \{ f^{-1}(v) : v \in V \} & \text{by \DEF{3.4.1}} \\
    \iff & x = f^{-1}(v) \text{\ for some\ } v \in V \\
    \iff & f(x) = f(f^{-1}(v)) & \text{by apply \(f\) on both sides} \\
    \iff & f(x) = v & \text{by cancellation law on RHS, \EXEC{3.3.6}} \\
    \iff & f(x) \in V & \text{since \(v \in V\)} \\
    \iff & x \in \text{inverse image of \(f\)} & \text{by \DEF{3.4.4}} \\
    \iff & x \in B.
\end{align*}
\end{proof}

\begin{exercise} \label{exercise 3.4.2}
Let \(f : X \to Y\) be a function from one set \(X\) to another set \(Y\), let \(S\) be a subset of \(X\), and let \(U\) be a subset of \(Y\).
What, in general, can one say about \(f^{-1}(f(S))\) and \(S\)?
What about \(f(f^{-1}(U))\) and \(U\)?
\end{exercise}

\begin{proof}
In general we can say \(S \subseteq f^{-1}(f(S))\) and \(f(f^{-1}(U)) \subseteq U\).

Suppose \(x \in S\), we have to show \(x \in f^{-1}(f(S))\).
By \DEF{3.4.1}, \(f(x) \in f(S)\), and by \DEF{3.4.4}, \(f(x) \in \BLUE{f(S)} \iff x \in f^{-1}( \BLUE{f(S)} )\), as desired.

Suppose \(y \in f(f^{-1}(U))\), we have to show \(y \in U\).
By \DEF{3.4.1}, there exists \(x \in f^{-1}(U)\) such that \(y = f(x)\), and by \DEF{3.4.4}, \(x \in f^{-1}(U) \iff f(x) \in U\), that is, \(y \in U\), as desired.
\end{proof}

\begin{exercise} \label{exercise 3.4.3}
Let \(A\), \(B\) be two subsets of a set \(X\), and let \(f : X \to Y\) be a function.
Show that \(f(A \cap B) \subseteq f(A) \cap f(B)\), that \(f(A) \setminus f(B) \subseteq f(A \setminus B)\), \(f(A \cup B) = f(A) \cup f(B)\).
For the first two statements, is it true that the \(\subseteq\) relation can be improved to \(=\)?
\end{exercise}

\begin{proof} Let \(X, Y, A, B, f\) satisfy the requirements in the description.

\begin{itemize}
    \item \(f(A \cap B) \subseteq f(A) \cap f(B)\):
        Suppose \(y \in f(A \cap B)\), we have to show \(y \in f(A) \cap f(B)\).
        By \DEF{3.4.1}, there exists \(x \in A \cap B\) such that \(f(x) = y\).
        But by \DEF{3.1.23}, \(x \in A \land x \in B\),
        and again by \DEF{3.4.1}, \(f(x) \in f(A) \land f(x) \in f(B)\), that is by \DEF{3.1.23}, \(f(x) \in f(A) \cap f(B)\),
        that is, since \(f(x) = y\), \(y \in f(A) \cap f(B)\), as desired.
        
        This \(\subseteq\) relation \textbf{can not} be improved to \(=\).
        If \(f\) is not one-to-one, e.g. \(y = f(x_1) = f(x_2), x_1 \neq x_2\) but \(A = \{ x_1 \}, B = \{ x_2 \}\),
        then \(A \cap B = \emptyset\), so \(f(A \cap B) = \emptyset\),
        but \(f(A) = \{ y \}\) and \(f(B) = \{ y \}\) and \(f(A) \cap f(B) = \{ y \} \neq \emptyset\).
    \item \(f(A) \setminus f(B) \subseteq f(A \setminus B)\): 
        Suppose \(y \in f(A) \setminus f(B)\), we have to show \(y \in f(A \setminus B)\).
        By \DEF{3.1.27}, \(y \in f(A) \land y \notin f(B)\).
        In particular, \(y \in f(A)\), so by \DEF{3.4.1}, there exists \(x \in A\) \MAROON{(1)} such that \(f(x) = y\) \MAROON{(2)};
        and in particular \(y \notin f(B)\), so for all \(b \in B\), \(f(b) \neq y\) \MAROON{(3)},
        that is, by \MAROON{(2)}, for all \(b \in B\), \(f(b) \neq f(x)\), so by \DEF{3.4.1}, \(x \notin B\) \MAROON{(4)}.
        So by \MAROON{(1) (4)}, and \DEF{3.1.27}, \(x \in A \setminus B\).
        By \DEF{3.4.1}, \(f(x) \in f(A \setminus B\), that is, \(y \in f(A \setminus B\), as desired.
        
        This \(\subseteq\) relation \textbf{can not} be improved to \(=\).
        If \(f\) is not one-to-one, e.g. \(y = f(x_1) = f(x_2), x_1 \neq x_2\) but \(A = \{ x_1 \}, B = \{ x_2 \}\),
        then \(f(A) \setminus f(B) = \{ y \} \setminus \{ y \} = \emptyset\),
        but \(f(A \setminus B) = f( \{ x_1 \} \setminus \{ x_2 \} ) = f( \{ x_1 \}) = \{ y \} \neq \emptyset\).
    \item \(f(A \cup B) = f(A) \cup f(B)\): Let \(y\) be an arbitrary object. Then
        \begin{align*}
                 & y \in f(A \cup B) \\
            \iff & \exists x \in A \cup B \text{\ s.t.\ } y = f(x) & \text{by \DEF{3.4.1}} \\
            \iff & x \in A \lor x \in B & \text{by \DEF{3.1.15}} \\
            \iff & f(x) \in f(A) \lor f(x) \in f(B) & \text{by \AXM{3.4}} \\
            \iff & f(x) \in f(A) \cup f(B) & \text{by \AXM{3.4}} \\
            \iff & y \in f(A) \cup f(B) & \text{since \(y = f(x)\)}
        \end{align*}
\end{itemize}
\end{proof}

\begin{exercise} \label{exercise 3.4.4}
Let \(f : X \to Y\) be a function from one set \(X\) to another set \(Y\), and let \(U\), \(V\) be subsets of \(Y\).
Show that \(f^{-1}(U \cup V) = f^{-1}(U) \cup f^{-1}(V)\) \MAROON{(1)},
that \(f^{-1}(U \cap V) = f^{-1}(U) \cap f^{-1}(V)\) \MAROON{(2)},
and that \(f^{-1}(U \setminus V) = f^{-1}(U) \setminus f^{-1}(V)\) \MAROON{(3)}.
\end{exercise}

\begin{proof} Let \(X, Y, U, V, f\) satisfy the requirements in the description.

\begin{itemize}
    \item \MAROON{(1)}: Let \(x\) be an arbitrary object. Then
        \begin{align*}
                 & x \in f^{-1}(U \cup V) \\
            \iff & f(x) \in U \cup V & \text{by \DEF{3.4.4}} \\
            \iff & f(x) \in U \lor f(x) \in V & \text{by \AXM{3.4}} \\
            \iff & x \in f^{-1}(U) \lor x \in f^{-1}(V) & \text{by \DEF{3.4.4}} \\
            \iff & x \in f^{-1}(U) \cup f^{-1}(V) & \text{by \AXM{3.4}}
        \end{align*}
        By \DEF{3.1.4}, \(f^{-1}(U \cup V) = f^{-1}(U) \cup f^{-1}(V)\)
    \item \MAROON{(2)}: Let \(x\) be an arbitrary object. Then
        \begin{align*}
                 & x \in f^{-1}(U \cap V) \\
            \iff & f(x) \in U \cap V & \text{by \DEF{3.4.4}} \\
            \iff & f(x) \in U \land f(x) \in V & \text{by \DEF{3.1.23}} \\
            \iff & x \in f^{-1}(U) \land x \in f^{-1}(V) & \text{by \DEF{3.4.4}} \\
            \iff & x \in f^{-1}(U) \cap f^{-1}(V) & \text{by \DEF{3.1.23}}
        \end{align*}
        By \DEF{3.1.4}, \(f^{-1}(U \cap V) = f^{-1}(U) \cap f^{-1}(V)\)
    \item \MAROON{(3)}: Let \(x\) be an arbitrary object. Then
        \begin{align*}
                 & x \in f^{-1}(U \setminus V) \\
            \iff & f(x) \in U \setminus V & \text{by \DEF{3.4.4}} \\
            \iff & f(x) \in U \land f(x) \notin V & \text{by \DEF{3.1.27}} \\
            \iff & x \in f^{-1}(U) \land x \notin f^{-1}(V) & \text{by \DEF{3.4.4}} \\
            \iff & x \in f^{-1}(U) \setminus f^{-1}(V) & \text{by \DEF{3.1.27}}
        \end{align*}
        By \DEF{3.1.4}, \(f^{-1}(U \setminus V) = f^{-1}(U) \setminus f^{-1}(V)\)
\end{itemize}

\end{proof}

\begin{exercise} \label{exercise 3.4.5}
Let \(f : X \to Y\) be a function from one set \(X\) to another set \(Y\).
Show that \(f(f^{-1}(S)) = S\) for every \(S \subseteq Y\) if and only if \(f\) is \emph{surjective} \MAROON{(1)}.
Show that \(f^{-1}(f(S)) = S\) for every \(S \subseteq X\) if and only if \(f\) is \emph{injective} \MAROON{(2)}.
\end{exercise}

\begin{proof}
In \EXEC{3.4.2} we have shown
\[\forall S \subseteq Y, f(f^{-1}(S)) \subseteq S\]
\[\forall S \subseteq X, S \subseteq f^{-1}(f(S))\]
This implies for \MAROON{(1)}, we only have to show
\[\forall S \subseteq Y, S \subseteq f(f^{-1}(S)) \iff f \text{\ is surjective \MAROON{(3)}}\]
For \MAROON{(2)}, we only have to show
\[\forall S \subseteq X, f^{-1}(f(S)) \subseteq S \iff f \text{\ is injective \MAROON{(4)}}\]

\begin{itemize}
    \item \MAROON{(3)}:
        \begin{itemize}
            \item \(\Longrightarrow\):
                Suppose \(\forall S \subseteq Y, S \subseteq f(f^{-1}(S))\) \GREEN{(1)}.
                Suppose for the sake of contradiction that \(f\) is not surjective.
                Then \(\exists y \in Y\) s.t. \(\forall x \in X, f(x) \neq y\).
                Now let \(S = \{ y \}\), so of course \(y \in S\).
                By \DEF{3.4.4}, \(f^{-1}(S) = \emptyset\), so \(f(f^{-1}(S)) = \emptyset\), so \(y \notin f(f^{-1}(S))\).
                Now we have \(y \in S\) and \(y \notin (f(f^{-1}(S))\), so \(S \not\subseteq f(f^{-1}(S))\), contradicting \GREEN{(1)}.
                So \(f\) must be surjective.
            \item \(\Longleftarrow\):
                Suppose \(f\) is surjective. Let \(S \subseteq Y\), we have to show \(S \subseteq f(f^{-1}(S))\),
                that is, if \(y \in S\), then \(y \in f(f^{-1}(S))\).
                So Suppose \(y \in S\). Then since \(f\) is surjective, \(\exists x \in X\) s.t. \(f(x) = y\).
                By \DEF{3.4.4}, \(x \in \GREEN{f^{-1}(S)}\),
                and by \DEF{3.4.1}, \(f(x) \in f(\GREEN{f^{-1}(S)})\), that is, \(y \in f(\GREEN{f^{-1}(S)})\).
        \end{itemize}
    \item \MAROON{(4)}:
        \begin{itemize}
            \item \(\Longrightarrow\):
                Suppose \(\forall S \subseteq X, f^{-1}(f(S)) \subseteq S\) \GREEN{(1)}.
                Suppose for the sake of contradiction that \(f\) is not injective.
                Then \(\exists x_1, x_2 \in X\) s.t. \(x_1 \neq x_2 \land f(x_1) = f(x_2)\).
                Now let \(S = \{ x_1 \} \), so \(x_{\RED{2}} \notin S\).
                Clearly \(f(S) = \{ y \}\) s.t. \(y = f(x_1)\),
                and also clearly by \DEF{3.4.4}, \(f^{-1}(f(S)) = f^{-1}( \{ y \} ) = \{ x_1, x_2 \}\),
                so \(x_{\RED{2}} \in f^{-1}(f(S))\).
                So we have \(x_2 \notin S\) but \(x_2 \in f^{-1}(f(S))\),
                so \(f^{-1}(f(S)) \not\subseteq S\), contradicting \GREEN{(1)},
                so \(f\) must be injective.
            \item \(\Longleftarrow\):
                Suppose \(f\) is injective \GREEN{(1)}.
                Now for the sake of contradiction, suppose \(\exists S \subseteq X\) s.t. \(f^{-1}(f(S)) \not\subseteq S\) \GREEN{(2)}.
                Then \(\exists x_1\) s.t. \(x_1 \in f^{-1}(f(S))\) (so \(f(x_1) \in f(S)\)), and \(x_1 \notin S\).
                But that means \(\exists x_2 \in S\) s.t. \(f(x_1) = f(x_2) \land x_1 \neq x_2\) \GREEN{(3)},
                (otherwise \(\forall x \in S, f(x) \neq f(x_1) \in S\),
                which by \DEF{3.4.4} implies \(x_1 \notin f^{-1}(f(S))\)).
                But \GREEN{(3)} by definition of injective means \(f\) is not injective, contradicting \GREEN{(1)}.
                So \GREEN{(2)} is false,
                that is, \(\forall S \subseteq X, f^{-1}(f(S)) \subseteq S\).
        \end{itemize}
\end{itemize}
\end{proof}

\begin{exercise} \label{exercise 3.4.6}
Prove \LEM{3.4.9}.
(Hint: start with the set \(\{0, 1\}^X\) and apply the replacement axiom \AXM{3.6}, replacing each function \(f\) with the object \(f^{-1}( \{1\} )\).)
See also \EXEC{3.5.11}.
\end{exercise}

\begin{proof}
Suppose \(X\) is a set. We have to find a set \(S\) s.t. every element of \(S\) is a subset of \(X\) and every subset of \(X\) is an element of \(S\).
By \AXM{3.10}, there exists a set \(\{0, 1\}^X\) which consists of all the functions \(f\) from \(X\) to \(\{0, 1\}\):
\[
    f \in \{0, 1\}^X \iff \text{\(f\) is a function with domain \(X\) and codomain \(\{0, 1\}\)}).
\]
By \AXM{3.6}, we can replace each \(f \in \{0, 1\}^X\) with \(f^{-1}(\{ 1 \})\), i.e. there exists a set
\[
    S = \{ f^{-1}(\{ 1 \}) : f \in \{ 0, 1 \}^X \}.
\]
By \DEF{3.4.4}, every element \(\in S\) is a subset of \(X\) since given \(f \in \{0, 1\}^X, f^{-1}(\{ 1 \}) \subseteq X\).
The last step is to prove every subset of \(X\) is in \(S\), i.e. if \(A \subseteq X\), then \(A \in S\).
Now given any subset \(A \subseteq X\), we can define a function \(f : X \rightarrow \{0, 1\}\), where
    \begin{equation*}
        f(x) =
        \begin{cases}
          0, & \text{if}\ x \notin A \\
          1, & \text{if}\ x \in A
        \end{cases}
    \end{equation*}
and \(f\) is trivially well-defined.
Then clearly \(f \in \{0, 1\}^X\) and \(f^{-1}(\{ 1 \}) = A\).
Also by definition of \(S\), \(f^{-1}(\{ 1 \}) \in S\), that is, \(A \in S\).
\end{proof}

\begin{exercise} \label{exercise 3.4.7}
Let \(X\), \(Y\) be sets.
Define a \emph{partial function} from \(X\) to \(Y\) to be any function \(f : X' \to Y'\) whose domain \(X'\) is a subset of \(X\), and whose codomain \(Y'\) is a subset of \(Y\).
Show that the collection of all partial functions from \(X\) to \(Y\) is itself a set.
\end{exercise}

\begin{exercise} \label{exercise 3.4.8}
Show that \AXM{3.4} can be deduced from \AXM{3.1}, \AXM{3.3} and \AXM{3.11}.
\end{exercise}

\begin{exercise} \label{exercise 3.4.9}
Show that if \(\beta\) and \(\beta'\) are two elements of a set \(I\), and to each \(\alpha \in I\) we assign a set \(A_{\alpha}\), then
\[
    \{x \in A_{\beta} : x \in A_{\alpha} \ \forall\ \alpha \in I\} = \{x \in A_{\beta'} : x \in A_{\alpha} \ \forall\ \alpha \in I\},
\]
and so the definition of \(\bigcap_{\alpha \in I} A_{\alpha}\) does not depend on \(\beta\).
Also explain why \ref{formula 3.4} is true.
\end{exercise}

\begin{exercise} \label{exercise 3.4.10}
Suppose that \(I\) and \(J\) are two sets, and for all \(\alpha \in I \cup J\) let \(A_{\alpha}\) be a set.
Show that \((\bigcup_{\alpha \in I} A_{\alpha}) \cup (\bigcup_{\alpha \in J} A_{\alpha}) = \bigcup_{\alpha \in I \cup J} A_{\alpha}\).
If \(I\) and \(J\) are non-empty, show that \((\bigcap_{\alpha \in I} A_{\alpha}) \cap (\bigcap_{\alpha \in J} A_{\alpha}) = \bigcap_{\alpha \in I \cup J} A_{\alpha}\).
\end{exercise}

\begin{exercise} \label{exercise 3.4.11}
Let \(X\) be a set, let \(I\) be a non-empty set, and for all \(\alpha \in I\) let \(A_{\alpha}\) be a subset of \(X\).
Show that
\[
    X \setminus \bigcup_{\alpha \in I} A_{\alpha} = \bigcap_{\alpha \in I} (X \setminus A_{\alpha})
\]
and
\[
    X \setminus \bigcap_{\alpha \in I} A_{\alpha} = \bigcup_{\alpha \in I} (X \setminus A_{\alpha}).
\]
This should be compared with de Morgan’s laws in \PROP{3.1.28}
(although one cannot derive the above identities directly from de Morgan’s laws, as \(I\) could be infinite).
\end{exercise}