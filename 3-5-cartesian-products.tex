\section{Cartesian products} \label{sec 3.5}

\begin{definition} [Ordered pair] \label{def 3.5.1}
If \(x\) and \(y\) are any objects (possibly equal), we define the \textbf{ordered pair} \((x, y)\) to be a new \emph{object}, consisting of \(x\) as its \emph{first} component and \(y\) as its \emph{second} component.
Two ordered pairs \((x, y)\) and \((x', y')\) are considered equal if and only if both their components match, i.e.
\begin{align} \label{formula 3.5}
    (x, y) = (x', y') \iff (x = x' \text{\ and\ } y = y').
\end{align}
\end{definition}

\begin{note}
\((3, 5) \neq (5, 3)\) but \(\{ 3, 5 \} = \{ 5, 3 \}\).
\end{note}

\begin{remark} \label{remark 3.5.2}
Strictly speaking, this definition is \emph{partly an axiom}, because we have simply \emph{postulated} that given any two objects \(x\) and \(y\), that an object of the form \((x, y)\) \textbf{exists}.
However, it is possible to \emph{define} an ordered pair using the axioms of set theory in such a way that we do not need any further postulates (see \EXEC{3.5.1}).
\end{remark}

\begin{remark} \label{remark 3.5.3}
We have now ``overloaded'' the parenthesis symbols \(()\) once again;
they now are not only used to denote grouping of operators and arguments of functions, but also to enclose ordered pairs.
This is usually not a problem in practice as one can still determine what usage the symbols \(()\) were intended for from context.
\end{remark}

\begin{definition} [Cartesian product] \label{def 3.5.4}
If \(X\) and \(Y\) are sets, then we define the \emph{Cartesian product} \(X \X Y\) to be the \textbf{collection} of ordered pairs, whose first component lies in \(X\) and second component lies in \(Y\),
thus
\[
    X \X Y = \{ (x, y): x \in X, y \in Y \}
\]
or equivalently
\[
    a \in (X \X Y ) \iff (a = (x, y) \text{\ for some \(x \in X\) and \(y \in Y\)}).
\]
\end{definition}

\begin{remark} \label{remark 3.5.5}
The definition use the term ``collection''.
However, one can show that the Cartesian product \(X \X Y\) is indeed a set; see \EXEC{3.5.1}.
\end{remark}

\begin{example}
幹話。
\end{example}

Strictly speaking, \(X \X Y\) and \(Y \X X\) are different sets, although they are very similar.
For instance, they always have the same number of elements (\EXEC{3.6.5}).

Let \(f : X \X Y \rightarrow Z\) be a function whose \emph{domain \(X \X Y\)} is \emph{a Cartesian product} of two other sets \(X\) and \(Y\).
Then \(f\) can \textbf{either} be thought of as a function of one variable, mapping the single input of an ordered pair \((x, y)\) in \(X \X Y\) to an output \(f( (x, y) )\)in \(Z\),
\textbf{or} as a function of two variables, mapping an input \(x \in X\) and another input \(y \in Y\) to a single output \(f(x, y)\) in \(Z\).
While the two notions are technically different, we will not bother to distinguish the two,
and think of \(f\) simultaneously as a function of one variable with domain \(X \X Y\)
and as a function of two variables with domains \(X\) and \(Y\).
Thus for instance the addition operation \(+\) on the natural numbers can now be re-interpreted as a function \(+ : \SET{N} \X \SET{N} \rightarrow \SET{N}\), defined by \((x, y) \rightarrow x + y\).

One can of course generalize the concept of ordered pairs to ordered
triples, ordered quadruples, etc:

\begin{definition} [Ordered n-tuple and n-fold Cartesian product] \label{def 3.5.7}
Let \(n\) be a natural number.
An ordered \(n\)-tuple \((x_i)_{1 \leq i \leq n} \) (also denoted \( (x_1, ..., x_n) \) ) is a collection of objects \(x_i\),
one for every natural number \(i\) between \(1\) and \(n\);
we refer to \(x_i\) as the \(i\)th component of the ordered \(n\)-tuple.
Two ordered \(n\)-tuples \((x_i)_{1 \leq i \le n}\) and \((y_i)_{1 \leq i \leq n}\) are said to be equal iff \(x_i = y_i\) for all \(1 \leq i \le n\).
If \((X_i)_{1 \leq i \leq n}\) is an ordered \(n\)-tuple of \emph{sets}, we define there \emph{Cartesian product} \(\prod_{1 \leq i \leq n}X_i\) (also denoted \(\prod_{i=1}^{n}X_i\) or \(X_1 \X ... \X X_n\)) by
\[
    \prod_{1 \leq i \leq n} X_i := \{ (x_i)_{1 \leq i \leq n} : x_i \in X_i \text{\ for all \(1 \leq i \leq n\)} \}.
\]
\end{definition}

Again, this definition simply \emph{postulates} that an ordered \(n\)-tuple and a Cartesian product \emph{always exist} when needed,
but using the axioms of set theory one can explicitly construct these objects (\EXEC{3.5.2}).

\begin{remark} \label{remark 3.5.8}
One can show that \(\prod_{1 \leq i \leq n} X_i\) is indeed a set.
Indeed, from the power set axiom \AXM{3.10} we can consider the set of all functions \(i \mapsto x_i\) from the \emph{domain} \(\{1 \leq i \leq n\}\) to the codomain \(\bigcup_{1 \leq i \leq n} X_i\),
and then we can restrict using the axiom of specification \AXM{3.5} to restrict to those functions \(i \mapsto x_i\) for which \(x_i \in X_i\) for all \(1 \leq i \leq n\).
One can generalize this construction to \emph{infinite} Cartesian products, see \DEF{8.4.1}.
\end{remark}

\begin{note}
上面的\ remark,用\ \AXM{3.10} 把想要\ \(n\)-tuple 都挑出來,
但是會挑到一些不屬於\ \(\prod_{1 \leq i \leq n} X_i\) 內的\ \(n\)-tuple。
所以再用\ \AXM{3.5} 把那些多餘的\ \(n\)-tuple 拿掉。
\end{note}

\begin{example}
The example shows the sets \(X_1 \times X_2 \times X_3\), \((X_1 \times X_2) \times X_3\), and \(X_1 \times (X_2 \times X_3)\) are distinct.
However, they are clearly very related to each other (for instance, there are obvious bijections between any two of the three sets),
and it is common in practice to neglect the minor distinctions between these sets and pretend that they are in fact equal.
Thus a function \(f : X_1 \times X_2 \times X_3 \to Y\) can be thought of as
a function of one variable \((x_1, x_2, x_3) \in X_1 \times X_2 \times X_3\),
or as a function of three variables \(x_1 \in X_1\), \(x_2 \in X_2\), \(x_3 \in X_3\),
or as a function of two variables \(x_1 \in X_1\), \((x_2, x_3) \in X_2 \times X_3\), and so forth;
we will not bother to distinguish between these different perspectives.
\end{example}

\begin{remark} \label{remark 3.5.10}
An ordered \(n\)-tuple \(x_1, ..., x_n\) of objects is also called an \emph{ordered sequence} of \(n\) elements, or a \textbf{finite} sequence for short.
In \CH{5} we shall also introduce the very useful concept of an \textbf{infinite}
sequence.
\end{remark}

\begin{example} \label{example 3.5.11}
If \(x\) is an object, then \((x)\) is a \(1\)-tuple, which we shall identify with \(x\) itself (even though the two are, strictly speaking, not the same object).
Then if \(X_1\) is any set, then the Cartesian product \(\prod_{1 \leq i \leq 1} X_i\) is just \(X_1\) (why? \MAROON{(1)}).
Also, the \emph{empty Cartesian product} \(\prod_{1 \leq i \leq 0} X_i\) gives, not the empty set \(\{\}\), but rather the singleton set \(\{()\}\) whose only element is the \emph{\(0\)-tuple} \(()\), also known as the \emph{empty tuple}.
\end{example}

\begin{proof} \MAROON{(1)}
\(\prod_{1 \leq i \leq 1} X_i\) by \DEF{3.5.7} is equal to
\[
    \{ (x_i)_{1 \leq i \leq 1} : x_i \in X_i \text{\ for all \(1 \leq i \leq 1\)} \}.
\]
that is,
\[
    \{ (x_1) : x_1 \in X_1 \}.
\]
But we shall identify \((x_1)\) as \(x_1\), so the set is equivalent to
\[
    \{ x_1 : x_1 \in X_1 \}.
\]
Which is equal to \(X_1\).
\end{proof}

\begin{note}
If \(n\) is a natural number, we often write \(X^n\) as shorthand for the \(n\)-fold Cartesian product \(X^n := \prod_{1 \leq i \leq n} X\).
Thus \(X^1\) is essentially the same set as \(X\) (if we ignore the distinction between an object \(x\) and the \(1\)-tuple \((x)\)), while \(X^2\) is the Cartesian product \(X \times X\).
The set \(X^0\) is a singleton set \(\{()\}\) (why? because it is just a shorthand of empty Cartesian product).
\end{note}

We can now generalize the single choice lemma (\LEM{3.1.6}) to allow for multiple (but finite) number of choices.

\begin{lemma} [Finite choice] \label{lem 3.5.12}
Let \(n \geq 1\) be a natural number, and for each natural number \(1 \leq i \leq n\), let \(X_i\) be a non-empty set.
Then \emph{there exists an \(n\)-tuple} \((x_i)_{1 \leq i \leq n}\) such that \(x_i \in X_i\) for all \(1 \leq i \leq n\).
In other words, if each \(X_i\) is non-empty, then the set \(\prod_{1 \leq i \leq n} X_i\) is also non-empty.
\end{lemma}

\begin{proof}
We induct on \(n\) (starting with the base case \(n = 1\); the claim is also vacuously true with \(n = 0\), because the hypothesis in \LEM{3.5.12} requires \(n \ge 1\) and therefore is false). When \(n = 1\) the claim follows from \LEM{3.1.6},
(why? because we identify \(\prod_{1 \leq 1 \leq n} X_1\) as just \(X_1\), see \EXAMPLE{3.5.11}; and by \LEM{3.1.6} we can choose element \(x_1\) from \(X_1\).)
Now suppose inductively that the claim has already been proven for some \(n\); we will now prove it for \(n\INC\).
Let \(X_1, ... , X_{n\INC}\) be a collection of nonempty sets.
By induction hypothesis, we can find an \(n\)-tuple \((x_i)_{1 \leq i \leq n}\) such that \(x_i \in X_i \text{\ for all \(1 \leq i \leq n\)}\).
Also, since \(X_{n\INC}\) is non-empty, by \LEM{3.1.6} we may find an object \(a\) such that \(a \in X_{n\INC}\).
If we thus define the \(n\INC\)-tuple \((y_i)_{1 \leq i \leq n\INC}\) by setting \(y_i := x_i\) when \(1 \leq i \leq n\) and \(y_i := a\) when \(i = n\INC\) it is clear that \(y_i \in X_i\) for all \(1 \leq i \leq n\INC\),
thus closing the induction.
\end{proof}

\begin{remark} \label{remark 3.5.13}
It is intuitively plausible that this lemma should be extended to allow for an \emph{infinite} number of choices, but this cannot be done automatically;
it requires an additional axiom, the \emph{axiom of choice}. See \SEC{8.4}.
\end{remark}

\exercisesection

\begin{exercise} \label{exercise 3.5.1}
(a) Suppose we define the ordered pair \((x, y)\) for any objects \(x\) and \(y\) by the formula \((x, y):= \{ \{x\}, \{x, y\} \}\) (thus using several applications of \AXM{3.3}).
Thus for instance \((1, 2)\) is the set \(\{ \{1\}, \{1, 2\} \}\), \((2, 1)\) is the set \( \{ \{2\}, \{2, 1\} \} \), and \((1, 1)\) is the set \( \{ \{1\}, \{1, 1\} = \{ \{1\}, \{1\} \} = \{ \{1\} \} \).
Show that such a definition indeed obeys the property \ref{formula 3.5}, and also whenever \(X\) and \(Y\) are sets, the Cartesian product \(X \X Y\) is also a set.
Thus this definition can be validly used as a definition of an ordered pair.

(b) For an additional challenge, show that the alternate definition \((x, y) := \{x, \{x, y\} \}\) also verifies \ref{formula 3.5} and is thus also an acceptable definition of ordered pair.
(For this latter task one needs the axiom of regularity \AXM{3.9}, and in particular \EXEC{3.2.2}.)

(c): Show that regardless of the definition of ordered pair, the Cartesian product \(X \X Y\) is a set.
(Hint: first use the axiom of replacement \AXM{3.6} to show that for any \(x \in X\), the set \(\{ (x, y): y \in Y \}\) is a set, then apply the axioms of replacement and union.).
\end{exercise}

