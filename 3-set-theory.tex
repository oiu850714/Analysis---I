\chapter{Set Theory} \label{ch3}

For now we pause to introduce the concepts and notation of set theory, as they will be used increasingly heavily in
later chapters.

While set theory is not the main focus of this text, almost every other branch of mathematics relies on set theory as part of its foundation, so it is important to get at least some grounding in set theory before doing other advanced areas of mathematics.

In this chapter we present the more elementary aspects of \emph{axiomatic} set theory, leaving more advanced topics such as a discussion of infinite sets and the axiom of choice to
\CH{8}.

\begin{note}
為何到第八章才講那些進階的東西?\ 因為直到第八章之前的定理都不用依靠那些東西。
\end{note}

\section{Fundamentals}\label{sec 3.1}

For pedagogical reasons, we will use a \emph{somewhat overcomplete list of axioms} for set theory, in the sense that some of the axioms can be used to deduce others, but there is \emph{no real harm} in doing this.

\begin{note}
就好像你可以說正整數除了滿足皮亞諾公理也自動滿足\ \(a + b = b + a\),但後者用皮亞諾就可證明。
\end{note}

\begin{definition}[\emph{Informal}] \label{def 3.1.1}
We define a set \(A\) to be any \emph{unordered collection} of objects, e.g., \( \{3, 8, 5, 2\} \) is a set. If \(x\) is an object, we say that \(x\) \emph{is an element of} \(A\) or \(x \in A\) if \(x\) lies in the collection; otherwise we say that \(x \notin A\). For instance, \(3 \in \{1, 2, 3, 4, 5\} \) but \(7 \notin \{1, 2, 3, 4, 5\} \).
\end{definition}

\begin{note}
\DEF{3.1.1} 有很多問題沒有回答,例如「什麼樣的\ collection」 才能被稱作集合,兩個集合怎麼判斷是否相等,怎麼對集合作操作(聯集、交集等等),集合可以做什麼,以及集合的元素(element)可以做什麼。
\end{note}

\begin{axiom}[Sets are objects]\label{axm 3.1}
If \(A\) is a set, then \(A\) is \emph{also an object}. In particular, given two sets \(A\) and \(B\), it is meaningful to ask whether \(A\) is also an element of \(B\).
\end{axiom}

\begin{example}[Informal]\label{example 3.1.2}
這個例子舉例\ \( \{3, \{3, 4\}, 4\} \) 裡面有一個元素也是集合,但敘述方式不嚴謹,要去看\ \SEC{3.6}
\end{example}

\begin{remark}\label{remark 3.1.3}
這裡在探討是否需要把所有\ object 都當成\  set。在邏輯的角度,這樣推論過程比較簡單,因為需要的東西的類型就只有一種,就是\ set,但是從概念上來看,將某些\ object 視為「不是\ set」則會比較單純,比方說給定一個自然數\ \(2\),將他視為一個集合(在\ Analysis 的範疇)沒什麼進一步的應用。是否將所有\ object 都當成集合,\ more or less 是等價的,所以,we shall take an agnostic position as to whether all objects are sets or not.
\end{remark}

\begin{note}
若已知\ \(x\) 是一個\ object 且\ \(A\) 是一個\ set,則要馬\ \(x \in A\) 為真,要馬\ \(x \notin A\) 為真。而若\ \(A\) 不是\ set,則我們視\ \(x \in A\) 為\ undefined。
\end{note}

\begin{definition}[Equality of sets] \label{def 3.1.4} 
Two sets \(A\) and \(B\) are equal, \(A = B\), if and only if every element of \(A\) is an element of \(B\) and vice versa. To put it another way, \(A = B\) if and only if every element \(x\) of \(A\) belongs also to \(B\), and every element \(y\) of \(B\) belongs also to \(A\). Or equivalently,
\[
  \forall\ x : x \in A \iff x \in B
\]
\end{definition}

\begin{example}
嘴砲。
\end{example}

\begin{note}
One can easily verify that this notion of equality is reflexive, symmetric, and transitive (See \EXEC{3.1.1}).
\end{note}

\begin{additional corollary}\label{ac 3.1.1}
The definition of equality in \DEF{3.1.4} is reflexive, symmetric and transitive.
\end{additional corollary}

\begin{proof}

Reflexive: Suppose \(A\) is a set. Then given any object \(x\), if \(x \in \GREEN{A}\), then \(x \in \BLUE{A}\), and given any object \(y\), if \(y \in \BLUE{A}\), then \(y \in \GREEN{A}\). So by \DEF{3.1.4}, \(\GREEN{A} = \BLUE{A}\).

Symmetric: Suppose \(A, B\) are sets and \(A = B\), then by \DEF{3.1.4},
\[
  \forall\ x : x \in A \iff x \in B
\]
But this statement is just equivalent to
\[
  \forall\ x : x \in B \iff x \in A
\]
and this by \DEF{3.1.4} implies \(B = A\).

Transitive: Suppose \(A, B, C\) are sets and \(A = B\) and \(B = C\). Then by \DEF{3.1.4}
\[
  \forall\ x : x \in A \iff x \in B
\]
\[
  \forall\ x : x \in B \iff x \in C
\]
And this implies
\[
  \forall\ x : x \in A \iff x \in B \iff x \in C
\]
And by logic this implies
\[
  \forall\ x : x \in A \iff x \in C
\]
By \DEF{3.1.4}, \(A = C\).
\end{proof}

\begin{note}
``is an element of'' relation \(\in\) 符合\ Axiom of Substitution \AXM{a.7.4},因為
\begin{center}
    if \(x \in A\) and \(A = B\), then \(x \in B\), by \DEF{3.1.4}.
\end{center}
這也代表那些完全以\ \(\in\) 定義的新的集合操作會自動符合\ Axiom of Substitution \AXM{a.7.4}。例如這一節剩下的所有\ operations 都是用\ \(\in\) 來定義的。
\end{note}
\begin{note}
接著\ \RMK{3.1.3},我們繼續來探討什麼\ object 是\ set,什麼不是。有點類似我們定哪些東西為自然數,哪些不是(\AXM{2.1},\(0\) 是自然數,然後用 \AXM{2.2} 來擴增/建構其他的自然數)。這邊在集合論就是先假設存在一個集合,叫「空集合」,然後再定義一些在集合上的操作來建構其他的集合。
\end{note}

\begin{axiom}[Empty set] \label{axm 3.2}
There exists a set \(\emptyset\), known as \emph{the} empty set, which \emph{contains no elements}, i.e., for every object \(x\) we have \(x \notin \emptyset\).
\end{axiom}

\begin{note}
\emph{The} empty set is also denoted \(\{\}\). Note that there can only be \textbf{one} empty set.
\end{note}

\begin{additional corollary} [The empty set is unique] \label{ac 3.1.2}
If there were two sets \(\emptyset\) and \(\emptyset'\) which were both empty, then by \DEF{3.1.4} they would be equal to each other.
\end{additional corollary}
\begin{proof}
Suppose \(\emptyset'\) is also empty. Then the statement
\[
  \forall\ x : x \in \emptyset' \implies x \in \emptyset
\]
is vacuously true because by \AXM{3.2} empty set contains no elements. And again by \AXM{3.2}, the statement
\[
  \forall\ x : x \in \emptyset \implies x \in \emptyset'
\]
is also vacuous. These imply
\[
  \forall\ x : (x \in \emptyset' \implies x \in \emptyset) \land (x \in \emptyset \implies x \in \emptyset')
\]
that is,
\[
  \forall\ x : x \in \emptyset' \iff x \in \emptyset
\]
by \DEF{3.1.4}, \(\emptyset' = \emptyset\)
\end{proof}

\begin{note}
If a set is not equal to the empty set, we call it \emph{non-empty}.
\end{note}

\begin{lemma}[Single choice]\label{lem 3.1.6}
Let \(A\) be a \emph{non-empty} set. Then there exists an object \(x\) such that \(x \in A\).
\end{lemma}
\begin{proof}
Suppose for the sake of contradiction that \(A\) be a non-empty set and for all object \(x\), \(x \notin A\). And by \AXM{3.2}, \(x \notin \emptyset\). Then similarly as \AC{3.1.2}, we can derive \(A = \emptyset\), which contradicts that \(A\) is \emph{non-empty}.
\end{proof}

\begin{remark}\label{remark 3.1.7}
\LEM{3.1.6} asserts that given any non-empty set \(A\), we are allowed to \emph{``choose''} an element \(x\) of \(A\) which demonstrates this non-emptyness. Later on (in \LEM{3.5.12}) we will show that given any \textbf{finite} number of non-empty sets, say \(A_1, \dots, A_n\), it is possible to choose one element \(x_1, \dots, x_n\) from each set \(A_1, \dots, A_n\); this is known as ``finite choice''. However, in order to choose elements from an \textbf{infinite} number of sets, we need an additional axiom, the \emph{axiom of choice} (\AXM{8.1}).
\end{remark}

\begin{note}
\RMK{3.1.7} 在講\ ``finite choice'' 的部分,看起來是說要被選的集合是有限個,但是沒有規定個別集合的元素數量要有限個,目前還不確定這意味著什麼。
\end{note}

\begin{remark} \label{remark 3.1.8}
Note that the empty set is not the same thing as the natural number \(0\). One is a set; the other is a number. However, it is true that the \emph{cardinality} of the empty set is \(0\); see \SEC{3.6}.
\end{remark}

We now present further axioms to enrich the class of sets available.

\begin{axiom}[Singleton sets and pair sets]\label{axm 3.3}
If \(a\) is an object, then there exists a set \( \{a\} \) whose \emph{only} element is \(a\), i.e., for every object \(y\), we have \(y \in \{a\}\) if and only if \(y = a\); we refer to \( \{a\} \) as the \emph{singleton set} whose element is \(a\). Furthermore, if \(a\) and \(b\) are objects, then there exists a set \( \{a, b\} \) whose only elements are \(a\) and \(b\); i.e., for every object \(y\), we have \( y \in \{a, b\} \) if and only if \(y = a\) or \(y = b\); we refer to this set as the \emph{pair set} formed by \(a\) and \(b\).
\end{axiom}

\begin{note}
\href{https://www.wikiwand.com/en/Axiom_of_pairing#/Consequences}{參考}: 這個公理實際說的是,給定兩個集合(這邊暫時當作所有物件都是集合)\ \(x\) 和\ \(y\),我們可以找到一個集合\ \(A\) ,它的成員就是\ \(x\) 和\ \(y\)。
\end{note}

\begin{note}
前方高能注意: \RMK{3.1.9} 在解釋\ \AXM{3.3} 裡面的\ singleton、\ pair,還有\ \AXM{3.4} 這三者,若假設``其中一部分''是公理,則剩下的可以從那個部分直接推得,不用當作是公理,i.e. 剩下的只須為定理,不用是公理。這種\ redundant\ 在本節開頭有講過,便於推導,but does no real harm。BTW 這個\ remark 裡面有一堆\ (why?) 全部都要自己推導。
\end{note}

\begin{remark} \label{remark 3.1.9}

Just as there is only one empty set, there is only one singleton set for each object \(a\), thanks to \DEF{3.1.4} (why? \MAROON{(1)}).

Similarly, given any two objects \(a\) and \(b\), there is only one pair set formed by \(a\) and \(b\).

Also, \DEF{3.1.4} also ensures that \( \{a, b\} = \{b, a\} \) (why? \MAROON{(2)}) and \( \{a, a\} = \{a\} \) (why? \MAROON{(3)}). Thus the \textbf{singleton} set axiom is in fact redundant, being \textbf{a consequence of} the \textbf{pair} set axiom.

\emph{Conversely}, the \textbf{pair set} axiom will \textbf{follow from} the \textbf{singleton} set axiom \textbf{and} the \textbf{pairwise union axiom \AXM{3.4}} (see  \LEM{3.1.13}).

One may wonder why we don’t go further and create triplet axioms, quadruplet axioms, etc.; however there will be no need for this once we introduce the pairwise union axiom
below.
\end{remark}

\begin{proof}
\MAROON{(1)}: given any object \(a\), suppose there exist two sets \(A\) and \(A'\) which are singleton sets of \(a\). Then we have:
\begin{align*}
         & (\forall\ x : x \in A \iff x = a) \land (\forall\ x : x \in A' \iff x = a) & \text{by \AXM{3.3}} \\
    \iff & (\forall\ x : x \in A \iff x = a) \land (\forall\ x : x = a \iff x \in A') & \text{by logic} \\
    \iff & (\forall\ x : x \in A \iff x = a \iff x \in A')                            & \text{by logic} \\
    \iff & (\forall\ x : x \in A \iff x \in A')                                       & \text{by simplifying logic} \\
    \iff & A = A'                                                                     & \text{by \DEF{3.1.4}}
\end{align*}

\MAROON{(2)}: given any objects \(a, b\), then
\begin{align*}
    & (x \in \{a, b\} \iff (x = a \lor x = b)) & \text{by \AXM{3.3}} \\
    \iff & (x \in \{a, b\} \iff (x = a \lor x = b) \iff (x = b \lor x = a)) & \text{by logic} \\
    \iff & (x \in \{a, b\} \iff (x = b \lor x = a)) & \text{by simplifying logic} \\
    \iff & (x \in \{a, b\} \iff (x = b \lor x = a) \iff x \in \{b, a\}) & \text{by \AXM{3.3}} \\
    \iff & (x \in \{a, b\} \iff x \in \{b, a\}) & \text{by simplifying logic} \\
    \iff & \{a, b\} = \{b, a\} & \text{by \DEF{3.1.4}}
\end{align*}

\MAROON{(3)}: given any object \(a\), then
\begin{align*}
    & (x \in \{a, a\} \iff (x = a \lor x = a)) & \text{by \AXM{3.3}} \\
    \iff & (x \in \{a, a\} \iff (x = a)) & \text{by logic} \\
    \iff & (x \in \{a, a\} \iff (x = a) \iff x \in \{a\})& \text{by \AXM{3.3}} \\
    \iff & (x \in \{a, a\} \iff x \in \{a\})& \text{by simplifying logic} \\
    \iff & \{a, a\} = \{a\} & \text{by \DEF{3.1.4}}
\end{align*}
\end{proof}

\begin{example} \label{example 3.1.10}
Since \(\emptyset\) is a set (and hence an object), so is singleton set \(\{ \emptyset \}\), i.e., the set whose only element is \( \emptyset \), is a set (and it is not the same set as \( \emptyset \), \( \{ \emptyset \} \neq  \emptyset \) (why? See \EXEC{3.1.2}). Similarly, the singleton set \( \{ \{ \emptyset \} \} \) and the pair set \( \{ \emptyset, \{ \emptyset \} \} \) are also sets. These three sets are not equal to each other (\EXEC{3.1.2}).
\end{example}

\begin{note}
現在有的三個公理已經可以讓我們建構出一堆集合了,但是他們的元素都不超過兩個,雖然元素本身可能長得很複雜。
\end{note}

\begin{axiom} [Pairwise union] \label{axm 3.4}
Given any two sets \(A, B\), there exists a set \(A \cup B\), called the \emph{union} \(A \cup B\) of \(A\) and \(B\), whose elements consists of all the elements which belong to \(A\) or \(B\) or both. In other words, for any object \(x\),
\[
    x \in A \cup B \iff (x \in A \lor x \in B)
\].
\end{axiom}

\begin{note}
注意\ Union 的定義完全是由\ \( \in \) (還有\ or) 兜出來的,所以符合替換公理。
\end{note}

\begin{example}
很廢。\( \{1, 2\} \cup \{ 2, 3 \} = \{ 1, 2, 3 \} \)
\end{example}

\begin{remark} \label{remark 3.1.12}
If \(A, B\) are sets, \(A'\) is also a set which is equal to \(A\), then \(A \cup B\) is equal to \(A' \cup B\) (why? \MAROON{(1)} One needs to use \AXM{3.4} and \DEF{3.1.4}). Similarly if \(B'\) is a set which is equal to \(B\), then \(A \cup B\) is equal to \(A \cup B'\). Thus the operation of union \emph{obeys the axiom of substitution}, and is thus well-defined on sets.
\end{remark}

\begin{proof}
\MAROON{(1)}: Suppose \(A, B, A'\) are sets such that \(A = A'\). Then given any object \(x\),
\begin{align*}
         & x \in A \cup B \\
    \iff & x \in A \lor x \in B & \text{by \AXM{3.4}} \\
    \iff & x \in A' \lor x \in B & \text{since \(A = A'\) and \(=\) satisfies \AXM{a.7.4}} \\
    \iff & x \in A' \cup B & \text{by \AXM{3.4}} \\
\end{align*}
So \(\forall x, x \in A \cup B \iff x \in A' \cup B\), so by \DEF{3.1.4} \(A \cup B = A' \cup B\).

Now suppose \(B' = B\), Then given any object \(x\),
\begin{align*}
         & x \in A \cup B \\
    \iff & x \in A \lor x \in B & \text{by \AXM{3.4}} \\
    \iff & x \in A \lor x \in B' & \text{since \(B = B'\) and \(=\) satisfies \AXM{a.7.4}} \\
    \iff & x \in A \cup B' & \text{by \AXM{3.4}} \\
\end{align*}
So similarly \(A \cup B = A \cup B'\).
\end{proof}

\begin{lemma} \label{lem 3.1.13}
If \(a\) and \(b\) are objects, then \( \{a, b\} = \{ a \} \cup \{ b \} \). If \(A, B, C\) are sets, then the \emph{union operation is commutative} (i.e., \(A \cup B = B \cup A\)) and \emph{associative} (i.e., \((A \cup B) \cup C = A \cup (B \cup C)\)). Also, we have \(A \cup A = A \cup \emptyset = \emptyset \cup A = A\).
\end{lemma}

\begin{proof}
\( \{a, b\} = \{ a \} \cup \{ b \} \): for any object \(x\),
\begin{align*}
         & x \in \{a, b\} \\
    \iff & x = a \lor x = b & \text{by \AXM{3.3}, pair-part} \\
    \iff & x \in \{a\} \lor x \in \{b\} & \text{by \AXM{3.3}, singleton-part} \\
    \iff & x \in \{a\} \cup \{b\} & \text{by \AXM{3.4}} \\
\end{align*}
So \(\forall x, x \in \{a, b\} \iff x \in \{a\} \cup \{b\} \), so \(\{a, b\} = \{a\} \cup \{b\} \) by \DEF{3.1.4}.

\(A \cup B = B \cup A\): for any object \(x\),
\begin{align*}
         & x \in A \cup B \\
    \iff & x \in A \lor x \in B & \text{by \AXM{3.4}} \\
    \iff & x \in B \lor x \in A & \text{by logic} \\
    \iff & x \in B \cup A & \text{by \AXM{3.4}}
\end{align*}

\((A \cup B) \cup C = A \cup (B \cup C)\): for any object \(x\),
\begin{align*}
         & x \in (A \cup B) \cup C \\
         \iff & x \in (A \cup B) \lor x \in C & \text{by \AXM{3.4}} \\
         \iff & (x \in A \lor x \in B) \lor x \in C & \text{by \AXM{3.4}} \\
         \iff & x \in A \lor (x \in B \lor x \in C) & \text{by logic} \\
         \iff & x \in A \lor (x \in B \cup C) & \text{by \AXM{3.4}} \\
         \iff & x \in A \cup (x \in B \cup C) & \text{by \AXM{3.4}} \\
\end{align*}
So first statement if and only if last statement, so \((A \cup B) \cup C = A \cup (B \cup C)\).

\(A \cup A = A \cup \emptyset = \emptyset \cup A = A\): for any object \(x\),
\begin{align*}
         & x \in A \cup A \\
    \iff & x \in A \lor x \in A & \text{by logic} \\
    \iff & x \in A & \text{by simplifying logic} \\
    \iff & \MAROON{A \cup A = A} & \text{by \DEF{3.1.4}} \\
    \iff & x \in A \lor x \in \emptyset & \text{by logic, something \(\lor\) something false = something} \\
    \iff & x \in A \cup \emptyset & \text{by \AXM{3.4}} \\
    \iff & \MAROON{A \cup A = A \cup \emptyset} & \text{by \DEF{3.1.4}} \\
    \iff & \MAROON{A \cup A = \emptyset \cup A} & \text{already proved commutative law}
\end{align*}
\end{proof}

\begin{remark}
幹話。
\end{remark}

\begin{note}
We are not yet in a position to define sets consisting of \(n\) objects for any given natural numbers \(n\). 老實說我不是很理解為什麼,文中敘述是說我們還沒有定義「做\ \(n\) 次操作」是什麼意思"(require iterating the above construction “\(n\) times”, but the concept of \(n\)-fold iteration has not yet been rigorously defined). 類似任意有限個元素的情況,我們目前也無法給出有無限個元素的集合的定義。這需要其他的公理。現在我們先定義什麼是「子集合」。
\end{note}

\begin{definition}[Subsets] \label{def 3.1.15}
Let \(A, B\) be sets. We say that \(A\) is a subset of \(B\), denoted \(A \subseteq B\), if and only if every element of \(A\) is also an element of \(B\), i.e. For any object \(x\), \(x \in A \implies x \in B\). We say that \(A\) is a proper subset of \(B\), denoted \(A \subsetneq B\), if \(A \subseteq B\) and \(A \neq B\).
\end{definition}

\begin{remark} \label{remark 3.1.16}
Because these definition\textbf{s}(both \(\subseteq\) and \(\subsetneq\)) involve only the notions of (set) equality and the “is an element of” relation, both of which already obey the axiom of substitution \AXM{a.7.4}, the notion of subset also automatically obeys the axiom of substitution. Thus for instance if \(A \subseteq B\) and \(A = A'\), then \(A' \subseteq B\).
\end{remark}

\begin{example} \label{example 3.1.17}
前面幹話。Given any set \(A\), we always have \(A \subseteq A\) (why?) and \(\emptyset \subseteq A\) (why?).
\end{example}

\begin{proof}
Let \(x\) be arbitrarily chosen object. Then if \(x \in A\) then \(x \in A\) is trivially true. By \DEF{3.1.15}, \(A \subseteq A\).

Let \(x\) be arbitrarily chosen object. Then if \(x \in \emptyset\) then \(x \in A\) is vacuously true. By \DEF{3.1.15}, \(\emptyset \subseteq A\).
\end{proof}

\begin{note}
下面的\ Proposition 要參考\ \DEF{8.5.1},旨在說明\ "set-inclusion" 這個\ relation 是\ partially ordered。也就是要證明它是\ reflexive(by \EXAMPLE{3.1.17}), anti-symmetric, transitive。
\end{note}

\begin{proposition} [Sets are partially ordered by set inclusion] \label{prop 3.1.18}
Let \(A, B, C\) be sets. If \(A \subseteq B\) and \(B \subsetneq C\) then \(A \subseteq C\). If \(A \subseteq B\) and \(B \subseteq A\), then \(A = B\). Finally, if \(A \subsetneq B\) and \(B \subsetneq C\) then \(A \subsetneq C\).
\end{proposition}

\begin{proof}
Transitive: Let \(A, B, C\) be sets such that \(A \subseteq B\) \MAROON{(1)} and \(B \subseteq C\) \MAROON{(2)}. Suppose object \(x \in A\), wanted: \(x \in C\). Then by \MAROON{(1)}, \(x \in B\), which with \MAROON{(2)} implies \(x \in C\).

Anti-Symmetric: Let \(A, B\) be sets such that \(A \subseteq B \land B \subseteq A\). Wanted: \(A = B\). Then we have
\begin{align*}
     & A \subseteq B \land B \subseteq A \\
\iff & (\forall\ x : x \in A \implies x \in B) \land (\forall\ x : x \in B \implies x \in A) & \text{by \DEF{3.1.15}} \\
\iff & (\forall\ x : x \in A \iff x \in B) & \text{by trivially simplifying logic} \\
\iff & A = B. & \text{by \DEF{3.1.4}}
\end{align*}
\end{proof}

Transitive for proper-inclusion: Let \(A, B, C\) be sets such that \(A \subsetneq B\) \MAROON{(1)} and \(B \subsetneq C\) \MAROON{(2)}. Wanted: \(A \subsetneq C\), that is,
\[
    A \subseteq C \land A \neq C
\] The former is trivially true since \(A \subsetneq B\) by definition implies \(A \subseteq B\) and \(B \subsetneq C\) by definition implies \(B \subseteq C\), and by transitivity of set-inclusion \(A \subseteq B\) and \(B \subseteq C\) implies \(A \subseteq C\). For the latter, suppose for the sake of contradiction that \(A = C\). Then by \MAROON{(1)} and Axiom of Substitution \AXM{a.7.4}, we have \(C \subsetneq B\), but that implies \(C \subseteq B\) and we have known \(B \subseteq C\), together implies \(B = C\) by anti-symmetry of set-inclusion. But this contradicts \(B \subsetneq C\) because that implies \(B \neq C\).

\begin{note}
根據本書的脈絡,我們直到\ \PROP{3.1.18} 才能用\ \(A \subseteq B \land B \subseteq A\) 的方法來證明\ \(A = B\)。
\end{note}

\begin{remark} \label{remark 3.1.19}
There is a relationship between subsets and unions: see for instance \EXEC{3.1.7}. 這就頭腦體操。
\end{remark}
\section{Russel's Paradox}\label{sec 3.2}

\begin{axiom} [Universal specification] \label{axm 3.8} (\RED{Dangerous!})
Suppose for every object \(x\) we have a property \(P(x)\) pertaining to \(x\) (so that for every \(x\), \(P(x)\) is either a true statement or a false statement). Then there exists a set \( \{x : P(x) \text{\ is true} \} \) such that for every object \(y\),
\[
    y \in \{x : P(x) \text{\ is true} \} \iff P(y) \text{\ is true}.
\]
This axiom is also called \emph{axiom of comprehension}. It asserts that \emph{every property corresponds to a set}. This axiom also implies most of the axioms in the previous section (\EXEC{3.2.1}, mind=blown).
\end{axiom}

\begin{note}
這條「公理」跟\ \AXM{3.5} \AXM{3.6} 的差別就是後兩個都是做用在一個\textbf{已知是集合}的東西上的。
\end{note}

\begin{note}
接下來的東西都是在聊羅素悖論,去查各種影片可能會比較有趣。
\end{note}

\begin{note}
We shall simply postulate an axiom which ensures that absurdities such as Russell’s paradox do not occur.
\end{note}

\begin{axiom} \label{axm 3.9}
If \(A\) is a non-empty set, then \textbf{there is} at least one element \(x\) of \(A\) which is either \textit{not a set}, \textit{or} is \textit{disjoint} from \(A\).
\end{axiom}

\begin{note}
BTW, the description of \AXM{3.9} uses the term \emph{disjoint}, which depends on the definition of intersection(\DEF{3.1.23}), which depends on the definition of \AXM{3.5}, so I assume \AXM{3.5} is also asserted when \AXM{3.9} is asserted.
\end{note}

\begin{note}
這個公理可能可以參考一下其他地方(e.g. \href{https://www.wikiwand.com/en/Axiom_of_regularity}{wiki})怎麼描述的,因為這本書是假設有些\ object 不是\ set,但是\ ZFC(或者\ pure set theory?) 實際上所有東西都是\ set。不過看起來其實就是把\ ``\(x\) is either not a set'' 拔掉而已。
\end{note}

\begin{note}
One particular consequence of this axiom is that \textbf{sets are no longer allowed to contain themselves}. (\EXEC{3.2.2}) So if something contains itself, then it is not a set.
\end{note}

\begin{note}
This axiom(\AXM{3.9}) is never needed for the purposes of doing analysis.
\end{note}

\exercisesection

\begin{exercise} \label{exercise 3.2.1}
Show that the universal specification axiom, \AXM{3.8}, if assumed to be true, would imply \AXM{3.2} (empty set), \AXM{3.3} (singleton and pair), \AXM{3.4} (union), \AXM{3.5} (specification), and \AXM{3.6} (replacement). (If we assume that all natural numbers are objects, we also obtain \AXM{3.7}.) Thus, this axiom, if permitted, would simplify the foundations of set theory tremendously (and can be viewed as one basis for an intuitive model of set theory known as ``\emph{naive set theory}''). Unfortunately, as we have seen, \AXM{3.8} is ``too good to be true''!
\end{exercise}

\begin{proof}
First the exercise does not say \AXM{3.8} implies \AXM{3.1}, so it seems that \AXM{3.1} is also needed.

For \AXM{3.2}, we just let \(P(x) := \text{false}\) for any object \(x\). Then by \AXM{3.8} there exists a set \( \emptyset := \{ x : P(x) \} \). Then we must have \(\forall x : x \notin \emptyset\), otherwise if \(x \in \emptyset\) then by definition of \(\emptyset\), \(P(x)\) is true, which contradicts that \(P(x)\) is false.

For \AXM{3.3}, given particular but arbitrary objects \(a, b\), we just let \(P(x) := x = a\) and \(Q(x) := x = a \lor x b\). Then by \AXM{3.8} there exist sets  \(A := \{ x : P(x) \} \) and \(B := \{ x : Q(x) \} \), i.e. \(A := \{ x : x = a \} \) and \(B := \{ x : x = a \lor x \ b \} \). Hence \AXM{3.3} (existence of singleton and pair set) is satisfied.

For \AXM{3.4} Let \(A, B\) be particular but arbitrary sets. And Let \(P(x) := x \in A \lor x \in B\). Then by \AXM{3.8}, there exists a set \(C := \{x : P(x)\} \), that is, \(C := \{x : x \in A \lor x \in B\}\). Thus the union of \(A, B\) exists. Hence \AXM{3.4} is satisfied.

For \AXM{3.5}. Let \(A\) be a particular but arbitrary set and \(P(x)\) be a particular but arbitrary statement that is either true or false for any object \(x \in A\). Then let \(Q(x) := x \in A \land P(x) \text{ is true}\). By \AXM{3.8}, there exists a set \(B := \{ x : Q(x) \} \), that is, \(B := \{x : x \in A \land P(x) \text{ is true}\}\). Thus the specification set \(B\) of \(A\) with statement \(P\) exists. Hence \AXM{3.5} is satisfied.

For \AXM{3.6}. Let \(A\) be a particular but arbitrary set and \(P(x, y)\) satisfy the hypothesis in \AXM{3.6}. By \AXM{3.8}, there exists a set \(B := \{y : P(x, y) \text{ is true}\} \land x \in A\), that is, \(B := \{y : P(x, y) \text{ is true for some \(x \in A\)} \} \). So B is the replacement set of \(A\). Hence \AXM{3.6} is satisfied.

Finally, for \AXM{3.7}, suppose all natural numbers are objects. Then let \(P(x)\) := ``\(x\) is a natural number'' :). Then there exists a set \( \SET{N} := \{ x : P(x) \} \).

\end{proof}

\begin{note}
The proof of implication of \AXM{3.7} is not rigorous(or escape the detail of the statement \(P\), escape what need to be satisfied to be a natural number). We just need to know they are object and hence can be elements of a set.
\end{note}

\begin{exercise} \label{exercise 3.2.2}
Use the axiom of regularity, \AXM{3.9} (and the singleton set axiom, \AXM{3.3}) to show that if \(A\) is a set, then \(A \notin A\). Furthermore, show that if \(A\) and \(B\) are two sets, then either \(A \notin B\) or \(B \notin A\) (or both).
\end{exercise}

\begin{proof}
Again, we use \AXM{3.1}. So if \(A\) is a set, then \(A\) is an object, and by \AXM{3.3}, \(\{A\}\) is a singleton set.

Suppose for the sake of contradiction that there exists an object \(A\) such that \(A \in A\) \MAROON{(1)}. Then because \(A \in \{A\} \)  \MAROON{(2)}, by \MAROON{(1) (2)} we know \(\{A\} \cap A = \{A\} \neq \emptyset\), i.e. not disjoint. But since the singleton set \( \{A\} \) has only one object \(A\), it implies there is no object \(x\) of \(\{A\}\) such that \(x\) and \( \{A\}\) are disjoint, which contradicts \AXM{3.9}. So the supposition is false, so for any object \(A\), \(A \notin A\).

Also, suppose for the sake of contradiction that there exist sets \(A, B\), such that
\begin{center}
    \(A \in B\) \MAROON{(1)} and \(B \in A\) \MAROON{(2)}.
\end{center}
Now consider the set \(\{A, B\}\) (which exists by pair set of \AXM{3.3}), it has two ``elements'' \BLUE{\(A\)} and \GREEN{\(B\)}. For element \BLUE{\(A\)}, since \(B \in \BLUE{A}\) by \MAROON{(2)} and \(B \in \{A, B\}\), \(B \in \BLUE{A} \cap \{A, B\}\), so \BLUE{\(A\)} and \(\{A, B\}\) are not disjoint. For element \GREEN{\(B\)}, since \(A \in \GREEN{B}\) by \MAROON{(1)} and \(A \in \{A, B\}\), \(A \in \GREEN{B} \cap \{A, B\}\), so \GREEN{\(B\)} and \(\{A, B\}\) are not disjoint. So for each element \(x \in \{A, B\}\), \(x\) and \(\{A, B\}\) is not disjoint, which contradicts with \AXM{3.9}.
\end{proof}

\begin{exercise} \label{exercise 3.2.3}
Show (assuming the other axioms of set theory) that the universal specification axiom, \AXM{3.8}, is equivalent to an axiom postulating the existence of a ``universal set'' \(\Omega\) consisting of all objects (i.e., for all objects \(x\), we have \(x \in \Omega\)). In other words, if \AXM{3.8} is true, then a universal set exists, and conversely, if a universal set exists, then \AXM{3.8} is true. (This may explain why \AXM{3.8} is called the axiom of universal specification.) Note that if a universal set \(\Omega\) existed, then we would have \(\Omega \in \Omega\) by \AXM{3.1} (\(\Omega\) is a set by \AXM{3.8} or universal specification, therefore by \AXM{3.1} it is also an object, and any object is \(\in \Omega\), so \(\Omega \in \Omega\)), contradicting \EXEC{3.2.2}. Thus the axiom of foundation specifically rules out the axiom of universal specification.
\end{exercise}

\begin{proof}
Suppose \AXM{3.8} is true. Then let \(P(x) := true\) for any object \(x\). Then by \AXM{3.8}, there exists at set \(\Omega := \{x : P(x) \text{ is true} \}\), and since \(P(x)\) is true for any object \(x\), \(x \in \Omega\).

Suppose the universal set \(\Omega\) exists. Then given an arbitrary property \(P(x)\) satisfiying the hypothesis of \AXM{3.8}, and by \AXM{3.5} (specification), since \(\Omega\) is a set, there exists the set \(A := \{ x \in \Omega : P(x) \text{ is true} \}\). Now we have to show that for every object \(y\),
\[
    y \in \{x \in \Omega : P(x) \text{ is true} \} \iff P(y) \text{ is true \BLUE{(1)}}
\]
to show that \(y \in \{x \in \Omega : P(x) \text{ is true} \}\) is in fact \(y \in \{x : P(x) \text{ is true} \}\) whose existence is guaranteed by \AXM{3.8}.
So let \(y\) be an object.
\begin{itemize}
    \item Suppose \(y \in \{x \in \Omega : P(x) \text{ is true} \). Then trivally \(P(y)\) is true.
    \item Suppose \(P(y)\) is true \MAROON{(1)}. Then since \(y\) is an object, \(y \in \Omega\) \MAROON{(2)}. By \MAROON{(1) (2)}, \(y \in \{x \in \Omega : P(x) \text{ is true} \} \).
\end{itemize}
So \BLUE{(1)} is proved.
\end{proof}

\section{Functions} \label{sec 3.3}

\begin{definition} [Functions] \label{def 3.3.1}
Let \(X, Y\) be sets, and let \(P(x, y)\) be a property pertaining to an object \(x \in X\) and an object \(y \in Y\), such that for every \(x \in X\), there is \textbf{exactly one} \(y \in Y\) for which \(P(x, y)\) is true (this is sometimes known as the \emph{vertical line test}). Then we define \emph{the function} \(f : X \longrightarrow Y\) \emph{defined by \(P\)} on the \textbf{domain} \(X\) and \textbf{codomain} \(Y\) to be the object which, given any input \(x \in X\), assigns an output \(f(x) \in Y\), defined to be the unique object \(f(x) \in Y\) for which \(P(x, f(x))\) is true. Thus, for any \(x \in X\) and \(y \in Y\),
\[
    y = f(x) \iff P(x, y) \text{ is true}.
\]
\end{definition}

\begin{note}
Functions are also referred to as \emph{maps} or \emph{transformations}, depending on the context. They are also sometimes called \emph{morphisms}, although to be more precise, a morphism refers to a more general class of object, which may or may not correspond to actual functions, depending on the context.

Implicit in the \DEF{3.3.1} is the assumption that whenever one is given two sets \(X, Y\) and a property \(P\) obeying the vertical line test, one can form a function object. Strictly speaking, this assumption of the existence of the function as a mathematical object \emph{should be stated as an explicit axiom}; however we will not do so here, as it turns out to be redundant. (More precisely, in view of \EXEC{3.5.10} below, it is always possible to encode a function \(f\) as an \emph{ordered triple} \((X, Y, \{ (x,f(x)): x \in X \})\) consisting of the domain, codomain, and graph of the function, which gives a way to build functions as objects using the operations provided by the ``preceding'' axioms(axioms declared up to \EXEC{3.5.10}.)
\end{note}

\begin{note}
\DEF{3.3.1} 看來可以跟\ \AXM{3.6} 比較一下?
\end{note}

\begin{example} \label{example 3.3.2}
Let \(X = \SET{N}\), \(Y = \SET{N}\), and let \(P(x, y)\) be the property that \(y = x\INC\). Then for each \(x \in \SET{N}\) (or \(X\)) there is \textbf{exactly one} \(y \in \SET{N}\) (or \(Y\)) for which \(P(x, y)\) is true, namely \(y = x\INC\). Thus we can define a function \(f : X \longrightarrow Y\) associated to this property, so that \(f(x)= x\INC\) for all \(x \in X\); this is the \textbf{increment function} on \(\SET{N}\), which takes a natural number as input and returns its increment as output. Thus for instance \(f(4) = 5\), \(f(2n + 3)\) = \(2n + 4\) (assuming \(n\) is a natural number) and so forth. One might also \emph{hope} to define a decrement function \(g : N \longrightarrow N\) associated to the property \(Q(x, y)\) defined by \(y\INC = x\), i.e., \(g(x)\) would be the number whose increment is \(x\). Unfortunately this does \emph{not} define a function, because when \(x = 0\) there is \emph{no} natural number \(y\) whose increment is equal to \(x\) (by \AXM{2.3}). On the other hand, we can legitimately define a decrement function \(h : N \setminus \{0\} \longrightarrow N\) associated to the property \(Q\), because when \(x \in N \setminus \{0\}\) there is indeed exactly one natural number \(y\) such that \(y\INC = x\), thanks to \LEM{2.2.10}. Thus for instance \(h(4) = 3\) and \(h(2n +3) = 2n + 2\) (assuming \(n\) is a natural number), but \(h(0)\) is \emph{undefined} since \(0\) is not in the domain \(N \setminus \{0\}\).
\end{example}
