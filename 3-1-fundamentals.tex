\section{Fundamentals}\label{sec 3.1}

For pedagogical reasons, we will use a \emph{somewhat overcomplete list of axioms} for set theory, in the sense that some of the axioms can be used to deduce others, but there is no real harm in doing this.

\begin{note}
就好像你可以說正整數除了滿足皮亞諾公理也自動滿足\ \(a + b = b + a\),但後者用皮亞諾就可證明。
\end{note}

\begin{definition}[\emph{Informal}] \label{def 3.1.1}
We define a set \(A\) to be any \emph{unordered collection} of objects, e.g., \( \{3, 8, 5, 2\} \) is a set. If \(x\) is an object, we say that \(x\) \emph{is an element of} \(A\) or \(x \in A\) if \(x\) lies in the collection; otherwise we say that \(x \notin A\). For instance, \(3 \in \{1, 2, 3, 4, 5\} \) but \(7 \notin \{1, 2, 3, 4, 5\} \).
\end{definition}

\begin{note}
\DEF{3.1.1} 有很多問題沒有回答,例如「哪些\ collection」 才能被稱作集合,兩個集合怎麼判斷是否相等,怎麼對集合作操作(聯集、交集等等),集合可以做什麼,以及集合的元素(element)可以做什麼。
\end{note}

\begin{axiom}[Sets are objects]\label{axm 3.1}
If \(A\) is a set, then \(A\) is \emph{also an object}. In particular, given two sets \(A\) and \(B\), it is meaningful to ask whether \(A\) is also an element of \(B\).
\end{axiom}

\begin{example}[Informal]\label{example 3.1.2}
這個例子舉例\ \( \{3, \{3, 4\}, 4\} \) 裡面有一個元素也是集合,但敘述方式不嚴謹,要去看\ \SEC{3.6}
\end{example}

\begin{remark}\label{remark 3.1.3}
這裡在探討是否需要把所有\ object 都當成\  set。在邏輯的角度,這樣推論過程比較簡單,因為只需要處理一種物件,就是\ set,但是從概念上來看,將某些\ object 視為「不是\ set」則會比較單純,比方說給定一個自然數\ \(2\),將他視為一個集合(在\ Analysis 的範疇)沒什麼進一步的應用。是否將所有\ object 都當成集合,\ more or less 是等價的,所以,we shall take an agnostic position as to whether all objects are sets or not.
\end{remark}

\begin{note}
若已知\ \(x\) 是一個\ object 且\ \(A\) 是一個\ set,則要馬\ \(x \in A\) 為真,要馬\ \(x \notin A\) 為真。而若\ \(A\) 不是\ set,則我們視\ \(x \in A\) 為\ undefined。
\end{note}

\begin{definition}[Equality of sets] \label{def 3.1.4} 
Two sets \(A\) and \(B\) are equal, \(A = B\), if and only if every element of \(A\) is an element of \(B\) and vice versa. To put it another way, \(A = B\) if and only if every element \(x\) of \(A\) belongs also to \(B\), and every element \(y\) of \(B\) belongs also to \(A\). Or equivalently,
\[
  \forall\ x : x \in A \iff x \in B
\]
\end{definition}

\begin{example}
嘴砲。
\end{example}

\begin{note}
One can easily verify that this notion of equality is reflexive, symmetric, and transitive (See \EXEC{3.1.1}).
\end{note}

\begin{additional corollary}\label{ac 3.1.1}
The definition of equality in \DEF{3.1.4} is reflexive, symmetric and transitive.
\end{additional corollary}

\begin{proof}

Reflexive: Suppose \(A\) is a set. Then \(\GREEN{A} = \BLUE{A}\) because if \(x \in \GREEN{A}\), then \(x \in \BLUE{A}\), and if \(y \in \BLUE{A}\), then \(y \in \GREEN{A}\).

Symmetric: Suppose \(A, B\) are sets and \(A = B\), then by \DEF{3.1.4},
\[
  \forall\ x : x \in A \iff x \in B
\]
But this statement is just equivalent to
\[
  \forall\ x : x \in B \iff x \in A
\]
and this by \DEF{3.1.4} implies \(B = A\).

Transitive: Suppose \(A, B, C\) are sets and \(A = B\) and \(B = C\). Then by \DEF{3.1.4}
\[
  \forall\ x : x \in A \iff x \in B
\]
\[
  \forall\ x : x \in B \iff x \in C
\]
And this implies
\[
  \forall\ x : x \in A \iff x \in B \iff x \in C
\]
And by logic this implies
\[
  \forall\ x : x \in A \iff x \in C
\]
By \DEF{3.1.4}, \(A = C\).
\end{proof}

\begin{note}
"is an element of" relation \(\in\) 符合\ Axiom of Substitution \AXM{a.7.4},因為
\begin{center}
    if \(x \in A\) and \(A = B\), then \(x \in B\), by \DEF{3.1.4}.
\end{center}
這也代表那些完全以\ \(\in\) 定義的新的集合操作會自動符合\ Axiom of Substitution \AXM{a.7.4}。真的要說的話這一節剩下的所有\ operations 都是用\ \(\in\) 來定義的。
\end{note}
\begin{note}
接著\ \RMK{3.1.3},我們繼續來探討什麼\ object 是\ set,什麼不是。有點類似我們定哪些東西為自然數,哪些不是(\AXM{2.1},\(0\) 是自然數,然後用 \AXM{2.2} 來擴增/建構其他的自然數)。這邊在集合論就是先假設存在一個集合,叫「空集合」,然後再定義一些在集合上的操作來建構其他的集合。
\end{note}

\begin{axiom}[Empty set] \label{axm 3.2}
There exists a set \(\emptyset\), known as \emph{the} empty set, which \emph{contains no elements}, i.e., for every object \(x\) we have \(x \notin \emptyset\).
\end{axiom}