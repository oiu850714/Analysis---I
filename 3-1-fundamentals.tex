\section{Fundamentals}\label{sec 3.1}

For pedagogical reasons, we will use a \emph{somewhat overcomplete list of axioms} for set theory, in the sense that some of the axioms can be used to deduce others, but there is \emph{no real harm} in doing this.

\begin{note}
就好像你可以說正整數除了滿足皮亞諾公理也自動滿足\ \(a + b = b + a\),但後者用皮亞諾就可證明。
\end{note}

\begin{definition}[\emph{Informal}] \label{def 3.1.1}
We define a set \(A\) to be any \emph{unordered collection} of objects, e.g., \( \{3, 8, 5, 2\} \) is a set. If \(x\) is an object, we say that \(x\) \emph{is an element of} \(A\) or \(x \in A\) if \(x\) lies in the collection; otherwise we say that \(x \notin A\). For instance, \(3 \in \{1, 2, 3, 4, 5\} \) but \(7 \notin \{1, 2, 3, 4, 5\} \).
\end{definition}

\begin{note}
\DEF{3.1.1} 有很多問題沒有回答,例如「什麼樣的\ collection」 才能被稱作集合,兩個集合怎麼判斷是否相等,怎麼對集合作操作(聯集、交集等等),集合可以做什麼,以及集合的元素(element)可以做什麼。
\end{note}

\begin{axiom}[Sets are objects]\label{axm 3.1}
If \(A\) is a set, then \(A\) is \emph{also an object}. In particular, given two sets \(A\) and \(B\), it is meaningful to ask whether \(A\) is also an element of \(B\).
\end{axiom}

\begin{example}[Informal]\label{example 3.1.2}
這個例子舉例\ \( \{3, \{3, 4\}, 4\} \) 裡面有一個元素也是集合,但敘述方式不嚴謹,要去看\ \SEC{3.6}
\end{example}

\begin{remark}\label{remark 3.1.3}
這裡在探討是否需要把所有\ object 都當成\  set。在邏輯的角度,這樣推論過程比較簡單,因為需要的東西的類型就只有一種,就是\ set,但是從概念上來看,將某些\ object 視為「不是\ set」則會比較單純,比方說給定一個自然數\ \(2\),將他視為一個集合(在\ Analysis 的範疇)沒什麼進一步的應用。是否將所有\ object 都當成集合,\ more or less 是等價的,所以,we shall take an agnostic position as to whether all objects are sets or not.
\end{remark}

\begin{note}
若已知\ \(x\) 是一個\ object 且\ \(A\) 是一個\ set,則要馬\ \(x \in A\) 為真,要馬\ \(x \notin A\) 為真。而若\ \(A\) 不是\ set,則我們視\ \(x \in A\) 為\ undefined。
\end{note}

\begin{definition}[Equality of sets] \label{def 3.1.4} 
Two sets \(A\) and \(B\) are equal, \(A = B\), if and only if every element of \(A\) is an element of \(B\) and vice versa. To put it another way, \(A = B\) if and only if every element \(x\) of \(A\) belongs also to \(B\), and every element \(y\) of \(B\) belongs also to \(A\). Or equivalently,
\[
  \forall\ x : x \in A \iff x \in B
\]
\end{definition}

\begin{example}
嘴砲。
\end{example}

\begin{note}
One can easily verify that this notion of equality is reflexive, symmetric, and transitive (See \EXEC{3.1.1}).
\end{note}

\begin{additional corollary}\label{ac 3.1.1}
The definition of equality in \DEF{3.1.4} is reflexive, symmetric and transitive.
\end{additional corollary}

\begin{proof}

Reflexive: Suppose \(A\) is a set. Then given any object \(x\), if \(x \in \GREEN{A}\), then \(x \in \BLUE{A}\), and given any object \(y\), if \(y \in \BLUE{A}\), then \(y \in \GREEN{A}\). So by \DEF{3.1.4}, \(\GREEN{A} = \BLUE{A}\).

Symmetric: Suppose \(A, B\) are sets and \(A = B\), then by \DEF{3.1.4},
\[
  \forall\ x : x \in A \iff x \in B
\]
But this statement is just equivalent to
\[
  \forall\ x : x \in B \iff x \in A
\]
and this by \DEF{3.1.4} implies \(B = A\).

Transitive: Suppose \(A, B, C\) are sets and \(A = B\) and \(B = C\). Then by \DEF{3.1.4}
\[
  \forall\ x : x \in A \iff x \in B
\]
\[
  \forall\ x : x \in B \iff x \in C
\]
And this implies
\[
  \forall\ x : x \in A \iff x \in B \iff x \in C
\]
And by logic this implies
\[
  \forall\ x : x \in A \iff x \in C
\]
By \DEF{3.1.4}, \(A = C\).
\end{proof}

\begin{note}
``is an element of'' relation \(\in\) 符合\ Axiom of Substitution \AXM{a.7.4},因為
\begin{center}
    if \(x \in A\) and \(A = B\), then \(x \in B\), by \DEF{3.1.4}.
\end{center}
這也代表那些完全以\ \(\in\) 定義的新的集合操作會自動符合\ Axiom of Substitution \AXM{a.7.4}。例如這一節剩下的所有\ operations 都是用\ \(\in\) 來定義的。
\end{note}
\begin{note}
接著\ \RMK{3.1.3},我們繼續來探討什麼\ object 是\ set,什麼不是。有點類似我們定哪些東西為自然數,哪些不是(\AXM{2.1},\(0\) 是自然數,然後用 \AXM{2.2} 來擴增/建構其他的自然數)。這邊在集合論就是先假設存在一個集合,叫「空集合」,然後再定義一些在集合上的操作來建構其他的集合。
\end{note}

\begin{axiom}[Empty set] \label{axm 3.2}
There exists a set \(\emptyset\), known as \emph{the} empty set, which \emph{contains no elements}, i.e., for every object \(x\) we have \(x \notin \emptyset\).
\end{axiom}

\begin{note}
\emph{The} empty set is also denoted \(\{\}\). Note that there can only be \textbf{one} empty set.
\end{note}

\begin{additional corollary} [The empty set is unique] \label{ac 3.1.2}
If there were two sets \(\emptyset\) and \(\emptyset'\) which were both empty, then by \DEF{3.1.4} they would be equal to each other.
\end{additional corollary}
\begin{proof}
Suppose \(\emptyset'\) is also empty. Then the statement
\[
  \forall\ x : x \in \emptyset' \implies x \in \emptyset
\]
is vacuously true because by \AXM{3.2} empty set contains no elements. And again by \AXM{3.2}, the statement
\[
  \forall\ x : x \in \emptyset \implies x \in \emptyset'
\]
is also vacuous. These imply
\[
  \forall\ x : (x \in \emptyset' \implies x \in \emptyset) \land (x \in \emptyset \implies x \in \emptyset')
\]
that is,
\[
  \forall\ x : x \in \emptyset' \iff x \in \emptyset
\]
by \DEF{3.1.4}, \(\emptyset' = \emptyset\)
\end{proof}

\begin{note}
If a set is not equal to the empty set, we call it \emph{non-empty}.
\end{note}

\begin{lemma}[Single choice]\label{lem 3.1.6}
Let \(A\) be a \emph{non-empty} set. Then there exists an object \(x\) such that \(x \in A\).
\end{lemma}
\begin{proof}
Suppose for the sake of contradiction that \(A\) be a non-empty set and for all object \(x\), \(x \notin A\). And by \AXM{3.2}, \(x \notin \emptyset\). Then similarly as \AC{3.1.2}, we can derive \(A = \emptyset\), which contradicts that \(A\) is \emph{non-empty}.
\end{proof}

\begin{remark}\label{remark 3.1.7}
\LEM{3.1.6} asserts that given any non-empty set \(A\), we are allowed to \emph{``choose''} an element \(x\) of \(A\) which demonstrates this non-emptyness. Later on (in \LEM{3.5.12}) we will show that given any \textbf{finite} number of non-empty sets, say \(A_1, \dots, A_n\), it is possible to choose one element \(x_1, \dots, x_n\) from each set \(A_1, \dots, A_n\); this is known as ``finite choice''. However, in order to choose elements from an \textbf{infinite} number of sets, we need an additional axiom, the \emph{axiom of choice} (\AXM{8.1}).
\end{remark}

\begin{note}
\RMK{3.1.7} 在講\ ``finite choice'' 的部分,看起來是說要被選的集合是有限個,但是沒有規定個別集合的元素數量要有限個,目前還不確定這意味著什麼。
\end{note}

\begin{remark} \label{remark 3.1.8}
Note that the empty set is not the same thing as the natural number \(0\). One is a set; the other is a number. However, it is true that the \emph{cardinality} of the empty set is \(0\); see \SEC{3.6}.
\end{remark}

We now present further axioms to enrich the class of sets available.

\begin{axiom}[Singleton sets and pair sets]\label{axm 3.3}
If \(a\) is an object, then there exists a set \( \{a\} \) whose \emph{only} element is \(a\), i.e., for every object \(y\), we have \(y \in \{a\}\) if and only if \(y = a\); we refer to \( \{a\} \) as the \emph{singleton set} whose element is \(a\). Furthermore, if \(a\) and \(b\) are objects, then there exists a set \( \{a, b\} \) whose only elements are \(a\) and \(b\); i.e., for every object \(y\), we have \( y \in \{a, b\} \) if and only if \(y = a\) or \(y = b\); we refer to this set as the \emph{pair set} formed by \(a\) and \(b\).
\end{axiom}

\begin{note}
\href{https://www.wikiwand.com/en/Axiom_of_pairing#/Consequences}{參考}: 這個公理實際說的是,給定兩個集合(這邊暫時當作所有物件都是集合)\ \(x\) 和\ \(y\),我們可以找到一個集合\ \(A\) ,它的成員就是\ \(x\) 和\ \(y\)。
\end{note}

\begin{note}
前方高能注意: \RMK{3.1.9} 在解釋\ \AXM{3.3} 裡面的\ singleton、\ pair,還有\ \AXM{3.4} 這三者,若假設``其中一部分''是公理,則剩下的可以從那個部分直接推得,不用當作是公理,i.e. 剩下的只須為定理,不用是公理。這種\ redundant\ 在本節開頭有講過,便於推導,but does no real harm。BTW 這個\ remark 裡面有一堆\ (why?) 全部都要自己推導。
\end{note}

\begin{remark} \label{remark 3.1.9}

Just as there is only one empty set, there is only one singleton set for each object \(a\), thanks to \DEF{3.1.4} (why? \MAROON{(1)}).

Similarly, given any two objects \(a\) and \(b\), there is only one pair set formed by \(a\) and \(b\).

Also, \DEF{3.1.4} also ensures that \( \{a, b\} = \{b, a\} \) (why? \MAROON{(2)}) and \( \{a, a\} = \{a\} \) (why? \MAROON{(3)}). Thus the \textbf{singleton} set axiom is in fact redundant, being \textbf{a consequence of} the \textbf{pair} set axiom.

\emph{Conversely}, the \textbf{pair set} axiom will \textbf{follow from} the \textbf{singleton} set axiom \textbf{and} the \textbf{pairwise union axiom \AXM{3.4}} (see  \LEM{3.1.13}).

One may wonder why we don’t go further and create triplet axioms, quadruplet axioms, etc.; however there will be no need for this once we introduce the pairwise union axiom
below.
\end{remark}

\begin{proof}
\MAROON{(1)}: given any object \(a\), suppose there exist two sets \(A\) and \(A'\) which are singleton sets of \(a\). Then we have:
\begin{align*}
         & (\forall\ x : x \in A \iff x = a) \land (\forall\ x : x \in A' \iff x = a) & \text{by \AXM{3.3}} \\
    \iff & (\forall\ x : x \in A \iff x = a) \land (\forall\ x : x = a \iff x \in A') & \text{by logic} \\
    \iff & (\forall\ x : x \in A \iff x = a \iff x \in A')                            & \text{by logic} \\
    \iff & (\forall\ x : x \in A \iff x \in A')                                       & \text{by simplifying logic} \\
    \iff & A = A'                                                                     & \text{by \DEF{3.1.4}}
\end{align*}

\MAROON{(2)}: given any objects \(a, b\), then
\begin{align*}
    & (x \in \{a, b\} \iff (x = a \lor x = b)) & \text{by \AXM{3.3}} \\
    \iff & (x \in \{a, b\} \iff (x = a \lor x = b) \iff (x = b \lor x = a)) & \text{by logic} \\
    \iff & (x \in \{a, b\} \iff (x = b \lor x = a)) & \text{by simplifying logic} \\
    \iff & (x \in \{a, b\} \iff (x = b \lor x = a) \iff x \in \{b, a\}) & \text{by \AXM{3.3}} \\
    \iff & (x \in \{a, b\} \iff x \in \{b, a\}) & \text{by simplifying logic} \\
    \iff & \{a, b\} = \{b, a\} & \text{by \DEF{3.1.4}}
\end{align*}

\MAROON{(3)}: given any object \(a\), then
\begin{align*}
    & (x \in \{a, a\} \iff (x = a \lor x = a)) & \text{by \AXM{3.3}} \\
    \iff & (x \in \{a, a\} \iff (x = a)) & \text{by logic} \\
    \iff & (x \in \{a, a\} \iff (x = a) \iff x \in \{a\})& \text{by \AXM{3.3}} \\
    \iff & (x \in \{a, a\} \iff x \in \{a\})& \text{by simplifying logic} \\
    \iff & \{a, a\} = \{a\} & \text{by \DEF{3.1.4}}
\end{align*}
\end{proof}

\begin{example} \label{example 3.1.10}
Since \(\emptyset\) is a set (and hence an object), so is singleton set \(\{ \emptyset \}\), i.e., the set whose only element is \( \emptyset \), is a set (and it is not the same set as \( \emptyset \), \( \{ \emptyset \} \neq  \emptyset \) (why? See \EXEC{3.1.2}). Similarly, the singleton set \( \{ \{ \emptyset \} \} \) and the pair set \( \{ \emptyset, \{ \emptyset \} \} \) are also sets. These three sets are not equal to each other (\EXEC{3.1.2}).
\end{example}

\begin{note}
現在有的三個公理已經可以讓我們建構出一堆集合了,但是他們的元素都不超過兩個,雖然元素本身可能長得很複雜。
\end{note}

\begin{axiom} [Pairwise union] \label{axm 3.4}
Given any two sets \(A, B\), there exists a set \(A \cup B\), called the \emph{union} \(A \cup B\) of \(A\) and \(B\), whose elements consists of all the elements which belong to \(A\) or \(B\) or both. In other words, for any object \(x\),
\[
    x \in A \cup B \iff (x \in A \lor x \in B)
\].
\end{axiom}

\begin{note}
注意\ Union 的定義完全是由\ \( \in \) (還有\ or) 兜出來的,所以符合替換公理。
\end{note}

\begin{example}
很廢。\( \{1, 2\} \cup \{ 2, 3 \} = \{ 1, 2, 3 \} \)
\end{example}

\begin{remark} \label{remark 3.1.12}
If \(A, B\) are sets, \(A'\) is also a set which is equal to \(A\), then \(A \cup B\) is equal to \(A' \cup B\) (why? \MAROON{(1)} One needs to use \AXM{3.4} and \DEF{3.1.4}). Similarly if \(B'\) is a set which is equal to \(B\), then \(A \cup B\) is equal to \(A \cup B'\). Thus the operation of union \emph{obeys the axiom of substitution}, and is thus well-defined on sets.
\end{remark}

\begin{proof}
\MAROON{(1)}: Suppose \(A, B, A'\) are sets such that \(A = A'\). Then given any object \(x\),
\begin{align*}
         & x \in A \cup B \\
    \iff & x \in A \lor x \in B & \text{by \AXM{3.4}} \\
    \iff & x \in A' \lor x \in B & \text{since \(A = A'\) and \(=\) satisfies \AXM{a.7.4}} \\
    \iff & x \in A' \cup B & \text{by \AXM{3.4}} \\
\end{align*}
So \(\forall x, x \in A \cup B \iff x \in A' \cup B\), so by \DEF{3.1.4} \(A \cup B = A' \cup B\).

Now suppose \(B' = B\), Then given any object \(x\),
\begin{align*}
         & x \in A \cup B \\
    \iff & x \in A \lor x \in B & \text{by \AXM{3.4}} \\
    \iff & x \in A \lor x \in B' & \text{since \(B = B'\) and \(=\) satisfies \AXM{a.7.4}} \\
    \iff & x \in A \cup B' & \text{by \AXM{3.4}} \\
\end{align*}
So similarly \(A \cup B = A \cup B'\).
\end{proof}

\begin{lemma} \label{lem 3.1.13}
If \(a\) and \(b\) are objects, then \( \{a, b\} = \{ a \} \cup \{ b \} \). If \(A, B, C\) are sets, then the \emph{union operation is commutative} (i.e., \(A \cup B = B \cup A\)) and \emph{associative} (i.e., \((A \cup B) \cup C = A \cup (B \cup C)\)). Also, we have \(A \cup A = A \cup \emptyset = \emptyset \cup A = A\).
\end{lemma}

\begin{proof}
\( \{a, b\} = \{ a \} \cup \{ b \} \): for any object \(x\),
\begin{align*}
         & x \in \{a, b\} \\
    \iff & x = a \lor x = b & \text{by \AXM{3.3}, pair-part} \\
    \iff & x \in \{a\} \lor x \in \{b\} & \text{by \AXM{3.3}, singleton-part} \\
    \iff & x \in \{a\} \cup \{b\} & \text{by \AXM{3.4}} \\
\end{align*}
So \(\forall x, x \in \{a, b\} \iff x \in \{a\} \cup \{b\} \), so \(\{a, b\} = \{a\} \cup \{b\} \) by \DEF{3.1.4}.

\(A \cup B = B \cup A\): for any object \(x\),
\begin{align*}
         & x \in A \cup B \\
    \iff & x \in A \lor x \in B & \text{by \AXM{3.4}} \\
    \iff & x \in B \lor x \in A & \text{by logic} \\
    \iff & x \in B \cup A & \text{by \AXM{3.4}}
\end{align*}

\((A \cup B) \cup C = A \cup (B \cup C)\): for any object \(x\),
\begin{align*}
         & x \in (A \cup B) \cup C \\
         \iff & x \in (A \cup B) \lor x \in C & \text{by \AXM{3.4}} \\
         \iff & (x \in A \lor x \in B) \lor x \in C & \text{by \AXM{3.4}} \\
         \iff & x \in A \lor (x \in B \lor x \in C) & \text{by logic} \\
         \iff & x \in A \lor (x \in B \cup C) & \text{by \AXM{3.4}} \\
         \iff & x \in A \cup (x \in B \cup C) & \text{by \AXM{3.4}} \\
\end{align*}
So first statement if and only if last statement, so \((A \cup B) \cup C = A \cup (B \cup C)\).

\(A \cup A = A \cup \emptyset = \emptyset \cup A = A\): for any object \(x\),
\begin{align*}
         & x \in A \cup A \\
    \iff & x \in A \lor x \in A & \text{by logic} \\
    \iff & x \in A & \text{by simplifying logic} \\
    \iff & \MAROON{A \cup A = A} & \text{by \DEF{3.1.4}} \\
    \iff & x \in A \lor x \in \emptyset & \text{by logic, something \(\lor\) something false = something} \\
    \iff & x \in A \cup \emptyset & \text{by \AXM{3.4}} \\
    \iff & \MAROON{A \cup A = A \cup \emptyset} & \text{by \DEF{3.1.4}} \\
    \iff & \MAROON{A \cup A = \emptyset \cup A} & \text{already proved commutative law}
\end{align*}
\end{proof}

\begin{remark}
幹話。
\end{remark}

\begin{note}
We are not yet in a position to define sets consisting of \(n\) objects for any given natural numbers \(n\). 老實說我不是很理解為什麼,文中敘述是說我們還沒有定義「做\ \(n\) 次操作」是什麼意思"(require iterating the above construction “\(n\) times”, but the concept of \(n\)-fold iteration has not yet been rigorously defined). 類似任意有限個元素的情況,我們目前也無法給出有無限個元素的集合的定義。這需要其他的公理。現在我們先定義什麼是「子集合」。
\end{note}

\begin{definition}[Subsets] \label{def 3.1.15}
Let \(A, B\) be sets. We say that \(A\) is a subset of \(B\), denoted \(A \subseteq B\), if and only if every element of \(A\) is also an element of \(B\), i.e. For any object \(x\), \(x \in A \implies x \in B\). We say that \(A\) is a proper subset of \(B\), denoted \(A \subsetneq B\), if \(A \subseteq B\) and \(A \neq B\).
\end{definition}

\begin{remark} \label{remark 3.1.16}
Because these definition\textbf{s}(both \(\subseteq\) and \(\subsetneq\)) involve only the notions of (set) equality and the “is an element of” relation, both of which already obey the axiom of substitution \AXM{a.7.4}, the notion of subset also automatically obeys the axiom of substitution. Thus for instance if \(A \subseteq B\) and \(A = A'\), then \(A' \subseteq B\).
\end{remark}

\begin{example} \label{example 3.1.17}
前面幹話。Given any set \(A\), we always have \(A \subseteq A\) (why?) and \(\emptyset \subseteq A\) (why?).
\end{example}

\begin{proof}
Let \(x\) be arbitrarily chosen object. Then if \(x \in A\) then \(x \in A\) is trivially true. By \DEF{3.1.15}, \(A \subseteq A\).

Let \(x\) be arbitrarily chosen object. Then if \(x \in \emptyset\) then \(x \in A\) is vacuously true. By \DEF{3.1.15}, \(\emptyset \subseteq A\).
\end{proof}

\begin{note}
下面的\ Proposition 要參考\ \DEF{8.5.1},旨在說明\ "set-inclusion" 這個\ relation 是\ partially ordered。也就是要證明它是\ reflexive(by \EXAMPLE{3.1.17}), anti-symmetric, transitive。
\end{note}

\begin{proposition} [Sets are partially ordered by set inclusion] \label{prop 3.1.18}
Let \(A, B, C\) be sets. If \(A \subseteq B\) and \(B \subsetneq C\) then \(A \subseteq C\). If \(A \subseteq B\) and \(B \subseteq A\), then \(A = B\). Finally, if \(A \subsetneq B\) and \(B \subsetneq C\) then \(A \subsetneq C\).
\end{proposition}

\begin{proof}
Transitive: Let \(A, B, C\) be sets such that \(A \subseteq B\) \MAROON{(1)} and \(B \subseteq C\) \MAROON{(2)}. Suppose object \(x \in A\), wanted: \(x \in C\). Then by \MAROON{(1)}, \(x \in B\), which with \MAROON{(2)} implies \(x \in C\).

Anti-Symmetric: Let \(A, B\) be sets such that \(A \subseteq B \land B \subseteq A\). Wanted: \(A = B\). Then we have
\begin{align*}
     & A \subseteq B \land B \subseteq A \\
\iff & (\forall\ x : x \in A \implies x \in B) \land (\forall\ x : x \in B \implies x \in A) & \text{by \DEF{3.1.15}} \\
\iff & (\forall\ x : x \in A \iff x \in B) & \text{by trivially simplifying logic} \\
\iff & A = B. & \text{by \DEF{3.1.4}}
\end{align*}
\end{proof}

Transitive for proper-inclusion: Let \(A, B, C\) be sets such that \(A \subsetneq B\) \MAROON{(1)} and \(B \subsetneq C\) \MAROON{(2)}. Wanted: \(A \subsetneq C\), that is,
\[
    A \subseteq C \land A \neq C
\] The former is trivially true since \(A \subsetneq B\) by definition implies \(A \subseteq B\) and \(B \subsetneq C\) by definition implies \(B \subseteq C\), and by transitivity of set-inclusion \(A \subseteq B\) and \(B \subseteq C\) implies \(A \subseteq C\). For the latter, suppose for the sake of contradiction that \(A = C\). Then by \MAROON{(1)} and Axiom of Substitution \AXM{a.7.4}, we have \(C \subsetneq B\), but that implies \(C \subseteq B\) and we have known \(B \subseteq C\), together implies \(B = C\) by anti-symmetry of set-inclusion. But this contradicts \(B \subsetneq C\) because that implies \(B \neq C\).

\begin{note}
根據本書的脈絡,我們直到\ \PROP{3.1.18} 才能用\ \(A \subseteq B \land B \subseteq A\) 的方法來證明\ \(A = B\)。
\end{note}

\begin{remark} \label{remark 3.1.19}
There is a relationship between subsets and unions: see for instance \EXEC{3.1.7}. 這就頭腦體操。
\end{remark}

\begin{remark} [Difference between the subset relation \(\subsetneq\) and the less than relation \(<\)] \label{remark 3.1.20}
根據自然數\ \(<\) 的三一律(\PROP{2.2.13}),若兩自然數\ \(a, b\) 不相等,則要馬\ \(a < b\) 要馬\ \(b < a\)。但任兩不相等集合\ \(A, B\),\(A \subsetneq B\) 跟\ \(B \subsetneq A\) 可以都不成立。根據\ \DEF{8.5.1},\DEF{8.5.3},自然數(這個集合)配上\ \(<\) relation 是\ totally ordered,而(所有的;這個形容詞實際上「目前」還不精確,別忘了我們還沒列完公理)集合配上\ \(\subsetneq\) 是\ partially ordered。
\end{remark}

\begin{remark} [Subset relation and element relation] \label{remark 3.1.21}
\(2 \in \{1, 2, 3\}\), but \(2 \not \subseteq \{1, 2, 3\}\). \(\{2\} \subseteq \{1, 2, 3\}\), but \(\{2\} \notin \{1, 2, 3\}\). The point is that the number \(2\) and the set \(\{2\}\) are distinct objects.

BTW, It is important to distinguish sets from their elements, as \textbf{they can have different properties}. For instance, it is possible to have an \textbf{infinite set} consisting \textbf{of finite} numbers (the set \(\SET{N}\) of natural numbers is one such example), and it is also possible to have a finite set consisting of infinite objects (consider for instance the finite set \(\{ \SET{N}, \SET{Z} , \SET{Q} , \SET{R} \} \), which has four elements, all of which are infinite)
\end{remark}

\begin{note}
上面\ remark 可參考 \href{https://math.stackexchange.com/questions/851599/why-set-of-natural-numbers-is-infinite-while-each-natural-number-is-finite}{這篇},可用數學歸納法推得\ every natural number 是\ finite。但是「set 是 infinite」這點可能要先定義\ set 的\ cardinality? 我還不是很確定。
\end{note}

\begin{note}
We now give an axiom which easily allows us to \emph{create subsets out of} larger sets.
\end{note}

\begin{axiom} [Axiom of specification] \label{axm 3.5}
Let \(A\) be a set, and for each \(x \in A\), let \(P(x)\) be a \textbf{property} pertaining to \(x\) (i.e., \(P(x)\) is either a true statement or a false statement). Then \textbf{there exists} a set, called \( \{ x \in A : P(x) \text{\ is true} \} \) (or simply \( \{ x \in A : P(x) \} \) for short), whose elements are precisely the elements \(x\) in \(A\) for which \(P(x)\) is true. In other words, for any object \(y\),
\[
    y \in \{ x \in A : P(x) \text{\ is true} \} \iff (y \in A \text{\ and \(P(y)\) is true}).
\]
\end{axiom}

\begin{note}
This axiom is also known as the \emph{axiom of separation}. Note that \( \{ x \in A : P(x) \text{\ is true} \} \) is \textbf{always a subset} of \(A\) (why? \MAROON{(1)}), though it could be as large as \(A\) or as small as the empty set.

One can verify that the axiom of substitution \emph{works for specification}, thus if \(A = A'\) then \( \{ x \in A : P(x) \} = \{ x \in A' : P(x) \} \) (why? \MAROON{(2)})
\end{note}

\begin{proof}
\MAROON{(1)}: Let \(y\) be an arbitrary object such that \(y \in \{ x \in A : P(x) \text{\ is true} \} \). By \AXM{3.5}, we have \(y \in A \text{\ and \(P(y)\) is true}\). In particular, \(y \in A\). Since \(y\) is arbitrary, by \DEF{3.1.15}, \( \{ x \in A : P(x) \text{\ is true} \} \subseteq A \).

\MAROON{(2)}:
\begin{align*}
         & y \in \{ x \in A : P(x) \} \\
    \iff & (y \in A \land P(y)) & \text{by \AXM{3.5}} \\
    \iff & (y \in A' \land P(y)) & \text{since A = A', and \(y \in A\) and \AXM{a.7.4}} \\
    \iff & y \in \{ x \in A' : P(x) \} & \text{by \AXM{3.5}}
\end{align*}
\end{proof}

\begin{example}
幹話。
\end{example}


\begin{note}
BTW,\AXM{3.5} 裡面的冒號\ ``:'' 常會改成\  ``\(\mid\)'',主要是冒號可能會拿來代表別的意思。例如標記\ function 的\ domain 和\ codomain \(f : X \xrightarrow{} Y\)。Function 的定義參考\ \SEC{3.3}。
\end{note}

\begin{note}
接下來我們用\ \AXM{3.5} 來定義其它的\ set operation: 交集和補集(聯集已經在\ \AXM{3.4} 定義了。但蠻奇妙的,聯集是公理,但交集是從\ specification 進一步定義的)。
\end{note}

\begin{definition} [Intersections] \label{def 3.1.23}
The intersection \(S_1 \cap S_2 \) of two sets is \emph{defined} to be the set
\[
    S_1 \cap S_2 := \{ x \in S_1 : x \in S_2 \}.
\]
\end{definition}

\begin{note}
注意這個定義看起來有點玄妙,我猜是因為目前用 \AXM{3.5},你必須從一個已經是\ set 的\ object 做\ specification,所以一定要先說要從哪個集合拿元素(i.e. \(S_1\)),然後再說要符合什麼\ property(i.e. \(x \in S_2\))。

實際上某種不精確的寫法\ \( \{ x : x \in S_1 \land x \in S_2 \} \) 是沒有定義的,而且還可能會遇到羅素悖論?
\end{note}

\begin{remark} \label{remark 3.1.24}
Note that this definition is obeys the axiom of substitution \AXM{a.7.4} because it is defined in terms of more primitive operations which were already known to obey the axiom of substitution. Similar remarks apply to future definitions in this chapter and will usually not be mentioned explicitly again.
\end{remark}

\begin{example}
幹話。
\end{example}

\begin{remark} \label{remark 3.1.26}
這裡在講英文的\ "and" 根據情境有可能是\ "or"(union),甚至是\ addition。但用在邏輯描述則一定是\ \(\land\)。
\end{remark}

\begin{note}
Two set \(A, B\) are said to be \emph{disjoint} if \(A \cap B = \emptyset\). Meanwhile, the sets \(\emptyset\) and \(\emptyset\) are disjoint but not distinct (why?).
\end{note}

\begin{proof}
It is trivially that for any set \(A\), \(A \cap \BLUE{\emptyset} = \emptyset\). In particular, \(\GREEN{\emptyset} \cap \BLUE{\emptyset} = \emptyset\), so by definition of disjoint, \(\GREEN{\emptyset}\) and \(\BLUE{\emptyset}\) are disjoint. But \(\GREEN{\emptyset} = \BLUE{\emptyset}\), so \(\GREEN{\emptyset}\) and \(\BLUE{\emptyset}\) are not distinct.
\end{proof}