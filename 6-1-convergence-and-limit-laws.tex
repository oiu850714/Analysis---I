\section{Convergence and limit laws}

In the previous chapter, we defined the real numbers as \emph{formal limits} of \emph{rational} (Cauchy) sequences, and we then defined various operations on the real numbers.
However, \emph{unlike} our work in constructing the integers (where we eventually replaced formal differences with actual differences) and rationals (where we eventually replaced formal quotients with actual quotients),
we never really finished the job of constructing the real numbers, because we never got around to replacing formal limits \(\LIM_{n \toINF} a_n\) with actual limits \(\lim_{n \toINF} a_n\).
In fact, we \emph{haven’t defined} limits at all yet.
This will now be rectified.

We begin by repeating much of the machinery of \(\varE\)-close sequences, etc. again
- but this time, we do it for sequences of \emph{real} numbers, not rational numbers.
Thus this discussion will \emph{supersede} what we did in the previous chapter.

\begin{definition} [Distance between two real numbers] \label{def 6.1.1}
Given two real numbers \(x\) and \(y\), we define their distance \(d(x, y)\) to be \(d(x, y):= \abs{x - y}\).
\end{definition}

\begin{note}
Clearly \DEF{6.1.1} is consistent with \DEF{4.3.2}.
Further, \PROP{4.3.3} works just as well for real numbers as it does for rationals, because the real numbers obey all the rules of algebra that the rationals do.
(In fact this had been discussed in the middle of page 114; it's somewhat redundant.)
\end{note}

\begin{definition} [\(\varE\)-close real numbers] \label{def 6.1.2}
Let \(\varE > 0\) be a real number.
We say that two real numbers \(x, y\) are \(\varE\)-close iff we have \(d(y, x) \le \varE\).
\end{definition}

\begin{note}
Again, it is clear that this \DEF{6.1.2} of \(\varE\)-close is consistent with \DEF{4.3.4}.
\end{note}

\begin{note}
Now let \((a_n)_{n = m}^\infty\) be a sequence of \emph{real} numbers;
i.e., we assign a real number \(a_n\) for every integer \(n \geq m\).
The starting index \(m\) is some integer;
usually this will be \(1\), but in some cases we will start from some index other than \(1\).
(The choice of \emph{label} used to index this sequence is unimportant; we could use for instance \((a_k)_{k = m}^{\infty}\) and this would represent exactly the same sequence as \((a_n)_{n = m}^{\infty}\).)
\end{note}

We can define the notion of a Cauchy sequence in the same manner as before.

\begin{definition} [Cauchy sequences of reals] \label{def 6.1.3}
(Annoyingly, we actually define three concepts at the same time.)
Let \(\varE > 0\) be a \emph{real} number.
A sequence \((a_n)_{n = N}^{\infty}\) of \emph{real} numbers starting at some integer index \(N\) is said to be \emph{\(\varE\)-steady} iff \(a_j\) and \(a_k\) are \(\varE\)-close for every \(j, k \ge N\).
A sequence \((a_n)_{n = m}^{\infty}\) starting at some integer index \(m\) is said to be \emph{eventually \(\varE\)-steady} iff there exists an \(N \ge m\) such that \((a_n)_{n = N}^{\infty}\) is \(\varE\)-steady.
We say that \((a_n)_{n = m}^{\infty}\) is a \emph{Cauchy sequence} iff it is eventually \(\varE\)-steady \emph{for every} \(\varE > 0\).
\end{definition}

\begin{note}
You can compare \DEF{5.1.3}, \DEF{5.1.6}, \DEF{5.1.8} with \DEF{6.1.3}.
Note that ``\(\varE\)-close'' in \DEF{6.1.3} is depend on \DEF{6.1.2}, \emph{not} \DEF{4.3.4}.

These definitions are consistent with the corresponding definitions for rational numbers, although \emph{verifying consistency} for Cauchy sequences takes a little bit of care:
\end{note}

\begin{proposition} \label{prop 6.1.4}
Let \((a_n)_{n = m}^{\infty}\) be a sequence of \emph{rational} numbers starting at some integer index \(m\).
Then \((a_n)_{n = m}^{\infty}\) is a Cauchy sequence in the sense of \DEF{5.1.8} if and only if it is a Cauchy sequence in the sense of \DEF{6.1.3}.
\end{proposition}

\begin{note}
這邊只要考慮\ sequence of ``rational'' 在新的定義跟舊的定義相容。 我們也不用考慮\ sequence of ``real'' 要\ ``相容'' \DEF{5.1.8},因為\ \DEF{5.1.8} 根本不\ care sequence of real。
\end{note}

\begin{proof}
Suppose first that \((a_n)_{n = m}^{\infty}\) is a Cauchy sequence in the sense of \DEF{6.1.3};
then it is eventually \(\varE\)-steady for every \emph{real} \(\varE > 0\).
In particular, given arbitrary rational \(\varE\), since rational is real, by \DEF{6.1.3} \((a_n)_{n = m}^{\infty}\) is eventually \(\varE\)-steady.
Since rational \(\varE > 0\) is arbitrary, \((a_n)_{n = m}^{\infty}\) is eventually \(\varE\)-steady for every rational \(\varE > 0\).
which makes it a Cauchy sequence in the sense of \DEF{5.1.8}.

Now suppose that \((a_n)_{n = m}^{\infty}\) is a Cauchy sequence in the sense of \DEF{5.1.8};
then it is eventually \(\varE\)-steady (in the sense of \DEF{5.1.6}) for every \emph{rational} \(\varE > 0\).
Now given arbitrary \emph{real} number \(\varE > 0\), then there exists a \emph{rational} \(\varE' > 0\) which is smaller than (or equal to) \(\varE\), by \PROP{5.4.12}.
Since \(\varE'\) is \emph{rational}, we know that \((a_n)_{n = m}^{\infty}\) is eventually \(\varE'\)-steady(since it is supposed to satisfy \DEF{5.1.8});
and since \(\varE' < \varE\), this implies that \((a_n)_{n = m}^{\infty}\) is eventually \(\varE\)-steady.
Since \(\varE\) is an arbitrary positive \emph{real} number, we thus see that \((a_n)_{n = m}^{\infty}\) is a Cauchy sequence in the sense of \DEF{6.1.3}.
\end{proof}

\begin{note}
Because of this proposition, we will no longer care about the distinction between \DEF{5.1.8} and \DEF{6.1.3}, and view the concept of a Cauchy sequence as a single unified concept.
\end{note}

Now we talk about what it means for a sequence of real numbers to \emph{converge} to some limit \(L\).

\begin{definition} [Convergence of sequences] \label{def 6.1.5}
(Again annoyingly, we actually define three concepts at the same time.)
Let \(\varE > 0\) be a \emph{real} number, and let \(L\) be a \emph{real} number.
A sequence \((a_n)_{n = N}^{\infty}\) of real numbers is said to be ``\(\varE\)-close'' to \(L\) iff \(a_n\) is \(\varE\)-close(in the sense of \DEF{6.1.2}) to \(L\) for every \(n \ge N\),
i.e., we have \(\abs{a_n - L} \le \varE\) for every \(n \ge N\).
We say that a sequence \((a_n)_{n = m}^{\infty}\) is ``eventually \(\varE\)-close'' to \(L\) iff there exists an \(N \ge m\) such that \((a_n)_{n = N}^{\infty}\) is \(\varE\)-close to \(L\).
We say that a sequence \((a_n)_{n = m}^{\infty}\) \textbf{converges to} \(L\) iff it is ``eventually \(\varE\)-close'' to \(L\) for every real \(\varE > 0\).
\end{definition}

\begin{note}
One can unwrap all the definitions here and write the concept of convergence more directly; see \EXEC{6.1.2}.
\end{note}

\begin{example} \label{examples 6.1.6}
The sequence
\[
    0.9, 0.99, 0.999, 0.9999,...
\]
is \(0.1\)-close to \(1\), but is not \(0.01\)-close to \(1\), because of the first element of the sequence.
However, it is \emph{eventually} \(0.01\)-close to \(1\).
In fact, for every real \(\varE > 0\), this sequence is \emph{eventually} \(\varE\)-close to \(1\), hence is \emph{convergent} to \(1\).
\end{example}

\begin{proposition} [Uniqueness of limits] \label{prop 6.1.7}
Let \((a_n)_{n = m}^{\infty}\) be a real sequence starting at some integer index \(m\),
and let \(L \neq L'\) be two distinct real numbers.
Then it is \emph{not} possible for \((a_n)_{n = m}^{\infty}\) to converge to \(L\) while also converging to \(L'\).
\end{proposition}

\begin{proof}
Suppose for sake of contradiction that \((a_n)_{n = m}^{\infty}\) was converging to both \(L\) and \(L'\).
Let \(\varE = \abs{L - L'}/3\) \MAROON{(*)};
note that \(\abs{L - L'}\) is positive \MAROON{(**)} since \(L \neq L'\), and hence \(\abs{L - L'}/3\) is positive so \(\varE\) is positive.
Since \((a_n)_{n = m}^{\infty}\) converges to \(L\), we know that \((a_n)_{n = m}^{\infty}\) is eventually \(\varE\)-close to \(L\);
thus there is an \(N \ge m\) such that \(d(a_n, L) \le \varE\) for all \(n \ge N\).
\emph{Similarly}, there is an \(M \ge m\) such that \(d(a_n, L') \le \varE\) for all \(n \ge M\).
In particular, if we set \(n := \max(N, M)\), then we have \(d(a_n, L) \le \varE\) and \(d(a_n, L') \le \varE\), hence by the triangle inequality \(d(L,L') \le d(L, a_n) + d(a_n, L') \le 2\varE\), which by \MAROON{(*)} is equal to \(2\abs{L - L'}/3\).
But then we have
\begin{align*}
             & d(L, L') = \abs{L - L'} \le 2\abs{L - L'}/3 \\
    \implies & \abs{L - L'}/3 \le 0 \\
    \implies & \abs{L - L'} \le 0
\end{align*}
which contradicts \MAROON{(**)}.
Thus it is not possible to converge to both \(L\) and \(L'\).
\end{proof}

Now that we know limits are unique, we can set up notation to specify them:

\begin{definition} [Limits of sequences] \label{def 6.1.8}
If a sequence \((a_n)_{n = m}^{\infty}\) converges to some real number \(L\), we say that \((a_n)_{n = m}^{\infty}\) is \emph{convergent} and that its limit is \(L\);
we write
\[
    L = \lim_{n \toINF} a_n
\]
to denote this fact.
If a sequence \((a_n)_{n = m}^{\infty}\) is \emph{not} converging to any real number \(L\), we say that the sequence \((a_n)_{n = m}^{\infty}\) is \emph{divergent} and we leave \(\lim_{n \toINF} a_n\) \emph{undefined}.
\end{definition}

\begin{note}
\PROP{6.1.7} ensures that a sequence can have \emph{at most one} limit.
Thus, if the limit exists, it is a single real number, otherwise
it is undefined.
\end{note}

\begin{remark} \label{remark 6.1.9}
The notation \(\lim_{n \toINF} a_n\) does \emph{not} give any indication about the \emph{starting index} \(m\) of the sequence,
but the starting index is irrelevant (\EXEC{6.1.3}).
Thus in the rest of this discussion we shall not be too careful as to where these sequences start, as we shall be mostly focused on their limits.
\end{remark}

\begin{note}
We sometimes use the phrase ``\(a_n \to x\) as \(n \to \infty\)'' as an alternate way of writing the statement ``\((a_n)_{n = m}^{\infty}\) converges to \(x\)''.
Bear in mind, though, that the \emph{individual} statements \(a_n \to x\) and \(n \to \infty\) do not have any rigorous meaning;
this phrase is just a convention, though of course a very suggestive one.
\end{note}

\begin{remark} \label{remark 6.1.10}
The exact choice of letter used to denote the index (in this case \(n\)) is \emph{irrelevant}:
the phrase \(lim_{n \toINF} a_n\) has exactly the same meaning as \(\lim_{k \toINF} a_k\), for instance.
Sometimes it will be convenient to change the label of the index to avoid conflicts of notation;
for instance, we might want to change \(n\) to \(k\) because \(n\) is simultaneously being used for some other purpose, and we want to reduce confusion.
\end{remark}

\begin{note}
這邊說要我們去看\ \EXEC{6.1.4} 來比較\ \RMK{6.1.10},但那個練習感覺沒有直接關聯,該練習在講一個\ sequence 拔掉前面有限項,跟它是否收斂無關。
\end{note}

\begin{proposition} \label{prop 6.1.11}
We have \(\lim_{n \toINF} 1/n = 0\).
\end{proposition}

\begin{note}
Compare to \EXEC{5.3.5}.
\end{note}

\begin{proof}
By \DEF{6.1.8}, we have to show that the sequence \((a_n)_{n = 1}^{\infty}\) converges to \(0\), where \(a_n := 1/n\).
In other words, by \DEF{6.1.5}, for every \(\varE > 0\), we need to show that the sequence \((a_n)_{n = 1}^{\infty}\) is eventually \(\varE\)-close to \(0\).

So, let \(\varE > 0\) be an arbitrary real number.
We have to find an \(N\) such that \(\abs{a_n - 0} \le \varE\) for every \(n \ge N\).
But \(\varE > 0\), and by Archimedean property(\CORO{5.4.13}), we can find an integer \(N\) s.t. \(N\varE > 1\) and
\begin{align*}
             & N \varE > 1 \\
    \implies & N > 1/\varE \\
    \implies & 1/N < \varE \MAROON{(*)}.
\end{align*}
But for all \(n \ge N\), \(1/n \le 1/N\).
With \MAROON{(*)}, we have \(1/n \le \varE\), and trivially we have \(\abs{1/n - 0} \le \varE\).
So for all \(n \ge N\), \(\abs{1/n - 0} \le \varE\).
So by \DEF{6.1.5} \((a_n)_{n = 1}^{\infty}\) is eventually \(\varE\)-close to \(0\).
Since \(\varE\) was arbitrary, again by \DEF{6.1.5}, \((a_n)_{n = 1}^{\infty}\) converges to \(0\).
\end{proof}

\begin{proposition} [Convergent sequences are Cauchy] \label{prop 6.1.12}
Suppose that \((a_n)_{n = m}^{\infty}\) is a \emph{convergent} sequence of real numbers.
Then \((a_n)_{n = m}^{\infty}\) is also a \emph{Cauchy} sequence.
\end{proposition}

\begin{proof}
Suppose \((a_n)_{n = m}^{\infty}\) is a convergent sequence of real numbers.
By \DEF{6.1.3}, we have to show that given arbitrary \(\varE > 0\), there exists integer \(N \ge m\) s.t. \(d(a_n, a_{n'}) \le \varE\) for all \(n, n' \ge N\).

So let \(\varE > 0\).
In particular \(\varE/2 > 0\).
And WLOG, we let \((a_n)_{n = m}^{\infty}\) converge to \(L\) (or \(L = \lim_{n \toINF} a_n\).).
And by \DEF{6.1.5}, we can find an integer \(N \ge m\) s.t. \(d(a_n, L) \le \varE/2\) for all \(n \ge N\) \MAROON{(*)}.
So in particular given integer \(n, m' \ge N\):
\begin{align*}
             & d(a_n, L) \le \varE/2 \land d(a_{n'}, L) \le \varE/2 & \text{by \MAROON{(*)}} \\
    \implies & d(a_n, L) + d(a_{n'}, L) \le \varE \\
    \implies & d(a_n, a_{n'}) \le d(a_n, L) + d(a_{n'}, L) \le \varE & \text{by \PROP{4.3.3}(g)} \\
    \implies & d(a_n, a_{n'}) \le \varE & \text{simplify}
\end{align*}
So by \DEF{6.1.3}, \((a_n)_{n = m}^{\infty}\) is eventually \(\varE\)-steady.
Since \(\varE > 0\) is arbitrary, \((a_n)_{n = m}^{\infty}\) is eventually-\(\varE\)-steady for all \(\varE > 0\).
Again by \DEF{6.1.3}, \((a_n)_{n = m}^{\infty}\) is a Cauchy sequence.
\end{proof}

\begin{example} \label{example 6.1.13}
The sequence \(1, -1, 1, -1, 1, -1,...\) is not a Cauchy sequence (because it is not eventually \(1\)-steady), and is hence not a
convergent sequence, by (contrapositive of) \PROP{6.1.12}.
\end{example}

\begin{remark} \label{remark 6.1.14}
For a \emph{converse} to \PROP{6.1.12}, see \THM{6.4.18} below.
\end{remark}

\begin{note}
開天眼,\THM{6.4.18} (in particular, \PROP{6.1.12} 的\ converse) 是想說實數系是個「完備賦距空間」。
實數的柯西序列會收斂到實數,但是有理數的柯西序列會跑出有理數\
(例如, informally, \(1, 1.4, 1.41, 1.414,...\) 會跑去無理數\ \(\sqrt{2}\))。
\end{note}

Now we show that \emph{formal} limits can be superseded by \emph{actual} limits.

\begin{proposition} [Formal limits are genuine limits] \label{prop 6.1.15}
Suppose that \((a_n)_{n = 1}^{\infty}\) is a Cauchy sequence of \emph{rational} numbers.
Then \((a_n)_{n = 1}^{\infty}\) converges to \(\LIM_{n \toINF} a_n\),
i.e.
\[
    \LIM_{n \toINF} a_n = \lim_{n \toINF} a_n.
\]
\end{proposition}

\begin{note}
Similar as \PROP{6.1.4}, we only need to consider the sequence of rationals because the symbol \(\LIM\) in \CH{5} was used only on the sequence of rationals.
\end{note}

\begin{proof}
Let \((a_n)_{n = m}^{\infty}\) be a Cauchy sequence of rationals, and write \(L := \LIM_{n \toINF} a_n\).
We have to show that \((a_n)_{n = m}^{\infty}\) converges to \(L\).
Let \(\varE > 0\).
Assume for sake of contradiction that sequence \(a_n\) is \emph{not} eventually \(\varE\)-close to \(L\) \MAROON{(*)}.
Then by \DEF{6.1.5}, for all \(N \ge m\), there exists an integer \(n \ge N\) s.t. \(d(a_n, L) > \varE\).
Also since \((a_n)_{n = m}^{\infty}\) is Cauchy, and in particular \(\varE/2 > 0\), there exists an integer \(N_1 \ge m\) s.t. \(d(a_n, a_k) \le \varE/2\) for all \(n, k \ge N_1\) \MAROON{(**)}.
We preserve this \(N_1\).
\emph{Again}, since \MAROON{(*)}, in particular for the integer \(N_1 \ge m\), there exists an integer \(N_2 \ge N_1\) s.t. \(d(a_{N_2}, L) > \varE\) \MAROON{(***)}.
We preserve this \(N_2\).
Also note that since \(N_2 \ge N_1\), in particular it can be replaced into the free variable \(k\) in \MAROON{(**)} so that we have:
there exists an integer \(N_1 \ge m\) s.t. \(d(a_n, a_{N_2}) \le \varE/2\) for all \(n \ge N_1\).
And from \EXEC{5.4.6}, we can rewrite it as the in equality \(-\varE/2 \le a_n - a_{N_2} \le \varE/2\), and hence \(-\varE/2 + a_{N_2} \le a_n \le \varE/2 + a_{N_2}\), for all \(n \ge N_1\) \BLUE{(*)}.

Now we consider the value of \(a_{N_2} - L\):
Since from \MAROON{(***)}, we know \(a_{N_2} - L \neq 0\), so there are two cases:
\begin{itemize}
    \item[>>] \(a_{N_2} - L > 0\). Then
        \begin{align*}
            a_{N_2} - L & = \abs{a_{N_2} - L} \\
                        & = d(a_{N_2}, L) \\
                        & > \varE & \text{from \MAROON{(***)}}
        \end{align*}
        So we have \(\varE < a_{N_2} - L\).
        And from \BLUE{(*)}, in particular we have \(-\varE/2 + a_{N_2} \le a_n\), for all \(n \ge N_1\).
        Add both sides of these two inequalities, we have:
        \begin{align*}
                     & -\varE/2 + a_{N_2} \le a_n \land \varE < a_{N_2} - L \\
            \implies & \varE/2 + a_{N_2} < a_n + a_{N_2} - L \\
            \implies & \varE/2 + L < a_n & \text{for all \(n \ge N_1\).}
        \end{align*}
        Now we defined a sequence \((b_n)_{n = 1}^{\infty}\) s.t. \(b_n := a_{n + N_1 - 1}\).
        Then it's trivial that \(\LIM b_n = \LIM a_n\) and \(\varE/2 + L < b_n\) for all \(n \ge 1\). (If we proved informally, we can just use \EXEC{5.4.8} directly on the sequence \(a_n\)).
        Then from \EXEC{5.4.8}, we have \(\varE/2 + L < \LIM_{n \toINF} b_n = L\),
        which implies \(\varE < 0\), a contradiction.
    \item[>>] \(a_{N_2} - L < 0\). Then
        \begin{align*}
            a_{N_2} - L & = -\abs{a_{N_2} - L} \\
                        & = -d(a_{N_2}, L) \\
                        & < -\varE & \text{from \MAROON{(***)}}
        \end{align*}
        So we have \(\varE < L - a_{N_2}\).
        And from \BLUE{(*)}, in particular we have \(a_n \le \varE/2 + a_{N_2}\), for all \(n \ge N_1\).
        Add both sides of these two inequalities, we have:
        \begin{align*}
                     & a_n \le \varE/2 + a_{N_2} \land \varE < L - a_{N_2} \\
            \implies & a_n + \varE < \varE/2 + L \\
            \implies & a_n < L - \varE/2 & \text{for all \(n \ge N_1\).}
        \end{align*}
        Then similarly with the previous case, from \EXEC{5.4.8} we have \(\LIM_{n \toINF} a_n < L - \varE/2\), or \(L < L - \varE/2\), which implies \(\varE < 0\), a contradiction.
\end{itemize}
So in all cases we have contradiction, so there is no \(\varE > 0\) s.t. the sequence \(a_n\) is not \(\varE\)-close to \(L\).
So the sequence \(a_n\) is convergent to \(L\), that is, \(\lim_{n \toINF} a_n = L\), that is, \(\lim_{n \toINF} a_n = \LIM_{n \toINF} a_n\).
\end{proof}

\begin{definition} [Bounded sequences] \label{def 6.1.16}
A sequence \((a_n)_{n = m}^{\infty}\) of \emph{real} numbers is \emph{bounded} by a \emph{real} number \(M\) iff we have \(\abs{a_n} \le M\) for all \(n \ge m\).
We say that \((a_n)_{n = m}^{\infty}\) is bounded iff it is \emph{bounded} by \(M\) for some real number \(M > 0\).
\end{definition}

\begin{note}
This definition is consistent with \DEF{5.1.12}; see \EXEC{6.1.7}.
\end{note}

Recall from \LEM{5.1.15} that every Cauchy sequence of \emph{rational} numbers is bounded.
An inspection of the proof of that Lemma shows that the same argument works for \emph{real} numbers;
every Cauchy sequence of real numbers is bounded.
In particular, from \PROP{6.1.12} we have
\begin{corollary} \label{corollary 6.1.17}
Every convergent sequence of real numbers by \PROP{6.1.12} is Cauchy, and Cauchy by the argument above(``real'' version of \LEM{5.1.15}) is bounded.
So every convergent sequence of real numbers is bounded.
\end{corollary}

\begin{example} \label{example 6.1.18}
The sequence \(1, 2, 3, 4, 5,...\) is not bounded, and hence (by contrapositive of \CORO{6.1.17}) is not convergent.
\end{example}

We can now prove the usual limit laws.
\begin{theorem} [Limit Laws] \label{thm 6.1.19}
Let \((a_n)_{n = m}^\infty\) and \((b_n)_{n = m}^\infty\) be convergent sequences of real numbers, and let \(x, y\) be the real numbers \(x := \lim_{n \to \infty} a_n\) and \(y := \lim_{n \to \infty} b_n\).
\begin{enumerate}
\item
    The sequence \((a_n + b_n)_{n = m}^\infty\) converges to \(x + y\);
    in other words,
    \[
        \lim_{n \to \infty} (a_n + b_n) = \lim_{n \to \infty} a_n + \lim_{n \to \infty} b_n.
    \]
\item
    The sequence \((a_n b_n)_{n = m}^\infty\) converges to \(xy\);
    in other words,
    \[
        \lim_{n \to \infty} (a_n b_n) = (\lim_{n \to \infty} a_n)(\lim_{n \to \infty} b_n).
    \]
\item
    For any real number \(c\), the sequence \((c a_n)_{n = m}^\infty\) converges to \(cx\);
    in other words,
    \[
        \lim_{n \to \infty} (c a_n) = c(\lim_{n \to \infty} a_n).
    \]
\item
    The sequence \((a_n - b_n)_{n = m}^\infty\) converges to \(x - y\);
    in other words,
    \[
        \lim_{n \to \infty} (a_n - b_n) = \lim_{n \to \infty} a_n - \lim_{n \to \infty} b_n.
    \]
\item
    Suppose that \(y \neq 0\), and that \(b_n \neq 0\) for all \(n \geq m\).
    Then the sequence \((b_n^{-1})_{n = m}^\infty\) converges to \(y^{-1}\);
    in other words,
    \[
        \lim_{n \to \infty} b_n^{-1} = (\lim_{n \to \infty} b_n)^{-1}.
    \]
\item
    Suppose that \(y \neq 0\), and that \(b_n \neq 0\) for all \(n \geq m\).
    Then the sequence \((a_n / b_n)_{n = m}^\infty\) converges to \(x / y\);
    in other words,
    \[
        \lim_{n \to \infty} \frac{a_n}{b_n} = \frac{\lim_{n \to \infty} a_n}{\lim_{n \to \infty} b_n}.
    \]
\item
    The sequence \((\max(a_n, b_n))_{n = m}^\infty\) converges to \(\max(x, y)\);
    in other words,
    \[
        \lim_{n \to \infty} \max(a_n, b_n) = \max(\lim_{n \to \infty} a_n, \lim_{n \to \infty} b_n).
    \]
\item
    The sequence \((\min(a_n, b_n))_{n = m}^\infty\) converges to \(\min(x, y)\);
    in other words,
    \[
        \lim_{n \to \infty} \min(a_n, b_n) = \min(\lim_{n \to \infty} a_n, \lim_{n \to \infty} b_n).
    \]
\end{enumerate}
\end{theorem}

\begin{note}
\begin{enumerate}
\item 各項相加而成的\ seq 的極限等於兩個\ seq 的極限相加。 
\item 各項相乘而成的\ seq 的極限等於兩個\ seq 的極限相乘。
\item 逐項乘以常數而成的\ seq 的極限等於\ seq 的極限乘以該常數。
\item 各項相減而成的\ seq 的極限等於兩個\ seq 的極限相減。
\item 若原\ seq 每一項都\ \(!= 0\) 且該\ seq 不會收斂到\ 0,
則逐項取倒數而成的\ seq 的極限等於原本的\ seq 的極限的倒數。
\item 若要被當分母的\ seq 每一項都\ \(!= 0\) 且該\ seq 不會收斂到\ 0,則各項相除而成的\ seq 的極限等於兩個\ seq 的極限相除。
\item 各項取最大值而成的\ seq 的極限等於兩個\ seq 的極限的最大值。
\item 各項取最小值而成的\ seq 的極限等於兩個\ seq 的極限的最小值。
\end{enumerate}
\end{note}

\begin{proof} (a)

Then by \DEF{6.1.5}, we have to show that \(\forall \varE > 0, \exists N \le n\) s.t. \(\abs{(a_i + b_i) - (x + y)} \le \varE\ \forall i \ge N\) \BLUE{(*)}.

So let arbitrary \(\varE > 0\).
In particular \(\varE/2 > 0\).
Since \((a_n)_{n = m}^{\infty}\) converges to \(x\), we can find \(N_1 \ge m\) s.t. \(\abs{a_i - x} \le \varE/2\ \forall i \ge N_1\) \MAROON{(1)}.
Similarly we can find \(N_2 \ge m\) s.t. \(\abs{b_j - y} \le \varE/2\ \forall j \ge N_2\) \MAROON{(2)}.
Now let \(N = \max(N_1, N_2)\).
Then from \MAROON{(1)(2)} we have
\begin{align*}
             & \abs{a_k - x} \le \varE/2 \land \abs{b_k - y} \le \varE/2\ \forall k \ge N \\
    \implies & \abs{a_k - x} + \abs{b_k - y} \le \varE/2 + \varE/2 = \varE\ \forall k \ge N \\
    \implies & \abs{(a_k - x) + (b_k - y)} \le \abs{a_k - x} + \abs{b_k - y} \le \varE\ \forall k \ge N & \text{by \PROP{4.3.3}(b)} \\
    \implies & \abs{(a_k - x) + (b_k - y)} \le \varE\ \forall k \ge N \\
    \implies & \abs{(a_k + b_k) - (x + y)} \le \varE\ \forall k \ge N
\end{align*}
satisfying \BLUE{(*)} as desired.
\end{proof}

\begin{proof} (b)

By \DEF{6.1.5} we have to show that \(\forall \varE > 0, \exists N \ge n\) s.t. \(\abs{a_n b_n - xy} \le \varE\) \BLUE{(1)}.

So let arbitrary \(\varE > 0\).
Since \((a_n)_{n = m}^{\infty}\) converges to \(x\) and \((b_n)_{n = m}^{\infty}\) converges to \(y\), by \CORO{6.1.17}, they are bounded.
So by \DEF{6.1.16} we can find \(A, B > 0\) s.t. \(\abs{a_n} \le A\) \MAROON{(*1)} and \(\abs{b_n} \le B\) \MAROON{(*2)} for all \(n \ge m\).

And in particular, from \MAROON{(*1)} we have \(x \le A\) \MAROON{(*3)}, otherwise if \(x > A\), then \(x - A > 0\), hence for \(\varE = (x - A)/2 > 0\), given any \(N \ge m\), we just have found \(N\) s.t.
\begin{align*}
    \abs{a_n - x} & = \abs{x - a_n} \\
                  & = \abs{x + (-a_n)} \\
                  & \ge \abs{x} - \abs{-a_n} & \text{by \AC{4.3.1}} \\
                  & \ge \abs{x} - \abs{a_n} \\
                  & \ge x - \abs{a_n} & \text{since \(x > A > 0\)} \\
                  & \ge x - A & \text{since \(\abs{a_n} \le A\)} \\
                  & > (x - A)/2 = \varE
\end{align*}
which implies \((a_n)_{n = m}^{\infty}\) does \emph{not} converge to \(x\), a contradiction.

Now in particular \(\varE/2A > 0\) and \(\varE/2B > 0\).
And since \((a_n)_{n = m}^{\infty}\) converges to \(x\) and \((b_n)_{n = m}^{\infty}\) converges to \(y\),
\(\exists N_1\) s.t. \(\abs{a_i - x} \le \varE/2\RED{B}\) for all \(i \ge N_1\) \MAROON{(**1)}, and \(\exists N_2\) s.t. \(\abs{b_j - y} \le \varE/2\RED{A}\) for all \(j \ge N_2\) \MAROON{(**2)}.

Now let \(N = \max(N_1, N_2)\) then for all \(n \ge N\),
\begin{align*}
    \abs{a_n b_n - xy} & = \abs{a_n b_n - xy + x b_n - x b_n}  \\
                       & = \abs{a_n b_n - x b_n + x b_n - xy} \\
                       & = \abs{b_n(a_n - x) + x(b_n - y)} \\
                       & \le \abs{b_n(a_n - x)} + \abs{x(b_n - y)} & \text{by \PROP{4.3.3}(b)} \\
                       & = \abs{b_n}\abs{a_n - x} + \abs{x}\abs{b_n - y} \\
                       & \le B \X \varE/2\RED{B} + \abs{x}\abs{b_n - y} & \text{by \MAROON{(*2) (**1)}} \\
                       & \le \varE/2 + A \X \varE/2\RED{A} & \text{by \MAROON{(*3) (**2)}} \\
                       & = \varE,
\end{align*}
which satisfies \BLUE{(1)}, as desired.
\end{proof}

\begin{proof} (c)
For a constant \(c\), \(c = \lim_{n \toINF} b_n\) where \(b_n = c\ \forall n \ge m\) \MAROON{(*)}. So
\begin{align*}
    \lim_{n \toINF} c a_n & = \lim_{n \toINF} b_n a_n & \text{by \MAROON{(*)}} \\
                          & = \lim_{n \toINF} b_n \lim_{n \toINF} a_n & \text{by part (b)} \\
                          & = cx
\end{align*}
\end{proof}

\begin{proof} (d)
\begin{align*}
\lim_{n \toINF}(a_n - b_n) & = \lim_{n \toINF}(a_n + (-b_n)) & \text{by algebra} \\
                           & = \lim_{n \toINF}(a_n + (-1) b_n) & \text{by algebra} \\
                           & = \lim_{n \toINF} a_n + \lim_{n \toINF}(-1) b_n & \text{by part(a)} \\
                           & = \lim_{n \toINF} a_n + (-1)\lim_{n \toINF} b_n & \text{by part(c)} \\
                           & = \lim_{n \toINF} a_n - \lim_{n \toINF} b_n & \text {by algebra}
\end{align*}
\end{proof}

\begin{proof} (e)
From the hint, we first show that \((b_n)_{n = m}^{\infty}\) is bounded away from zero.
But that is equivalent to \((b_n^{-1})_{n = m}^{\infty}\) is \emph{bounded}:
Suppose \((b_n)_{n = m}^{\infty}\) is bounded away from zero.
Then \(\exists M > 0\) s.t. \(\abs{b_n} \ge M\ \forall n \ge m\), which implies \(\frac1{\abs{b_n}} \le \frac1{M}\ \forall n \ge m\).
But it's trivial that \(\frac1{\abs{b_n}} = \abs{\frac1{b_n}} = \abs{b_n^{-1}}\).
So we have \(\abs{b_n^{-1}} \le \frac1{M}\ \forall n \ge m\), and by \DEF{6.1.16}, \((b^{-1})_{n = m}^{\infty}\) is bounded.
Now suppose \((b_n^{-1})_{n = m}^{\infty}\) is bounded.
Then from \DEF{6.1.16} \(\exists M > 0\) s.t. \(\abs{b^{-1}} \le M\ \forall n \ge m\), which again implies \(\frac1{\abs{b_n}} \le M\ \forall n \ge m\);
which implies \(\abs{b_n} \ge \frac1{M}\ \forall n \ge m\).
So we have found \(\frac1{M} > 0\) s.t. \(\abs{b_n} \ge \frac1{M}\ \forall n \ge m\). So \((b_n)_{n = m}^{\infty}\) is bounded away from zero.

Now, we first show \((b_n^{-1})_{n = m}^{\infty}\) is bounded.
Since \(y \neq 0\), \(\abs{y} > 0\) and \(\abs{y}/2 > 0\).
Also, since \((b_n)_{n = m}^{\infty}\) converges to \(y\), \(\exists N \ge m\) s.t. \(\abs{b_n - y} \le \abs{y}/2\ \forall n \ge N\).
But since \(\abs{y} - \abs{b_n} \le \abs{b_n - y}\) (by some absolute value algebra and \AC{4.3.1}), we have \(\abs{y} - \abs{b_n} \le \abs{y}/2\),
or \(\abs{y}/2 \le \abs{b_n}\), which implies \(2/\abs{y} \ge 1/\abs{b_n} = \abs{1/b_n}\ \forall n \ge N\), or \(\abs{b_n^{-1}} \le 2/\abs{y}\ \forall n \ge N\).
So currently we have shown \(b_n^{-1}\) is bounded for \(n \ge N\).
But by \LEM{5.1.14}, the \emph{finite} \((b_n^{-1})_{n = m}^{N - 1}\) is also bounded.
Let them bounded by \(M_0\).
Then the whole sequence \((b_n^{-1})_{n = m}^{\infty}\) is bounded by \(\max(\frac{2}{\abs{y}}, M_0)\); label it as \(M\) \MAROON{(*)}.

Now we start the actual proof.
That is, \(\forall \varE > 0, \exists N \ge m\) s.t. \(\abs{b_n^{-1} - y^{-1}} \le \varE\ \forall n \ge N\).
So let arbitrary \(\varE > 0\).
Since \((b_n)_{n = m}^{\infty}\) converges to \(y\), and in particular \(\frac{\varE \abs{y}}{M} > 0\),
\(\exists N_1 \ge m\) s.t. \(\abs{b_n - y} \le \frac{\varE \abs{y}}{M}\ \forall n \ge N_1\) \MAROON{(**)}.
Then, given this \(N_1 \ge m\), we have for all \(n \ge N_1\),
\begin{align*}
    \abs{b_n^{-1} - y^{-1}} & = \abs{\frac{1}{b_n} - \frac{1}{y}} \\
                            & = \abs{\frac{y - b_n}{b_n y}} \\
                            & = \frac{1}{\abs{b_n y}} \abs{y - b_n} & \text{by absolute value algebra} \\
                            & = \frac{1}{\abs{b_n} \abs{y}} \abs{y - b_n} & \text{by absolute value algebra} \\
                            & = \abs{\frac{1}{b_n}} \frac{1}{\abs{y}} \abs{y - b_n} & \text{by absolute value algebra} \\
                            & \le M \frac{1}{\abs{y}} \abs{y - b_n} & \text{by \MAROON{(*)}} \\
                            & = M \frac{1}{\abs{y}} \abs{b_n - y} & \text{by absolute value algebra} \\
                            & \le M \frac{1}{\abs{y}} \frac{\varE \abs{y}}{M} & \text{by \MAROON{(**)}} \\
                            & = \varE,
\end{align*}
as desired.
\end{proof}

\begin{proof}(f)
\begin{align*}
\lim_{n \toINF} \frac{a_n}{b_n} & = \lim_{n \toINF} a_n b_n^{-1} & \text{by algebra} \\
                                & = \lim_{n \toINF} a_n \lim_{n \toINF} b_n^{-1} & \text {by part(b)} \\
                                & = (\lim_{n \toINF} a_n) (\frac{1}{\lim_{n \toINF} b_n}) & \text {by part(e)} \\
                                & = \frac{\lim_{n \toINF} a_n}{\lim_{n \toINF} b_n} & \text{of course}
\end{align*}
\end{proof}

\begin{proof}(g)
We split into two cases: \(x \ge y\) and \(x < y\).
\begin{itemize}
\item[\(x \ge y\)]: Then \(\max(x, y) = x\).
    So we have to show \((\max(a_n, b_n))_{n = m}^{\infty}\) converges to \(x\).
    That is, \(\forall \varE > 0, \exists N \ge m\) s.t. \(\abs{\max(a_n, b_n) - x} \le \varE\ \forall n \ge N\).
    
    Sot let arbitrary \(\varE > 0\).
    Since \((a_n)_{n = m}^{\infty}\) converges to \(x\), \(\exists N_1 \ge m\) s.t. \(\abs{a_n - x} \le \varE\ \forall n \ge N_1\) \MAROON{(*)}.
    Since \((b_n)_{n = m}^{\infty}\) converges to \(y\), \(\exists N_2 \ge m\) s.t. \(\abs{b_n - y} \le \varE\ \forall n \ge N_2\) \MAROON{(**)}.
    Now let \(N = \max(N_1, N_2)\).
    Then with \MAROON{(*)(**)} and \EXEC{5.4.6}, \(-\varE \le a_n - x \le \varE \land -\varE \le b_n - y \le \varE\ \forall n \ge N\),
    or \(x - \varE \le a_n \le x + \varE \land y -\varE \le b_n \le y + \varE\ \forall n \ge N\) \MAROON{(***)}.
    
    But since \(x \ge y\), we have \(x + \varE \ge y + \varE\), so from \MAROON{(***)} we have \(a_n \le x + \varE\) and \(b_n \le y + \varE \le x + \varE\).
    So both \(a_n, b_n \le x + \varE\), so \(\max(a_n, b_n) \le x + \varE\) \BLUE{(*)}.
    Also from \MAROON{(***)}, we know \(x - \varE \le a_n\), which \(\le \max(a_n, b_n)\), so we have \(x - \varE \le \max(a_n, b_n)\) \BLUE{(**)}.
    By \BLUE{(*)(**)} we have \(x - \varE \le \max(a_n, b_n) \le x + \varE\).
    By \EXEC{5.4.6}, we have \(\abs{\max(a_n, b_n) - x} \le \varE\).
    So we have found \(N \ge m\) s.t. \(\abs{\max(a_n, b_n) - x} \le \varE\ \forall n \ge N\), as desired.
\item[\(x < y\)]: Then \(\max(x, y) = y\).
    So similarly we have to show \(\forall \varE > 0, \exists N \ge m\) s.t. \(\abs{\max(a_n, b_n) - y} \le \varE\ \forall n \ge N\).
    
    So let arbitrary \(\varE > 0\).
    Again, with the same argument up to \MAROON{(***)}, we have \(x - \varE \le a_n \le x + \varE \land y -\varE \le b_n \le y + \varE\ \forall n \ge N\) \MAROON{(****)}.
    
    Since \(x < y\), we have \(x + \varE < y + \varE\).
    So from \MAROON{(****)}, we have \(a_n \le x + \varE < y + \varE\) and \(b_n \le y + \varE\).
    So both \(a_n, b_n \le y + \varE\), so \(\max(a_n, b_n) \le y + \varE\) \BLUE{(***)}.
    Also, from \MAROON{(****)}, we have \(y - \varE \le b_n\), which implies \(y - \varE \le \max(a_n, b_n)\) \BLUE{(****)}.
    So from \BLUE{(***)(****)} we have \(y - \varE \le \max(a_n, b_n) \le y + \varE\).
    By \EXEC{5.4.6}, we get \(\abs{\max(a_n, b_n) - y} \le \varE\).
    So we have found \(N \ge m\) s.t. \(\abs{\max(a_n, b_n) - y} \le \varE\ \forall n \ge N\), as desired.
\end{itemize}
\end{proof}

\begin{proof}(h)
First we give a fact that \(\min(a, b) = -\max(-a, -b)\) \BLUE{(*)};
we split into three cases:
\begin{itemize}
    \item[\(a > b\)]: then \(\min(a, b) = b\).
        Also, \(-a < -b\), so \(\max(-a, -b) = -b\), so \(-\max(-a, -b) = -(-b) = b\).
    \item[\(a = b\)]: then \(\min(a, b) = b\).
        Also, \(-a = -b\), so \(\max(-a, -b) = -b\), so \(-\max(-a, -b) = -(-b) = b\).
    \item[\(a < b\)]: then \(\min(a, b) = a\).
        Also, \(-a > -b\), so \(\max(-a, -b) = -a\), so \(-\max(-a, -b) = -(-a) = a\).
\end{itemize}
So in all cases, LHS = RHS. That is, \(\min(a, b) = -\max(-a, -b)\).

So we have
\begin{align*}
    \lim_{n \toINF} \min(a_n, b_n) & = \lim_{n \toINF} -\max(-a_n, -b_n) & \text{by \BLUE{(*)}} \\
                                   & = -\lim_{n \toINF} \max(-a_n, -b_n) & \text{by part(c)} \\
                                   & = -\max(\lim_{n \toINF} -a_n, \lim_{n \toINF} -b_n) & \text{by part(g)} \\
                                   & = -\max(-\lim_{n \toINF} a_n, -\lim_{n \toINF} b_n) & \text{by part(c)} \\
                                   & = \min(\lim_{n \toINF} a_n, \lim_{n \toINF} b_n) & \text{by \BLUE{(*)}}
\end{align*}
\end{proof}

\exercisesection

\begin{exercise} \label{exercise 6.1.1}
Let \((a_n)_{n = 0}^{\infty}\) be a sequence of real numbers, such that \(a_{n + 1} > a_n\) for each natural number \(n\).
Prove that whenever \(n\) and \(m\) are natural numbers such that \(m > n\), then we have \(a_m > a_n\).
(We refer to these sequences as \emph{increasing} sequences.)
\end{exercise}

\begin{note}
每一項的下一項都比自己大,則比自己還要後面的任何項都比自己大。
\end{note}

\begin{proof}
We prove this by induction.

What we want: Given any natural number \(n\), \(a_{n + k} > a_n\) for any natural number \(k \ge 1\). So let \(n\) be arbitrary natural number.

For base case \(k = 1\), \(a_{n + 1} > a_n\) is guaranteed by the property of the sequence.

Suppose for some natural number \(k \ge 1\), \(a_{n + k} > a_n\);
we have to show \(a_{(n + 1) + k} > a_n\).
But again by the property of the sequence we have \(a_{n + k + 1} > a_{n + k}\), or \(a_{(n + 1) + k} > a_{n + k}\).
With the hypothesis \(a_{n + k} > a_n\), by order transitivity  we have \(a_{(n + 1) + k} > a_n\), as desired.
This closes the induction.

So given any two natural number \(m > n\), \(m = n + k\) for some positive natural number, so by the statement we have shown, \(a_{n + k} > a_n\), or \(a_m > a_n\).
\end{proof}

\begin{exercise} \label{exercise 6.1.2}
Let \((a_n)_{n = m}^{\infty}\) be a sequence of real numbers, and let \(L\) be a real number.
Show that \((a_n)_{n = m}^{\infty}\) converges to \(L\) if and only if,
given any real \(\varE > 0\), one can find an \(N \ge m\) such that \(\abs{a_n - L} \le \varE\) for all \(n \ge N\).
\end{exercise}

\begin{align*}
         & (a_n)_{n = m}^{\infty} \text{ converges to } L \\
    \iff & \forall \varE > 0, (a_n)_{n = m}^{\infty} \text{ is eventually } \varE\text{-close to } L & \text{by \DEF{6.1.5}} \\
    \iff & \forall \varE > 0, \exists N \ge n \text{ s.t. } (a_n)_{n = N}^{\infty} \text{ is } \varE\text{-close to } L & \text{by \DEF{6.1.5}} \\
    \iff & \forall \varE > 0, \exists N \ge n \text{ s.t. } \abs{a_n - L} \le \varE\ \forall n \ge N & \text{by \DEF{6.1.5}}
\end{align*}

\begin{exercise} \label{exercise 6.1.3}
Let \((a_n)_{n = m}^{\infty}\) be a sequence of real numbers, let \(c\) be a real number, and let \(m' \ge m\) be an integer.
Show that \((a_n)_{n = m}^{\infty}\) converges to \(c\) if and only if \((a_n)_{n = m'}^{\infty}\) converges to \(c\).
\end{exercise}

\begin{proof}
Suppose \((a_n)_{n = m}^{\infty}\) converges to \(c\), we have to show \((a_n)_{n = m'}^{\infty}\) converges to \(c\).

So given arbitrary \(\varE > 0\), we have to find an integer \(N \ge m'\) s.t. \(\abs{a_n - c} \le \varE\) for all \(n \ge N\).
But since \((a_n)_{n = m}^{\infty}\) converges to \(c\), by \DEF{6.1.5}(or unwrapped \EXEC{6.1.2}), we can find an \(N_1 \ge m\) s.t. \(\abs{a_n - c} \le \varE\ \forall n \ge N_1\).
Now there are two cases: \(m' \ge N_1\) or \(m' < N_1\).
\begin{itemize}
    \item[\(m' \ge N_1\)]:
        Then since \(\abs{a_n - c} \le \varE\ \forall n \ge N_1\), we of course have \(\abs{a_n - c} \le \varE\ \forall n \ge m'\).
        So by \DEF{6.1.5} \((a_n)_{n = m'}^{\infty}\) is \(\varE\)-close (not just eventually \(\varE\)-close) to \(c\).
    \item[\(m' < N_1\)]:
        Then again we have found \(N_1 \ge m'\) s.t. \(\abs{a_n - c} \le \varE\ \forall n \ge N_1\).
        So by \DEF{6.1.5} \((a_n)_{n = m'}^{\infty}\) is eventually \(\varE\)-close) to \(c\).
\end{itemize}
So in all cases, \((a_n)_{n = m'}^{\infty}\) is (eventually) \(\varE\)-close to \(c\).
Since \(\varE > 0\) is arbitrary, by \DEF{6.1.5} \((a_n)_{n = m'}^{\infty}\) converges to \(c\).

Now suppose \((a_n)_{n = m'}^{\infty}\) converges to \(c\), we have to show \((a_n)_{n = m}^{\infty}\) converges to \(c\).
So let arbitrary \(\varE > 0\).
Since \((a_n)_{n = m'}^{\infty}\) converges to \(c\), we can find an integer \(N_2 \ge m'\) s.t. \(\abs{a_n - c} \le \varE\) for all \(n \ge N_2\).
But since \(m' \ge m\), we have \(N_2 \ge m\).
So we just also find an integer \(N_2 \ge m\) s.t. \(\abs{a_n - c} \le \varE\) for all \(n \ge N_2\).
So by \DEF{6.1.5} \((a_n)_{n = m}^{\infty}\) is eventually \(\varE\)-close to \(c\).
Since \(\varE > 0\) is arbitrary, by \DEF{6.1.5}, \((a_n)_{n = m}^{\infty}\) converges to \(c\).
\end{proof}

\begin{exercise} \label{exercise 6.1.4}
Let \((a_n)_{n = m}^{\infty}\) be a sequence of real numbers, let \(c\) be a real number, and let \(k \ge 0\) be a non-negative integer.
Show that \((a_n)_{n = m}^{\infty}\) converges to \(c\) if and only if \((a_{n + k})_{n = m}^{\infty}\) converges to \(c\).
\end{exercise}

\begin{proof}
With some reindexing, we have \BLUE{\((a_{n + k})_{n = m}^{\infty} = (a_n)_{n = m + k}^{\infty}\)}.
Also since \(m + k \geq m\), by \EXEC{6.1.3} \((a_n)_{n = m}^{\infty}\) converges to \(c\) if and only if \BLUE{\((a_n)_{n = m + k}^{\infty}\)} converges to \(c\).
Thus \((a_n)_{n = m}^{\infty}\) converges to \(c\) if and only if \BLUE{\((a_{n + k})_{n = m}^{\infty}\)} converges to \(c\).
\end{proof}

\begin{exercise} \label{exercise 6.1.5}
Prove \PROP{6.1.12}.
(Hint: use the triangle inequality, or \PROP{4.3.7}.)
\end{exercise}

\begin{proof}
See \PROP{6.1.12}.
\end{proof}

\begin{exercise} \label{exercise 6.1.6}
Prove \PROP{6.1.15}, using the following outline.
Let \((a_n)_{n = m}^{\infty}\) be a Cauchy sequence of rationals, and write \(L := \LIM_{n \toINF} a_n\).
We have to show that \((a_n)_{n = m}^{\infty}\) converges to \(L\).
Let \(\varE > 0\).
Assume for sake of contradiction that sequence an is not eventually \(\varE\)-close to \(L\).
Use this, and the fact that \((a_n)_{n = m}^{\infty}\) is Cauchy, to show that there is an \(N \ge m\) such that either \(a_n > L + \varE/2\) \emph{for all} \(n \ge N\), or \(a_n < L - \varE/2\) for all \(n \ge N\).
Then use \EXEC{5.4.8}.
\end{exercise}

\begin{proof}
See \PROP{6.1.15}.
\end{proof}

\begin{exercise} \label{exercise 6.1.7}
Show that \DEF{6.1.16} is consistent with \DEF{5.1.12}
(i.e., prove an analogue of \PROP{6.1.4} for bounded sequences instead of Cauchy sequences).
\end{exercise}

\begin{proof}
I think the point is, the sequence of \emph{rationals} is bounded by a \emph{real} if and only if the sequence is bounded by a \emph{rational}.

So let \((a_n)_{n = m}^{\infty}\) be a sequence of \emph{rationals}.

Suppose the sequence is bounded by a \emph{real} number \(M\).
Then by \CORO{5.4.13} for \(\varE = 1\) we can find a positive integer \(N\) s.t. \(N \varE > M\), or \(N \X 1 > M\), or \(N > M\);
and of course the sequence is bounded by \(N\).
But \(N\) is integer, so it's \emph{rational}, so we have found a rational by which the sequence if bounded.

Now suppose the sequence is bounded by a \emph{rational} number \(M\).
Then \(M\) is just also a \emph{real} number, so the sequence is bounded by a real number.
\end{proof}

\begin{exercise} \label{exercise 6.1.8}
Prove \THM{6.1.19}.
(Hint: you can use some parts of the theorem to prove others, e.g., (b) can be used to prove (c);
(a), (c) can be used to prove (d);
and (b), (e) can be used to prove (f).
The proofs are similar to those of \LEM{5.3.6}, \PROP{5.3.10}, and \LEM{5.3.15}.
For (e), you may need to first prove the \emph{auxiliary result} that any sequence whose elements are non-zero, and which converges to a non-zero limit, is \emph{bounded away from zero}.)
\end{exercise}

\begin{proof}
See \THM{6.1.19}.
\end{proof}

\begin{exercise} \label{exercise 6.1.9}
Explain why \THM{6.1.19}(f) fails when the limit of the denominator is \(0\).
(To repair that problem requires L’\^{H}opital’s rule, see \SEC{10.5}.)
\end{exercise}

\begin{proof}
For the sake of contradiction suppose \THM{6.1.19}(f) also works when \((b_n)_{n = m}^{\infty}\) converges to \(0\).
Then in particular, let \(a_n = b_n := 1/n\) for all \(n \ge m\); we have shown in \EXEC{5.3.5} that both \((a_n)_{n = m}^{\infty}\) and \((b_n)_{n = m}^{\infty}\) converge to \(0\).
So \(\lim_{n \toINF} \frac{a_n}{b_n} = 1\) since \(\frac{a_n}{b_n} = \frac{\frac{1}{n}}{\frac{1}{n}} = 1\) for all \(n \ge m\).
But \(\frac{\lim_{n \toINF} a_n}{\lim_n \toINF b_n} = \frac{0}{0}\), which is undefined.
So the two sides of equation is \emph{not} consistent(and the RHS is even undefined).
So \THM{6.1.19}(f) cannot be applied when the sequence in the denominator converges to \(0\).
\end{proof}

\begin{exercise} \label{exercise 6.1.10}
Show that the concept of equivalent Cauchy sequence, as defined in \DEF{5.2.6}, does not change if \(\varE\) is required to be positive \emph{real} instead of positive \emph{rational}.
More precisely, if \((a_n)_{n = 0}^{\infty}\) and \((b_n)_{n = 0}^{\infty}\) are sequences of reals, show that \((a_n)_{n = 0}^{\infty}\) and \((b_n)_{n = 0}^{\infty}\) are eventually \(\varE\)-close for every \emph{rational} \(\varE > 0\) if and only if they are eventually \(\varE\)-close for every \emph{real} \(\varE > 0\).
(Hint: modify the proof of \PROP{6.1.4}.)
\end{exercise}

\begin{note}
It's mind=blowing that we have not rigorously defined the equivalence between to Cauchy sequences of \emph{real}s.
What we have defined is that they can converge to some real numbers, and the real number may be the same.
\end{note}

\begin{proof}
Let \((a_n)_{n = 0}^{\infty}\) and \((b_n)_{n = 0}^{\infty}\) be sequences of reals.

Suppose \((a_n)_{n = 0}^{\infty}\) and \((b_n)_{n = 0}^{\infty}\) are eventually \(\varE\)-close for every \emph{rational} \(\varE > 0\).
Now let \(\varE'\) be arbitrary \emph{real} s.t. \(\varE' > 0\).
Then by \PROP{5.4.12} we can find a \emph{rational} \(\varE\) s.t. \(0 < \varE < \varE'\).
And by supposition, \((a_n)_{n = 0}^{\infty}\) and \((b_n)_{n = 0}^{\infty}\) are eventually \(\varE\)-close for this \emph{rational}.
But since \(\varE < \varE'\), \((a_n)_{n = 0}^{\infty}\) and \((b_n)_{n = 0}^{\infty}\) are of course eventually \(\varE'\)-close for the real \(\varE'\).
Since \(\varE' > 0\) is arbitrary, \((a_n)_{n = 0}^{\infty}\) and \((b_n)_{n = 0}^{\infty}\) are eventually \(\varE\)-close for every \(real > 0\).

Now suppose \((a_n)_{n = 0}^{\infty}\) and \((b_n)_{n = 0}^{\infty}\) are eventually \(\varE\)-close for every \emph{real} \(\varE > 0\).
Now given any \emph{rational} \(\varE' > 0\), since \(\varE'\) are also real, by supposition \((a_n)_{n = 0}^{\infty}\) and \((b_n)_{n = 0}^{\infty}\) are eventually \(\varE'\)-close.
Since \(\varE' > 0\) is arbitrary, \((a_n)_{n = 0}^{\infty}\) and \((b_n)_{n = 0}^{\infty}\) are eventually \(\varE\)-close for every \emph{rational} \(\varE > 0\).
\end{proof}