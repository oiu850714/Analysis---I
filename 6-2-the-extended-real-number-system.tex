\section{The Extended real number system} \label{sec 6.2}

To deal with the sequences that does \emph{not} converge, e.g.
\[
    1, 2, 3, 4, 5,...
\]
\[
    -1, -2, -3, -4, -5,...
\]
\[
    1, -1, 1, -1, 1, -1,...
\]
(, which does not converge to anything but has \(-1\) and \(1\) as ``limit point'')
\[
    1, -2, 3, -4, 5, -6,...
\]
we need to talk about something called the \emph{extended real number system}.

\begin{definition} [Extended real number system] \label{def 6.2.1}
The \emph{extended real number system} \(\SET{R}^*\) is the real line \(\SET{R}\) with two additional elements attached, called \(+\infty\) and \(-\infty\).
These elements are \emph{distinct} from each other and \emph{also distinct} from every real number.
An extended real number \(x\) is called \emph{finite} iff it is a real number,
and \emph{infinite} iff it is equal to \(+\infty\) or \(-\infty\).
(This definition is \emph{not} directly related to the notion of finite and infinite sets in \SEC{3.6}, though it is of course similar in spirit.)
\end{definition}

\begin{note}
注意,\(+\infty\) or \(-\infty\) 各自都是「一個」符號,而不是\ \(+\) 作用在\ \(\infty\) 之類的;在\ \DEF{6.2.1} 之下,\(\infty\) 是無意義的東西。這件事情不搞清楚的話,\DEF{6.2.2} 就會感到問號。
\end{note}

These new symbols, \(+\infty\) and \(-\infty\), at present do not have much meaning, since we have no operations to manipulate them
(other than equality \(=\) and inequality \(\neq\)).
Now we place a few operations on the extended real number system.

\begin{definition} [Negation of extended reals] \label{def 6.2.2}
The operation of negation \(x \to -x\) on \(\SET{R}\), we now extend to \(\SET{R}^*\) by defining \(-(+\infty) := -\infty\) and \(-(-\infty) := +\infty\).
\end{definition}

Thus every \emph{extended} real number \(x\) has a negation, and \(-(-x)\) is always equal to \(x\) \MAROON{(*)}.

\begin{note}
\MAROON{(*)} is tricky.
It does \emph{not} mean \(-(-\infty) = \infty\), in fact \(\infty\) is undefined in \DEF{6.2.1}.
It in fact means \(-(-(\RED{-\infty})) = \RED{-\infty}\) and \(-(-(\RED{+\infty})) = +\infty\), why?
\begin{align*}
    -(-(\RED{-\infty})) & = -(+\infty) & \text{by \DEF{6.2.2}} \\
                        & = -\infty & \text{by \DEF{6.2.2}}
\end{align*}
and
\begin{align*}
    -(-(\RED{+\infty})) & = -(-\infty) & \text{by \DEF{6.2.2}} \\
                        & = +\infty & \text{by \DEF{6.2.2}}
\end{align*}
\end{note}

\begin{definition} [Ordering of extended reals] \label{def 6.2.3}
Let \(x\) and \(y\) be \emph{extended real} numbers.
We say that \(x \le y\), i.e., \(x\) is less than or equal to \(y\), iff \textbf{one of} the following three statements is true:
\begin{enumerate}
    \item \(x\) and \(y\) are \emph{real} numbers, and \(x \le y\) as \emph{real} numbers.
    \item \(y = +\infty\).
    \item \(x = -\infty\).
\end{enumerate}
We say that \(x < y\) if we have \(x \le y\) and \(x \ne y\).
We sometimes write \(x < y\) as \(y > x\), and \(x \le y\) as \(y \ge x\).
\end{definition}

\begin{note}
注意一下最後一句話;對於\ \(\SET{R}^*\),我們「又」把\ \(x < y\) 定義成\ \(x \le y \land x \ne y\) 了。
在這本書裡,自然數跟整數的\ order 也是這樣定義的;但是有理數跟實數就是定義成相減為正數(所以他們的定義相依於正數的定義)。
另外對於自然數、整數、\(\SET{R}^*\),\(x < y\) 跟\ \(y > x\) 我們直接定義成是「一樣的」;
但對於有理數跟實數,\(x < y\) 跟\ \(y > x\) 等價是需要證明的。
\end{note}

\begin{example} \label{example 6.2.4}
\(3 \le 5\) (satisfying \DEF{6.2.3}(a)), \(3 < +\infty\) (satisfying (b)), and \(-\infty < +\infty\) (satisfying (b), (c)), but \(3 \not \le -\infty\) (all (a), (b), (c) are not satisfied).
\end{example}

\begin{proposition} \label{prop 6.2.5}
Let \(x, y, z\) be \emph{extended real} numbers.
Then the following statements are true:
\begin{enumerate}
    \item (Reflexivity) We have \(x \le x\).
    \item (Trichotomy) Exactly one of the statements \(x < y\), \(x = y\), or \(x > y\) is true.
    \item (Transitivity) If \(x \le y\) and \(y \le z\), then \(x \le z\).
    \item (Negation reverses order) If \(x \le y\), then \(-y \le -x\).
\end{enumerate}
\end{proposition}

\begin{proof}
We already know the \emph{real numbers} has these properties, so we only need to prove the cases that one of \(x, y, z\) is \(+\infty\) or \(-\infty\).

Reflexivity: if \(x = +\infty\), then by \DEF{6.2.3}(b) \(x \le x\). if \(x = -\infty\), then by \DEF{6.2.3}(c), \(x \le x\), as desired.

Trichotomy: we prove by cases, and in each case we show ``at least'' and ``at most'' one of the trichotomy is true:
\begin{itemize}
\item \(x \in \SET{R}\) and \(y \in \SET{R}\): This fallbacks to \PROP{4.2.9}(a).
\item \(x \in \SET{R}\) and \(y = +\infty\):
    Then by \DEF{6.2.3}(b), \(x \le y\).
    Also, \(x \ne y\) since by \DEF{6.2.1} \(+\infty\) is distinct from any real number.
    Together, by \DEF{6.2.3}, we have \(x < y\), so the ``at least'' part is shown.
    Now since we have shown \(x \ne y\), we only need to show \(x > y\) is false to show the ``at most'' part.
    For the sake of contradiction, suppose \(x > y\) is true. Then by \DEF{6.2.3} this means \(y < x\), and in particular \(y \le x\).
    But \(y \le x\) is false because it satisfies none of the conditions in \DEF{6.2.3}, so we have contradiction.
\item \(x \in \SET{R}\) and \(y = -\infty\):
    Then by \DEF{6.2.3}(c), \(y \le x\).
    Also, \(y \ne x\) since by \DEF{6.2.1} \(-\infty\) is distinct from any real number.
    Together, by \DEF{6.2.3}, we have \(y < x\), so the ``at least'' part is shown.
    Now since we have shown \(y \ne x\), we only need to show \(y > x\) is false to show the ``at most'' part.
    For the sake of contradiction, suppose \(y > x\) is true. Then by \DEF{6.2.3} this means \(x < y\), and in particular \(x \le y\).
    But \(x \le y\) is false because it satisfies none of the conditions in \DEF{6.2.3}, so we have contradiction.
\item \(x = +\infty\) and \(y \in \SET{R}\):
    Then by \DEF{6.2.3}(b), \(y \le x\).
    Also, \(y \ne x\) since by \DEF{6.2.1} \(+\infty\) is distinct from any real number.
    Together, by \DEF{6.2.3}, we have \(y < x\), so the ``at least'' part is shown.
    Now since we have shown \(y \ne x\), we only need to show \(y > x\) is false to show the ``at most'' part.
    For the sake of contradiction, suppose \(y > x\) is true. Then by \DEF{6.2.3} this means \(x < y\), and in particular \(x \le y\).
    But \(x \le y\) is false because it satisfies none of the conditions in \DEF{6.2.3}, so we have contradiction.
\item \(x = +\infty\) and \(y = +\infty\):
    Then we have \(x = y\), so we cannot have \(x \ne y\) so by \DEF{6.2.3} it is not possible to have \(x < y\) or \(x > y\) since both need \(x \ne y\).
\item \(x = +\infty\) and \(y = -\infty\):
    Then by \DEF{6.2.3}(b) or (c), \(y \le x\).
    Also, \(y \ne x\) since by \DEF{6.2.1} \(+\infty\) and \(-\infty\) are distinct from each other.
    Together, by \DEF{6.2.3}, we have \(y < x\), so the ``at least'' part is shown.
    Now since we have shown \(y \ne x\), we only need to show \(y > x\) is false to show the ``at most'' part.
    For the sake of contradiction, suppose \(y > x\) is true. Then by \DEF{6.2.3} this means \(x < y\), and in particular \(x \le y\).
    But \(x \le y\) is false because it satisfies none of the conditions in \DEF{6.2.3}, so we have contradiction.
\item \(x = -\infty\) and \(y \in \SET{R}\):
    Then by \DEF{6.2.3}(c), \(x \le y\).
    Also, \(x \ne y\) since by \DEF{6.2.1} \(-\infty\) is distinct from any real number.
    Together, by \DEF{6.2.3}, we have \(x < y\), so the ``at least'' part is shown.
    Now since we have shown \(x \ne y\), we only need to show \(x > y\) is false to show the ``at most'' part.
    For the sake of contradiction, suppose \(x > y\) is true. Then by \DEF{6.2.3} this means \(y < x\), and in particular \(y \le x\).
    But \(y \le x\) is false because it satisfies none of the conditions in \DEF{6.2.3}, so we have contradiction.
\item \(x = -\infty\) and \(y = +\infty\):
    Then by \DEF{6.2.3}(b) or (c), \(x \le y\).
    Also, \(x \ne y\) since by \DEF{6.2.1} \(+\infty\) and \(-\infty\) are distinct from each other.
    Together, by \DEF{6.2.3}, we have \(x < y\), so the ``at least'' part is shown.
    Now since we have shown \(x \ne y\), we only need to show \(x > y\) is false to show the ``at most'' part.
    For the sake of contradiction, suppose \(x > y\) is true. Then by \DEF{6.2.3} this means \(y < x\), and in particular \(y \le x\).
    But \(y \le x\) is false because it satisfies none of the conditions in \DEF{6.2.3}, so we have contradiction.
\item \(x = -\infty\) and \(y = -\infty\):
    Then we have \(x = y\), so we cannot have \(x \ne y\) so by \DEF{6.2.3} it is not possible to have \(x < y\) or \(x > y\) since both need \(x \ne y\).
\end{itemize}

Transitivity: Suppose \(x \le y\) and \(y \le z\).
We split into cases by the value of \(z\):
\begin{itemize}
\item \(z = +\infty\):
    Then by \DEF{6.2.3}(b) we just have \(x \le z\).
\item \(z = -\infty\):
    Then since \(y \le z\), and both (a), (b) of \DEF{6.2.3} cannot be satisfied, so (c) must be satisfied, i.e. \(y = -\infty\).
    Similarly since \(x \le y\), we have \(x = -\infty\).
    And by \DEF{6.2.3}(c), we have \(x \le z\).
\item \(z \in \SET{R}\):
    Then since \(y \le z\), by \DEF{6.2.3}, \(y\) can only be real numbers or \(y = -\infty\). Again we split these into cases:
    \begin{itemize}
        \item
            If \(y \in \SET{R}\), then again since \(x \le y\), \(x\) can only be real numbers or \(x = -\infty\).
            If \(x \in \SET{R}\), then now all \(x, y, z\) are real numbers and \(x \le y\) and \(y \le z\), so we fallbacks to the transitivity of real numbers, so \(x \le z\).
            If \(x = -\infty\), then by \DEF{6.2.3}(c), we have \(x \le z\).
        \item
            If \(y = -\infty\), then since \(x \le y\) and both (a), (b) of \DEF{6.2.3} cannot be satisfied, (c) must be satisfied, i.e. \(x = -\infty\).
            So by \DEF{6.2.3}(c), we have \(x \le z\).
    \end{itemize}
\end{itemize}
So in all cases, we have \(x \le z\) as desired.

Negation reverses order: Suppose \(x \le y\), we have to show \(-y \le -x\).
Since \(x \le y\), we can split into cases according to \DEF{6.2.3}:
\begin{itemize}
    \item Both \(x \in \SET{R}\) and \(y \in \SET{R}\):
        Then it fallbacks to the case of real numbers, so clearly we have \(-y \le -x\).
    \item \(y = +\infty\):
        Then by \DEF{6.2.2} \(-y = -\infty\), and by \DEF{6.2.3}(c), \(-y \le -x\).
    \item \(x = -\infty\):
        Then by \DEF{6.2.2} \(-x = +\infty\), and by \DEF{6.2.3}(b), \(-y \le -x\).
\end{itemize}
So in all cases, we have \(-y \le -x\) as desired.
\end{proof}

\begin{note}
One could also introduce other operations on the extended real number system, such as addition, multiplication, etc.
However, this is somewhat dangerous as \emph{these operations will almost certainly fail to obey the familiar rules of algebra}.
For instance, to define addition it seems reasonable (given one’s intuitive notion of infinity) to set \(+\infty + 5 = +\infty\) and \(+\infty + 3 = +\infty\), but then this implies that \(+\infty + 5 = +\infty + 3\), while \(5 \neq 3\).
So things like the cancellation law begin to break down once we try to operate involving infinity.
To avoid these issues \emph{we shall simply not define any arithmetic operations} on the extended real number system other than negation and order.
\end{note}

\begin{definition} [Supremum of sets of \emph{extended reals}] \label{def 6.2.6}
Let \(E\) be a subset of \(\SET{R}^*\).
Then we define the \emph{supremum} \(\sup(E)\) or \emph{least upper bound} of \(E\) by the following rule.
\begin{enumerate}
    \item If \(E\) is \emph{contained} in \(\SET{R}\) (i.e., \(+\infty\) and \(-\infty\) are \emph{not} elements of \(E\)), then we let \(\sup(E)\) be as defined in \DEF{5.5.10}.
    \item If \(E\) contains \(+\infty\), then we set \(\sup(E) := +\infty\).
    \item If \(E\) does not contain \(+\infty\) but does contain \(-\infty\), then we set \(\sup(E) := \sup(E \setminus \{-\infty\})\) (which is a subset of \(\SET{R}\) and thus falls under case (a)).
\end{enumerate}
We also define the \emph{infimum} \(\inf(E)\) of \(E\) (also known as the \emph{greatest lower bound} of \(E\)) by the formula \(\inf(E) := -\sup(-E)\)
where \(-E\) is the set \(-E := \{ -x : x \in E \}\).
\end{definition}

\begin{note}
Just a reminder, although \DEF{6.2.6}(a) use \DEF{5.5.10}, from that definition if \(E\) (is empty and) has no upper bound, then \(\sup(E)\) is again \(+\infty\).
\end{note}

\begin{note}
Note that ``\(-x\)'' in the last sentence used negation of \emph{extended} real numbers, i.e. \DEF{6.2.2}.
\end{note}

\begin{note}
Note that \DEF{6.2.6} is \emph{well-defined}, i.e. the supremum is unique.
For case (a), it fallbacks to \DEF{5.5.10}, which is well defined.
For case (b), \(\sup(E) := +\infty\), which is also unique.
For case (c), it again fallbacks to case (a) so is well-defined.
So it is impossible that \(\sup(E) < \sup(E)\); this fact is used in the proof of \EXAMPLE{6.2.10}.
\end{note}

\begin{example} \label{example 6.2.7}
Let \(E\) be the negative integers, \emph{together with} \(-\infty\):
\[
    E = \{-1, -2, -3, -4,...\} \cup \{-\infty\}.
\]
Then by \DEF{6.2.6}(c), \(\sup(E) = \sup(E \setminus \{-\infty\}) = -1\), while
\begin{align*}
    \inf(E) & = -\sup(-E) & \text{by \DEF{6.2.6}} \\
           & = -(+\infty) & \text{by \DEF{6.2.6}(b)} \\
           & = -\infty & \text{by \DEF{6.2.2}}
\end{align*}
\end{example}

\begin{note}
Although currently we are discussing the \emph{extended} real numbers, the notation \(\{1, 2, 3, ...\}\) still does \emph{not} contain the \(+\infty\) as member.
\end{note}

\begin{example} \label{example 6.2.8}
The set \(\{0.9, 0.99, 0.999, 0.9999,...\}\) has infimum \(0.9\) and supremum \(1\).
(Indeed this case fallbacks to \DEF{5.5.10}, with the set non-empty and having upper bound.)
Note that in this case the supremum does \emph{not} actually
belong to the set, but it is in some sense ``touching the set'' from the right.
\end{example}

\begin{example} \label{example 6.2.9}
The set \(\{1, 2, 3, 4, 5 ...\}\) has infimum \(1\) and supremum \(+\infty\). (Indeed this case fallbacks to \DEF{5.5.10}, with the set non-empty and having \emph{no} upper bound.)
But again, \(+\infty\) does \emph{not} belong to the set.
\end{example}

\begin{example} \label{example 6.2.10}
Let \(E\) be the empty set.
Then \(\sup(E) = -\infty\) and \(\inf(E) = +\infty\) (why?).
This is the \emph{only case} in which the supremum can be \emph{less than} the infimum (why?).
\end{example}

\begin{proof}
Since \(E\) is contained in \(\SET{R}\), by \DEF{6.2.6}, \(\sup(E)\) fallbacks to \DEF{5.5.10}.
By \DEF{5.5.10}, since \(E\) is empty, in that definition we have \(\sup(E) = -\infty\).
Now, by \DEF{6.2.6}, it's clear that \(-E := \{ -x : x \in E \}\) is also empty, so again by \DEF{5.5.10}, \(\sup(-E) = -\infty\).
And by \DEF{6.2.2}, \(-\sup(-E) = +\infty\), that is, by \DEF{6.2.6}, \(\inf(E) = +\infty\).

Now we show that the only case in which the supremum can be less than the infimum is when \(E = \emptyset\).
Suppose for the sake of contradiction that there exists an non-empty set \(E'\) s.t. \(\sup(E') < \inf(E')\) \MAROON{(1)}.

Since \(E'\) is non-empty, (by \LEM{3.1.6}) we can find element \(x \in E'\).
And by \THM{6.2.11}, we have \(x \le \sup(E')\) \MAROON{(2)} and \(\inf(E') \le x\) \MAROON{(3)}.
Then by \MAROON{(1)(2)(3)} we have \(\sup(E') < \inf(E') \le x \le \sup(E')\), in particular \(\sup(E') < \sup(E')\), which is impossible(see the third note of \DEF{6.2.6}).
\end{proof}

\begin{note}
The proof above used \THM{6.2.11}, but it's ok because the theorem does not depend on this example.
\end{note}

\begin{note}
One can intuitively think of the supremum of \(E\) as follows.
Imagine the real line with \(+\infty\) ``somehow'' on the far right, and \(-\infty\) on the far left.
Imagine a \emph{piston} at \(+\infty\) moving leftward until it is stopped by the presence of a set \(E\);
\emph{the location where it stops is the supremum of \(E\)}.
Similarly if one imagines a piston at \(-\infty\) moving rightward until it is stopped by the presence of \(E\), the location where it stops is the infimum of \(E\).
\emph{In the case when \(E\) is the empty set}, \textbf{the pistons pass through each other}, the supremum landing at \(-\infty\) and the infimum landing at \(+\infty\).
\end{note}

\begin{note}
我其實不知道作者為什麼用\ ``piston'' 來形容... 感覺應該比較像``子彈''才對...
\end{note}

The following \emph{theorem} justifies the terminology \sout{``least upper bound''} ``supremum'' and \sout{``greatest lower bound''} ``infimum'':

\begin{note}
In my opinion the theorem ``justifies'' sup and inf, not LUB and GLB.
From \DEF{6.2.6}, sup/inf are always defined, but LUB/GLB may not exist.
\end{note}

\begin{theorem} \label{thm 6.2.11}.
Let \(E\) be a subset of \(\SET{R}^*\).
Then the following statements are true.
\begin{enumerate}
    \item For every \(x \in E\) we have \(x \le \sup(E)\) and \(x \ge \inf(E)\).
    \item Suppose that \(M \in \SET{R}^*\) is an upper bound for \(E\), i.e., \(x \le M\) for all \(x \in E\).
          Then we have \(\sup(E) \le M\).
    \item Suppose that \(M \in \SET{R}^*\) is a lower bound for \(E\), i.e., \(x \ge M\) for all \(x \in E\).
          Then we have \(\inf(E) \ge M\).
\end{enumerate}
\end{theorem}

\begin{note}
會需要證明這個定理,i.e. 要確定任意(廣義)實數子集合內所有\ element 小於等於該集合的\ supremum 的原因,我認為是這本書的 ``最小上界''跟\ ``supremum'' 不是等價名詞(去看\ \DEF{5.5.10} 還有 \DEF{6.2.6});最小上界不一定會存在,但是\ supremum 一定存在(可能是\ \(+\infty\) 或 \(-\infty\))。
可能要去看看其他分析書是怎麼區分這兩者的。
\end{note}

\begin{proof}
\begin{enumerate}
\item
    We first show \(\forall x \in E, x \le \sup(E)\) \BLUE{(1)}.
    There are three cases:
    \begin{itemize}
    \item \(E \subseteq \SET{R}\):
        We further split into two cases: \(E\) is empty or not.
        If \(E\) is empty, then \BLUE{(1)} is vacuously true.
        If \(E\) is non-empty \emph{and has an upper bound}, then by \DEF{5.5.10}, \(\sup(E)\) is \emph{the} least upper bound of \(E\), i.e. an upper bound of \(E\), so by definition of upper bound, \BLUE{(1)} is true.
        If \(E\) is non-empty \emph{but has no upper bound}, then again by \DEF{5.5.10}, \(\sup(E) = +\infty\). And by \DEF{6.2.3}(b), \BLUE{(1)} is true.
    \item \(E\) contains \(+\infty\):
        Then by \DEF{6.2.6}, \(\sup(E) = +\infty\).
        And by \DEF{6.2.3}(b), \BLUE{(1)} is true.
    \item \(E\) does not contain \(+\infty\) but contains \(-\infty\):
        Then we consider the value of \(x\): \(x = -\infty\) or \(x\) is a real number.
        If \(x = -\infty\), then by \DEF{6.2.3}(c), \(x\) is less than any extended real number.
        In particular, \(x \le \sup(E)\).
        Now suppose \(x\) is a real number.
        We first consider \(\sup(E)\).
        By \DEF{6.2.6}(c), \(\sup(E) = \sup(E \setminus \{-\infty\})\), so we only need to show \(x \le \sup(E \setminus \{-\infty\})\).

        Clearly \(x \in E \setminus \{-\infty\}\) because \(x\) is a real number that belongs to \(E\) \MAROON{(1)}.
        Again we have two cases: \(E \setminus \{-\infty\}\) has an upper bound or not.
        If \(E \setminus \{-\infty\}\) has an upper bound, then by \DEF{5.5.10} \(\sup(E \setminus \{-\infty\})\) is the least upper bound, i.e. an upper bound.
        And by definition of upper bound and \MAROON{(1)}, \(x \le \sup(E \setminus \{-\infty\})\).
        If \(E \setminus \{-\infty\}\) has no upper bound, then again by \DEF{5.5.10}, \(\sup(E \setminus \{-\infty\}) = +\infty\), and by \DEF{6.2.3}(b), \(x \le \sup(E \setminus \{-\infty\})\).
        So in all cases, \(x \le \sup(E \setminus \{-\infty\})\), as desired.
    \end{itemize}
    Now we show \(\forall x \in E, x \ge \inf(E)\) \BLUE{(2)}.
    But given any \(x \in E\), by \DEF{6.2.6} we know that \(-x \in -E\), and by the previous case, \(-x \le \sup(-E)\).
    By \PROP{6.2.5}(d) (negation reverses order), \(x \ge -(\sup(-E))\), that is, by \DEF{6.2.6}, \(x \ge \inf(E)\).
\item
    Again we have Three cases:
    \begin{itemize}
    \item \(E \subseteq \SET{R}\):
        There are three cases.

        If \(E\) is empty, then by \DEF{5.5.10}, \(\sup(E) = -\infty\), and by \DEF{6.2.3}(c), \(\sup(E) \le M\).
        
        If \(E\) is non-empty and has an upper bound, then by \DEF{5.5.10} \(\sup(E)\) is the least upper bound of \(E\), so for any upper bound \(M\) of \(E\) we have \(\sup(E) \le M\).
        
        If \(E\) is non-empty and has no upper bound, then by \DEF{5.5.10}, \(\sup(E) = +\infty\), and \(M\) must equal to \(+\infty\) (otherwise trivially we have contradiction that \(M\) is not an upper bound of \(E\).
        So by \DEF{6.2.3}(b), we have \(\sup(E) \le M\).
    \item \(E\) contains \(+\infty\):
        Then we must have \(+\infty \le M\) since \(M\) is an upper bound of \(E\) (otherwise if \(+\infty > M\) we can find element \(+\infty \in E\) s.t. \(M < +\infty\), contradicting \(M\) is an upper bound of \(E\)).
        And by \DEF{6.2.6}(b), \(\sup(E) = +\infty\), so together we have \(\sup(E) = +\infty \le M\).
    \item \(E\) does not contain \(+\infty\) but contains \(-\infty\):
        Then by \DEF{6.2.6}(c), \(\sup(E) = \sup(E \setminus \{-\infty\})\).
        So we only have to show \(\sup(E \setminus \{-\infty\}) \le M\).
        Now we claim that \(M\) is \emph{also} an upper bound of \(E \setminus \{-\infty\})\), since \(E \setminus \{-\infty\} \subseteq E\).
        Also, since \(E \setminus \{-\infty\}\) is contained in \(\SET{R}\), by the first case and the upper bound \(M\) of this set, we have \(\sup(E \setminus \{-\infty\}) \le M\), as desired.
    \end{itemize}
\item
    Let arbitrary \(y \in -E\), then \(y = -x\) for some \(x \in E\).
    Since \(x \in E\) and \(M\) is a lower bound of \(E\), we have \(M \le x\), which by \PROP{6.2.5}(d) implies \(-M \ge -x\), or \(-M \ge y\).
    Since \(y\) is arbitrary, that implies for all \(y \in -E\), \(-M \ge y\), so by definition, \(-M\) is an upper bound of \(-E\).
    Then by part(b), we have \(\sup(-E) \le -M\), which by \PROP{6.2.5}(d) implies \(-\sup(-E) \ge M\), or by \DEF{6.2.6}, \(\inf(E) \ge M\), as desired.
\end{enumerate}
\end{proof}

\exercisesection

\begin{exercise} \label{exercise 6.2.1}
Prove \PROP{6.2.5}.
(Hint: you may need \PROP{5.4.7}.)
\end{exercise}

\begin{proof}
See \PROP{6.2.5}.
\end{proof}

\begin{exercise} \label{exercise 6.2.2}
Prove \THM{6.2.11}.
(Hint: you may need to break into cases depending on whether \(+\infty\) or \(-\infty\) belongs to \(E\).
You can of course use \DEF{5.5.10}, provided that \emph{\(E\) consists only of real numbers}.)
\end{exercise}

\begin{proof}
See \THM{6.2.11}.
\end{proof}