\section{The least upper bound property} \label{sec 5.5}

We now give one of the most basic advantages of the real numbers over the rationals;
one can take the \emph{least upper bound} \(\sup(E)\) of any subset \(E\) of the real numbers \(\SET{R}\).

\begin{note}
開天眼,我覺得上一段話有點嘴,又不是每個實數子集合都有最小上界(\THM{5.5.9})。
\end{note}

\begin{definition} [Upper bound] \label{def 5.5.1}  Let \(E\) be a subset of \(\SET{R}\), and let \(M\) be a real number.
We say that \(M\) is ``an'' \emph{upper bound} for E, iff we have \(x \le M\) for every element \(x\) in \(E\).
\end{definition}

\begin{example}
Simple example.
\end{example}

\begin{example} \label{example 5.5.3}
Let \(\SET{R}^+\) be the set of positive reals:
\(\SET{R}^+ := \{x \in \SET{R} : x > 0\} \).
Then \(\SET{R}^+\) does \emph{not} have any upper bounds at all.
\end{example}

\begin{proof}
Suppose we have an upper bound \(M\) for the set.
Then by definition for all \(x \in \SET{R}^+, x \le M\), and that implies, \(M > 0\) and \(M + 1 > 0\).
So by definition of \(\SET{R}^+\), \(M + 1 \in \SET{R}^+\), so we have found a real number \(M + 1 \in \SET{R}^+\) and \(M < M + 1\), which contradicts that \(M\) is an upper bound for \(\SET{R}^+\).
\end{proof}

\begin{example} \label{example 5.5.4}
Let \(\emptyset\) be the empty set.
Then \emph{every} number \(M\) is an upper bound for \(\emptyset\), because \(M\) is greater than ``every element'' of the ``empty'' set
(this is a vacuously true statement, but still true).
\end{example}

\begin{note}
It is clear that if \(M\) is an upper bound of \(E\), then any larger number \(M' \geq M\) is also an upper bound of \(E\).
On the other hand, it is \emph{not so clear} whether it is also possible for any number \emph{smaller} than \(M\) to also be an upper bound of \(E\).
\end{note}

\begin{definition} [Least upper bound] \label{definition 5.5.5}
Let \(E\) be a subset of \(\SET{R}\), and \(M\) be a real number.
We say that \(M\) is ``a'' least upper bound for \(E\) iff
\begin{enumerate}
    \item \(M\) is an upper bound for \(E\), and also
    \item any other upper bound \(M'\) for \(E\) must be larger than or equal to \(M\).
\end{enumerate}
\end{definition}

\begin{note}
We will prove the least upper bound is \emph{unique}(if exist) (\PROP{5.5.8}).
\end{note}

\begin{example} \label{example 5.5.6}
Let \(E\) be the interval \(E := \{x \in \SET{R} : 0 \le x \le 1 \} \).
Then, as noted before, \(E\) has many upper bounds, indeed every number greater than or equal to \(1\) is an upper bound.
But only \(1\) is \emph{the} least upper bound;
all other upper bounds are larger than \(1\).
\end{example}

\begin{example} \label{example 5.5.7}
The empty set does not have a least upper bound.
\end{example}

\begin{proof}
Suppose not.
Then we have a least upper bound \(M\) for the empty set.
And since \(M\) is real, \(M - 1\) is real, by \EXAMPLE{5.5.4}, \(M - 1\) is also an upper bound of the empty set.
But \(M - 1 < M\), which contradicts that \(M\) is a \emph{least} upper bound for the empty set.
\end{proof}

\begin{proposition} [Uniqueness of least upper bound] \label{prop 5.5.8}
Let \(E\) be a subset of \(R\).
Then \(E\) can have \emph{at most one} least upper bound.
\end{proposition}

\begin{proof}
Let \(M_1\) and \(M_2\) be two \emph{least} upper bounds.
Since \(M_1\) is a \emph{least} upper bound and \(M_2\) is (also just) an upper bound, then by definition of least upper bound we have \(M_2 \ge M_1\).
Since \(M_2\) is a \emph{least} upper bound and \(M_1\) is (also just) an upper bound, we similarly have \(M_1 \ge M_2\).
Thus \(M_1 = M_2\).
Thus there is at most one least upper bound.
\end{proof}

Now we come to an important(worth calling ``theorem'') property of the real numbers:

\begin{theorem} [Existence of least upper bound] \label{thm 5.5.9}
Let \(E\) be a nonempty subset of \(\SET{R}\).
\emph{If} \(E\) has an upper bound, (i.e., \(E\) has some upper bound \(M\)), \emph{then it must have exactly one \emph{least} upper bound}.
\end{theorem}

\begin{proof}
This theorem will take quite a bit of effort to prove, and many of the steps will be left as exercises(\EXEC{5.5.2}, \EXEC{5.5.3}, \EXEC{5.5.4}).

Let \(E\) be a non-empty subset of \(R\) with an upper bound \(M\).
By \PROP{5.5.8}, we know that \(E\) has at most one least upper bound;
we have to show that \(E\) \emph{has at least one} least upper bound.

Since \(E\) is non-empty, we can choose some element \(x_0\) in \(E\).
Let \(n\) be arbitrary positive integer(i.e. \(n \ge 1\)).
We know that \(E\) has an upper bound \(M\).
Since \(n\) is an integer, so trivially \(1/n\) is rational.
By the Archimedean property (\CORO{5.4.13}), we can find an integer \(K\) such that \(K \X (1/n) = K/n > M\), and hence \(K/n\) is also an upper bound for \(E\) \MAROON{(1)}.
Now we consider the value of \(x_0\):
\begin{itemize}
    \item [>>] \(x_0 \ge 0\):
        Then let \(L = -1\), so \(L/n\) is negative, so \(L/n < 0\).
    \item [>>] \(x_0 < 0\):
        Then \(-x_0 > 0\), so again by the Archimedean property (\CORO{5.4.13}), we can find an integer \(L'\) s.t. \(L' \X 1/n > -x_0\), or \(-L' \X 1/n < x_0\).
        So let \(L = -L'\), then we have \(L/n < x_0\).
\end{itemize}
So in all cases we can find \(L\) s.t. \(L/n < x_0\).
That is, we can find an element \(x_0 \in E\) s.t. \(x_0 < L/n\), so \(L/n\) is \emph{not} an upper bound for \(E\).
But by \MAROON{(1)}, \(K/n\) is an upper bound for \(E\), so \(K/n \ge L/n\), so \(K \ge L\).

Since \(K/n\) is an upper bound for \(E\) and \(L/n\) is not, we can find an integer \(L < m_n \le K\) with the property that \(m_n/n\) \emph{is} an upper bound for \(E\), but \((m_n - 1)/n\) \emph{is not} (see \EXEC{5.5.2}) \MAROON{(2)}.
In fact, this integer \(m_n\) is unique (\EXEC{5.5.3}).
We \emph{subscript} \(m_n\) by \(n\) to emphasize the fact that \emph{this integer \(m\) depends on the choice of \(n\).}
So for any integer \(n \ge 1\), the \emph{mapping} \(m_n\) is unique.
So by definition of sequence (\DEF{5.1.1}), this gives a well-defined (and unique) sequence \(m_1, m_2, m_3,...\) of integers, and by \MAROON{(2)},
with each of the \(m_n/n\) being upper bounds(of \(E\)) and each of the \((m_n-1)/n\) not being upper bounds(of \(E\)).

(BTW, the choice of \(K\) and \(L\) are also dependent on \(n\), but we don't use them later, so we do not add subscript on them.)

Now let \(N \ge 1\) be a positive integer, and let \(n, n' \ge N\) be integers.
So of course \(n, n' \ge 1\), so in particular for \(n\), \(m_n/n\) is an upper bound for \(E\), and for \(n'\), \((m_{n'} - 1)/n'\) is not, so we must have \(m_n/n > (m_{n'} - 1)/n'\)
\begin{itemize}
    \item
        why?
        Because \((m_{n'} - 1)/n'\) is not an upper bound, we can find another real number \(a \in E\) s.t. \(a > (m_{n'} - 1)/n'\).
        But \(m_n/n\) is an upper bound for \(E\), so by definition \(m_n/n \ge a\), so we have \(m_n/n \ge a > (m_{n'} - 1)/n'\), so \(m_n/n > (m_{n'} - 1)/n'\).
\end{itemize}
And from that we have:
\begin{align*}
             & \frac{m_n}{n} > \frac{m_{n'} - 1}{n'} \\
    \implies & \frac{m_n}{n} - \frac{m_{n'}}{n'} > \frac{-1}{n'} \\
    \implies & \frac{m_n}{n} - \frac{m_{n'}}{n'} > \frac{-1}{n'} \ge \frac{-1}{N} \MAROON{\ (3)} & \text{since \(n' \ge N\), we have \(1/n' \le 1/N\), and \(-1/n' \ge -1/N\)}
\end{align*}
Similarly, since \(m_{n'}/n'\) is an upper bound for \(E\) and \((m_n - 1)/n\)) is not, we have \(m_{n'}/n' > (m_n - 1)/n\), and,
\begin{align*}
             & \frac{m_{n'}}{n'} > \frac{m_n - 1}{n} \\
    \implies & \frac{m_{n'}}{n'} - \frac{m_n}{n} > \frac{-1}{n} \\
    \implies & -(\frac{m_{n'}}{n'} - \frac{m_n}{n}) < -(\frac{-1}{n}) \\
    \implies & \frac{m_n}{n} - \frac{m_{n'}}{n'} < \frac{1}{n} \\
    \implies & \frac{m_n}{n} - \frac{m_{n'}}{n'} < \frac{1}{n} \le \frac{1}{N} \MAROON{\ (4)} & \text{since \(n \ge N\), we have \(1/n \le 1/N\)}
\end{align*}
So with \MAROON{(3) (4)} and \PROP{4.3.3}(c)(note that all operand are rationals, so we just use property for rationals)we have
\[
    \abs{\frac{m_n}{n} - \frac{m_{n'}}{n'}} \le \frac{1}{N} \text{ for all } n, n' \ge N \ge 1
\]
This implies that \((m_n/n)_{n = 1}^{\infty}\) is a Cauchy sequence (\EXEC{5.5.4}).
So since \((m_n/n)_{n = 1}^{\infty}\) is Cauchy, we can take the formal limit of it and let
\[
    S := \LIM_{n \toINF}\frac{m_n}{n}.
\]
From \EXEC{5.3.5} we also have \(\LIM_{n \toINF}1/n = 0\), So we can derive that
\begin{align*}
    S & = \LIM_{n \toINF}\frac{m_n}{n} \\
      & = \LIM_{n \toINF}\frac{m_n}{n} - 0 \\
      & = \LIM_{n \toINF}\frac{m_n}{n} - \LIM_{n \toINF}\frac{1}{n} \\
      & = \LIM_{n \toINF}\frac{m_n - 1}{n}
\end{align*}

To finish the proof of the theorem, we need to show that \(S\) is the least upper bound for \(E\).
First we show that it is an upper bound.
Let \(x\) be arbitrary element of \(E\).
Then, \emph{for all} \(n \ge 1\), since \(m_n/n\) is an upper bound for \(E\), we have \(x \le m_n/n\) \emph{for all} \(n \ge 1\).
Applying \EXEC{5.4.8}, we conclude that \(x \le \LIM_{n \toINF} m_n/n = S\).
Thus \(S\) is indeed an upper bound for \(E\).

Now we show it is a \emph{least} upper bound.
Suppose \(y\) is an arbitrary upper bound for \(E\).
Since \emph{for all} \(n \ge 1\), \((m_n - 1)/n\) is \emph{not} an upper bound, we conclude that \(y \ge (m_n - 1)/n\) for all \(n \ge 1\).
Applying \EXEC{5.4.8}, we conclude that \(y \ge \LIM_{n \toINF}(m_n - 1)/n = S\).
Thus the upper bound \(S\) is less than or equal to \emph{every} upper bound of \(E\), and \(S\) is thus a least upper bound of \(E\).
\end{proof}

\begin{definition}[Supremum]\label{5.5.10}
Let \(E\) be a subset of the real numbers.
If \(E\) is non-empty and has some upper bound, we define \(\sup(E)\) to be \emph{the} least upper bound of \(E\)
(this is well-defined by \THM{5.5.9}).
We introduce two additional symbols, \(+\infty\) and \(-\infty\).
If \(E\) is non-empty and \emph{has no} upper bound, we set \(\sup(E) := +\infty\);
if \(E\) is empty, we set \(\sup(E) := -\infty\).
We refer to \(\sup(E)\) as the \emph{supremum} of \(E\), and also denote it by \(\sup E\).
\end{definition}

\begin{remark} \label{remark 5.5.11}
At present, \(+\infty\) and \(-\infty\) are meaningless symbols; 
we have no operations on them at present, and none of our results involving \emph{real numbers} apply to \(+\infty\) and \(-\infty\), because these are \emph{not} real numbers.
In \SEC{6.2} we add \(+\infty\) and \(-\infty\) to the reals to form the \emph{extended real number system}, but this system is not as convenient to work with as the real number system, because many of the laws of algebra break down.
For instance, it is not a good idea to try to define \(+\infty + -\infty\); setting this equal to 0 causes some problems.
\end{remark}

Now we give an example of how the least upper bound property is useful.

\begin{proposition} \label{prop 5.5.12}
There exists a positive \emph{real} number \(x\) such that \(x^2 = 2\).
\end{proposition}

\begin{remark} \label{remark 5.5.13}
Comparing \PROP{5.5.12} with \PROP{4.4.4}, we see that certain numbers are real but not rational.
The proof of this proposition also shows that the rationals \(\SET{Q}\) do not obey the least upper bound property, \emph{otherwise} one could use that property to construct a (\emph{rational} that is) square root of \(2\), which by \PROP{4.4.4} is not possible.
\end{remark}

\begin{proof}
Let \(E\) be the set \( \{ y \in \SET{R} : y \ge 0 \text{ and } y^2 < 2 \} \);
thus \(E\) is the set of all non-negative real numbers whose square is less than \(2\).
Observe that \(E\) has an upper bound of \(2\) (because if \(y > 2\), then \(y^2 > 4 > 2\) and hence \(y \notin E\)).
Also, \(E\) is non-empty (for instance, \(1\) is an element of \(E\)).
Thus by the least upper bound property \THM{5.5.9}, we have a real number \(x := \sup(E)\) which is the least upper bound of \(E\) \MAROON{(1)}.
Then \(x\) is greater than or equal to \(1\)(since \(1 \in E\)) and less than or equal to \(2\) (since \(2\) is an upper bound for E).
So \(x\) is positive.
Now we show that \(x^2 = 2\).

We argue this by contradiction.
We show that both \(x^2 < 2\) and \(x^2 > 2\) lead to contradictions.
First suppose that \(x^2 < 2\). Let \(0 < \varE < 1\) be a small number;
then we have
\begin{align*}
    (x + \varE)^2 & = x^2 + 2\varE x + \varE^2 & \text{by algebra} \\
                  & \le x^2 + 4\varE + \varE^2 & \text{since \(x \le 2\), an upper bound of \(E\)} \\
                  & \le x^2 + 4\varE + \varE & \text{since \(0 < \varE < 1\), so \(\varE^2 < \varE\) }\\
                  & = x^2 + 5\varE \MAROON{(2)}
\end{align*}
Now from \(x^2 < 2\), we have
\begin{align*}
             & x^2 < 2 \\
    \implies & 0 < 2 - x^2 \\
    \implies & 0/5 < (2 - x^2)/5 \\
    \implies & 0 < (2 - x^2)/5
\end{align*}
And by \PROP{5.4.14} we can find a rational \(\varE\) s.t. \(0 < \varE < (2 - x^2)/5\), and
\begin{align*}
             & 0 < \varE < (2 - x^2)/5 \\
    \implies & 0 < 5\varE < (2 - x^2) \\
    \implies & x^2 + 0 < x^2 + 5\varE < 2 \\
    \implies & x^2 < x^2 + 5\varE < 2 \MAROON{(3)}
\end{align*}
So we can choose \(\varE\) s.t. \MAROON{(3)} is true.
And with the derivation in \MAROON{(2)} we have \((x + \varE)^2 < 2\).
By construction(definition) of \(E\), this means \(x + \varE \in E\); but this contradicts \MAROON{(1)} that \(x\) is an upper bound of \(E\) since \(x < x + \varE\).

Now suppose that \(x^2 > 2\).
Let \(0 < \varE < 1\) be a small number;
then we have
\begin{align*}
    (x - \varE)^2 & = x^2 - 2\varE x + \varE^2 & \text{by algebra} \\
                  & \ge x^2 - 2\varE x & \text{of course} \\
                  & \ge x^2 - 4\varE \MAROON{(4)} & \text{since \(x \le 2\), an upper bound of \(E\)}
\end{align*}
Again, since \(x^2 > 2\), we can choose \(0 < \varE < 1\) such that by \PROP{5.4.14} \(x^2 > 2 + 4\varE > 2\), which implies \(x^2 - 4\varE > 2\), and thus by the derivation in \MAROON{(4)} we have \((x - \varE)^2 > 2\) \MAROON{(5)}.
But then this implies that \(x - \varE \ge y\) for all \(y \in E\). (Why? If \(x - \varE < y\) then \((x - \varE)^2 < y^2 \le 2\) by construction of \(E\), so \((x - \varE)^2 < 2\), contradicting \MAROON{(5)}.)
Thus \(x - \varE\) is an upper bound for \(E\), which contradicts the fact that \(x\) is the \emph{least} upper bound of \(E\).

From these two contradictions we see that \(x^2 = 2\), as desired.
\end{proof}

\begin{note}
這裡有一個很\ mind=blown 的論述: 若要證明一個實數\ \(r\) 是某集合的\ upper bound,你可以證明所有比\ \(r\) 大的實數都不在這個集合裡。
\end{note}

\begin{remark} \label{remark 5.5.14}
In \CH{6} we will use the least upper bound property to develop the theory of limits, which allows us to do many more
things than just take square roots.
\end{remark}

\begin{remark} \label{remark 5.5.15}
We can of course talk about lower bounds, and \emph{greatest lower bounds}, of sets \(E\);
the greatest lower bound of a set \(E\) is also known as the \emph{infimum} of \(E\) and is denoted \(\inf(E)\) or \(\inf E\).
Everything we say about suprema has a counterpart for infima;
we will usually leave such statements to the reader.
A precise relationship between the two notions is given by \EXEC{5.5.1}.
See also \SEC{6.2}.
\end{remark}

\begin{note}
Supremum means ``highest'' and infimum means ``lowest'', and the plurals are suprema and infima.
Supremum is to superior, and infimum to inferior, as maximum is to major, and minimum to minor.
The root words are ``super'', which means ``above'', and ``infer'', which means ``below''
(this usage only survives in a few rare English words such as ``infernal'', with the Latin prefix ``sub'' having mostly replaced ``infer'' in English).
\end{note}

\exercisesection

\begin{exercise} \label{exercise 5.5.1}
Let \(E\) be a subset of the real numbers \(\SET{R}\), and suppose that \(E\) has a least upper bound \(M\) which is a real number, i.e., \(M = \sup(E)\).
Let \(-E\) be the set \(-E := \{-x : x \in E\} \).
Show that \(-M\) is the greatest lower bound of \(-E\), i.e., \(-M = \inf(-E)\).
\end{exercise}

\begin{proof}
We first show \(-M\) is a lower bound of \(-E\).
Let arbitrary \(x' \in -E\).
By construction of \(-E\), \(x' = -x\) for some \(x \in E\).
So we have \(x \le M\) \MAROON{(1)} since \(M\) is a least upper bound of \(E\).
And \MAROON{(1)} implies \(-M \le -x\), that is, \(-M \le x'\).
So for all \(x' \in -E\), \(-M \le x'\), so \(-M\) is a lower bound of \(-E\).

Now we show \(-M\) is a greatest lower bound of \(-E\).
Suppose not, then there exists another lower bound \(-M'\) of \(-E\) s.t. \(-M' > -M\), which implies \(M' < M\).
But now given arbitrary \(x \in E\), since \(-x \in -E\), and we have \(-M'\) as a lower bound, we can conclude that \(-x \ge -M'\), which implies \(x \le M'\).
So for all \(x \in E\), \(x \le M'\), so \(M'\) is also an upper bound of \(E\);
but \(M' < M\), which contradicts that \(M\) is a least upper bound of \(E\).

It is similar as \PROP{5.5.8} to show the uniqueness of ``the'' greatest lower bound.
\end{proof}

\begin{exercise} \label{exercise 5.5.2}
Let \(E\) be a non-empty subset of \(\SET{R}\), let \(n \ge 1\) be an integer, and let \(L < K\) be integers.
Suppose that \(K/n\) is an upper bound for E, but that \(L/n\) is \emph{not} an upper bound for \(E\).
Without using \THM{5.5.9}(\emph{otherwise circular}), show that there exists an integer \(L < m \le K\) such that \(m/n\) is an upper bound for \(E\), but that \((m - 1)/n\) is not an upper bound for \(E\).
(Hint: prove by contradiction, and use induction.
It may also help to draw a picture of the situation.)
\end{exercise}

\begin{proof}
\href{https://taoanalysis.wordpress.com/2020/05/14/exercise-5-5-2/}{Reference}

The statement of the exercise:
\begin{align*}
    \forall\ n \in \SET{N} \text{ s.t. } \ge 1 & \\
             & (\exists L, K \in \SET{Z} \text{ s.t. } L < K \\
             &  \land K/n \text{ is an upper bound of } E \\
             &  \land L/n \text{ is not an upper bound of } E) \\
    \implies & \\
             & (\exists m \in \SET{Z} \text{ s.t. } L < m \le K \\
             &  \land\ m/n \text{ is an upper bound of } E \\
             &  \land\ (m - 1)/n \text{ is not an upper bound of } E).
\end{align*}

So for the sake of contradiction \BLUE{(1)}: Suppose there exists an integer \(n \ge 1\) s.t. there exist integers \(L < K\) s.t. \(K/n\) is an upper bound of \(E\) but \(L/n\) is not, \emph{but, for the sake of contradiction}, there is no integer \(m\) such that \(L < m \le K\), and \(m/n\) is an upper bound of \(E\) and \((m - 1)/n\) is not an upper bound of \(E\)

We will show by induction on \(j\) that \BLUE{(2)} \((K - j)/n\) is an upper bound for \(E\) for all natural numbers \(j\).

For base case \(j = 0\), \((K - j)/n = (K - 0)/n = K/n\), which is an upper bound of \(E\) by supposition, so the base case is true.

Now suppose inductively for some \(j \ge 0\) we have \((K - j)/n\) is an upper bound for \(E\).
We have to show \((K - (j + 1))/n\) is also an upper bound for \(E\) \MAROON{(1)}.
Now by supposition, \(L/n\) is \emph{not} an upper bound of \(E\), so we can find \(x \in E\) s.t. \(L/n < x\).
But since by inductive hypothesis \((K - j)/n\) is an upper bound for \(E\), \(x \le (K - j)/n\).
So together we have \(L/n < x \le (K - j)/n\), so \(L/n < (K - j)/n\), so \(L < K - j\) \MAROON{(2)}.
And since \(j\) is a natural number, i.e. \(j \ge 0\), so \(K - j \le K - 0 = K\), so can further derive that \(L < K - j \le K\).
Suppose for the sake of contradiction that \MAROON{(1)} is false, i.e. \((K - (j + 1))/n\) is \emph{not} an upper bound of \(E\).
Then currently we have
\begin{enumerate}
    \item[(1)] (by inductive hypothesis) \((K - j)/n\) is an upper bound for \(E\)
    \item[(2)] (suppose for contradiction) \((K - (j + 1))/n\) is not an upper bound for \(E\) \MAROON{(3)}.
\end{enumerate}
Now let \(m := K - j\), then with \MAROON{(2)} we have \(L < m\),
and replacing \(m\) into above, we have \(m/n\) is an upper bound for \(E\) and \((m - 1)/n\) is not an upper bound for \(E\), which contradicts the very first supposition \BLUE{(1)}.
So \((K - (j + 1))/n\) must also be an upper bound for \(E\).
This closes the induction, so \BLUE{(2)} is true.
In particular, since integers \(L < K\), \(K - L > 0\) i.e. \(K - L\) is a natural number;
so from \BLUE{(2)} we have \((K - (K - L))/n\) is an upper bound for \(E\), that is, \(L/n\) is an upper bound for \(E\), which is a contradiction!
So the very first supposition \BLUE{(1)} is false, so the statement of the exercise is true.
\end{proof}

\begin{exercise} \label{exercise 5.5.3}
Let \(E\) be a non-empty subset of \(R\), let \(n \ge 1\) be an integer, and let \(m, m'\) be integers with the properties that \(m/n\) and \(m'/n\) are upper bounds for \(E\), but \((m - 1)/n\) and \((m' - 1)/n\) are \emph{not} upper bounds for \(E\).
Show that \(m = m'\).
This shows that the integer \(m\) constructed in \EXEC{5.5.2} is \emph{unique}.
(Hint: again, drawing a picture will be helpful.)
\end{exercise}

\begin{proof}
Since \(\frac{m' - 1}{n}\) is \emph{not} an upper bound for \(E\), there exists \(x \in E\) s.t. \(\frac{m' - 1}{n} < x\);
and since \(\frac{m}{n}\) is an upper bound for \(E\), \(x \le \frac{m}{n}\), so together we have \(\frac{m' - 1}{n} < x \le \frac{m}{n}\), so \(\frac{m' - 1}{n} < \frac{m}{n}\) \MAROON{(1)}.
Similarly since \(\frac{m - 1}{n}\) is \emph{not} an upper bound for \(E\) and \(\frac{m'}{n}\) is an upper bound for \(E\), we have \(\frac{m - 1}{n} < \frac{m'}{n}\) \MAROON{(2)}.
And since \(n \ge 1\), by multiplying \(n\) on both sides of \(\MAROON{(1) (2)}\) we have \(m' - 1 < m\) and \(m - 1 < m'\) \MAROON{(3)}.
But since \(m, m'\) are integer, by similar property of \PROP{2.2.12}(e)(this is for natural numbers, but the property can be proved for integers), from \MAROON{(3)} we have \(m' \le m\) and \(m \le m'\), which implies \(m' = m\).
\end{proof}

\begin{exercise} \label{exercise 5.5.4}
Let \(q_1, q_2, q_3,...\) be a sequence of rational numbers with the property that \(\abs{q_n - q_{n'}} \le \frac{1}{M}\) whenever \(M \ge 1\) is an integer and \(n, n' \ge M\) \BLUE{(property 1)}.
Show that \(q_1, q_2, q_3,...\) is a Cauchy sequence.
Furthermore, if \(S := \LIM_{n \toINF} q_n\), show that \(\abs{q_M - S} \le \frac{1}{M}\) for every \(M \ge 1\).
(Hint: use \EXEC{5.4.8}.)
\end{exercise}

\begin{proof}
We first show \(q_1, q_2, q_3,...\) is a Cauchy sequence.
So let arbitrary \(\varE > 0\), we have to show there exists \(N \ge 1\) s.t. \(\abs{q_n - q_{n'}} \ge \varE\) for all \(n, n' \ge N\).
But from \EXEC{5.4.4}, there exists a positive integer \(N\) s.t. \(\varE > 1/N > 0\) \MAROON{(1)}.
And from \BLUE{(property 1)} of \(q_1, q_2, q_3,...\), \(\abs{q_n - q_{n'}} \le \frac{1}{N}\) \MAROON{(2)} for all \(n, n' \ge N\).
So with \MAROON{(1) (2)}, we can conclude that \(\abs{q_n - q_{n'}} \le 1/N < \varE\) for all \(n, n' \ge N\), as desired.
So the sequence \(q_1, q_2, q_3,...\) is Cauchy.

Now we let \(S\) be the formal limit of the sequence \(q_1, q_2, q_3,...\):
\[
    S := \LIM_{n \toINF} q_n.
\]
Let arbitrary integer \(M \ge 1\), we have to show \(\abs{q_M - S} \le \frac{1}{M}\);
Now from \BLUE{(property 1)}, given \(n \ge M\), we have \(\abs{q_M - q_n} \le \frac{1}{M}\).
By \EXEC{5.4.6} we have \(-\frac{1}{M} \le q_M - q_n \le \frac{1}{M}\).
Now we define another sequence \((q'_n)_{n = 1}^{\infty}\) that \(q'_n := q_M\) if \(n < M\) else \(q'_n := q_n\).
Then the sequence \((q_n)_{n = 1}^{\infty}\) and \((q'_n)_{n = 1}^{\infty}\) are equivalent since they are eventually the same(similar argument as \LEM{5.3.14}).
So \(S\) is also equal to the formal limit of \((q'_n)_{n = 1}^{\infty}\) \MAROON{(3)}, and it's trivial that \(-\frac{1}{M} \le q_M - q'_n \le \frac{1}{M}\) for all \(n \ge 1\)(not from \(M\), but from \(1\)).
With \EXEC{5.4.8}, that implies \(-\frac{1}{M} \le q_M - \LIM_{n \toINF} (q'_n) \le \frac{1}{M}\).
That is, by \MAROON{(3)}, \(-\frac{1}{M} \le q_M - S \le \frac{1}{M}\).
So again by \EXEC{5.4.6}, \(\abs{q_M - S} \le \frac{1}{M}\).

Since \(M \ge 1\) is arbitrary, we have \(\abs{q_M - S} \le \frac{1}{M}\) for all \(M \ge 1\).
\end{proof}

\begin{note}
第二個敘述是說,這個序列的第\ \(M\) 項跟這個序列的極限的距離小於\ \(1/M\)。
\end{note}

\begin{exercise} \label{exercise 5.5.5}
Establish an analogue of \PROP{5.4.14}, in which ``rational'' is replaced by ``irrational''.
That is, given any two real numbers \(x < y\), we can find a irrational number \(r\) such that \(x < r < y\).
\end{exercise}

\begin{proof}
Let \(x, y\) be arbitrary real numbers s.t. \(x < y\).
Let \(r = \sqrt{2}\), which is well-defined by \PROP{5.5.12}.
And since \(x < y\), we have \(x - r < y - r\) (by \PROP{5.4.7}).
Now by \PROP{5.4.14}, we can find a \emph{rational} \(r'\) s.t. \(x - r < r' < y - r\), so we have \(x < r' + r < y\).
Then \(r' + r\) must be \emph{irrational}, otherwise if \(r' + r\) is rational, then \((r' + r) - r'\) is also a rational by definition of the \emph{rational} subtraction, so \((r' + r) - r' = r\) is a rational, which is impossible since \(r = \sqrt{2}\). 
\end{proof}
