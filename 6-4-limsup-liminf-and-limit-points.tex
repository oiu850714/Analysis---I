\section{Limsup, Liminf, and limit points} \label{sec 6.4}

Consider the sequence
\[
    1.1, -1.01, 1.001, -1.0001, 1.00001,....
\]
If one plots this sequence, then one sees (informally, of course) that this sequence does \emph{not} converge;
\emph{half} the time the sequence is getting close to \(1\), and \emph{half} the time the sequence is getting close to \(-1\), but it is not converging to either of them;
for instance, it never gets eventually \(1/2\)-close to \(1\), and never gets eventually \(1/2\)-close to \(-1\).
However, even though \(-1\) and \(+1\) are not quite limits of this sequence, it does seem that \emph{in some vague way they “want” to be limits}.
To make this notion precise we introduce the notion of a \emph{limit point}.

\begin{definition} [Limit points] \label{def 6.4.1}
Let \((a_n)_{n = m}^{\infty}\) be a sequence of real numbers, let \(x\) be a real number, and let \(\varE > 0\) be a real number.

\BLUE{(1)} We say that \(x\) is \emph{\(\varE\)-adherent to} \((a_n)_{n = m}^{\infty}\) iff there \emph{exists} an \(n \ge m\) such that \(a_n\) is \(\varE\)-close to \(x\).

\BLUE{(2)} We say that \(x\) is \emph{continually \(\varE\)-adherent} to \((a_n)_{n = m}^{\infty}\) iff it is \(\varE\)-adherent to \((a_n)_{n = \textbf{N}}^{\infty}\) \emph{for every} \(N \ge m\).

\BLUE{(3)} We say that \(x\) is a(i.e. can be many) \emph{limit point} or \emph{adherent point} of \((a_n)_{n = m}^{\infty}\) iff it is continually \(\varE\)-adherent to \((a_n)_{n = m}^{\infty}\) \emph{for every} \(\varE > 0\).
\end{definition}

\begin{note}
\(\varE\)-adherent 只要\ sequence "存在"一個\ element 跟該點距離\ \(\varE\) 即可。
continually \(\varE\)-adherent 則是不管拿掉多少\ sequence 前面的\ elements,剩下的\ sequence 都還是存在\ element 跟該點距離\ \(\varE\)。感覺有點永不放棄。
limit point 則是不管給我多小的\ \(\varE\),sequence 都跟該點是\ continually \(\varE\)-adherent。
\end{note}

\begin{remark} \label{remark 6.4.2}
The verb ``to adhere'' means much the same as ``to stick to'';
hence the term ``adhesive''.
\end{remark}

\begin{additional corollary} \label{ac 6.4.1}
\emph{Unwrapping} all the definitions, we see that \(x\) is a limit point of \((a_n)_{n = m}^{\infty}\) if, for every \(\varE > 0\) and every \(N \ge m\), there exists an \(n' \ge N\) such that \(\abs{a_{n'} - x} \le \varE\).
Why?
\begin{align*}
             & x \text{ is a limit point of } (a_n)_{n = m}^{\infty} \\
    \implies & \forall \varE > 0, (a_n)_{n = m}^{\infty} \text{ is continually } \varE \text{-adherent to } (a_n)_{n = m}^{\infty} \text{, by \DEF{6.4.1}(3)} \\
    \implies & \forall \varE > 0, \forall N \ge m, x \text{ is \(\varE\)-adherent to } (a_n)_{n = \textbf{N}}^{\infty} \text{, by \DEF{6.4.1}(2)} \\
    \implies & \forall \varE > 0, \forall N \ge m, \exists n' \ge N \text{ s.t. } x \text{ is \(\varE\)-close to \(a_{n'}\), by \DEF{6.4.1}(1)}
\end{align*}
\end{additional corollary}

\begin{note}
Note the difference between a sequence being \(\varE\)-close to \(L\) (which means that \emph{all the elements} of the sequence stay within a distance \(\varE\) of \(L\))
and \(L\) being \(\varE\)-adherent to the sequence (which \emph{only needs a single element} of the sequence to stay within a distance \(\varE\) of \(L\)).
Also, for \(L\) to be continually \(\varE\)-adherent to \((a_n)_{n = m}^{\infty}\), it has to be \(\varE\)-adherent to \((a_n)_{n = N}^{\infty}\) \emph{for all} \(N \ge m\),
whereas for \((a_n)_{n = m}^{\infty}\) to be eventually \(\varE\)-close to \(L\), we only need \((a_n)_{n = N}^{\infty}\) to be \(\varE\)-close to \(L\) \emph{for some} \(N \ge m\).
Thus there are some subtle differences in \emph{quantifiers} between limits and limit points.
\end{note}

\begin{note}
Currently limit points are only defined (in \DEF{6.4.1}) for \emph{finite} real numbers.
It is also possible to rigorously define the concept of \(+\infty\) or \(-\infty\) being a limit point;
see \EXEC{6.4.8}.
\end{note}

\begin{example} \label{example 6.4.3}
Let \((a_n)_{n = 1}^{\infty}\) denote the sequence
\[
    0.9, 0.99, 0.999, 0.9999, 0.99999,....
\]
The number \(0.8\) is \(0.1\)-adherent to this sequence, since \(0.8\) is \(0.1\)-close to 0.9, which is a member (the first member) of this sequence.
However, \(0.8\) is not \emph{continually} \(0.1\)-adherent to this sequence, since once one discards the first element of this sequence there is no member of the sequence to be \(0.1\)-close to \(0.8\).
In particular, \(0.8\) is not a limit point of this sequence.
On the other hand, the number \(1\) is \(0.1\)-adherent to this sequence, and in fact \(0.1\) is \emph{continually} \(0.1\)-adherent to this sequence,
since no matter how many \emph{initial members} of the sequence one discards, there is still \emph{something}(i.e. ``some'') for \(1\) to be \(0.1\)-close to.
In fact, it is continually \(\varE\)-adherent for every \(\varE > 0\), and is hence a limit point of this sequence.
\end{example}

\begin{example} \label{example 6.4.4}
Now consider the sequence
\[
    1.1, -1.01, 1.001, -1.0001, 1.00001,....
\]
The number \(1\) is \(0.1\)-adherent to this sequence;
in fact \(1\) is continually \(0.1\)-adherent to this sequence,
because no matter how many elements of the sequence one discards, there are \emph{some} elements of the sequence that \(1\) is \(0.1\)-close to.
(As discussed earlier, one does \emph{not} need \emph{all} the elements to be \(0.1\)-close to 1, just \emph{some}; 
thus \(0.1\)-adherent is weaker than \(0.1\)-close, and continually \(0.1\)-adherent is a different notion from eventually \(0.1\)-close.)
In fact, for every \(\varE > 0\), the number \(1\) is continually \(\varE\)-adherent to this sequence, and is thus a limit point of this sequence.
Similarly \(-1\) is a limit point of this sequence;
however \(0\) is not a limit point of this sequence, since the sequence is not continually \(0.1\)-adherent to \(0\).
\end{example}

Limits are \emph{of course a special case} of limit points:

\begin{proposition} [Limits are limit points] \label{prop 6.4.5}
Let \((a_n)_{n = m}^{\infty}\) be a sequence which \emph{converges} to a real number \(c\).
Then \(c\) is a limit point of \((a_n)_{n = m}^{\infty}\), and in fact it is \emph{the only limit point} of \((a_n)_{n = m}^{\infty}\).
\end{proposition}

\begin{proof}
To show \(c\) is a limit point of \((a_n)_{n = m}^{\infty}\), by \AC{6.4.1} we have to show:
for every \(\varE > 0\) and every \(N \ge m\), there exists \(n' \ge N\) such that \(\abs{a_{n'} - c} \le \varE\).

So let arbitrary \(\varE > 0\) and arbitrary \(N \ge m\).
We have to find \(n' \ge N\) such that \(\abs{a_{n'} - c} \le \varE\) \BLUE{(1)}.

Since the sequence converges to \(c\), by \DEF{6.1.5} (or \EXEC{6.1.2}),
there exists \(N' \ge m\) s.t. \(\abs{a_{n'} - c} \le \varE\) \emph{for all} \(n' \ge N'\) \MAROON{(1)}.
Now there are two cases: \(N' < N\) or \(N' \ge N\).

If \(N' < N\), then in particular we can replace \(n' = N\) into \MAROON{(1)} s.t. \(\abs{a_{N} - c} \le \varE\),
so we have found \(n' = N \ge N\) s.t. \(\abs{a_{n'} - c} \le \varE\).

If \(N' \ge N\), then in particular we just use \(n' = N'\) into \MAROON{(1)} s.t. \(\abs{a_{N'} - c} \le \varE\),
so we have found \(n' = N' \ge N\) s.t. \(\abs{a_{n'} - c} \le \varE\).

In all cases, \BLUE{(1)} is satisfied, as desired.

Now we show that \(c\) is the \emph{only} limit point of the sequence.
For the sake of contradiction suppose \(c' \ne c\) is also a limit point of the sequence.

Then let \(\varE = \abs{c - c'}/2 > 0\).
\emph{In particular}, \(\abs{c - c'}/\BLUE{3} > 0\).
Since the sequence converges to \(c\), there exists \(N \ge m\) s.t. \(\abs{a_n - c} \le \abs{c - c'}/\BLUE{3}\) \MAROON{(2)} for all \(n \ge N\).
And we have for all \(n \ge N\),
\begin{align*}
    \abs{c - c'} & = \abs{c - c' + a_n - a_n} \\
                 & = \abs{(c - a_n) + (a_n - c')} \\
                 & \le \abs{c - a_n} + \abs{a_n - c'} \\
                 & = \abs{a_n - c} + \abs{a_n - c'},
\end{align*}
which implies \(\abs{a_n - c'} \ge \abs{c - c'} - \abs{a_n - c}\) for all \(n \ge N\).
And for all \(n \ge N\),
\begin{align*}
    \abs{a_n - c'} & \ge \abs{c - c'} - \abs{a_n - c} \\
                   & \ge \abs{c - c'} - \abs{c - c'}/\BLUE{3} & \text{by \MAROON{(2)}} \\
                   & = 2\abs{c - c'}/3 \\
                   & \RED{>} \abs{c - c'}/2 \\
                   & = \varE.
\end{align*}
So for all \(n \ge N\), \(a_n\) is not \(\varE\)-close to \(c'\),
which implies the sequence is not continually \(\varE\)-adherent to \(c'\),
which implies \(c'\) is not a limit point of the sequence, a contradiction.
\end{proof}

Now we will look at two special types of limit points: the limit superior (\(\limsup\)) and limit inferior (\(\liminf\)).
(\RED{warning}: I think it's too early to say they are limit points;
by \PROP{6.4.12}(e) they are limit points only if they are finite;
by \EXEC{6.4.8} they are limit points only if we allow limit points to be infinite.)

\begin{definition} [Limit superior and limit inferior]  \label{def 6.4.6}
Suppose that \((a_n)_{n = m}^{\infty}\) is a sequence.

\BLUE{(1)}  We define a \emph{new} sequence \((a_N^+)_{N = m}^{\infty}\) by the formula
\[
    a_{\textbf{N}}^+ := \sup(a_n)_{n = \textbf{N}}^{\infty}.
\]
More informally, \(a_N^+\) is \emph{the supremum of all the elements} in the sequence \textbf{from \(a_N\) onwards}.

\BLUE{(2)} We then define the \emph{limit superior} of the sequence \((a_n)_{n = m}^{\infty}\), denoted \(\limsup_{n \toINF} a_n\), by the formula
\[
    \limsup_{n \toINF} a_n := \inf(a_N^+)_{N = m}^{\infty}.
\]

\BLUE{(3)} Similarly, we can define \(a_N^- := \inf(a_n)_{n = N}^{\infty}\) and \BLUE{(4)} define the \emph{limit inferior} of the sequence \((a_n)_{n = m}^{\infty}\), denoted \(\liminf_{n \toINF} a_n\), by the formula
\[
    \liminf_{n \toINF} a_n := \sup(a_N^-)_{N = m}^{\infty}.
\]

Note that the definition of new sequence in \BLUE{(1)(2)} are \emph{well-defined}, since supremum and infimum are well-defined(i.e. unique; see \DEF{6.2.6}).
\end{definition}

\begin{note}
``From \(a_N\) onwards'': 從\ \(a_N\) 開始繼續往下數。

不精確的講,\(\limsup\) 是所有(給定任意\ \(N \ge m\),從第\ \(N\) 個\ element 開始算的\ sequence 所成的集合的) supremum 所成的集合的\ infimum(不正確但好記的說法: 所有\ sup 裡面最小的那個)。

而\ \(\liminf\) 是所有(給定任意\ \(N \ge m\),從第\ \(N\) 個\ element 開始算的\ sequence 所成的集合的) infimum 所成的集合的\ supremum(不正確但好記的說法: 所有\ inf 裡面最大的那個)。
\end{note}

\begin{example} \label{example 6.4.7}
Let \(a_1, a_2, a_3,...\) denote the sequence
\[
    1.1, -1.01, 1.001, -1.0001, 1.00001,....
\]
Then \(a_1^+, a_2^+, a_3^+, ...\) is the sequence
\[
    1.1, 1.001, 1.001, 1.00001, 1.00001,...
\]
Why?

For \(n = 1\), \(1.1\) is \(\sup(a_n)_{n = 1}^{\infty}\), or \(\sup \{1.1, -1.01, 1.001, -1.0001,...\} \).

For \(n = 2\), \(1.001\) is \(\sup(a_n)_{n = 2}^{\infty}\), or \(\sup \{-1.01, 1.001, -1.0001,... \}\).

For \(n = 3\), \(1.001\) is \(\sup(a_n)_{n = 3}^{\infty}\), or \(\sup \{1.001, -1.0001, 1.00001,...\}\).

For \(n = 4\), \(1.00001\) is \(\sup(a_n)_{n=4}^{\infty}\), or \(\sup \{-1.0001, 1.00001, ...\}\).

For \(n = 5\), \(1.00001\) is \(\sup(a_n)_{n = 5}^{\infty}\), or \(\sup \{1.00001, ...\}\).

And the infimum of the sequence \(a_1^+, a_2^+, a_3^+, ...\) is \(1\).
Hen the \emph{limit superior} of \(a_1, a_2, a_3,...\) is \(1\).

Similarly, \(a_1^-, a_2^-, a_3^- ,...\) is the sequence
\[
    -1.01, -1.01, -1.0001, -1.0001, -1.000001,...
\]
and the supremum of the sequence \(a_1^-, a_2^-, a_3^- ,...\) is \(-1\).
Hence the \emph{limit inferior} of \(a_1, a_2, a_3,...\) is \(-1\).

One should compare this with the \emph{supremum} and \emph{infimum} of the sequence, which are \(1.1\) and \(-1.01\) respectively.
\end{example}

\begin{example} \label{example 6.4.8}
Trivial.
\end{example}

\begin{example} \label{example 6.4.9}
Trivial.
\end{example}

\begin{example} \label{example 6.4.10}
Let \(a_1, a_2, a_3,...\) denote the sequence
\[
    1, 2, 3, 4, 5, 6,...
\]
Then \(a_1^+, a_2^+ ,...\) is the sequence
\[
    +\infty, +\infty, +\infty,...
\]
so the limit superior (of \(1, 2, 3, 4, 5, 6,...\)) is \(+\infty\).
Similarly, \(a_1^-, a_2^-, ...\) is the sequence
\[
    1, 2, 3, 4, 5,...
\]
which has a \emph{supremum} of \(+\infty\).
So the limit inferior (of \(1, 2, 3, 4, 5, 6,...\)) is also \(+\infty\).
\end{example}

\begin{remark} \label{remark 6.4.11}
\sloppy Some authors use the notation \(\overline{\lim}_{n \toINF} a_n\) instead of \(\limsup_{n \toINF} a_n\), and \(\underline{\lim}_{n \toINF} a_n\) instead of \(\liminf_{n \toINF} a_n\).
Note that the starting index \(m\) of the sequence is irrelevant (see \EXEC{6.4.2}).
\end{remark}

\begin{note}
Returning to the piston analogy, imagine a piston at \(+\infty\) moving leftward until it is stopped by the presence of the sequence \(a_1, a_2, \dots\).
The place it will stop is the \emph{supremum} of \(a_1, a_2, a_3, \dots\), which in our new notation is \(a_1^+\).
Now let us \emph{remove the first element} \(a_1\) from the sequence;
this may cause our piston to slip leftward, to a new point \(a_2^+\)
(though in many cases the piston will not move and \(a_2^+\) will just be the same as \(a_1^+\)).
Then we remove the second element \(a_2\), causing the piston to slip a little more.
If we keep doing this the piston will keep slipping, but \emph{there will be some point where it cannot go any further}, and \emph{this is the limit superior of the sequence}.
A similar analogy can describe the limit inferior of the sequence.
\end{note}

\begin{additional corollary} [Rigorous proof for piston analogy] \label{ac 6.4.2}
Given sequence \((a_n)_{n = m}^{\infty}\), and integer \(m_1, m_2\) s.t. \(m_1 \ge m_2 \ge m\).
Then \(\sup(a_n)_{n = m_1}^{\infty} \le \sup(a_n)_{n = m_2}^{\infty}\).
Similarly, \(\inf(a_n)_{n = m_1}^{\infty} \ge \inf(a_n)_{n = m_2}^{\infty}\).
\end{additional corollary}

\begin{proof}
For the sake of contradiction, suppose \(\sup(a_n)_{n = m_1}^{\infty} > \sup(a_n)_{n = m_2}^{\infty}\) for some \(m_1 \ge m_2 \ge m\).
Then by \PROP{6.3.6}(1), for all \(n \ge m_2\), \(a_n \le \sup(a_n)_{n = m_2}^{\infty}\);
in particular, for all \(n \ge m_1\) where \(m_1 \ge m_2\), \(a_n \le \sup(a_n)_{n = m_2}^{\infty}\).
That is, \(\sup(a_n)_{n = m_2}^{\infty}\) is an upper bound of \(\{a_n : n \ge m_1\}\),
and \(\sup(a_n)_{n = m_2}^{\infty} < \sup(a_n)_{n = m_1}^{\infty}\),
which contradicts that \(\sup(a_n)_{n = m_1}^{\infty}\) is the supremum of \(\{a_n : n \ge m_1\}\).

Now for the sake of contradiction, suppose \(\inf(a_n)_{n = m_1}^{\infty} < \inf(a_n)_{n = m_2}^{\infty}\) for some \(m_1 \ge m_2 \ge m\).
Then by \RMK{6.3.7}(1), for all \(n \ge m_2\), \(a_n \ge \inf(a_n)_{n = m_2}^{\infty}\);
in particular, for all \(n \ge m_1\) where \(m_1 \ge m_2\), \(a_n \ge \inf(a_n)_{n = m_2}^{\infty}\).
That is, \(\inf(a_n)_{n = m_2}^{\infty}\) is a lower bound of \(\{a_n : n \ge m_1\}\),
and \(\inf(a_n)_{n = m_1}^{\infty} < \inf(a_n)_{n = m_2}^{\infty}\),
which contradicts that \(\inf(a_n)_{n = m_1}^{\infty}\) is the infimum of \(\{a_n : n \ge m_1\}\).
\end{proof}

We now describe some basic properties of limit superior and limit inferior.

\begin{proposition} \label{prop 6.4.12}
Let \((a_n)_{n = m}^{\infty}\) be a sequence of real numbers, let \(L^+\) be the \emph{limit superior} of this sequence, and let \(L^-\) be the \emph{limit inferior} of
this sequence (thus both \(L^+\) and \(L^-\) are \emph{extended} real numbers, because supremum and infimum can be extended real numbers, by \DEF{6.2.6}).
\begin{enumerate}
\item For every \(x > L^+\), there exists an \(N \ge m\) such that \(a_n < x\) \emph{for all} \(n \ge N\).
      (In other words, for every \(x > L^+\), the elements of the sequence \((a_n)_{n = m}^{\infty}\) are \emph{eventually less than} \(x\).)
      Similarly, for every \(y < L^-\) there exists an \(N \ge m\) such that \(a_n > y\) \emph{for all} \(n \ge N\).
\item For every \(x < L^+\), and every \(N \ge m\), there exists an \(n \ge N\) such that \(a_n > x\).
      (In other words, for every \(x < L^+\), the elements of the sequence \((a_n)_{n = m}^{\infty}\) exceed \(x\) \emph{infinitely often}.)
      Similarly, for every \(y > L^-\) and every \(N \ge m\), there exists an \(n \ge N\) such that \(a_n < y\).
\item We have \(\inf(a_n)_{n = m}^{\infty} \le L^- \le L^+ \le \sup(a_n)_{n = m}^{\infty}\).
\item If \(c\) is any \emph{limit point} of \((a_n)_{n = m}^{\infty}\), then we have \(L^- \le c \le L^+\).
\item If \(L^+\) is finite, then it \emph{is a limit point} of \((a_n)_{n = m}^{\infty}\).
      Similarly, if \(L^-\) is finite, then it is a limit point of \((a_n)_{n = m}^{\infty}\).
\item Let \(c\) be a real number.
      If \((a_n)_{n = m}^{\infty}\) converges to \(c\), then we must have \(L^+ = L^- = c\).
      Conversely, if \(L^+ = L^- = c\), then \((a_n)_{n = m}^{\infty}\) converges to \(c\).
\end{enumerate}
\end{proposition}

\begin{note}
\begin{enumerate}
\item 若某個數字比上極限大,則序列最終會永遠比該數字小;若某個數字比下極限小,則序列最終會永遠比該數字大。
\item 若某數字比上極限小,則不管怎麼丟掉序列前面的項,剩下的序列都還是存在一個項比該數字大(這叫做\ ``infinitely often``);若某個數字比下極限大,則不管怎麼丟掉序列前面的數字,剩下的序列都還是存在一個項比該數字小。
\item 序列的最大下界,小於等於下極限,小於等於上極限,小於等於序列的最小上界。
\item 極限點在下極限跟上極限之間
\item 若上下極限是有限的,則他們也是極限點。
\item 數列收斂到某實數,若且唯若上極限跟下極限都等於該實數。
\end{enumerate}
\end{note}

\begin{proof}
\begin{enumerate}
\item Suppose first that \(x > L^+\).
      Then by definition(\DEF{6.4.6}(2)) of \(L^+\), we have \(x > \inf(a_N^+)_{N = m}^{\infty}\).
      By \PROP{6.3.6}(actually, infimum version, \RMK{6.3.7}(3)), there must then exist an integer \(N' \ge m\) such that \(x > a_{N'}^+\).
      And by \DEF{6.4.6}(1), this means \(x > \sup(a_n)_{n = N'}^{\infty}\).
      Finally, by \PROP{6.3.6}(2) again, we have \(x > \sup(a_n)_{n = N'}^{\infty} \ge a_n\) for all \(n \ge N'\).
      So all in all, we have found \(N' \ge m\) s.t. \(x > a_n\), for all \(n \ge N'\), as desired.
      
      Now suppose first that \(y < L^-\).
      Then by definition(\DEF{6.4.6}(4)) of \(L^-\), we have \(y < \sup(a_N^-)_{N = m}^{\infty}\).
      By \PROP{6.3.6}(3), there must then exist an integer \(N' \ge m\) such that \(y < a_{N'}^-\).
      And by \DEF{6.4.6}(3), this means \(y < \inf(a_n)_{n = N'}^{\infty}\).
      Finally, by \RMK{6.3.7}(2) again, we have \(y < \inf(a_n)_{n = N'}^{\infty} \le a_n\) for all \(n \ge N'\).
      So all in all, we have found \(N' \ge m\) s.t. \(y < a_n\), for all \(n \ge N'\), as desired.
\item Suppose that \(x < L^+\).
      Then again by \DEF{6.4.6}(2) we have \(x < \inf(a_N^+)_{N = m}^{\infty}\) \MAROON{(1)}.
      Then first, by \RMK{6.3.7}(1), we have \(a_N^+ \ge \inf(a_N^+)_{N = m}^{\infty}\) for all \(N \ge m\).
      \GREEN{Let \(N'\) be arbitrary integer} s.t. \(N' \ge m\).
      Then in particular \(a_{N'}^+ \ge \inf(a_N^+)_{N = m}^{\infty}\) \MAROON{(2)}.
      With \MAROON{(1)(2)} we have \(x < a_{N'}^+\).
      And by \DEF{6.4.6}(1), this means \(x < \sup(a_n)_{n = N'}^{\infty}\).
      Then by \PROP{6.3.6}(3) again, \GREEN{there must exists} \(n' \ge N'\) s.t. \(a_{n'} > x\), \GREEN{as desired}.
      
      Now suppose that \(y > L^-\).
      Then again by \DEF{6.4.6}(4) we have \(y > \sup(a_N^-)_{N = m}^{\infty}\) \MAROON{(3)}.
      Then first, by \PROP{6.3.6}(1), we have \(a_N^- \le \sup(a_N^-)_{N = m}^{\infty}\) for all \(N \ge m\).
      \GREEN{Let \(N'\) be arbitrary integer} s.t. \(N' \ge m\).
      Then in particular \(a_{N'}^- \le \inf(a_N^+)_{N = m}^{\infty}\) \MAROON{(4)}.
      With \MAROON{(3)(4)} we have \(y > a_{N'}^-\).
      And by \DEF{6.4.6}(3), this means \(y > \inf(a_n)_{n = N'}^{\infty}\).
      Then by \RMK{6.3.7}(3) again, \GREEN{there must exists} \(n' \ge N'\) s.t. \(a_{n'} < y\), \GREEN{as desired}.
\item We first show that \(\inf(a_n)_{n = m}^{\infty} \le L^-\).
      Then
      \begin{align*}
        L^- & = \sup(a_N^-)_{N = m}^{\infty} & \text{by \DEF{6.4.6}(4)} \\
            & \ge a_m^- & \text{in particular for \(N = m\), and by \PROP{6.3.6}(1)} \\
            & = \inf(a_n)_{n = m}^{\infty} & \text{by \DEF{6.4.6}(3) and \(a_m^-\)}
      \end{align*}
      
      Now we show that \(L^+ \le \sup(a_n)_{n = m}^{\infty}\).
      Then
      \begin{align*}
        L^+ & = \inf(a_N^+)_{N = m}^{\infty} & \text{by \DEF{6.4.6}(2)} \\
            & \ge a_m^+ & \text{in particular for \(N = m\), and by \RMK{6.3.7}(1)} \\
            & = \sup(a_n)_{n = m}^{\infty} & \text{by \DEF{6.4.6}(1) and \(a_m^+\)}
      \end{align*}
      
      Finally we show that \(L^- \le L^+\).
      For the sake of contradiction, suppose \(L^+ < L^-\).
      Then
      \begin{align*}
                 & L^+ < L^- \\
        \implies & \inf(a_N^+)_{N = m}^{\infty} < \sup(a_N^-)_{N = m}^{\infty} & \text{by \DEF{6.4.6}(2)(4)} \\
        \implies & \exists N_1 \ge m \text{ s.t. } \inf(a_N^+)_{N = m}^{\infty} \le a_{N_1}^+ < \sup(a_N^-)_{N = m}^{\infty} & \text{by \RMK{6.3.7}(3),} \\
                 & & \text{by replacing \(x = \inf(a_N^+)_{N = m}^{\infty}\),} \\
                 & & \text{\(y = \sup(a_N^-)_{N = m}^{\infty}\)} \\
        \implies & a_{N_1}^+ < \sup(a_N^-)_{N = m}^{\infty} & \text{in particular} \\
        \implies & \exists N_2 \ge m \text{ s.t. } a_{N_1}^+ < a_{N_2}^- \le \sup(a_N^-)_{N = m}^{\infty} & \text{by \PROP{6.3.6}(3),} \\
                 & & \text{by replacing \(x = \sup(a_N^-)_{N = m}^{\infty}\),} \\
                 & & \text{\(y = a_{N_1}^+\)} \\
        \implies & a_{N_1}^+ < a_{N_2}^- & \text{in particular} \\
        \implies & \sup(a_n)_{n = N_1}^{\infty} < \inf(a_n)_{n = N_2}^{\infty} & \text{by \DEF{6.4.6}(1)(3)} \\
      \end{align*}
      Now, \AC{6.4.2}(rigorous piston analogy), let \(N = max(N_1, N_2)\), then we have
      \[
        \sup(a_n)_{n = N}^{\infty} \le \sup(a_n)_{n = N_1}^{\infty} < \inf(a_n)_{n = N_2}^{\infty} \le \inf(a_n)_{n = N}^{\infty},
      \]
      or \(\sup(a_n)_{n = N}^{\infty} < \inf(a_n)_{n = N}^{\infty}\).
      But that is impossible! (Since that implies \(a_N \ge \inf(a_n)_{n = N}^{\infty} > \sup(a_n)_{n = N}^{\infty}\), so \(a_N\) is greater than supremum, impossible.)
      So \(L^- \le L^+\).

      So all in all we have \(\inf(a_n)_{n = m}^{\infty} \le L^- \le L^+ \le \sup(a_n)_{n = m}^{\infty}\).
\item Suppose \(c\) is any limit point of \((a_n)_{n = m}^{\infty}\).

      We first show that \(L^- \le c\).
      For the sake of contradiction, suppose \(c < L^-\).
      So \(L^- - c > 0\), in particular we let \(\varE = (L^- - c)/2 > 0\).
      Since \(c\) is a limit point, by \DEF{6.4.1}(3) (or unwrapping, \AC{6.4.1}), \(\forall N \ge m, \exists n \ge N\) s.t. \(\abs{a_n - c} \le \varE\),
      or \(-\varE \le a_n - c \le \varE\), and thus \(a_n \le c + \varE\) \GREEN{(1)}.
      And since \(L^- - \varE < L^-\), by part(a) there exists \(N' \ge m\) s.t. \(a_{n'} > L^- - \varE\) for all \(n' \ge N'\) \GREEN{(2)}.
      In particular, given this \(N'\), by \GREEN{(1)} we can find \(n_2 \ge N'\) s.t. \(a_{n_2} \le c + \varE\), and use that \(n_2\) in \GREEN{(2)} we have \(a_{n_2} > L^- - \varE\).
      Together, we have \(L^- - \varE < a_{n_2} \le c + \varE\), or \(L^- - \varE < c + \varE\).
      But substitute \(\varE\) back to \((L^- - c)/2\), we then have \(L^- - (L^- - c)/2 < c + (L^- - c)/2\),
      or \(L^-/2 + 2/c < 2/c + L^-/2\), which is impossible.
      So it must be true that \(L^- \le c\).
      
      Now we show that \(c \le L^+\).
      For the sake of contradiction, suppose \(c > L^+\).
      So \(c - L^+ > 0\), in particular we let \(\varE = (c - L^+)/2 > 0\).
      Since \(c\) is a limit point, by \DEF{6.4.1}(3) (or unwrapping, \AC{6.4.1}), \(\forall N \ge m, \exists n \ge N\) s.t. \(\abs{a_n - c} \le \varE\),
      or \(-\varE \le a_n - c \le \varE\), and thus \(a_n \ge c - \varE\) \GREEN{(3)}.
      And since \(L^+ + \varE > L^+\), by part(a) there exists \(N' \ge m\) s.t. \(a_{n'} < L^+ + \varE\) for all \(n' \ge N'\) \GREEN{(4)}.
      In particular, given this \(N'\), by \GREEN{(3)} we can find \(n_2 \ge N'\) s.t. \(a_{n_2} \ge c - \varE\), and use that \(n_2\) in \GREEN{(4)} we have \(a_{n_2} < L^+ + \varE\).
      Together, we have \(c - \varE \le a_{n_2} < L^+ + \varE\), or \(c - \varE < L^+ + \varE\).
      But substitute \(\varE\) back to \((c - L^+)/2\), we then have \(c - (c - L^+)/2 < L^+ + (c - L^+)/2\),
      or \(2/c + L^+/2 < L^+/2 + 2/c\), which is impossible.
      So it must be true that \(c \le L^+\).

      So together, we have \(L^- \le c \le L^+\).
\item We first prove the case for \(L^+\).
      Suppose \(L^+\) is finite.
      We have to show that \(L^+\) is a limit point of the sequence.

      So by \DEF{6.4.1}(3)(or unwrapping, \AC{6.4.1}), we have to show for any \(\varE > 0\), and for any \(N \ge m\), there exists \(n \ge N\) s.t. \(\abs{a_n - L^+} \le \varE\).
      So let arbitrary \(\varE > 0\).
      And let arbitrary \(N \ge m\).
      Then since \(L^+\) is finite, \(L^+ + \varE\) is also finite so of course \(L^+ + \varE > L^+\).
      And by part(a), we can find \(N' \ge m\) s.t. \(L^+ + \varE > a_n\) \emph{for all} \(n \ge N'\).
      And in particular, given \(\max(N, N')\), \(L^+ + \varE > a_n\) \emph{for all} \(n \ge \max(N, N')\) \GREEN{(5)}.
      Also, \(L^+ - \varE\) is finite so of course \(L^+ - \varE < L^+\).
      By part(b), given that \(\max(N, N')\), there exists \(n' \ge \max(N, N')\) s.t. \(a_{n'} > L^+ - \varE\). \GREEN{(6)}.
      So by \GREEN{(5)(6)}, we have found \(n' \ge \max(N, N')\), which of course \(\ge N\), s.t. \(L^+ - \varE < a_{n'} < L^+ + \varE\),
      or \(-\varE < a_{n'} - L^+ < \varE\), in particular \(\abs{a_n - L^+} \le \varE\), as desired.
      
      Now we prove the case for \(L^-\).
      Suppose \(L^-\) is finite.
      We have to show that \(L^-\) is a limit point of the sequence.

      So by \DEF{6.4.1}(3)(or unwrapping, \AC{6.4.1}), we have to show for any \(\varE > 0\), and for any \(N \ge m\), there exists \(n \ge N\) s.t. \(\abs{a_n - L^-} \le \varE\).
      So let arbitrary \(\varE > 0\).
      And let arbitrary \(N \ge m\).
      Then since \(L^-\) is finite, \(L^- - \varE\) is also finite so of course \(L^- - \varE < L^-\).
      And by part(a), we can find \(N' \ge m\) s.t. \(L^- - \varE < a_n\) \emph{for all} \(n \ge N'\).
      And in particular, given \(\max(N, N')\), \(L^- - \varE < a_n\) \emph{for all} \(n \ge \max(N, N')\) \GREEN{(7)}.
      Also, \(L^- + \varE\) is finite so of course \(L^- + \varE > L^-\).
      By part(b), given that \(\max(N, N')\), there exists \(n' \ge \max(N, N')\) s.t. \(a_{n'} < L^- + \varE\). \GREEN{(8)}.
      So by \GREEN{(7)(8)}, we have found \(n' \ge \max(N, N')\), which of course \(\ge N\), s.t. \(L^- - \varE < a_{n'} < L^- + \varE\),
      or \(-\varE < a_{n'} - L^- < \varE\), in particular \(\abs{a_n - L^-} \le \varE\), as desired.
\item Suppose \((a_n)_{n = m}^{\infty}\) converges to \(c\), we have to show \(L^- = c = L^+\).
      Now since the sequence converges to \(c\), by \PROP{6.4.5} \(c\) is also a limit point, and \emph{the only limit point}.
      And by \CORO{6.1.17}, the sequence are bounded, and by \AC{6.3.1}, the supremum/infimum of the sequence are finite.
      And again, by \AC{6.4.2}(rigorous piston analogy), \emph{\(L^-\) and \(L^+\) are also finite}.
      So by part(e), \(L^-\) and \(L^+\) are also limit points, together that implies \(L^- = c = L^+\), otherwise the converged sequence has multiple limit points, contradicting \PROP{6.4.5}.

      Now suppose \(L^- = c = L^+\), we have to show the \((a_n)_{n = m}^{\infty}\) converges to \(c\).
      So let arbitrary \(\varE > 0\).
      We have to show there exists \(N \ge m\) s.t. \(\abs{a_n - c} \le \varE\) for all \(n \ge N\).
      Then of course \(c - \varE < c < c + \varE\).
      And of course \(c < c + \varE\), that is, \(L^+ < L^+ + \varE\).
      And by part(a), there exists \(N_1 \ge m\) s.t. \(a_n < L^+ + \varE\) for all \(n \ge N_1\) \GREEN{(9)}.
      And again of course \(c - \varE < c\), that is, \(L^- - \varE < L^-\).
      And by part(a) again, there exists \(N_2 \ge m\) s.t. \(L^- - \varE < a_n\) for all \(n \ge N_2\) \GREEN{(10)}.
      Then by \GREEN{(9)(10)}, we have found \(N = \max(N_1, N_2)\) s.t. \(L^- - \varE < a_n < L^+ + \varE\) for all \(n \ge N\).
      That is, \(c - \varE < a_n < c + \varE\) for all \(n \ge N\).
      In particular, \(\abs{a_n - c} \le \varE\) for all \(n \ge N\), as desired.
\end{enumerate}
\end{proof}

\begin{note}
Parts (c) and (d) of \PROP{6.4.12} say, in particular, that \(L^+\) is the \emph{largest} limit point of \((a_n)_{n = m}^{\infty}\), and \(L^-\) is the \emph{smallest} limit point
(providing that \(L^+\) and \(L^-\) are \emph{finite}).
Part (f) then says that if \(L^+\) and \(L-\) coincide (so there is only \emph{one} limit point), then the sequence in fact \emph{converges}.
\emph{This gives a way to test if a sequence converges}:
compute its limit superior and limit inferior, and see if they are equal.
\end{note}

We now give a basic comparison property of limit superior and limit inferior.

\begin{lemma} [Comparison principle] \label{lem 6.4.13}
Suppose that \((a_n)_{n = m}^{\infty}\) and \((b_n)_{n = m}^{\infty}\) are two sequences of real numbers such that \(a_n \le b_n\) for all
\(n \ge m\).
Then we have the inequalities
\begin{enumerate}
    \item \(\sup(a_n)_{n = m}^{\infty} \le \sup(b_n)_{n = m}^{\infty}\)
    \item \(\inf(a_n)_{n = m}^{\infty} \le \inf(b_n)_{n = m}^{\infty}\)
    \item \(\limsup_{n \toINF} a_n \le \limsup_{n \toINF} b_n\)
    \item \(\liminf_{n \toINF} a_n \le \liminf_{n \toINF} b_n\)
\end{enumerate}
\end{lemma}

\begin{proof}
We first show \BLUE{(a)}.
Suppose for the sake of contradiction that \(\sup(b_n)_{n = m}^{\infty} < \sup(a_n)_{n = m}^{\infty}\).
Then by \PROP{6.3.6}(3), there exists integer \(n' \ge m\) s.t. \(\sup(b_n)_{n = m}^{\infty} < a_{n'} \le \sup(a_n)_{n = m}^{\infty}\) \MAROON{(1)}.
But again by \PROP{6.3.6}(1), we have \(b_n \le \sup(b_n)_{n = m}^{\infty}\) for all \(n \ge m\).
In particular \(b_{n'} \le \sup(b_n)_{n = m}^{\infty}\) \MAROON{(2)}.
By \MAROON{(1)(2)}, we have \(b_{n'} \le \sup(b_n)_{n = m}^{\infty} < a_{n'}\), in particular \(b_{n'} < a_{n'}\), which contradicts \(a_n \le b_n\) for all \(n \ge m\).

For \BLUE{(b)}, suppose for the sake of contradiction that \(\inf(a_n)_{n = m}^{\infty} > \inf(b_n)_{n = m}^{\infty} \).
Then by \RMK{6.3.7}(3), there exists integer \(n' \ge m\) s.t. \(\inf(a_n)_{n = m}^{\infty} > b_{n'} > \inf(b_n)_{n = m}^{\infty} \) \MAROON{(3)}.
But again by \RMK{6.3.7}(1), we have \(a_n \ge \inf(a_n)_{n = m}^{\infty}\) for all \(n \ge m\).
In particular \(a_{n'} \ge \inf(a_n)_{n = m}^{\infty}\) \MAROON{(4)}.
By \MAROON{(3)(4)}, we have \(a_{n'} \ge \inf(a_n)_{n = m}^{\infty} > b_{n'}\), in particular \(a_{n'} > b_{n'}\), which contradicts \(a_n \le b_n\) for all \(n \ge m\).

For \BLUE{(c)}, suppose for the sake of contradiction that \(\limsup_{n \toINF} b_n < \limsup_{n \toINF} a_n\).
Then by \DEF{6.4.6}(2), we have \(\inf (b_N^+)_{N = m}^{\infty} < \inf (a_N^+)_{N = m}^{\infty}\).
By \RMK{6.3.7}(3), (with \(\inf (b_N^+)_{N = m}^{\infty}\) as \(x\), \(\inf (a_N^+)_{N = m}^{\infty}\) as \(y\),) there exists \(n' \ge m\) s.t. \(\inf (a_N^+)_{N = m}^{\infty} > b_{n'}^+ \ge \inf (b_N^+)_{N = m}^{\infty}\) \MAROON{(5)}.
But given that \(n' \ge m\), by \RMK{6.3.7}(1) we have \(a_{n'}^+ \ge \inf (a_N^+)_{N = m}^{\infty}\) \MAROON{(6)}.
So by \MAROON{(5)(6)} we have \(a_{n'}^+ \ge \inf (a_N^+)_{N = m}^{\infty} > b_{n'}^+\), in particular \(a_{n'}^+ > b_{n'}^+\).
Finally, by \DEF{6.4.6}(1), that implies \(\sup(a_n)_{n = n'}^{\infty} > \sup(b_n)_{n = n'}^{\infty}\).
But again, since \(n' \ge m\), in particular from the supposition of this lemma, \(a_n \le b_n\) for all \(n \ge n'\).
And by part(a) we have \(\sup(a_n)_{n = n'}^{\infty} \le \sup(b_n)_{n = n'}^{\infty}\), so we get a contradiction.

For \BLUE{(d)}, suppose for the sake of contradiction that \(\liminf_{n \toINF} b_n < \liminf_{n \toINF} a_n\).
Then by \DEF{6.4.6}(4), we have \(\sup (b_N^-)_{N = m}^{\infty} < \sup (a_N^-)_{N = m}^{\infty}\).
By \PROP{6.3.6}(3), (with \(\sup (a_N^-)_{N = m}^{\infty}\) as \(x\), \(\sup (b_N^-)_{N = m}^{\infty}\) as \(y\),) there exists \(n' \ge m\) s.t. \(\sup (b_N^-)_{N = m}^{\infty} < a_{n'}^- \le \sup (a_N^-)_{N = m}^{\infty}\) \MAROON{(7)}.
But given that \(n' \ge m\), by \PROP{6.3.6}(1) we have \(b_{n'}^- \le \sup (b_N^-)_{N = m}^{\infty}\) \MAROON{(8)}.
So by \MAROON{(7)(8)} we have \(b_{n'}^- \le \sup (b_N^-)_{N = m}^{\infty} < a_{n'}^-\), in particular \(b_{n'}^- < a_{n'}^-\).
Finally, by \DEF{6.4.6}(3), that implies \(\inf(b_n)_{n = n'}^{\infty} < \inf(a_n)_{n = n'}^{\infty}\).
But again, since \(n' \ge m\), in particular from the supposition of this lemma, \(a_n \le b_n\) for all \(n \ge n'\).
And by part(b) we have \(\inf(a_n)_{n = n'}^{\infty} \le \inf(b_n)_{n = n'}^{\infty}\), so we get a contradiction.
\end{proof}

\begin{corollary} [Squeeze test] \label{corollary 6.4.14}
Let \((a_n)_{n = m}^{\infty}\), \((b_n)_{n = m}^{\infty}\), and \((c_n)_{n = m}^{\infty}\) be sequences of real numbers such that
\[
    a_n \le b_n \le c_n
\]
for all \(n \ge m\).
Suppose also that \((a_n)_{n = m}^{\infty}\) and \((c_n)_{n = m}^{\infty}\) both converge to the same limit \(L\).
Then \((b_n)_{n = m}^{\infty}\) is also convergent to \(L\).
\end{corollary}

\begin{proof}
\sloppy First, since both \((a_n)_{n = m}^{\infty}\) and \((c_n)_{n = m}^{\infty}\) converge to \(L\), by \PROP{6.4.12},
we have \(\limsup_{n \toINF} a_n = \liminf_{n \toINF} a_n = L\) and \(\limsup_{n \toINF} c_n = \liminf_{n \toINF} c_n = L\).
That is, \(\limsup_{n \toINF} a_n = \liminf_{n \toINF} a_n = \limsup_{n \toINF} c_n = \liminf_{n \toINF} c_n = L\).

Then since \(b_n \le c_n\) for all \(n \ge m\), by \LEM{6.4.13}(c) we have \(\limsup_{n \toINF} b_n \le \limsup_{n \toINF} c_n = L\) \MAROON{(1)}.
And since \(a_n \le b_n\) for all \(n \ge m\), by \LEM{6.4.13}(d) we have \(L = \liminf_{n \toINF} a_n \le \liminf_{n \toINF} b_n\) \MAROON{(2)}.

Also, by \PROP{6.4.12}(C), we have \(\liminf_{n \toINF} b_n \le \limsup_{n \toINF} b_n\) \MAROON{(3)}.

So by \MAROON{(1)(2)(3)}, we have \(L \le \liminf_{n \toINF} b_n \le \limsup_{n \toINF} b_n \le L\),
which, of course, implies \(\liminf_{n \toINF} b_n = \limsup_{n \toINF} b_n = L\),
which by \PROP{6.4.12}(f) implies \((b_n)_{n = m}^{\infty}\) converges to \(L\).
\end{proof}

\begin{example} \label{example 6.4.15}
We already know (see \PROP{6.1.11}) that \(\lim_{n \toINF} 1/n = 0\).
By the limit laws (\THM{6.1.19}), this also implies that \(\lim_{n \toINF} 2/n = 0\) and \(\lim_{n \toINF} -2/n = 0\). 
The \emph{squeeze test then shows that} any sequence \((b_n)_{n = 1}^{\infty}\) for which \(-2/n \le b_n \le 2/n\) for all \(n \ge 1\) is convergent to \(0\).
For instance, we can use this to show that the sequence \((-1)^n/n + 1/n^2\) converges to zero,
or that \(2^{-n}\) converges to zero.
Note one can use induction to show that \(0 \le 2^{-n} \le 1/n\) for all \(n \ge 1\).
\end{example}

\begin{remark} \label{remark 6.4.16}
The squeeze test, combined with the limit laws and the principle that monotone bounded sequences always have limits(\PROP{6.3.8}),
allows one to compute a large number of limits.
We give some examples in the next chapter.
\end{remark}

\begin{corollary} [Zero test for sequences] \label{corollary 6.4.17}
Let \((a_n)_{n = m}^{\infty}\) be a sequence
of real numbers.
Then the limit \(\lim_{n \toINF} a_n\) exists and is equal to zero if and only if the limit \(\lim_{n \toINF} \abs{a_n}\) exists and is equal to zero.
\end{corollary}

\begin{proof}
Suppose \(\lim_{n \toINF} a_n = 0\). Then
\begin{align*}
             & \lim_{n \toINF} a_n = 0 \\
    \implies & \forall \varE > 0, \exists N \ge m \text{ s.t. } \abs{a_n - 0} \le \varE\ \forall n \ge N \\
    \implies & \forall \varE > 0, \exists N \ge m \text{ s.t. } \abs{a_n} \le \varE\ \forall n \ge N \\
    \implies & \forall \varE > 0, \exists N \ge m \text{ s.t. } \abs{\abs{a_n}} \le \varE\ \forall n \ge N & \text{nasty but of course} \\
    \implies & \forall \varE > 0, \exists N \ge m \text{ s.t. } \abs{\abs{a_n} - 0} \le \varE\ \forall n \ge N \\
    \implies & \lim_{n \toINF} \abs{a_n} = 0
\end{align*}
Now suppose \(\lim_{n \toINF} \abs{a_n} = 0\). Then
\begin{align*}
             & \lim_{n \toINF} \abs{a_n} = 0 \\
    \implies & -1 \X \lim_{n \toINF} \abs{a_n} = -1 \X 0 = 0 \\
    \implies & \lim_{n \toINF} -\abs{a_n} = 0 & \text{by \THM{6.1.19}(c)}
\end{align*}
So we have \(\lim_{n \toINF} \abs{a_n} = \lim_{n \toINF} -\abs{a_n} = 0\). \\
And (trivially) \(-\abs{a_n} \le a_n \le \abs{a_n}\) for all \(n \ge m\).
So by squeeze test \CORO{6.4.14}, \(\lim_{n \toINF} a_n = 0\).
\end{proof}

We close this section with the following \emph{improvement} to \PROP{6.1.12}.

\begin{theorem} [Completeness of the reals] \label{thm 6.4.18}
A sequence \((a_n)_{n = 1}^{\infty}\) of \textbf{real} numbers is a Cauchy sequence \textbf{if and only if} it is convergent.
\end{theorem}

\begin{remark} \label{remark 6.4.19}
Note that while this is very similar in spirit to \PROP{6.1.15}, it is a bit more general, since \PROP{6.1.15} refers to Cauchy sequences of \emph{rationals} instead of real numbers.

(Self note), \PROP{6.1.12} says convergent sequences of \emph{real} are Cauchy, \PROP{6.1.15} says Cauchy sequences of \emph{rational} are convergent.
So the last piece of the puzzle is ``Cauchy sequences of \emph{real} are also convergent''.
\end{remark}

\begin{proof}
\PROP{6.1.12} already tells us that every convergent sequence(of \emph{real}) is Cauchy, so it suffices to show that every Cauchy sequence(of \emph{real}) is convergent.

Let \((a_n)_{n = 1}^{\infty}\) be any Cauchy sequence(of \emph{real} numbers).
We know from \LEM{5.1.15} (or more precisely, from the extension of this lemma to the \emph{real} numbers, which is proven in exactly the same fashion) that the sequence \((a_n)_{n = 1}^{\infty}\) is bounded;
So by \AC{6.3.1}, the supremum and infimum of the sequence are finite numbers.
And by \PROP{6.4.12}(c), this implies that \(L^- := \liminf_{n \toINF} a_n\) and \(L^+ := \limsup_{n \toINF} a_n\) of the sequence are both finite.
To show that the sequence converges, it will suffice by \PROP{6.4.12}(f) to show that \(L^- = L^+\).

Now let \(\varE > 0\) be any real number.
Since \((a_n)_{n = 1}^{\infty}\) is a Cauchy
sequence, it must be eventually \(\varE\)-steady(by \DEF{6.1.3}), so in particular there exists an \(N \ge 1\) such that the sequence \((a_n)_{n = N}^{\infty}\) is \(\varE\)-steady.
That is(again by \DEF{6.1.3}), \(\abs{a_n - a_{n'}} \le \varE\) for all \(n, n' \ge N\).
In particular, fix \(n' = N\), we have \(\abs{a_n - a_N} \le \varE\), or \(-\varE \le a_n - a_N \le \varE\), or
\begin{center}
    \(a_N - \varE \le a_n \le a_N + \varE\), \emph{for all} \(i \ge N\).
\end{center}
Then by \PROP{6.3.6}(2), we have \(\sup(a_n)_{n = N}^{\infty} \le a_N + \varE\), and by \RMK{6.3.7}(2), we have \(a_N - \varE \le \inf(a_n)_{n = N}^{\infty}\).
But by \PROP{6.4.12}(c), \(\inf(a_n)_{n = N}^{\infty} \le L^- \le L^+ \le \sup(a_n)_{n = N}^{\infty}\), together we have
\[
    a_N - \varE \le \inf(a_n)_{n = N}^{\infty} \le L^- \le L^+ \le \sup(a_n)_{n = N}^{\infty} \le a_N + \varE.
\]
In particular,
\[
    a_N - \varE \le L^- \le L^+ \le a_N + \varE.
\]
And that implies \(0 \le L^- - (a_N - \varE) \le L^+ - (a_N - \varE) \le 2\varE\).
Just for readability, in particular: \(L^+ - (a_N - \varE) \le 2\varE\) and \(L^- - (a_N - \varE) \ge 0\).
Together that implies \(L^+ - (a_N - \varE) - (L^- - (a_N - \varE)) \le 2\varE - (L^- - (a_N - \varE)) \le 2\varE\).
Simplifying, we have \(L^+ - L^- \le 2\varE\).
And by \PROP{6.4.12}(c), \(L^+ \ge L^-\), or \(0 \le L^+ - L^-\), so together we have
\[
    0 \le L^+ - L^- \le 2\varE.
\]
Since \(\varE > 0\) is arbitrary, this is true for all \(\varE > 0\), and \(L^+\) and \(L^-\) do not depend on \(\varE\);
so we must therefore have \(L^+ = L^-\).
(If \(L^+ > L^-\) then we could set \(\varE := (L^+ - L^-)/3\) such that,
\begin{align*}
             & L^+ - L^- \le 2\varE \\
    \implies & L^+ - L^- \le 2(L^+ - L^-)/3 \\
    \implies & L^+/3 \le L^-/3 \\
    \implies & L^+ \le L^-
\end{align*}
and obtain a contradiction.)
By \PROP{6.4.12}(f) we thus see that the sequence converges.
\end{proof}

\begin{remark} \label{remark 6.4.20}
In the \emph{language of metric spaces} (see \CH{B.2}), \THM{6.4.18} asserts that the real numbers are a \textbf{complete metric space}
- that they do not contain ``holes'' the same way the rationals do.
(Certainly the rationals have lots of \emph{Cauchy} sequences \textbf{which do not converge to other rationals};
take for instance the sequence \(1, 1.4, 1.41, 1.414, 1.4142,...\) which converges to the irrational \(\sqrt{2}\).) This property is \textbf{closely related to the least upper bound property} (\THM{5.5.9}),
and is one of the principal characteristics which make the real numbers superior to the rational numbers for the purposes of doing analysis
(taking limits, taking derivatives and integrals, finding zeroes of functions, that kind of thing),
as we shall see in later chapters.
\end{remark}

\exercisesection

\begin{exercise} \label{exercise 6.4.1}
Prove \PROP{6.4.5}.
\end{exercise}

\begin{proof}
See \PROP{6.4.5}.
\end{proof}

\begin{exercise} \label{exercise 6.4.2}
State and prove analogues of \EXEC{6.1.3} and \EXEC{6.1.4} for limit points, limit superior, and limit inferior.

That is, Let \((a_n)_{n=m}^{\infty}\) be a sequence of real numbers, let \(c\) be a real number, and \(m' \ge m\) be an integer, and \(k \ge 0\) be an non-negative integer. 
\begin{enumerate}
\item \(c\) is a limit point of \((a_n)_{n=m}^{\infty}\) if and only if \(c\) is a limit point of \((a_n)_{n=m'}^{\infty}\).
\item \(\limsup_{n \toINF} a_n = c\) for \((a_n)_{n = m}^{\infty}\) if and only if \(\limsup_{n \toINF} a_n = c\) for \((a_n)_{n = m'}^{\infty}\).
\item \(\liminf_{n \toINF} a_n = c\) for \((a_n)_{n = m}^{\infty}\) if and only if \(\liminf_{n \toINF} a_n = c\) for \((a_n)_{n = m'}^{\infty}\).
\item \(c\) is a limit point of \((a_n)_{n = m}^{\infty}\) if and only if \(c\) is a limit point of \((a_{n + k})_{n = m}^{\infty}\).
\item \(\limsup_{n \toINF} a_n = c\) for \((a_n)_{n = m}^{\infty}\) if and only if \(\limsup_{n \toINF} a_n = c\) for \((a_{n + k})_{n = m}^{\infty}\).
\item \(\liminf_{n \toINF} a_n = c\) for \((a_n)_{n = m}^{\infty}\) if and only if \(\liminf_{n \toINF} a_n = c\) for \((a_{n + k})_{n = m}^{\infty}\).
\end{enumerate}
\end{exercise}

\begin{proof}
\begin{enumerate}
\item
Suppose \(c\) is a limit point of \((a_n)_{n = m}^{\infty}\).
By \DEF{6.4.1}(3), (or unwrapping, by \AC{6.4.1}), \(\forall \varE > 0, \forall N \ge m, \exists n \ge N\) s.t. \(\abs{a_n - c} \le \varE\).
In particular, since \(m' \ge m\),\(\forall \varE > 0, \forall N \ge m'\), where \(m' \ge m\), \(\exists n \ge N\) s.t. \(\abs{a_n - c} \le \varE\).
Then again by \DEF{6.4.1}(3), \(c\) is a limit point of \((a_n)_{n = m'}^{\infty}\).

Now suppose \(c\) is a limit point of \((a_n)_{n = m'}^{\infty}\).
So similarly by \DEF{6.4.1}(3), \(\forall \varE > 0, \forall N \ge m', \exists n \ge N\) s.t. \(\abs{a_n - c} \le \varE\) \MAROON{(1)}.
And since \(m' \ge m\), what we left is showing \(\forall \varE > 0, \forall N'\) s.t. \(m' > N' \ge m, \exists n' \ge N'\) s.t. \(\abs{a_{n'} - c} \le \varE\) \MAROON{(2)}.
But from \MAROON{(1)}, in particular for \(N = m'\) we can find \(n' \ge N = m'\) s.t. \(\abs{a_{n'} - c} \le \varE\), and that \(n' \ge N'\);
so we can always find such \(n' \ge N'\) for \MAROON{(2)}, so together we have \(\forall \varE > 0, \forall N \ge m, \exists n \ge N\) s.t. \(\abs{a_n - c} \le \varE\), i.e. \(c\) is a limit point of \((a_n)_{n = m}^{\infty}\).

\item
We have for all \(m' \ge m\):
\begin{align*}
             & m \le m' \\
    \implies & \sup(a_n)_{n = m}^{\infty} \ge \sup(a_n)_{n = m'}^{\infty} & \text{by \AC{6.4.2}} \\
    \implies & a_m^+ \ge a_{m'}^+ \MAROON{(3)} & \text{by \DEF{6.4.6}(1)}
\end{align*}
And for all \(m' \ge m\):
\begin{align*}
             & m \le m' \land a_m^+ \ge a_{m'}^+ \\
    \implies & \inf(a_N^+)_{N = m}^{\infty} \le \inf(a_N^+)_{N = m'}^{\infty} \land a_m^+ \ge a_{m'}^+ & \text{by \AC{6.4.2}}
\end{align*}
Now we claim that \(\inf(a_N^+)_{N = m}^{\infty} = \inf(a_N^+)_{N = m'}^{\infty}\),
for if \(\inf(a_N^+)_{N = m}^{\infty} < \inf(a_N^+)_{N = m'}^{\infty}\), then there exists \(M\) s.t. \(m \le M < m'\) and \(a_M^+ < a_{m'}^+\).
(Otherwise \(\inf(a_N^+)_{N = m}^{\infty} < \inf(a_N^+)_{N = m'}^{\infty}\) cannot be true.)
But by \MAROON{(3)} we have \(a_M^+ \ge a_{m'}^+\), so we have a contradiction.
So \(\inf(a_N^+)_{N = m}^{\infty} = \inf(a_N^+)_{N = m'}^{\infty}\);
that is, by \DEF{6.4.6}(2), \(\limsup\) for \((a_n)_{n = m}^{\infty}\) is equal to \(\limsup\) for \((a_n)_{n = m'}^{\infty}\).

\item
We have for all \(m' \ge m\):
\begin{align*}
             & m \le m' \\
    \implies & \inf(a_n)_{n = m}^{\infty} \le \inf(a_n)_{n = m'}^{\infty} & \text{by \AC{6.4.2}} \\
    \implies & a_m^- \le a_{m'}^- \MAROON{(4)} & \text{by \DEF{6.4.6}(3)}
\end{align*}
And for all \(m' \ge m\):
\begin{align*}
             & m \le m' \land a_m^- \le a_{m'}^- \\
    \implies & \sup(a_N^-)_{N = m}^{\infty} \ge \sup(a_N^-)_{N = m'}^{\infty} \land a_m^- \le a_{m'}^- & \text{by \AC{6.4.2}}
\end{align*}
Now we claim that \(\sup(a_N^-)_{N = m}^{\infty} = \sup(a_N^-)_{N = m'}^{\infty}\),
for if \(\sup(a_N^-)_{N = m}^{\infty} > \sup(a_N^-)_{N = m'}^{\infty}\), then there exists \(M\) s.t. \(m \le M < m'\) and \(a_M^- > a_{m'}^-\).
(Otherwise \(\sup(a_N^-)_{N = m}^{\infty} > \sup(a_N^-)_{N = m'}^{\infty}\) cannot be true.)
But by \MAROON{(4)} we have \(a_M^- \le a_{m'}^-\), so we have a contradiction.
So \(\sup(a_N^-)_{N = m}^{\infty} = \sup(a_N^-)_{N = m'}^{\infty}\);
that is, by \DEF{6.4.6}(4), \(\liminf\) for \((a_n)_{n = m}^{\infty}\) is equal to \(\liminf\) for \((a_n)_{n = m'}^{\infty}\).

\item
Since \((a_{n + k})_{n = m}^{\infty} = (a_n)_{n = m + k}^{\infty}\) \MAROON{(5)} so for \(m' = m + k\),
by part (a) we conclude that \(c\) is a limit point of \((a_n)_{n = m}^{\infty}\)
(iff \(c\) is a limit point of \((a_n)_{n = m + k}^{\infty}\),)
iff \(c\) is a limit point of \((a_{n + k})_{n = m}^{\infty}\).

\item
Again by \MAROON{(5)}, for \(m' = m + k\),
by part (b) we conclude that \(c\) is the \(\limsup\) for \((a_n)_{n = m}^{\infty}\)
(iff \(c\) is the \(\limsup\) for\((a_n)_{n = m + k}^{\infty}\),)
iff \(c\) is the \(\limsup\) for \((a_{n + k})_{n = m}^{\infty}\).

\item
Again by \MAROON{(5)}, for \(m' = m + k\),
by part (c) we conclude that \(c\) is the \(\liminf\) for \((a_n)_{n = m}^{\infty}\)
(iff \(c\) is the \(\liminf\) for\((a_n)_{n = m + k}^{\infty}\),)
iff \(c\) is the \(\liminf\) for \((a_{n + k})_{n = m}^{\infty}\).
\end{enumerate}
\end{proof}

\begin{exercise} \label{exercise 6.4.3}
Prove parts (c), (d), (e), (f) of \PROP{6.4.12}.
(Hint: you can use earlier parts of the proposition to prove later ones.)
\end{exercise}

\begin{proof}
See \PROP{6.4.12}.
\end{proof}

\begin{exercise} \label{exercise 6.4.4}
Prove \LEM{6.4.13}.
\end{exercise}

\begin{proof}
See \LEM{6.4.13}.
\end{proof}

\begin{exercise} \label{exercise 6.4.5}
Use \LEM{6.4.13} to prove \CORO{6.4.14}.
\end{exercise}

\begin{proof}
See \CORO{6.4.14}.
\end{proof}

\begin{exercise} \label{exercise 6.4.6}
Give an example of two \emph{bounded} sequences \((a_n)_{n = 1}^{\infty}\) and \((b_n)_{n = 1}^{\infty}\) such that \(a_n < b_n\) for all \(n \ge 1\),
but that \(\sup(a_n)_{n = 1}^{\infty} \not < \sup(b_n)_{n = 1}^{\infty}\).
Explain why this does not contradict \LEM{6.4.13}.
\end{exercise}

\begin{proof}
Let \(a_n := 1/n\), \(b_n := 2/n\) for all \(n \ge 1\).
Then \(a_n < b_n\) for all \(n \ge 1\).
But by what have shown in \EXAMPLE{6.4.15}, \(\lim_{n = 1}^{\infty} (a_n) = \lim_{n = 1}^{\infty} = 0\) so \(\lim_{n = 1}^{\infty} (a_n) \not < \lim_{n = 1}^{\infty}\).

But this does not contradict \LEM{6.4.13}, since the lemma does not say anything for ``strict less than'' relation.
\end{proof}

\begin{exercise} \label{exercise 6.4.7}
Prove \CORO{6.4.17}. Is the corollary still true if we replace zero in the statement of this Corollary by some other number?
\end{exercise}

\begin{proof}
See \CORO{6.4.17}.
Also, zero cannot be replaced by another real number \(c\).
because given that \(c\), we have a counter example:
\(\lim_{n \toINF} \abs{c \X (-1)^n)} = \lim_{n \toINF} (\abs{c} \X \abs{(-1)^n}) =  \abs{c}\),
but \(\lim_{n \toINF} c \X (-1)^n = c \X \lim_{n \toINF} (-1)^n\), which is undefined.
\end{proof}

\begin{exercise} \label{exercise 6.4.8}
Let us say that a sequence \((a_n)_{n = m}^{\infty}\) of real numbers has \(+\infty\) \emph{as a limit point} iff it has no finite upper bound,
and that it has \(-\infty\) as a limit point iff it has no finite lower bound.
With this definition, show that \(\limsup_{n \toINF} a_n\) is a limit point of \((a_n)_{n = m}^{\infty}\), and furthermore that it is larger than all the other limit points of \((a_n)_{n = m}^{\infty}\);
in other words, the limit superior is the largest limit point of a sequence.
Similarly, show that the limit inferior is the smallest limit point of a sequence.
(One can use \PROP{6.4.12} in the course of the proof.)
(\RED{warning}: To prove this exercise, any statements that have been proved \RED{but depend on the limit point in terms of \DEF{6.4.1}(3)} can not be used, or should be used with limitation,
because the ``limit points'' in these statements are only finite.)
\end{exercise}

\begin{proof}
For the case that the sequence is bounded above(below), by \AC{6.3.1}, the supremum(infimum) is finite, so by \PROP{6.4.12}(c), \(L^+\)(\(L^-\)) is finite,
and by \PROP{6.4.12}(e) we know they are limit points, and by \PROP{6.4.12}(d) we know they are largest and smallest limit points, respectively.

So we only need to show the case when the sequence is not bounded above/below.

Suppose \((a_n)_{n = m}^{\infty}\) has no upper bound, then by definition of this exercise, it has \(+\infty\) as a limit point.
By \DEF{6.2.3}(b), any limit point, which is a \emph{extended} real number, is less than or equal to \(+\infty\).
Now what we left is showing \(L^+\) is also \(\ge\) any limit point, including \(+\infty\);
that is, \(L^+ = +\infty\).
For the sake of contradiction, suppose \(L^+\) is finite.
Then \(L^+ + 1\) is also finite, and \(L^+ < L^+ + 1\).
So, by \PROP{6.4.12}(a), there exists \(N \ge m\) s.t. \(a_n < L^+ + 1\) \emph{for all} \(n \ge N\),
which implies \(a_n\) \emph{eventually} cannot exceed a finite number \(L^+ + 1\),
which contradicts that \((a_n)_{n = m}^{\infty}\) has no upper bound!.
So \(L^+ = +\infty\), so \(L^+ \ge\) any limit points.

Now suppose \((a_n)_{n = m}^{\infty}\) has no lower bound, then by definition of this exercise, it has \(-\infty\) as a limit point.
By \DEF{6.2.3}(c), any limit point, which is a \emph{extended} real number, is greater than or equal to \(-\infty\).
Now what we left is showing \(L^-\) is also \(\le\) any limit point, including \(-\infty\);
that is, \(L^- = -\infty\).
For the sake of contradiction, suppose \(L^-\) is finite.
Then \(L^+ - 1\) is also finite, and \(L^- - 1 < L^-\).
So, by \PROP{6.4.12}(a), there exists \(N \ge m\) s.t. \(L^- - 1 < a_n\) \emph{for all} \(n \ge N\),
which implies \(a_n\) \emph{eventually} cannot less than a finite number \(L^- - 1\), which contradicts that \((a_n)_{n = m}^{\infty}\) has no lower bound!.
So \(L^- = -\infty\), so \(L^- \le\) any limit points.
\end{proof}

\begin{exercise} \label{exercise 6.4.9}
Using the definition in \EXEC{6.4.8}, construct a sequence \((a_n)_{n = 1}^{\infty}\) which has exactly three limit points, at \(-\infty\), \(0\), and \(+\infty\).
\end{exercise}

\begin{proof}
Let
\begin{equation} \label{eq 6.1}
    a_n =
    \begin{cases}
      n, & \text{ if } n = 3k + 1 \text{ for some integer } k \\
      1/n, & \text{ if } n = 3k + 2 \text{ for some integer } k \\
      -n, & \text{ if } n = 3k \text{ for some integer } k
    \end{cases}
\end{equation}
Then by basic number theory, the definition is well-defined.

Furthermore, it is trivial to show that \(-\infty, 0, +\infty\) are limit points of this sequence by \DEF{6.4.1} and definition of \EXEC{6.4.8}.
\end{proof}

\begin{exercise}
\label{exercise 6.4.10}
Let \((a_n)_{n = N}^{\infty}\) be a sequence of real numbers, and let \((b_m)_{m = M}^{\infty}\) be another sequence of real numbers such that \emph{each \(b_m\) (may repeat) is a limit point of} \((a_n)_{n = N}^{\infty}\).
Let \(c\) be a limit point of \((b_m)_{m = M}^{\infty}\).
Prove that \(c\) is also a limit point of \((a_n)_{n = N}^{\infty}\).
(In other words, limit points of sequence of limit points of the original sequence are themselves limit points of the original sequence.)
\end{exercise}

\begin{proof}
Suppose \(c\) is a limit point of \((b_m)_{m = M}^{\infty}\), we have to show that \(c\) is also a limit point of \((a_n)_{n = N}^{\infty}\).
That is, by \DEF{6.4.1}, we have to show that \(\forall \varE > 0, \forall N' \ge N, \exists n \ge N'\) s.t. \(\abs{a_n - c} \le \varE\).

So let arbitrary \(\varE > 0\) and arbitrary \(N' \ge N\) \BLUE{(1)}.
In particular, \(\varE/2 > 0\).
And since \(c\) is a limit point of \((b_m)_{m = M}^{\infty}\), so again by \DEF{6.4.1}, \(\forall M' \ge M, \exists m \ge M'\) s.t. \(\abs{b_m - c} \le \varE/2\) \MAROON{(1)}.
We preserve that \(m \ge M'\) for later use.
In particular \(b_m\) is a limit point of \((a_n)_{n = N}^{\infty}\).
So again by \DEF{6.4.1}, given the particular \(N' \ge N\) from \BLUE{(1)}, \(\exists n \ge N'\) s.t. \(\abs{a_n - b_m} \le \varE/2\). \MAROON{(2)}
Again we preserve that \(n \ge N'\) for later use.
Then we have
\begin{align*}
    \abs{a_n - c} & = \abs{a_n - c + b_m - b_m} \\
                  & = \abs{(a_n - b_m) + (b_m - c)} \\
                  & \le \abs{a_n - b_m} + \abs{b_m - c} \\
                  & \le \varE/2 + \varE/2 & \text{by \MAROON{(1)(2)}} \\
                  & = \varE
\end{align*}
So all in all, from the given \(\varE > 0\) and \(N' \ge N\), we have found \(n \ge N'\) s.t. \(\abs{a_n - c} \le \varE\), so by \DEF{6.4.1}, \(c\) is a limit point of \((a_n)_{n = N}^{\infty}\).
\end{proof}
