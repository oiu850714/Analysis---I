\section{The construction of the real numbers} \label{sec 5.3}

\begin{definition} [Real numbers] \label{def 5.3.1}
A \emph{real number} is defined to be an \emph{object} of the \emph{form} \(\LIM_{n \toINF} a_n\),
where \((a_n)_{n = 1}^{\infty}\) is a Cauchy sequence of \emph{rational} numbers.
Two real numbers \(\LIM_{n \toINF} a_n\) and \(\LIM_{n \toINF} b_n\) are said to be \emph{equal} iff \((a_n)_{n = 1}^{\infty}\) and \((b_n)_{n = 1}^{\infty}\) are equivalent Cauchy sequences.
The set of all real numbers is denoted \(\SET{R}\).
\end{definition}

\begin{example} [Informal] \label{example 5.3.2}
Let \(a_1, a_2, a_3, ...\) denote the sequence
\[
    1.4, 1.41, 1.414, 1.4142, 1.41421,...
\]
and let \(b_1, b_2, b_3,...\) denote the sequence
\[
    1.5, 1.42, 1.415, 1.4143, 1.41422,...
\]
then \(\LIM_{n \toINF} a_n\) is a real number, and is \emph{the same} real number as \(\LIM_{n \toINF} b_n\),
because \((a_n)_{n = 1}^{\infty}\) and \((b_n)_{n = 1}^{\infty}\) are equivalent Cauchy sequences:
\(\LIM_{n \toINF} a_n = \LIM_{n \toINF} b_n\).
\end{example}
\begin{note}
到目前為止,我們還沒有證明:
\begin{itemize}
    \item 這兩個\ sequence 都是\ Cauchy sequences。
    \item 進一步來說,也無法知道他們對應的\ \LIM\ object 是\ real number。
    \item 所以也無法知道討論他們是否相等。
\end{itemize}
而且開天眼,這整節看完之後我們好像還是不能知道他們是\ Cauchy...。
\end{note}

\begin{note}
We will refer to \(\LIM_{n \toINF} a_n\) as the \emph{formal limit} of the sequence \((a_n)_{n = 1}^{\infty}\).
Later on we will define a genuine notion of limit, and show that the formal limit of a Cauchy sequence is the same as the limit of that sequence;
after that, we will not need formal limits ever again.
(The situation is much like what we did with \emph{formal} subtraction \(\M\) and \emph{formal} division \(\D\).)
\end{note}

\begin{note}
we need to check that the notion of equality in the definition obeys the first three laws of
equality:
\end{note}

\begin{proposition} [Formal limits are well-defined] \label{prop 5.3.3}
Let \(x = \LIM_{n \toINF} a_n,\ y = \LIM_{n \toINF} b_n\), and \(z = \LIM_{n \toINF} c_n\) be real numbers.
(This implies the corresponding sequences are Cauchy.)
Then, with the above \DEF{5.3.1} of equality for real numbers, we have \(x = x\).
Also, if \(x = y\), then \(y = x\).
Finally, if \(x = y\) and \(y = z\), then \(x = z\).
\end{proposition}

\begin{proof}
Reflexive: Since given \(\varE > 0\), \(\varE > 0 = d(a_n, a_n)\) for all \(n \ge 1\), by \DEF{5.2.6}, \((a_n)_{n = 1}^{\infty}\) and \((a_n)_{n = 1}^{\infty}\) are equivalent.
And by \DEF{5.3.1}, \(\LIM_{n \toINF} a_n = \LIM_{n \toINF} a_n\), that is, \(x = x\).

Symmetric: Suppose \(x = y\).
By \DEF{5.3.1}, \((a_n)_{n = 1}^{\infty}\) and \((b_n)_{n = 1}^{\infty}\) are equivalent.
And by \DEF{5.2.6}, For all \(\varE > 0\), there exists \(N \ge 1\) s.t. for all \(n \ge N\), \(d(a_n, b_n) \le \varE\) \MAROON{(1)}.
But by \PROP{4.3.3}(f), \(d(a_n, b_n) = d(b_n, a_n)\).
So with \MAROON{(1)}, we have For all \(\varE > 0\), there exists \(N \ge 1\) s.t. for all \(n \ge N\), \(\MAROON{d(b_n, a_n)} \le \varE\).
By \DEF{5.2.6}, \((b_n)_{n = 1}^{\infty}\) and \((a_n)_{n = 1}^{\infty}\) are equivalent.
By \DEF{5.3.1}, \(\LIM_{n \toINF} b_n = \LIM_{n \toINF} b_n\), that is, \(y = x\).

Transitive: Suppose \(x = y\) and \(y = z\).
By \DEF{5.3.1}, \((a_n)_{n = 1}^{\infty}, (b_n)_{n = 1}^{\infty}\) are equivalent and \((b_n)_{n = 1}^{\infty}, (c_n)_{n = 1}^{\infty}\) are equivalent.
Now given arbitrary \(\varE > 0 \), so \emph{in particular \(\varE/2 > 0\)}, we have
\begin{itemize}
    \item There exists \(N_1 \ge 1\) s.t. for all \(n \ge N_1\), \(d(a_n, b_n) \le \varE/2\) \MAROON{(2)}.
    \item There exists \(N_2 \ge 1\) s.t. for all \(n \ge N_2\), \(d(b_n, c_n) \le \varE/2\) \MAROON{(3)}.
\end{itemize}
Let \(N_3 := max(N_1, N_2)\), then by \MAROON{(2) (3)}, we have for all \(n \ge N_3\), \(d(a_n, b_n) \le \varE/2\) and \(d(b_n, c_n) \le \varE/2\).
But by \DEF{4.3.4}, that implies \(a_n, b_n\) and \(b_n, c_n\) are \(\varE/2\)-close.
By \PROP{4.3.7}, we have \(a_n, c_n\) are \(\varE/2 + \varE/2 = \varE\)-close.
By \DEF{4.3.4}, \(d(a_n, c_n) \le \varE\).
So for all \(n \ge N_3\), \(d(a_n, c_n) \le \varE\).
By \DEF{5.2.6}, \((a_n)_{n = 1}^{\infty}, (c_n)_{n = 1}^{\infty}\) are equivalent.
By \DEF{5.3.1}, \(\LIM_{n \toINF} a_n = \LIM_{n \toINF} c_n\), that is, \(x = z\).
\end{proof}

\begin{note}
By \PROP{5.3.3}, we now have well-defined equality between two real numbers.
Of course, when we define other operations on the reals, \emph{we have to check that they obey the law of substitution} \AXM{a.7.4}:
two real number inputs which are \emph{equal} should give equal outputs when applied to any operation on the real numbers.

Now we want to define all the usual arithmetic operations on the real numbers, such as addition and multiplication. 
\end{note}

\begin{definition} [Addition of reals] \label{def 5.3.4} 
Let \(x = \LIM_{n \toINF} a_n\) and \(y = \LIM_{n \toINF} b_n\) be real numbers.
Then we \emph{define} the \emph{sum} \(x + y\) to be \(x + y := \LIM_{n \toINF} (a_n + b_n)\).
\end{definition}

\begin{example} \label{example 5.3.5}
The sum of \(\LIM_{n \toINF} 1 + 1/n\) and \(\LIM_{n \toINF} 2 + 3 / n\) is \(\LIM_{n \toINF} 3 + 4/n\).
\end{example}

We now check that \DEF{5.3.4} is valid.
The first thing we need to do is to confirm that the sum of two real numbers is in fact a real number:

\begin{lemma} [Sum of Cauchy sequences is Cauchy] \label{lem 5.3.6}
Let \(x = \LIM_{n \toINF} a_n\) and \(y = \LIM_{n \toINF} b_n\) be real numbers.
Then \(x + y\) is also a real number. (i.e., \((a_n + b_n)_{n = 1}^{\infty}\) is a Cauchy sequence of rationals.)
\end{lemma}


\begin{proof}
We need to show that for every \(\varE > 0\), the sequence \((a_n + b_n)_{n = 1}^{\infty}\) is eventually \(\varE\)-steady.
Now from hypothesis and \DEF{5.3.1} we know that \((a_n)_{n = 1}^{\infty}\) is Cauchy and hence by \DEF{5.1.8} is eventually \(\varE\)-steady, and similarly \((b_n)_{n = 1}^{\infty}\) is eventually \(\varE\)-steady;
but it turns out that this is not quite enough
(this can be used to imply that \((a_n + b_n)_{n = 1}^{\infty}\) is eventually \(2\varE\)-steady, but that's not what we want).
So we need to do a little trick, which is to play with the value of \(\varE\).

We know that \((a_n)_{n = 1}^{\infty}\) is eventually \(\delta\)-steady for every value of \(\delta > 0\).
This implies not only that \((a_n)_{n = 1}^{\infty}\) is eventually \(\varE\)-steady, but it is also eventually \(\varE / 2\)-steady(just in particular let \(\delta = \varE/2\)).
Similarly, the sequence \((b_n)_{n = 1}^{\infty}\) is also eventually \(\varE / 2\)-steady.
This will turn out to be enough to conclude that \((a_n + b_n)_{n = 1}^{\infty}\) is eventually \(\varE\)-steady.

Since \((a_n)_{n = 1}^{\infty}\) is eventually \(\varE / 2\)-steady, we know that there exists an \(N \geq 1\) such that \((a_n)_{n = N}^{\infty}\) is \(\varE / 2\)-steady, i.e., \(a_n\) and \(a_m\) are \(\varE / 2\)-close for every \(n, m \geq N\).
Similarly there exists an \(M \geq 1\) such that \((b_n)_{n = M}^{\infty}\) is \(\varE / 2\)-steady, i.e., \(b_n\) and \(b_m\) are \(\varE / 2\)-close for every \(n, m \geq M\).

Let \(\max(N, M)\) be the larger of \(N\) and \(M\)
(we know from \PROP{2.2.13} that one has to be greater than or equal to the other).
If \(n, m \geq \max(N, M)\), then we know that \(a_n\) and \(a_m\) are \(\varE / 2\)-close, and \(b_n\) and \(b_m\) are \(\varE / 2\)-close, and so by \PROP{4.3.7}(d) we see that \(a_n + b_n\) and \(a_m + b_m\) are \(\varE/2 + \varE/2 = \varE\)-close for every \(n, m \geq \max(N, M)\).
This implies that the sequence \((a_n + b_n)_{n = 1}^{\infty}\) is eventually \(\varE\)-steady, as desired.
\end{proof}

The other thing we need to check is the axiom of substitution \AXM{a.7.4}:
if we replace a real number \(x\) by another number \(x'\) \emph{equal} to \(x\),  then we must have \(x' + y = x + y\)
(and similarly if we substitute \(y\) by another number \(y'\) equal to \(y\)).

\begin{lemma} [Sums of equivalent Cauchy sequences are equivalent]\label{lem 5.3.7}.
Let \(x = \LIM_{n \toINF} a_n, y = \LIM_{n \toINF} b_n\), and \(x' = \LIM_{n \toINF} a'_n\) be real numbers.
Suppose that \(x = x'\).
Then we have \(x + y = x' + y\).
\end{lemma}

\begin{proof}
We need to show \(x + y = x' + y\), 
that is by \DEF{5.3.4} \(\LIM_{n \toINF} (a_n + b_n) = \LIM_{n \toINF} (a'_n + b_n)\), 
that is by \DEF{5.3.1}, the sequences \((a_n + b_n)_{n = 1}^{\infty}\) and \((a'_n + b_n)_{n = 1}^{\infty}\) are eventually \(\varE\)-close for each \(\varE > 0\) \MAROON{(1)}. 
But since \(x = x'\), we know that the Cauchy sequences \((a_n)_{n = 1}^{\infty}\) and \((a'_n)_{n = 1}^{\infty}\) are equivalent, by similar definition derivation we know that there is an \(N \ge 1\) such that \((a_n)_{n = N}^{\infty}\) and \((a'_n)_{n = N}^{\infty}\) are \(\varE\)-close,
i.e., that \(a_n\) and \(a'_n\) are \(\varE\)-close for each \(n \ge N\).
Since \(b_n\) is of course \(0\)-close to \(b_n\)
(where we extend the notion of \(\varE\)-closeness(The author seems to mean \DEF{4.3.4}, \(\varE\)-closeness for two numbers) to include \(\varE=0\) in the obvious fashion),
we thus see from \PROP{4.3.7}(d)
(extended in the obvious manner to the \(\delta =  0\) case, or extended to cover the \(0\)-close case)
that \(a_n + b_n\) and \(a'_n + b_n\) are \(\varE + 0 = \varE\)-close for each \(n \ge N\).
This implies \MAROON{(1)} is true, and we are done.
\end{proof}

\begin{remark} \label{remark 5.3.8}
The above lemma verifies the axiom of substitution \AXM{a.7.4} for the ``\(x\)'' variable in \(x + y\),
but one can similarly prove the \AXM{a.7.4} for the ``\(y\)'' variable.
(A quick way is to observe from the definition of \(x + y\) that we certainly have \(x + y = y + x\), since \(a_n + b_n\) = \(b_n + a_n\).)
\end{remark}

We can define multiplication of real numbers in a manner similar to that of addition:

\begin{definition} [Multiplication of reals] \label{def 5.3.9}
Let \(x = \LIM_{n \toINF} a_n\) and \(y = \LIM_{n \toINF} b_n\) be real numbers.
Then we \emph{define} the product \(xy\) to be \(xy :=  \LIM_{n \toINF} a_n b_n\).
\end{definition}

\begin{proposition} [Multiplication is well defined] \label{prop 5.3.10}
Let \(x = \LIM_{n \toINF} a_n, y = \LIM_{n \toINF} b_n\), and \(x' = \LIM_{n \toINF} a'_n\) be real numbers.
Then \(xy\) is also a real number.
Furthermore, if \(x = x'\), then \(xy = x'y\). 
\end{proposition}

\begin{proof}
\begin{itemize}
    \item
        First we show \(xy\) is real;
        that is, \((a_n b_n)_{n = 1}^{\infty}\) is Cauchy;
        that is, for all \(\varE > 0\), there exists \(N \ge 1\) s.t. \(d(a_i b_i, a_j b_j) \le \varE\) for all \(i, j \ge N\).
        
        So let \(3 > \varE > 0\). \BLUE{(trick 1)}
        (The purpose of the upper bound is to make some trick in the latter proof).
        It's trivial that if \((a_n b_n)_{n = 1}^{\infty}\) is eventually \(\varE\)-close, then given any \(\delta \ge 3\), \((a_n b_n)_{n = 1}^{\infty}\) is also \(\delta\)-close.
        (By \PROP{4.3.7}(e) and the definition of closeness of the sequences and rational numbers.)
        So together we can conclude \((a_n b_n)_{n = 1}^{\infty}\) is eventually \(\varE\)-close for all \(\varE > 0\).
        
        Since \((a_n)_{n = 1}^{\infty}\) and \((b_n)_{n = 1}^{\infty}\) are Cauchy, by \LEM{5.1.15} they are bounded by some rational \(M_1, M_2 \ge 0\),
        that is, \(\abs{a_n} \le M_1\) and \(\abs{b_n} \le M_2\) for all \(n \ge 1\).
        Now let \(M = max(M_1, M_2, 1)\), and that trivially implies \(\abs{a_n} \le M\) \MAROON{(1)} and \(\abs{b_n} \le M\) \MAROON{(2)} for all \(n \ge 1\).
        (Note that \(M \ge 1\), this is also used as a trick in the latter proof \BLUE{(trick 2)}).
        
        And since \((a_n)_{n = 1}^{\infty}\) and \((b_n)_{n = 1}^{\infty}\) are Cauchy, there exists \(N_1 \ge 1\) s.t. \(d(a_i , a_j) \le \varE\) for all \(i, j \ge N_1\),
        and there exists \(N_2 \ge 1\) s.t. \(d(b_i , b_j) \le \varE\) for all \(i, j \ge N_2\).
        Let \(N = max(N_1, N_2)\), then \(d(a_i , a_j) \le \varE\) and \(d(b_i, b_j)\le \varE\) for all \(i, j \ge N\).
        And by \PROP{4.3.7}(h), we have \(d(a_i b_i, a_j b_j) \le \varE\abs{b_i} + \varE\abs{a_i} + \varE^2\).
        And by \MAROON{(1) (2)}
        \begin{align*}
                & \varE\abs{b_i} + \varE\abs{a_i} + \varE^2 \\
            \le &  \varE M + \varE M + \varE^2 \\
              = & \varE(2M + \varE),
        \end{align*}
        so we have \(d(a_i b_i, a_j b_j) \le \varE(2M + \varE)\) for all \(i, j \ge N\) \MAROON{(3)}.
        
        \sloppy Now let \(\varE' = \frac{\varE}{3M}\), then by similar argument until \MAROON{(3)},
        there exists \(N_3 \ge 1\) s.t. for all \(i, j \ge N_3\).
        \begin{align*}
            d(a_i b_i, a_j b_j) & \le \varE'(2M + \varE') \\
                                & = \frac{\varE}{3M}(2M + \frac{\varE}{3M}) \\
                                & = \frac{2\varE}{3} + \frac{\varE^2}{9M^2} \\
                                & \le \frac{2\varE}{3} + \frac{\varE^2}{9\X1^2} & \text{since \(M \ge 1\), trick \BLUE{(2)}} \\
                                & = \frac{2\varE}{3} + \frac{\varE^2}{9} \\
                                & = \frac{2\varE}{3} + \varE \X \frac{\varE}{9} \\
                                & \le \frac{2\varE}{3} + 3 \X \frac{\varE}{9} & \text{since \(\varE < 3\), trick \BLUE{(1)}} \\
                                & = \frac{2\varE}{3} + \frac{\varE}{3} \\
                                & = \varE
        \end{align*}
        By \DEF{5.1.6}, \((a_n b_n)_{n = 1}^{\infty}\) is eventually \(\varE\)-close.
        Since \(\varE\) is arbitrary between \(0\) and \(3\), \((a_n b_n)_{n = 1}^{\infty}\) is eventually \(\varE\)-close for all \(0 < \varE < 3\),
        and by the previous discussion it's trivial that \((a_n b_n)_{n = 1}^{\infty}\) is eventually \(\varE\)-close for all \(\varE > 0\).
        So by \DEF{5.1.8}, \((a_n b_n)_{n = 1}^{\infty}\) is Cauchy.
    \item
        Now we prove if \(x = x'\), then \(xy = x'y\).
        Since \(x = x'\), \((a_n)_{n = 1}^\infty\) and \((a'_n)_{n = 1}^\infty\) are equivalent sequences.
        Since \((b_n)_{n = 1}^\infty\) is Cauchy, there is some number \(M' \ge 0\) which bounds it.
        Now let \(M = max(M', 1)\), then it's trivial that \(M\) also bounds \((b_n)_{n = 1}^\infty\) and \(M\) is positive.

        Now let \(\varE > 0\), so in particular \(\varE/M > 0\).
        Then since \((a_n)_{n=1}^\infty\) and \((a'_n)_{n=1}^\infty\) are equivalent sequences, they are eventually \(\varE/M\)-close.
        So there exists \(N \ge 1\) s.t. \(a_i, a'_i\) are \(\varE/M\)-close for all \(i \ge N\).
        And by \PROP{4.3.7}(g), \(a_i b_i, a'_i b_i\) are \((\varE/M)\abs{b_i}\)-close for all \(i \ge N\).
        But since \(M\) bounds \((b_n)_{n = 1}^\infty\), we have \(\abs{b_i} \le M\), so \((\varE/M)\abs{b_i} \le (\varE/M)M = \varE\).
        So \(a_i b_i, a'_i b_i\) are \(\varE\)-close for all \(i \ge N\).
        So \((a_n b_n)_{n = 1}^\infty\) and \((a'_n b_n)_{n = 1}^\infty\) is eventually \(\varE\)-close.
        Since \(\varE > 0\) is arbitrary, by \DEF{5.2.6} \((a_n b_n)_{n = 1}^\infty\) and \((a'_n b_n)_{n = 1}^\infty\) are equivalent, that is, \(xy = x'y\).
\end{itemize}

\end{proof}

Of course we can prove a similar substitution rule when \(y\) is replaced by a real number \(y'\) which is equal to \(y\).

\begin{note}
At this point we \emph{embed the rationals back into the reals}, by equating every \emph{rational} number \(q\) with the \emph{real} number \(\LIM_{n \toINF} q\).
(That is, \(q \equiv \LIM_{n \toINF} q\).)
For instance, if \(a_1, a_2, a_3, ...\) is the sequence
\[
	0.5, 0.5, 0.5, 0.5, 0.5,...
\]
Then we let \(\LIM_{n \toINF} a_n\) equal to \(0.5\).

This embedding is consistent with our definitions of addition and multiplication, since for any rational numbers \(a, b\) we have
\begin{align*}
	a + b & = \LIM_{n \toINF} a + \LIM_{n \toINF} b & \text{by equating real and rational} \\
          & = \LIM_{n \toINF}(a + b) & \text{by \DEF{5.3.4}} \\
          & = a + b & \text{by equating again}
\end{align*}
And
\begin{align*}
	ab & = \LIM_{n \toINF} a \X \LIM_{n \toINF} b & \text{by equating} \\
          & = \LIM_{n \toINF}(ab) & \text{by \DEF{5.3.9}} \\
          & = ab & \text{by equating}
\end{align*}
(Note that the element \(a\) or \(b\) of the formula \(\LIM_{n \toINF} a\) or \(\LIM_{n \toINF} b\) is independent of the index \(n\).)

This means that when one wants to add or multiply two rational numbers \(a, b\) it does \emph{not} matter whether one thinks of these numbers as \emph{rationals} or as the \emph{real} numbers \(\LIM_{n \toINF} a, \LIM_{n \toINF} b\).
Also, this identification of rational numbers and real numbers is consistent with our definitions of equality(\EXEC{5.3.3}).
\end{note}

We can now easily define negation \(-x\) for real numbers \(x\) by the formula
\[
    -x := (-1) \X x,
\]
since \(-1\) is a rational number and is hence real.
(This definition is dependent on \DEF{5.3.9}, and we give it a label: \DEF{5.3.18}.)
Note that this is clearly consistent with our \emph{negation for rational} numbers since we have \(-q = (-1) \X q\) (by \AC{4.2.3}) for all rational numbers \(q\).
Also, from our definitions it is clear (why? see below) that
\begin{align*}
      & -(\LIM_{n \toINF} a_n) \\
    = & (-1) \X (\LIM_{n \toINF} a_n) & \text{by \DEF{5.3.18}, negation} \\
    = & \LIM_{n \toINF} (-1) \X \LIM_{n \toINF} a_n & \text{by equating rational and real} \\
    = & \LIM_{n \toINF} (-1) a_n & \text{by \DEF{5.3.9}} \\
    = & \LIM_{n \toINF} (-a_n)  & \text{by \AC{4.2.3}}
\end{align*}

Once we have addition and negation, we can define subtraction as usual by
\[
    x - y = x + (-y),
\]
(We label it as \DEF{5.3.19}.) Note that
\begin{align*}
      & \LIM_{n \toINF} a_n - \LIM_{n \toINF} b_n \\
    = & \LIM_{n \toINF} a_n + (-\LIM_{n \toINF} b_n) & \text{by \DEF{5.3.19}} \\
    = & \LIM_{n \toINF} a_n + \LIM_{n \toINF} (-b_n) & \text{shown above} \\
    = & \LIM_{n \toINF} (a_n + (-b_n)) & \text{by \DEF{5.3.4}} \\
    = & \LIM_{n \toINF} (a_n - b_n) & \text{by \DEF{4.2.13}, subtraction of rationals}
\end{align*}

We can now easily show that the \emph{real} numbers obey all the usual rules of algebra (\emph{except} perhaps for the laws involving \emph{division}
(since we currently have not defined the \emph{reciprocal} of the \emph{reals}), which we shall address shortly):

\begin{proposition} \label{prop 5.3.11}
All the laws of algebra from \PROP{4.1.6} (not \PROP{4.2.4}, not defining division yet) hold not only for the integers, but for the reals as well.
\end{proposition}

\begin{proof}
(The proof use the algebra laws of \emph{rationals}.)
We illustrate this with one such rule: \(x(y + z) = xy + xz\).
Let \(x = \LIM_{n \toINF} a_n\), \(y = \LIM_{n \toINF} b_n\),
and \(z = \LIM_{n \toINF} c_n\) be real numbers.
Then by \DEF{5.3.9}, \(xy = \LIM_{n \toINF} a_n b_n\) and \(xz = \LIM_{n \toINF} a_n c_n\),
and so by \DEF{5.3.4} \(xy + xz = \LIM_{n \toINF} (a_n b_n + a_n c_n)\).
A similar line of reasoning shows that \(x(y + z) = \LIM_{n \toINF} a_n (b_n + c_n)\).
But we already know that \(a_n (b_n + c_n)\) is equal to \(a_n b_n + a_n c_n\) for the \emph{rational} numbers \(a_n, b_n, c_n\), and the claim follows.
The other laws of algebra are proven similarly.
\end{proof}

The last basic arithmetic operation we need to define is reciprocation: \(x \to x^{-1}\).
This one is a little \emph{more subtle}.
On obvious first guess for how to proceed would be define
\[
    (\LIM_{n \toINF} a_n)^{-1} := \LIM_{n \toINF} a^{-1},
\]
but there are \emph{a few problems} with this.
For instance, let \(a_1, a_2, a_3,...\) be the \emph Cauchy sequence
\[
    0.1, 0.01, 0.001, 0.0001,...,
\]
and let \(x := \LIM_{n \toINF} a_n\).
Then by this definition, \(x^{-1}\) would be \(\LIM_{n \toINF} b_n\), where \(b_1, b_2, b_3,...\) is the sequence
\[
    10, 100, 1000, 10000,...
\]
but this is \emph{not} a Cauchy sequence (it isn't even bounded).
Of course, the problem here is that our original Cauchy sequence \((a_n)_{n = 1}^{\infty}\) was equivalent to the \emph{zero sequence} \((0)_{n = 1}^{\infty}\) (why? \MAROON{(1)}),
and hence that our real number \(x\) was in fact equal to \(0\).
So we should only allow the operation of reciprocal \emph{when \(x\) is non-zero}.

\begin{proof}
\MAROON{(1)} Suppose for the sake of contradiction, \((a_n)_{n = 1}^{\infty}\) was \emph{not} equivalent to \((0)_{n = 1}^{\infty}\).
Then by \DEF{5.2.6}, there exists \(\varE > 0\) s.t. \textbf{for all \(N \ge 1\)} there exists \(i \ge N\) s.t. \(\abs{a_i - 0} = \abs{a_i} = \abs{10^{-i}} > \varE\), or \(10^{-i} > \varE\), or \(10^i < 1/\varE\).
But since \(10^x\) an increasing function, and \(i \ge N\), we have \(10^N \le 10^i\), so we have \(10^N \le 1/\varE\).
And by \EXEC{4.3.5}, we have \(N \le 10^N\), wo we have \(N \le 1/\varE\).
So we conclude for all \(N \ge 1\), \(N \le 1/\varE\), which contradicts \PROP{4.4.1} that we can always find a natural number \(N\) s.t. \(N \ge 1/\varE\).
So the two sequences must be equivalent.
\end{proof}

However, even when we restrict ourselves to non-zero real numbers, we have a slight problem, because a non-zero real number might be the formal limit of a Cauchy sequence which \emph{contains zero elements}.
For instance, the number \(1\), which is rational and hence real, is the formal limit \(1 = \LIM_{n \toINF} a_n\) of the Cauchy sequence
\[
	0, 0.9, 0.99, 0.999, 0.9999,...
\]
(It can be proved that \(0, 0.9, 0.99,...\) and \(1, 1, 1, ...\) is equivalent.) but using our naive definition of reciprocal, we \emph{cannot} invert the real number \(1\), because we can’t invert the first element \(0\) of this Cauchy sequence!

To get around these problems we need to \emph{keep our Cauchy sequence away from zero}.
To do this we first need a definition.

\begin{definition} [Sequences bounded away from zero]  \label{def 5.3.12}
A sequence \((a_n)_{n = 1}^{\infty}\) of rational numbers is said to be \emph{bounded away from zero} iff there exists a \emph{rational} number \(c > 0\) such that \(\abs{a_n} \ge c\) for all \(n \ge 1\).
\end{definition}

\begin{example} \label{example 5.3.13}
The sequence \(1, -1, 1, -1, 1, -1, 1,...\) is bounded away from zero (all the coefficients have absolute value at least 1). 
But the sequence \(0.1, 0.01, 0.001,...\) is not bounded away from zero(proved by contradiction),
and neither is \(\textbf{0}, 0.9, 0.99, 0.999, 0.9999,....\)
The sequence \(10, 100, 1000,...\) is bounded away from zero, but is not bounded.
\end{example}

We now show that every \emph{non-zero} real number is the formal limit of a Cauchy sequence \emph{bounded away from zero}:
\begin{lemma} \label{lem 5.3.14}
Let \(x\) be a \emph{non-zero} real number.
Then \(x = \LIM_{n \toINF} a_n\) for some Cauchy sequence \((a_n)_{n = 1}^{\infty}\) which is \emph{bounded away from zero}.
\end{lemma}

\begin{proof}
Since \(x\) is real, we know that \(x = \LIM_{n \toINF} b_n\) for some Cauchy sequence \((b_n)_{n = 1}^{\infty}\).
But we are not yet done, because we do not know that \((b_n)_{n = 1}^{\infty}\) is bounded away from zero.

On the other hand, we are given that \(x \ne 0 = \LIM_{n \toINF} 0\), which means that the sequence \((b_n)_{n = 1}^{\infty}\) is \textbf{not} equivalent to \((0)_{n = 1}^{\infty}\).
Thus by \DEF{5.2.6} the sequence \((b_n)_{n = 1}^{\infty}\) \emph{cannot} be eventually \(\varE\)-close to \((0)_{n = 1}^{\infty}\) for every \(\varE > 0\).
Therefore we \emph{can find} an \(\varE >  0\) such that \((b_n)_{n = 1}^{\infty}\) is not eventually \(\varE\)-close to \((0)_{n = 1}^{\infty}\).

So let arbitrary \(\varE > 0\).
Since \((b_n)_{n = 1}^{\infty}\) is Cauchy, it is eventually \(\varE\)-steady.
Moreover, since \(\varE/2\) also \(> 0\) so \((b_n)_{n = 1}^{\infty}\) is eventually \(\varE/2\)-steady.
Thus there is an \(N \ge 1\) such that \(\abs{b_n - b_m} \le \varE/2\) for all \(n, m \ge N\) \MAROON{(1)}.
On the other hand, we cannot have \(\abs{b_n - 0} = \abs{b_n} \le \varE\) for all \(n \ge N\), 
since this would imply that \((b_n)_{n = 1}^{\infty}\) is eventually \(\varE\)-close to \((0)_{n = 1}^{\infty}\).
Thus there must be some \(n_0 \ge N\) for which \(\abs{b_{n_0} - 0} = \abs{b_{n_0}} > \varE\) \MAROON{(2)}.
Now from \MAROON{(1)}, in particular we have \(\abs{b_{n_0} - b_n} \le \varE/2\) for all \(n \ge N\),
and from the triangle inequality:
\begin{align*}
	         & \abs{b_{n_0} - b_n} \le \varE/2 \\
    \implies & -\abs{b_{n_0} - b_n} \ge -\varE/2 \\
    \implies & \abs{b_{n_0}} - \abs{b_{n_0} - b_n} \ge \abs{b_{n_0}} - \varE/2 \\
    \implies & \abs{b_{n_0}} - \abs{b_{n_0} - b_n} \ge \abs{b_{n_0}} - \varE/2 \ge \varE - \varE/2 = \varE/2 & \text{by \MAROON{(2)}}\\
    \implies & \abs{b_{n_0}} - \abs{b_{n_0} - b_n} \ge \varE/2 \\
    \implies & \abs{b_{n_0}} - \abs{b_n - b_{n_0}} \ge \varE/2 \\
    \implies & \abs{b_{n_0} + (b_n - b_{n_0})} \ge \abs{b_{n_0}} - \abs{b_n - b_{n_0}} \ge \varE/2 & \text{by \AC{4.3.1}} \\
    \implies & \abs{b_{n_0} + (b_n - b_{n_0})} \ge \varE/2 \\
    \implies & \abs{b_n} \ge \varE/2
\end{align*}
So \(\abs{b_n} \ge \varE/2\) for all \(n \ge N\).

This almost proves that \((b_n)_{n = 1}^{\infty}\) is bounded away from zero.
Actually, what it does is show that \((b_n)_{n = 1}^{\infty}\) is \emph{eventually} bounded away from zero.
But this is easily fixed, by \emph{defining a new sequence} \(a_n\), by setting \(a_n := \varE/2\) if \(n < N\) and \(a_n := b_n\) if \(n \ge N\).
Since \(b_n\) is a Cauchy sequence, it is not hard to verify that \(a_n\) is also a Cauchy sequence which is \emph{equivalent} to \(b_n\) (because the two sequences are eventually the same),
and so \(x = \LIM_{n \toINF} a_n\).
And since \(\abs{b_n} \ge \varE/2\) for all \(n \ge \textbf{N}\), we know that \(\abs{a_n} \ge \varE/2\) for all \(n \ge \textbf{1}\)
(splitting into the two cases \(n \ge N\) and \(n < N\) separately).
Thus we have a Cauchy sequence \(a_n\) which is bounded away from zero (by \(\varE/2\) instead of \(\varE\), but that’s still OK since \(\varE/2 > 0\)) and which has \(x\) as a formal limit, and so we are done.
\end{proof}

Once a sequence is bounded away from zero, we can take its \emph{reciprocal} without any difficulty:

\begin{lemma} \label{lem 5.3.15}
Suppose that \((a_n)_{n = 1}^{\infty}\) is a Cauchy sequence which is bounded away from zero.
Then the sequence \((a_n^{-1})_{n = 1}^{\infty}\) is also a Cauchy sequence.
\end{lemma}

\begin{proof}
Since \((a_n)_{n = 1}^{\infty}\) is bounded away from zero, by \DEF{5.3.12} we know that there is a \(c > 0\) such that \(\abs{a_n} \ge c\) for all \(n \ge 1\).

Now we need to show that \((a_n^{-1})_{n = 1}^{\infty}\) is eventually \(\varE\)-steady for each \(\varE > 0\).
Thus let arbitrary \(\varE > 0\);
our task is now to find an \(N \ge 1\) such that \(\abs{a_n^{-1} - a_m^{-1}} \le \varE\) for all \(n, m \ge N\).
But
\begin{align*}
    \abs{a_n^{-1} - a_m^{-1}} & = \abs{\frac1{a_n} - \frac1{a_m}} \\
                              & = \abs{\frac{a_m - a_n}{a_n a_m}} \\
                              & = \frac{\abs{a_m - a_n}}{\abs{a_n a_m}} \\
                              & = \frac{\abs{a_m - a_n}}{\abs{a_n}\abs{a_m}} \\
                              & \le \frac{\abs{a_m - a_n}}{c^2} & \text{since \(\abs{a_m}, \abs{a_n} \ge c\)}
\end{align*}
and so to make \(\abs{a_n^{-1} - a_m^{-1}}\) less than or equal to
\(\varE\), it will suffice to make \(\abs{a_m - a_n}\) less than or equal to \(c^2\varE\).

But since \((a_n)_{n = 1}^{\infty}\) is a Cauchy sequence, and \(c^2\varE > 0\), we can certainly find an \(N\) such that the sequence \((a_n)_{n = N}^{\infty}\) is \(c^2\varE\)-steady, 
i.e., \(\abs{a_m - a_n} \le c^2\varE\) for all \(m, n \ge N\).
By what we have said above, this shows that \(\abs{a_n^{-1} - a_m^{-1}} \le \varE\) for all \(m, n \ge N\), and hence the sequence \((a_n^{-1})_{n = 1}^{\infty}\) is eventually \(\varE\)-steady.
Since we have proven this for arbitrary \(\varE > 0\), we have that \((a_n^{-1})_{n = 1}^{\infty}\) is a Cauchy sequence, as desired.
\end{proof}

We are now ready to define reciprocation:
\begin{definition} [Reciprocals of \emph{real} numbers] \label{def 5.3.16}
Let \(x\) be a \emph{non-zero} real number.
Let \((a_n)_{n = 1}^{\infty}\) be a Cauchy sequence bounded away from zero such that \(x = \LIM_{n \toINF} a_n\)
(such a sequence exists by \LEM{5.3.14}).
Then we define the reciprocal \(x^{-1}\) by the formula \(x^{-1} := \LIM_{n \toINF} a_n^{-1}\).
(From \LEM{5.3.15} we know that \(x^{-1}\) is still a real number.)
\end{definition}

And we need to check that the reciprocals of the two different but \emph{equivalent}(in the sense of \DEF{5.2.6}) Cauchy sequences are still equivalent.

\begin{lemma} [Reciprocation is well defined] \label{lem 5.3.17}
Let \((a_n)_{n = 1}^{\infty}\) and \((b_n)_{n = 1}^{\infty}\) be two Cauchy sequences bounded away from zero such that \(\LIM_{n \toINF} a_n = \LIM_{n \toINF} b_n\)
(i.e., the two sequences are \emph{equivalent}).
Then \(\LIM_{n \toINF} a_n^{-1} = \LIM_{n \toINF} b_n^{-1}\).
\end{lemma}

\begin{proof}
Consider the following product \(P\) of three real numbers:
\[
    P := (\LIM_{n \toINF} a_n^{-1}) \X \BLUE{(\LIM_{n \toINF} a_n)} \X (\LIM_{n \toINF} b_n^{-1}).
\]
If we multiply (by \DEF{5.3.9}) this out, we obtain
\[
    P = \LIM_{n \toINF} a_n^{-1} a_n b_n^{-1} = \LIM_{n \toINF} b_n^{-1}.
\]
On the other hand, since \BLUE{\(\LIM_{n \toINF} a_n = \LIM_{n \toINF} b_n\)}, by \PROP{5.3.10}(multiplication is well-defined), we can write \(P\) in another way as
\[
     P = (\LIM_{n \toINF} a_n^{-1}) \X \BLUE{(\LIM_{n \toINF} b_n)} \X (\LIM_{n \toINF} b_n^{-1}).
\]
Multiplying things out again, we get
\[
    P = \LIM_{n \toINF} a_n^{-1} b_n b_n^{-1} = \LIM_{n \toINF} a_n^{-1}.
\]
Comparing our different formulae for \(P\) we see that \(\LIM_{n \toINF} a_n^{-1} = \LIM_{n \toINF} b_n^{-1}\), as desired.
\end{proof}

Thus reciprocal is well-defined (for each non-zero real number \(x\), we have \emph{exactly one definition of} the reciprocal \(x^{-1}\)).

\begin{note}
Note it is clear from the definition that \(xx^{-1} = x^{-1}x = 1\) (from the commutative law in \PROP{5.3.11} we only have to show \(xx^{-1} = 1\)):
\begin{align*}
    xx^{-1} & = \LIM_{n \toINF} a_n \LIM{n \toINF} a_n^{-1} \\
            & = \LIM_{n \toINF} a_n a_n^{-1} & \text{by \DEF{5.3.9}} \\
            & = \LIM_{n \toINF} 1 & \text{by \PROP{4.2.4}(10)} \\
            & = 1 & \text{by equating real to rational}
\end{align*}
Thus with this property, all the \emph{field} axioms (\PROP{4.2.4}) apply to the \emph{reals} as well as to the rationals.
\end{note}

\begin{note}
We of course cannot give \(0\) a reciprocal, since \(0\) multiplied by anything gives \(0\), not \(1\). 
\end{note}

\begin{note}
Also note that if \(q\) is a non-zero \emph{rational}, and hence equal to the real number \(\LIM_{n \toINF} q\),
then the reciprocal of \(\LIM_{n \toINF} q\) is \(\LIM_{n \toINF} q^{-1} = q^{-1}\);
That is,
\begin{align*}
    q^{-1} & = (\LIM_{n \toINF} q)^{-1} & \text{by equating rational and real} \\
           & = \LIM_{n \toINF} q^{-1} & \text{by \DEF{5.3.16}} \\
           & = q^{-1} & \text{by equating rational and real}
\end{align*}
thus the operation of \emph{reciprocal on real} numbers is consistent with the operation of \emph{reciprocal on rational} numbers.
\end{note}

Once one has reciprocal, one can define \emph{division} \(x/y\) of two real numbers \(x, y\), provided \(y\) is non-zero, by the formula
\[
    x/y := x \X y^{-1},
\]
just as we did with the rationals. (We label it as \DEF{5.3.20}.)
In particular, we have the cancellation law (label as \LEM{5.3.21}): if \(x, y, z\) are real numbers such that \(xz = yz\), and \(z\) is non-zero, then by dividing by \(z\) we conclude that \(x = y\).
Note that this cancellation law does not work when \(z\) is zero.

We now have all four of the basic arithmetic operations on the reals: addition, subtraction, multiplication, and division, with all the usual
rules of algebra.

\begin{definition} \label{def 5.3.18}
We define negation \(-x\) for real numbers \(x\) by the formula
\[
    -x := (-1) \X x,
\]
\end{definition}

\begin{definition} \label{def 5.3.19}
We define subtraction for real numbers \(x - y\) by the formula
\[
    x - y := x + (-y).
\]
\end{definition}

\begin{definition} \label{def 5.3.20}
We define division for real numbers \(x/y\), provided \(y \ne 0\), by the formula
\[
    x/y := x \X y^{-1}.
\]
\end{definition}

\begin{lemma} [Cancellation law for real number] \label{lem 5.3.21}
If \(x, y, z\) are real numbers such that \(xz = yz\), and \(z\) is non-zero, then by dividing by \(z\) we conclude that \(x = y\).
\end{lemma}

\exercisesection

\begin{exercise} \label{exercise 5.3.1}
Prove \PROP{5.3.3}.
\end{exercise}

\begin{proof}
See \PROP{5.3.3}.
\end{proof}

\begin{exercise} \label{exercise 5.3.2}
Prove \PROP{5.3.10}.
\end{exercise}

\begin{proof}
See \PROP{5.3.10}.
\end{proof}

\begin{exercise} \label{exercise 5.3.3}
Let \(a, b\) be \emph{rational} numbers.
Show that \(a = b\) if and only if \(\LIM_{n \toINF} a = \LIM_{n \toINF} b\)
(i.e., the Cauchy sequences \(a, a, a, a, ...\) and \(b, b, b, b, ...\) are equivalent if and only if \(a = b\)).
This allows us to \emph{embed the rational numbers inside the real numbers in a well-defined manner}.
\end{exercise}

\begin{proof}
\begin{align*}
         & a = b \\
    \iff & a - b = 0 \\
    \iff & \abs{a - b} = 0 \\
    \iff & \abs{a - b} < \varE\ \forall \varE > 0 \\
    \iff & \abs{a - b} < \varE\ \forall \varE > 0, \forall i \ge 1 & \text{just give a dummy index} \\
    \iff & (a)_{n = 1}^{\infty} = (b)_{n = 1}^{\infty} & \text{by \DEF{5.2.6}} \\
    \iff & \LIM_{n \toINF} a_n = \LIM_{n \toINF} b_n & \text{by \DEF{5.3.1}}
\end{align*}
\end{proof}

\begin{exercise} \label{exercise 5.3.4}
Let \((a_n)_{n = 0}^{\infty}\) be a sequence of rational numbers which is bounded.
Let \((b_n)_{n = 0}^{\infty}\) be another sequence of rational numbers which is \emph{equivalent} to \((a_n)_{n = 0}^{\infty}\).
Show that \((b_n)_{n = 0}^{\infty}\) is also bounded.
(Hint: use \EXEC{5.2.2}.)
\end{exercise}

\begin{proof}
Since \((a_n)_{n = 0}^{\infty}\) and \((b_n)_{n = 0}^{\infty}\) are equivalent, they are eventually \(\varE\)-closes for all \(\varE > 0\).
And since \((a_n)_{n = 0}^{\infty}\) is bounded, the conditions in the \EXEC{5.2.2} are satisfied, and by \EXEC{5.2.2}, \((b_n)_{n = 0}^{\infty}\) is bounded.
\end{proof}

\begin{note}
若\ seq A 是\ bounded,且\ seq B 等價於\ seq A,則\ seq B 也是\ bounded。
\end{note}

\begin{exercise} \label{exercise 5.3.5}
Show that \(\LIM_{n \toINF} 1 / n = 0\).
\end{exercise}

\begin{proof}
We have to show \(\LIM_{n \toINF} 1 / n = 0 = \LIM_{n \toINF} 0\).
Suppose for the sake of contradiction that \(\LIM_{n \toINF} 1 / n \ne \LIM_{n \toINF} 0\).
Then by \DEF{5.2.6}, There exists \(\varE > 0\), such that \textbf{for all \(N \ge 1\)}, there exists \(i \ge N\) s.t. \(\abs{1/i - 0} > \varE\).
And \(\abs{1/i - 0} = \abs{1/i}\) is positive, so we have \(1/i > \varE\), or \(i < 1/\varE\).
But since \(i \ge N\), we have \(N < 1/\varE\).
So we have for all \(N \ge 1\), \(N < 1/\varE\), which contradicts \PROP{4.4.1} that we can always find a natural number \(N\) s.t. \(N > 1/\varE\).
So \(\LIM_{n \toINF} 1 / n = 0\).
\end{proof}