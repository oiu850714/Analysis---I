\section{The integers} \label{sec 4.1}

\begin{definition} [Integers] \label{def 4.1.1}
An integer is an expression of the form \(a \M b\), where \(a\) and \(b\) are natural numbers.
Two integers are \emph{considered}(defined) to be equal, \(a \M b = c \M d\), if and only if \(a + d = c + b\).
We let \(\SET{Z}\) denote the set of all integers.
\end{definition}

\begin{note}
\DEF{4.1.1} 有一段註解在課本該頁下方,在講述\ \(a \M b\) 這種\ expression 跟集合論還有\ equivalence relation 的關係,可以讀一下。
之後建構有理數跟實數時也會有對應的\ equivalence relation 出現。
\end{note}

Thus for instance \(3 \M 5\) is an integer, and is equal to \(2 \M 4\), because \(3 + 4 = 2 + 5\).

This notation is strange looking, and has a few deficiencies; for instance, \(3\) is \emph{not yet an integer}, because it is not of the form \(a \M b\)! We will rectify these problems later(on page 77 of the textbook).

We have to check that the equality in \DEF{4.1.1} is a legitimate notion of equality.
We need to verify the reflexivity, symmetry, transitivity, and substitution axioms \AXM{a.7.4}.
Now we first prove the equivalence relation.

\begin{additional corollary} \label{ac 4.1.1}
The definition of equality on the integers is reflexive, symmetric and transitive.
\end{additional corollary}

\begin{proof}
Reflexive: Let \(a \M b\) be an integer where \(a, b\) are natural numbers.
Then \(\BLUE{a \M b} = \GREEN{a \M b}\) since \(\BLUE{a} + \GREEN{b} = \GREEN{a} + \BLUE{b}\).

Symmetric: Suppose \(a \M b = c \M d\) where \(a, b, c, d\) are natural numbers, we have to show \(c \M d = a \M b\).
From the supposition \(a \M b = c \M d\) and \DEF{4.1.1}, \(a + d = b + c\), and by the equality and addition of natural numbers, we can get \(b + c = a + d\), and \(c + b = a + d\).
By \DEF{4.1.1}, \(c + b = a + d \iff c \M d = a \M b\), so \(c \M d = a \M b\), as desired.

Transitive: Suppose \(a \M b = c \M d\) \MAROON{(1)} and \(c \M d = e \M f\) \MAROON{(2)} where \(a, b, c, d, e, f\) are natural numbers,
we have to show \(a \M b = e \M f\), or equivalently by \DEF{4.1.1}, we have to show \(a + f = e + b\).
By \DEF{4.1.1}, from \MAROON{(1)} we can get \(a + d = c + b\) \MAROON{(3)}, from \MAROON{(2)} we can get \(c + f = e + d\) \MAROON{(4)}.
Now add both sides of \MAROON{(3) (4)} we can get \((a + d) + (c + f) = (c + b) + (e + d)\), from which we can get \((a + f) + (c + d) = (e + b) + (c + d)\).
By cancellation law(\PROP{2.2.6}), we have \(a + f = e + b\), as desired.
\end{proof}

\begin{note}
The \AXM{a.7.4} will be showed for each basic operation for integers, such as addition, multiplication, and order.
More advanced operations will be defined in terms of these basic operations, so automatically satisfy \AXM{a.7.4}
\end{note}

\begin{definition} [Addition and multiplication] \label{def 4.1.2}
The \emph{sum} of two integers, \((a \M b) + (c \M d)\), is defined by the formula
\[
    (a \M b) + (c \M d) := (a + c) \M (b + d).
\]
The \emph{product} of two integers, \((a \M b) \X (c \M d)\), is defined by
\[
    (a \M b) \X (c \M d) := (ac + bd) \M (ad + bc).
\]
\end{definition}

\begin{note}
Both integer addition and multiplication are defined by that meaningless placeholder \(\M\) and operations of natural numbers, so, the definition is not circular.
\end{note}

Thus for instance, \((3 \M 5) + (1 \M 4) = (3 + 1) \M (5 + 4) =  (4 \M 9)\).

Now we show the addition and multiplication of integers satisfy the \AXM{a.7.4}.

\begin{lemma} [Addition and multiplication are well-defined] \label{lem 4.1.3}
Let \(a, b, a', b', c, d\) be natural numbers.
If \((a \M b) = (a' \M b')\), then \((a \M b) + (c \M d) = (a' \M b') + (c \M d)\) and \((a \M b) \X (c \M d) = (a' \M b') \X (c \M d)\),
and also \((c \M d) + (a \M b) = (c \M d) + (a' \M b')\) and \((c \M d) \X (a \M b) = (c \M d) \X (a' \M b')\).
Thus addition and multiplication are well-defined operations (equal inputs give equal outputs).
\end{lemma}

\begin{note}
看起來是要先證明\ operator \textbf{兩邊}都符合替換公理,才可以推導交換律結合律那些性質(列在\ \PROP{4.1.6})。
第二章自然數的時候沒有這樣推導是因為"所有"的自然數的\ operator 的性質基本上都是從數學歸納法開始推的。
\end{note}

\begin{proof}
Suppose \((a \M b) = (a' \M b')\).

\MAROON{(1)}: \((a \M b) + (c \M d) = (a' \M b') + (c \M d)\)
\begin{align*}
         & (a \M b) + (c \M d) = (a' \M b') + (c \M d) \\
    \iff & (a + c) \M (b + d) = (a' + c) \M (b' + d) & \text{by \DEF{4.1.2}} \\
    \iff & (a + c) + (b' + d) = (a' + c) + (b + d) & \text{by \DEF{4.1.1}} \\
    \iff & (a + b') + (c + d) = (a' + b) + (c + d) \GREEN{\ (1)} & \text{by properties of natural number} \\
    \iff & a + b' = a' + b & \text{\(\Rightarrow\) by \PROP{2.2.6}, \(\Leftarrow\) by \AXM{a.7.4}} \\
    \iff & a \M b = a' \M b' & \text{by \DEF{4.1.1}}
\end{align*}
which is true by supposition, so \((a \M b) + (c \M d) = (a' \M b') + (c \M d)\), as desired.
    
\MAROON{(2)}: \((a \M b) \X (c \M d) = (a' \M b') \X (c \M d)\)
\begin{align*}
         & (a \M b) \X (c \M d) = (a' \M b') \X (c \M d) \\
    \iff & (ac + bd) \M (ad + bc) = (a'c + b'd) \M (a'd + b'c) & \text{by \DEF{4.1.2}} \\
    \iff & (ac + bd) + (a'd + b'c) = (a'c + b'd) + (ad + bc) & \text{by \DEF{4.1.1}} \\
    \iff & ac + b'c + a'd + bd = a'c + bc + ad + b'd & \text{by properties of natural numbers} \\
    \iff & c(a + b') + d(a' + b) = c(a' + b) + d(a + b') \GREEN{\ (2)} & \text{by \PROP{2.3.4}, distributive}
\end{align*}
which is true since \(a + b' = a' + b\), or equivalently by \DEF{4.1.1}, \(a \M b = a' \M b'\), so \((a \M b) \X (c \M d) = (a' \M b') \X (c \M d)\), as desired.

\MAROON{(3)}: \((c \M d) + (a \M b) = (c \M d) + (a' \M b')\)
\begin{align*}
         & (c \M d) + (a \M b) = (c \M d) + (a' \M b') \\
    \iff & (c + a) \M (d + b) = (c + a') \M (d + b') & \text{by \DEF{4.1.2}} \\
    \iff & (c + a) + (d + b') = (c + a') + (d + b) & \text{by \DEF{4.1.1}} \\
    \iff & (a + b') + (c + d) = (a' + b) + (c + d) & \text{by properties of natural number}
\end{align*}
which is already shown by \GREEN{(1)}, so \((c \M d) + (a \M b) = (c \M d) + (a' \M b')\), as desired.

\MAROON{(4)}: \((c \M d) \X (a \M b) = (c \M d) \X (a' \M b')\)
\begin{align*}
         & (c \M d) \X (a \M b) = (c \M d) \X (a' \M b')\\
    \iff & (ca + db) \M (cb + da) = (ca' + db') \M (cb' + da') & \text{by \DEF{4.1.2}} \\
    \iff & (ca + db) + (cb' + da') = (ca' + db') + (cb + da) & \text{by \DEF{4.1.1}} \\
    \iff & ca + cb' + da' + db = ca' + cb + da + db' & \text{by properties of natural numbers} \\
    \iff & c(a + b') + d(a' + b) = c(a' + b) + d(a + b') & \text{by \PROP{2.3.4}, distributive}
\end{align*}
which is already shown by \GREEN{(2)}, so \((c \M d) \X (a \M b) = (c \M d) \X (a' \M b')\), as desired.
\end{proof}

\begin{note}
The \emph{integer} \(n \M 0\) \textbf{behave} in the same way as the \emph{natural number} \(n\);
indeed one can check that \((n \M 0) + (m \M 0) = (n + m) \M 0\), and \( (n \M 0) \X (m \M 0) = nm \M0\).
\end{note}

\begin{note}
Furthermore, \((n \M 0)\) is equal to \((m \M 0)\) (if and only if \(n + 0 = m + 0\) by \DEF{4.1.1},) if and only if \(n = m\).
(The mathematical term for this is that there is an \emph{isomorphism} between the \emph{natural} numbers \(n\) and those \emph{integers of the form} \(n \M 0\).
Thus we may \emph{identify} the \emph{natural} numbers with \emph{integers} by \textbf{setting \(n \equiv n \M 0\)}; this does not affect our definitions of addition or multiplication or equality since they are consistent with each other(i.e. addition or multiplication \textbf{behave} in the same way).
And the natural number \(n\) will also be equal to any other integer which is equal to \(n \M 0\),
for instance (natural number) \(3\) is equal not only to (integer) \(3 \M 0\), but also to (integer) \(4 \M 1\), (integer) \(5 \M 2\), etc.
\end{note}

\begin{note}
We can now define incrementation on the \emph{integers} by defining \(x\INC := x+1\) for any \emph{integer} \(x\);
this is of course consistent with our definition of the increment operation for \emph{natural} numbers.
However, this is no longer an important operation for us, as it has been now superseded by the more general notion of addition.
\end{note}

\begin{definition} [Negation of integers] \label{def 4.1.4}
If \((a \M b)\) is an integer, we \emph{define} the \emph{negation} \( -(a \M b) \) to be the integer \((b \M a)\).
In particular if \(n = n \M 0\) is a \emph{positive natural} number, we can define its negation \(-n = 0 \M n\).
\end{definition}

For instance \(-(3 \M 5) = (5 \M 3)\).
One can check this definition is well-defined (\EXEC{4.1.2}).

\begin{additional corollary} \label{ac 4.1.2}
The definition of negation on the integers is well-defined.
\end{additional corollary}

\begin{proof}
Let \(a, b, a', b'\) be arbitrary natural numbers such that \(a \M b = a' \M b'\), we have to show \(-(a \M b) = -(a' \M b')\).
Then
\begin{align*}
         & -(a \M b) = -(a' \M b') \\
    \iff & b \M a = b' \M a' & \text{by \DEF{4.1.4}} \\
    \iff & b + a' = b' + a & \text{by \DEF{4.1.1}} \\
    \iff & a' + b = b' + a & \text{by properties of natural numbers} \\
    \iff & a' \M b' = a \M b & \text{by \DEF{4.1.1}} \\
\end{align*}
which is true by supposition, so \(-(a \M b) = -(a' \M b')\), as desired.
Therefore, \DEF{4.1.4} is well-defined.
\end{proof}

\begin{lemma} [Trichotomy of integers] \label{lem 4.1.5}
Let \(x\) be an integer.
Then \emph{exactly one} of the following three statements is true:
(a) \(x\) is zero;
(b) \(x\) is equal to a \emph{positive} natural number \(n\);
or (c) \(x\) is the negation \(-n\) of a \emph{positive} natural number \(n\).
\end{lemma}

\begin{proof}
We first show that at least one of (a), (b), (c) is true.
By definition, \(x = a \M b\) for some natural numbers \(a\), \(b\).
By trichotomy of natural numbers(\PROP{2.2.13}), We have three cases: \(a > b\), \(a = b\), or \(a < b\).

If \(a > b\) then \(a = b + c\) for some positive natural number \(c\)(by \PROP{2.2.12}(f)),
which is trivially equivalent to \(a + 0 = b + c\),
and by \DEF{4.1.1}, is equivalent to \(a \M b = c \M 0\),
which by the isomorphism of \(c\) and \(c \M 0\), is equal to natural number \(c\), which is positive,
so \(x = a \M b\) is equal to a positive natural number \MAROON{(1)}, satisfying (b).

If \(a = b\), then \(a \M b = a \M a\), which is by \DEF{4.1.1} trivially equal to \(0 - 0\),
which by the isomorphism of \(0\) and \(0 \M 0\) is equal to the natural number \(0\).
So \(a \M b\) is equal to the natural number \(0\), satisfying (a).

If \(a < b\) then \(b > a\), and by reasoning in \MAROON{(1)}, \(b \M a = n\) for some positive natural number \(n\),
and thus by \DEF{4.1.4}, \(a \M b = -(b \M a)\), which \(= -n\), satisfying (c).

Now we show that no more than one of (a), (b), (c) can hold at a
time.
That is, (a), (b) cannot be simultaneously true, and (a), (c) cannot be simultaneously true, and (b), (c) cannot be simultaneously true.

By \DEF{2.2.7} (a), (b) cannot be simultaneously true, otherwise \(x\) is equal to the natural number \(0\) and a positive natural number, contradicting \DEF{2.2.7}.

Suppose (a), (c) are both true, then \(x = 0\) by (a) and also \(x = -n\) for some positive natural number \(n\) \MAROON{(2)},
So \(0 = -n\).
By the isomorphism of \(c\) and \(c - 0\) for any natural number \(c\), that can be written as \(0 - 0 = -n\), \(0 - 0 = -(n - 0)\), \(0 - 0 = 0 - n\),
from which by \DEF{4.1.1} we can get \(0 + n = 0 + 0\), that is, \(n = 0\), that is, \(n\) is not positive \MAROON{(3)}, contradicting \MAROON{(2)},
so (a), (c) are not both true.

Suppose (b), (c) are both true, then by (b) \(x\) is equal to a positive natural number \(m\), by (c) \(x = -n\) for some positive natural number \(n\), so \(m = -n\).
By the isomorphism of \(c\) and \(c - 0\) for any natural number \(c\), that can be written as \(m - 0 = -n\), \(m - 0 = -(n - 0)\), \(m - 0 = 0 - n\),
from which by \DEF{4.1.1} we can get \(m + n = 0 + 0 = 0\). But \(m, n\) are both positive integer, so by \PROP{2.2.8} \(m + n\) is positive, a contradiction.
So (b), (c) are not true.
\end{proof}

\begin{note}
If \(n\) is a \emph{positive} \emph{natural} number, we call \(n\) a \emph{positive} integer and \(-n\) a \emph{negative} integer.
Thus every integer is positive, zero, or negative, but not more than one of these at a time.
\end{note}

\begin{note}
One could well ask why we don’t use \LEM{4.1.5} to \emph{define} the integers;
i.e., why didn’t we just say an integer is anything which is either a positive natural number, zero, or the negative of a natural number.
The reason is that if we did so, the rules for adding and multiplying integers would split into many different cases
(e.g., negative times positive equals negative; negative plus positive is either negative, positive, or zero, depending on which term is larger, etc.)
and to verify all the properties would end up being much messier.
\end{note}

We now summarize the algebraic properties of the integers.

\begin{proposition} [Laws of algebra for integers] \label{prop 4.1.6}
Let \(x, y, z\) be integers.
Then we have
\begin{align*}
          x + y & = y + x \MAROON{\ (1)} \\
    (x + y) + z & = x + (y + z) \MAROON{\ (2)} \\
          x + 0 & = 0 + x = x \MAROON{\ (3)} \\
       x + (-x) & = (-x) + x = 0 \MAROON{\ (4)} \\
             xy & = yx \MAROON{\ (5)} \\
          (xy)z & = x(yz) \MAROON{\ (6)} \\
        x1 = 1x & = x \MAROON{\ (7)} \\
       x(y + z) & = xy + xz \MAROON{\ (8)} \\
       (y + z)x & = yx + zx \MAROON{\ (9)}
\end{align*}
\end{proposition}

\begin{remark} \label{remark 4.1.7}
The above set of nine identities have a name; they are asserting that the integers form a \emph{commutative ring}.
(If one deleted the identity \(xy = yx\), then they would only assert that the integers form a \emph{ring}).
Note that some of these identities were already proven for the natural numbers, \emph{but this does not automatically mean that they also hold for the integers} because the integers are a larger set than the natural numbers.
On the other hand, this proposition supersedes many of the propositions derived earlier for natural numbers.
\end{remark}

\begin{proof}
The book says you can use \LEM{4.1.5} to prove these identities, but that will split into many many cases depending on whether \(x, y, z\) are zero, positive, or negative.
This becomes very messy.
A shorter way is to write \(x =(a \M b), y = (c \M d)\), and \(z = (e \M f)\) for some natural numbers \(a, b, c, d, e, f\), and expand these identities in terms of \(a, b, c, d, e, f\) and \emph{use the algebra of the natural numbers}.

So let \(x =(a \M b), y = (c \M d)\), and \(z = (e \M f)\) for some natural numbers \(a, b, c, d, e, f\).

\MAROON{(1)} \(x + y = y + x\):
\begin{align*}
    x + y & = (a \M b) + (c \M d) \\
          & = (a + c) \M (b + d) & \text{by \DEF{4.1.2}} \\
          & = (c + a) \M (d + b) & \text{by \PROP{2.2.4}} \\
          & = (c \M d) + (a \M b) & \text{by \DEF{4.1.2}} \\
          & = y + x
\end{align*}

\MAROON{(2)} \((x + y) + z = x + (y + z)\):
\begin{align*}
    (x + y) + z & = ((a \M b) + (c \M d)) + (e \M f) \\
                & = ((a + c) \M (b + d)) + (e \M f) & \text{by \DEF{4.1.2}} \\
                & = ((a + c) + e) \M ((b + d) + f) & \text{by \DEF{4.1.2}} \\
                & = (a + (c + e)) \M (b + (d + f)) & \text{by \PROP{2.2.5}} \\
                & = (a \M b) + ((c + e) \M (d + f)) & \text{by \DEF{4.1.2}} \\
                & = (a \M b) + ((c \M d) + (e \M f)) & \text{by \DEF{4.1.2}} \\
                & = x + (y + z)
\end{align*}

\MAROON{(3)} \(x + 0 = 0 + x = x\):
By \MAROON{(1)} we have \(x + 0 = 0 + x\), so we only have to show \(x + 0 = x\).
Then
\begin{align*}
    x + 0 & = (a \M b) + 0 \\
          & = (a \M b) + (0 - 0) & \text{since (natural) \(0 \equiv (0 \M 0)\) (integer)} \\
          & = (a + 0) \M (b + 0) & \text{by \DEF{4.1.2}} \\
          & = a \M b & \text{by \LEM{2.2.2}} \\
          & = x
\end{align*}

\MAROON{(4)} \(x + (-x) = (-x) + x = 0\):
By \MAROON{(1)} we have \(x + (-x) = (-x) + x\), so we only have to show \(x + (-x) = 0\).
Then
\begin{align*}
    x + (-x) & = (a \M b) + -(a \M b) \\
             & = (a \M b) + (b \M a) & \text{by \DEF{4.1.4}} \\
             & = (a + b) \M (b + a) & \text{by \DEF{4.1.2}} \\
             & = (a + b) \M (a + b) & \text{by \PROP{2.2.4}, commutative} \\
             & = (a \M a) + (b \M b) & \text{by \DEF{4.1.2}} \\
             & = (0 \M 0) + (0 \M 0) & \text{by \DEF{4.1.1}} \\
             & = 0 + 0 & \text{since (natural) \(0 \equiv (0 \M 0)\) (integer)} \\
             & = 0 & \text{by \DEF{2.2.1}}
\end{align*}

\MAROON{(5)} \(xy = yx\):
\begin{align*}
    xy & = (a \M b)(c \M d) \\
       & = (ac + bd) \M (ad + bc) \text{\ \GREEN{(1)}} & \text{by \DEF{4.1.2}}
\end{align*}
And 
\begin{align*}
    yx & = (c \M d)(a \M b) \\
       & = (ca + db) \M (cb + da) & \text{by \DEF{4.1.2}} \\
       & = (ac + bd) \M (ad + bc) & \text{by \PROP{2.2.4}, \PROP{2.2.5}} \\
       & = \text{\GREEN{(1)}}
\end{align*}
So \(xy = yx\).

\MAROON{(6)} \((xy)z = x(yz)\):
\begin{align*}
    (xy)z & = ((a \M b)(c \M d))(e \M f) \\
          & = ((ac + bd) \M (ad + bc))(e \M f) & \text{by \DEF{4.1.2}} \\
          & = ((ace + bde + adf + bcf) \M (acf + bdf + ade + bce)) \text{\ \GREEN{(2)}} & \text{by \DEF{4.1.2}}
\end{align*}
And
\begin{align*}
    x(yz) & = (a \M b)((c \M d)(e \M f)) \\
          & = (a \M b)((ce + df) \M (cf + de)) & \text{by \DEF{4.1.2}} \\
          & = ((ace + adf + bcf + bde) \M (acf + ade + bce + bdf)) & \text{by \DEF{4.1.2}} \\
          & = ((ace + bde + adf + bcf) \M (acf + bdf + ade + bce)) \\
          & = \text{\GREEN{(2)}}
\end{align*}
so \((xy)z = x(yz)\)

\MAROON{(7)} \(x1 = 1x = x\):
Again by \MAROON{(1)}, we only have to show \(x1 = x\).
\begin{align*}
    x1 & = (a \M b) \X 1 \\
       & = (a \M b) \X (1 - 0) & \text{since (natural) \(1 \equiv 1 - 0\) (integer)} \\
       & = (a1 + b0) \M (a0 + b1) & \text{by \DEF{4.1.2}} \\
       & = a \M b & \text{by properties of natural numbers} \\
       & = x
\end{align*}

\MAROON{(8)} \(x(y + z) = xy + xz\):
\begin{align*}
    x(y + z) & = (a \M b)((c \M d) + (e \M f)) \\
             & = (a \M b)((c + e) \M (d + f)) & \text{by \DEF{4.1.2}} \\
             & = (a(c + e) + b(d + f)) \M (b(c + e) + a(d + f)) & \text{by \DEF{4.1.2}} \\
             & = (ac + ae + bd + bf) \M (bc + be + ad + af)) \text{ \GREEN{(4)}} & \text{by \PROP{2.3.4}}
\end{align*}
And
\begin{align*}
    xy + xz & = (a \M b)(c \M d) + (a \M b)(e \M f) \\
            & = ((ac + bd) \M (ad + bc)) + ((ae + bf) \M (af + be)) & \text{by \DEF{4.1.2}} \\
            & = (ac + bd + ae + bf) \M (ad + bc + af + be) & \text{by \DEF{4.1.2}} \\
            & = \GREEN{(4)}
\end{align*}
So \(x(y + z) = xy + xz\).

\MAROON{(9)} \((y + z)x = yx + zx\):
\begin{align*}
    (y + z)x & = x(y + z) & \text{by \MAROON{(1)}} \\
             & = xy + xz & \text{by \MAROON{(8)}} \\
             & = yx + zx & \text{by \MAROON{(1)}}
\end{align*}
\end{proof}

\begin{note}
仔細觀察\ \PROP{4.1.6},這些都不需要整數減法的定義(只需要那個\  meaningless placeholder 的定義)。
\end{note}

\begin{note}
Below refer to \href{https://github.com/ProFatXuanAll/terence-tao-analysis/blob/33a8568dce2474f82e381dcb1309f9ce685ce2f9/Analysis-I/4-1-the-integers.tex#L320-L358}{ProFatXuanAll's repo}
\end{note}

\begin{additional corollary} \label{ac 4.1.3}
Let \(x\) be an integer.
Then \(-x = (-1)x\).
\end{additional corollary}

\begin{proof}
Let \(x = (a \M b)\) where \(a, b\) are natural numbers.
Then
\begin{align*}
    -x & = -(a \M b) \\
       & = (b \M a) & \text{by \DEF{4.1.4}} \\
       & = (1b \M 1a) & \text{by \AC{2.3.4}} \\
       & = (0 + 1b) \M (0 + 1a) & \text{by \DEF{2.2.1}} \\
       & = (0a + 1b) \M (0b + 1a) & \text{by \LEM{2.3.3}} \\
       & = (0 \M 1)(a \M b) & \text{by \DEF{4.1.2}} \\
       & = (-(1 \M 0))(a \M b) & \text{by \DEF{4.1.4}} \\
       & = (-1)(a \M b) & \text{since (natural number) \(-1 \equiv -1 \M 0\) (integer)} \\
       & = (-1)x.
\end{align*}
\end{proof}

\begin{additional corollary} \label{ac 4.1.4}
Let \(x\) be an integer.
Then \(x = -(-x)\).
\end{additional corollary}

\begin{proof}
Let \(x = (a \M b)\) where \(a, b\) are natural numbers.
Then
\begin{align*}
    -(-x) & = -(-(a \M b)) \\
          & = -(b \M a) & \text{by \DEF{4.1.4}} \\
          & = (a \M b)  & \text{by \DEF{4.1.4}} \\
          & = x.
\end{align*}
\end{proof}

\begin{additional corollary} \label{ac 4.1.5}
Let \(x, y\) be integers.
Then \((-x)(-y) = xy\).
\end{additional corollary}

\begin{proof}
\begin{align*}
    (-x)(-y) & = ((-1)x)((-1)y) & \text{by \AC{4.1.3}} \\
             & = (-1)(x(-1))y & \text{by \PROP{4.1.6}(6))} \\
             & = (-1)((-1)x)y & \text{by \PROP{4.1.6}(5)} \\
             & = ((-1)(-1))(xy) & \text{by \PROP{4.1.6}(6)} \\
             & = (-(-1))(xy) & \text{by \AC{4.1.3}} \\
             & = 1(xy) & \text{by \AC{4.1.4}} \\
             & = xy & \text{by \PROP{4.1.6}(7)}
\end{align*}
\end{proof}

Finally, we now \textbf{define the operation of subtraction}. See \DEF{4.1.12}.

We do not need to verify the substitution axiom \AXM{a.7.4} for this operation, since we have defined subtraction \emph{in terms of two other operations on integers}, namely \emph{addition} and \emph{negation}, and we have already verified that those
operations are well-defined(\LEM{4.1.3} and \AC{4.1.2}).

One can easily check now that if \(a\) and \(b\) are \emph{natural numbers} then
\begin{align*}
      a - b & = a + (-b) & \text{by \DEF{4.1.12}} \\
            & = (a \M 0) + (-b) & \text{since (natural) \(a \equiv a \M 0\) (integer)} \\
            & = (a \M 0) + (-(b \M 0)) & \text{since (natural) \(b \equiv b \M 0\) (integer)} \\
            & = (a \M 0) + (0 \M b) & \text{by \DEF{4.1.4}} \\
            & = (a + 0) \M (0 + b) & \text{by \DEF{4.1.2}} \\
            & = a \M b & \text{trivial}
\end{align*}
so \(a \M b\) is just the same thing as \(a - b\).
Because of this we can now discard the \(\M\) notation, and use the familiar operation of subtraction instead.

\begin{note}
Someone else (and me) do not quite get this argument. Refer to \href{https://math.stackexchange.com/questions/3976309/terence-taos-definition-of-subtraction-operation-to-build-integers/3976320}{this} or \href{https://www.wikiwand.com/en/Integer#/Construction}{wiki}.
\end{note}

We can now generalize \LEM{2.3.3} and \CORO{2.3.7} from the \emph{natural} numbers to the \emph{integers}:

\begin{proposition} [Integers have no zero divisors] \label{prop 4.1.8}
Let \(x\) and \(y\) be integers such that \(xy = 0\).
Then either \(x = 0\) or \(y = 0\) (or both).
\end{proposition}

\begin{proof}
Suppose \(xy = 0\) but both \(x \neq 0\) and \(y \neq 0\).
By \LEM{4.1.5}, there are four cases:

\MAROON{(1)} both \(x = a,  y = b\) for some positive natural numbers \(a, b\).
Then by \PROP{2.2.8}, \(xy = ab \neq 0\), a contradiction.

\MAROON{(2)} \(x = a, y = -b\) for some positive natural numbers \(a, b\)
Then
\begin{align*}
    xy & = a \X -b \\
       & = (a - 0) \X -b & \text{by \(a \equiv a - 0\)} \\
       & = (a - 0) \X -(b - 0) & \text{by \(b \equiv b - 0\)} \\
       & = (a - 0) \X (0 - b) & \text{by \DEF{4.1.4}} \\
       & = (a0 + 0b) - (ab + 00) & \text{by \DEF{4.1.2}} \\
       & = 0 - ab & \text{by properties of natural numbers} \\
       & = -(ab - 0) & \text{by \DEF{4.1.4}} \\
       & = -(ab) & \text{\(ab \equiv ab - 0\)}
\end{align*}
So \(xy = -(ab)\), where \(ab\) by \PROP{2.2.8} is a positive natural number.
So by \LEM{4.1.5}, \(xy \neq 0\), a contradiction.

\MAROON{(3)} \(x = -a, y = b\) for some positive natural numbers \(a, b\).
But since by \PROP{4.1.6}(5), \(xy = yx\), and \(yx\) falls back to case \MAROON{(2)}, so \(xy = yx \neq 0\), a contradiction.

\MAROON{(4)} \(x = -a, y = -b\) for some positive natural numbers \(a, b\).
\begin{align*}
    xy & = -a \X -b \\
       & = -(a - 0) \X -b & \text{\(a \equiv a - 0\)} \\
       & = -(a - 0) \X -(b - 0) & \text{\(b \equiv b - 0\)} \\
       & = (0 - a) \X (0 - b) & \text{by \DEF{4.1.4}} \\
       & = (00 + ab) - (a0 + 0b) & \text{by \DEF{4.1.2}} \\
       & = ab - 0 & \text{by properties of natural numbers} \\
       & = ab & \text{\(ab \equiv ab - 0\)}
\end{align*}
So \(xy = ab \neq 0\), a contradiction.

So all cases lead to contradiction, so at least one of \(x\) or \(y\) must be \(0\).
\end{proof}

\begin{corollary} [Cancellation law for integers] \label{corollary 4.1.9}
If \(a, b, c\) are integers such that \(ac = bc\) and \(c\) is non-zero, then \(a = b\).
\end{corollary}

\begin{proof}
\begin{align*}
             & ac = bc \\
    \implies & ac + -(ac) = bc + -(ac) \\
    \implies & 0 = bc + -(ac) & \text{by \PROP{4.1.6}(4)} \\
    \implies & 0 = bc + (-1)(ac) & \text{by \AC{4.1.3}} \\
    \implies & 0 = bc + ((-1)a)c & \text{by \PROP{4.1.6}(6)} \\
    \implies & 0 = bc + (-a)c & \text{by \AC{4.1.3}} \\
    \implies & 0 = (b + (-a))c & \text{by \PROP{4.1.6}(9)} \\
    \implies & 0 = b + (-a) & \text{by \PROP{4.1.8} and \(c \neq 0\)} \\
    \implies & 0 + a = (b + (-a)) + a \\
    \implies & a = (b + (-a)) + a & \text{by \PROP{4.1.6}(3)} \\
    \implies & a = b + ((-a) + a) & \text{by \PROP{4.1.6}(2)} \\
    \implies & a = b + 0 & \text{by \PROP{4.1.6}(4)} \\
    \implies & a = b & \text{by \PROP{4.1.6}(3)}
\end{align*}
\end{proof}

We now extend the notion of order, which was defined on the natural numbers, to the integers by repeating the definition verbatim:

\begin{definition} [Ordering of the integers] \label{def 4.1.10}
Let \(n\) and \(m\) be integers.
We \emph{say} that \(n\) is greater than or equal to \(m\), and write \(n \ge m\) or \(m \le n\), iff we have \(n = m + a\) for some \textbf{natural} number \(a\).
We say that \(n\) is strictly greater than \(m\), and write \(n > m\) or \(m < n\), iff \(n \ge m\) and \(n \neq m\).
\end{definition}

\begin{note}
Clearly this definition is consistent with the notion of order on the \emph{natural} numbers, since we are using the same definition.
\end{note}

\begin{lemma} [Properties of order] \label{lem 4.1.11}
Let \(a, b, c\) be integers.
\begin{enumerate}
    \item \(a > b\) if and only if \(a - b\) is a positive natural number.
    \item (Addition preserves order) If \(a > b\), then \(a + c > b+ c\).
    \item (\emph{Positive} multiplication preserves order) If \(a > b\) and \(c\) is positive, then \(ac > bc\).
    \item (Negation reverses order) If \(a > b\), then \(-a < -b\).
    \item (Order is transitive) If \(a > b\) and \(b > c\), then \(a > c\).
    \item (Order trichotomy) Exactly one of the statements \(a > b\), \(a < b\), or \(a = b\) is true.
\end{enumerate}
\end{lemma}

\begin{proof}
\begin{enumerate}
\item \MAROON{(1)}
    \begin{align*}
             & a > b \\
        \iff & a \ge b \land a \neq b & \text{by \DEF{4.1.10}} \\
        \iff & \exists c \in \SET{N}\ a = b + c \land a \neq b \land c \neq 0 & \text{by \DEF{4.1.10}, and \(c \neq 0\) otherwise \(a = b\)} \\
        \iff & a - b = b + c - b \\
        \iff & a - b = b + c + (-b) & \text{by \DEF{4.1.12}} \\
        \iff & a - b = b + (-b) + c & \text{by \PROP{4.1.6}(1)} \\
        \iff & a - b = 0 + c = c & \text{by \PROP{4.1.6}(4)(3)} \\
    \end{align*}
    \(\iff a - b\) is equal to a positive natural number.

\item
    \begin{align*}
                 & a > b \\
        \implies & \exists d \in \SET{N}\ a = b + d \land a \neq b \land d \neq 0 & \text{same argument in \MAROON{(1)}} \\
        \implies & a + c = b + d + c \\
        \implies & a + c = b + c + d & \text{by \PROP{4.1.6}(1)} \\
        \implies & a + c > b + c & \text{by \DEF{4.1.10}}
    \end{align*}

\item
    \begin{align*}
                 & a > b \\
        \implies & \exists d \in \SET{N}\ a = b + d \land a \neq b \land d \neq 0 & \text{same argument in \MAROON{(1)}} \\
        \implies & ac = (b + d)c \\
        \implies & ac = bc + dc & \text{by \PROP{4.1.6}(9)} \\
        \implies & ac - bc = bc + dc - bc \\
        \implies & ac - bc = dc & \text{by \PROP{4.1.6}(3)(4), \DEF{4.1.4}}
    \end{align*}
    where \(dc\) is a positive natural number because of \LEM{2.3.3}, and both \(d, c\) are equal to a positive natural number.
    So by \MAROON{(1)}, \(ac > bc\).

\item
    \begin{align*}
                 & a > b \\
        \implies & \exists d \in \SET{N}\ a = b + d \land a \neq b \land d \neq 0 & \text{same argument in \MAROON{(1)}} \\
        \implies & (-a) = -(b + d) \\
        \implies & (-a) = (-1)(b + d) & \text{by \AC{4.1.3}} \\
        \implies & (-a) = (-1)b + (-1)d & \text{by \PROP{4.1.6}(8)} \\
        \implies & (-a) = (-b) + (-d) & \text{by \AC{4.1.3}} \\
        \implies & (-a) + d = (-b) + (-d) + d \\
        \implies & (-a) + d = (-b) + 0 & \text{by \PROP{4.1.6}(4)} \\
        \implies & (-a) + d = (-b) & \text{by \PROP{4.1.6}(3)} \\
        \implies & (-a) + d + (-(-a)) = (-b) + (-(-a)) \\
        \implies & d = (-b) + (-(-a)) & \text{by \PROP{4.1.6}(3)(4)} \\
        \implies & d = (-b) - (-a) & \text{by \DEF{4.1.12}}
    \end{align*}
    So \((-b) - (-a)\) is a positive integer. So by \MAROON{(1)}, \(-b > -a\).

\item (Order is transitive) If \(a > b\) and \(b > c\), then \(a > c\):
    \begin{align*}
                 & a > b \land b > c \\
        \implies & a - b = d_1 \land b - c = d_2 \text{\ where \(d_1, d_2\) is positive} & \text{same argument in \MAROON{(1)}} \\
        \implies & (a - b) + (b - c) = d_1 + d_2 \\
        \implies & a + (-b) + b + (-c) = d_1 + d_2 & \text{by \DEF{4.1.12}} \\
        \implies & a + (-c) = d_1 + d_2 & \text{by \PROP{4.1.6}(3)(4)} \\
        \implies & a - c = d_1 + d_2 & \text{by \DEF{4.1.12}}
    \end{align*}
    where \(d_1 + d_2\) by \PROP{2.2.8} is positive.
    So \(a - c\) is positive, so by \MAROON{(1)}, \(a > c\).

\item
    Given any two natural numbers \(a, b\), by \LEM{4.1.5}, \emph{exactly one of} the three cases below is true:
    \begin{enumerate}
        \item \(a - b = 0\): then
            \begin{align*}
                     & a - b = 0 \\
                \iff & a - b + b = 0 + b \\
                \iff & a + (-b) + b = 0 + b & \text{by \DEF{4.1.12}} \\
                \iff & a = b & \text{by \PROP{4.1.6}(3)(4)}
            \end{align*}
            
            Suppose for the sake of contradiction that \(a > b\) or \(b > a\), then by \MAROON{(1)} \(a - b\) or \(b - a\) is a positive natural number, contradicting \(a - b = 0\).
        \item \(a - b = n\) where \(n\) is a positive natural number:
            Then by \MAROON{(1)}, \(a - b\) is equal to a positive natural number if and only if \(a > b\).
            
            It is trivial that the supposition of \(a = b\) or \(a < b\) lead to contradiction.
        \item \(a - b = -n\) where \(n\) is a positive natural number:
            \begin{align*}
                     & a - b = -n \\
                \iff & -(a - b) = -(-n) \\
                \iff & b - a = -(-n) & \text{by \DEF{4.1.4}} \\
                \iff & b - a = n & \text{by \AC{4.1.4}} \\
                \iff & b > a & \text{by \MAROON{(1)}}
            \end{align*}
            It is trivial that the supposition of \(a = b\) or \(a > b\) lead to contradiction.
    \end{enumerate}
\end{enumerate}
\end{proof}

\begin{note}
The book does not label the definition of subtraction of integers, so I define it below.
\end{note}
\begin{definition} [on page 79] \label{def 4.1.12}
We now define the operation of subtraction \(x - y\) of two integers by the formula
\[
    x - y := x + (-y)
\]
\end{definition}


\exercisesection

\begin{exercise} \label{exercise 4.1.1}
Verify that the definition of equality on the integers is both reflexive and symmetric.
\end{exercise}

\begin{proof}
See \AC{4.1.1}.
\end{proof}

\begin{exercise} \label{exercise 4.1.2}
Show that the definition of negation on the integers is well-defined in the sense that if \((a \M b) = (a' \M b')\), then \(-(a \M b) = -(a' \M b')\)
(so equal integers have equal negations).
\end{exercise}

\begin{proof}
See \AC{4.1.2}.
\end{proof}

\begin{exercise} \label{exercise 4.1.3}
Show that \((-1) \X a = -a\) for every integer \(a\).
\end{exercise}

\begin{proof}
See \AC{4.1.3}.
\end{proof}

\begin{exercise} \label{exercise 4.1.4}
Prove the remaining identities in \PROP{4.1.6}.
\end{exercise}

\begin{proof}
See \PROP{4.1.6}.
\end{proof}

\begin{exercise} \label{exercise 4.1.5}
Prove \PROP{4.1.8}.
\end{exercise}

\begin{proof}
See \PROP{4.1.8}.
\end{proof}

\begin{exercise} \label{exercise 4.1.6}
Prove \CORO{4.1.9}.
\end{exercise}

\begin{proof}
See \CORO{4.1.9}.
\end{proof}

\begin{exercise} \label{exercise 4.1.7}
Prove \LEM{4.1.11}.
\end{exercise}

\begin{proof}
See \LEM{4.1.11}.
\end{proof}

\begin{exercise} \label{exercise 4.1.8}
Show that the principle of induction (\AXM{2.5}) does not apply directly to the integers.
More precisely, give an example of a property \(P(n)\) pertaining to an \emph{integer} \(n\) such that \(P(0)\) is true, and that \(P(n)\) implies \(P(n\INC)\) for all integers \(n\), but that \(P(n)\) is not true for all \emph{integers} \(n\).
Thus induction is not as useful a tool for dealing with the \emph{integers} as it is with the \emph{natural} numbers.
(The situation becomes even worse with the rational and real numbers, which we shall define shortly.)
\end{exercise}

\begin{proof}
Let \(P(n) := n \text{ is equal to a positive natural number or \(0\)}\).
    
Then \(P(0)\) is true.
    
Suppose \(P(n)\) is true, then \(n\) is positive or \(0\); if \(n = 0\), and \(n\INC = 1\), which is positive, if \(n\) is positive, then \(n\INC = n + 1\) is positive by \PROP{2.2.8}. So in both cases \(n\INC\) is positive, so \(P(n\INC)\) is true.
    
But \(P(-1)\) is false because if \(n = -1\), then by \LEM{4.1.5} \(n\) cannot be equal to \(0\) or a positive natural number.
\end{proof}